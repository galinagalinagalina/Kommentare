\section{Positive Aspekte der Kommentarfunktion}

Was macht Kommentare wertvoll und was bringen sie eigentlich? Dieser Frage wird
im folgenden Abschnitt nachgegangen.


\subsection{Popularität der Interaktion}

Obwohl nur ein kleiner Teil der Leser Kommentare verfasst und ein ebenso kleiner
Teil diese auch liest (Reich, 2011, S. 96ff), schlägt diese Art der
Nutzerbeteiligung wie eine Bombe ein. Das merken die Redaktionen, die von nicht
mehr zu bewältigenden Kommentaren erreicht werden. Dies ist dem Umstand
geschuldet, dass es auf einmal (2005 entstehen die ersten Kommentare) die
Möglichkeit gibt, rauszulassen, was einem gerade durch den Kopf geht beim Lesen
eines Artikels.

Es ist aber nicht nur die Gelegenheit spontan etwas zum eben Gelesenen zu sagen,
was die Kommentare attraktiv macht. Menschen haben selten die Chance, ihren
Unmut kund zu tun oder Zustimmung auszusprechen, weil sie auf das nur Zuhören
beschränkt sind (Reich, 2011, S. 99). Diese auferlegte Passivität wird
aufgehoben.

Für manche Leser werden die Kommentare sogar genauso wichtig, wie die Nachricht
selbst. Und manchen Leser gefällt es einfach, sich selbst veröffentlicht zu
sehen.


\subsection{Öffentlicher Diskurs}

Eine ganz wichtige Rolle nehmen die Kommentare als ein Platz oder Ort ein, wo
sich jeder zum Diskutieren treffen und wo man auch die Breite und
Verschiedenheit der Ansichten erfahren kann. Für den demokratischen Prozess ist
es wichtig, dass bestimmte Themen besprochen werden. Jede Stimme soll gehört
werden, auch diejenigen, die sonst leicht untergehen, wie Minderheiten oder
Leute, die sich nicht trauen, oder weil es an Unterstützung fehlt oder weil die
Angelegenheit einfach kontrovers ist (Santana, 2014, S. 12). Außerdem gibt es erst mal 
keine Beschränkungen, wer mit sprechen darf und wer nicht.
Zu einer gesunden Demokratie gehört das einfach dazu: 
``By allowing citizens to participate,
journalists behave ethically, and hence they democratize journalism and
the web.'' (Singer, 2011, S. 125)

Dieser neu geschaffene öffentliche Ort soll auch dazu dienen, herauszufinden,
wie die Leute im Moment „ticken“ (``serving as a gauge of society’s pulse“ (Loke,
2013, S. 181). Man kann herauslesen, wo die neuralgischen Punkte in der Gesellschaft liegen und
auf diesen Zug ausspringen, wenn nötig.



Über die Kommentare können Einspruch erhoben oder Bedenken geäußert werden.
Ebenso funktionieren Kommentare als Anregung für weitere Diskussionen oder
greifen korrigierend  in die Redaktionen ein. Auf diesem Weg erhalten die
Journalisten auch direktes Feedback: „Comments can confirm that the website is
doing a good job [\ldots] they can help improve accuracy [\ldots]“ (Reich, 2011,
S. 105).

Ufern die Diskussionen jedoch ins Unendliche aus oder werden aggressiv und
beleidigend, dann wird aus diesem liberalen Aspekt ein Problem der Kommentare.
„[\ldots] free expression and exposure to differing views can hold deliberative
potential only when participants were respectful toward each other“ (Santana,
2011, S. 7). Das nächste Kapitel wird sich damit beschäftigen.


\subsection{Neue oder zusätzliche Inhalte}

Die Kommentierenden begeistern sich für ihr Thema, lassen sich als Experten
erkennen, liefern Hinweise und sprechen ganz einfach aus, was ihnen wichtig ist.
Dort kann der Journalist ansetzen und solche Beiträge als ``journalistisches Werkzeug'' 
(Robinson, 2010, S. 133) benutzen .

Kommentare dienen als Informationsquelle oder als Inspiration für Themen.
Manche Journalisten sehen sie als die Möglichkeit, Kontext herzustellen (durch Hyperlinks) 
oder um bestimmte Punkte zu klären. 
Andere Perspektiven tun sich auf und der Journalist wird sensibilisiert, was die
Leser hören wollen (Santana, 2014, S.12).

Dafür müssen die Schreiber jedoch alle Kommentare durchforsten: entweder nur von
ihren eigenen Texten oder von der gesamten Redaktion. Es liegt nahe, dass dies
sehr schnell ausarten kann. \glqq If a journalist were to read all 300 comments, he
wouldn’t be able to do anything else [\ldots]“ (Singer, S. 84). 
So kommt es, dass nur wenige Journalisten Kommentare überhaupt lesen.\footnote{Only 35
percent of national journalists and 36 percent of local journalists said
they have a positive view of citizens posting news content on news
organizations’ websites. [\ldots] Even fewer said they take the time to read the
comments and submissions they receive. (Meyer \& Carey, 2014, S. 216)‚}
Das kommt auch, weil sie eine schlechte Meinung darüber entwickeln.\footnote{The study
finds that the more comments a news website receives, the more negative
attitudes the journalists who work at those sites have toward the value and
utility of comments.}


\subsection{Leserbindung}

\begin{quote}
„Newspaper websites compete in a marketplace where a rival news source is simply
a click away, so gaining and retaining the attention of readers is more
important than ever. [\ldots] It’s not just getting the eyes on your site
[\ldots] It’s getting them to stay on your site. ” (Singer, 2011, S.144)
\end{quote}

Schaffen es die Online-Redaktionen, dass die Leser dabei bleiben, dann erreichen
sie ihre Ziele. Und die Kommentare sind dabei ein ganz wesentlicher Faktor. Die
Teilnahme am Kommentieren kann sich zu einem Wir-Gefühl entwickeln. Das steigert
wiederum die Besuche auf der Website, sowie das Vertrauen und die
Glaubwürdigkeit der Seite (Meyer-Carey, 2014, S. 215). So bekommen die
Medienhäuser den gewünschten Verkehr im Netz und große Klickzahlen. Die Leser
entwickeln eine gewisse Loyalität gegenüber ihrer Zeitung. Das Medienprodukt
wird zur (begehrten) Marke und zieht positive wirtschaftliche Aspekte mit sich.

Alle positiven Aspekte können sich durch eine große Masse an Kommentaren jedoch
auch umkehren, z.B. werden die Leser der Online-Zeitung den Rücken zukehren,
wenn sie von den Äußerungen der anderen regelrecht erschlagen werden.


\subsection{Gatekeeper Funktion}


Den Nutzern geht es nicht darum, hauptsächlich neue Inhalte zu schaffen. Sie
bedienen sich nur an der Nachrichtenauswahl und verändern dann die Gewichtung. 
SIe suchen sich aus, was sie für gut finden und teilen dies anderen mit. Das Material wird
ein zweites Mal veröffentlicht und verteilt (Singer, 2014, S. 66) .
Damit übernehmen die Nutzer neue Rollen und zwar die des „active redistributor“, des
„active recipient“ und in einer gewissen Weise die des „gatekeepers“ (Singer,
2014, S. 57). Dadurch, dass ihnen aber die Autonomie fehlt, was wirklich zur
Veröffentlichung frei gegeben wird, werden sie immer hinter dem eigentlichen
gatekeeper, dem Journalisten, stehen. Auch weil sie sich an Regeln halten müssen
(wie diese Regeln konkret aussehen, wird in den folgenden Abschnitten
beschrieben). Diese Regeln bekräftigen den eigentlichen gatekeeper, indem sie
ihm eine neues „gate“ liefern (Santana, 2014, S. 13).

Der Journalist besitzt zu Recht aufgrund seiner Professionalität die
Entscheidungsgewalt. Aber der Nutzer wird zumindest zu einem „secondary
gatekeeper“ (Santana, 2014, S. 5), der Inhalte verändern kann, indem er sie
hervorheben, markieren, mit anderen teilen, weiter verschickten, zugänglich
machen kann (Singer, 2014, S. 57). 
Es gibt im Internet unglaublich viele Nachrichten und Informationen. Diese sind da,
aber irgendwo ``da draußen''. Deswegen ist das ``Sichtbarmachen'' von Nachrichten
umso wichtiger geworden. 










Bereicherung der public agenda? Die Nachrichten werden einem präsentiert, aber
nun gibt es auch die Möglichkeit darüber zu sprechen. Denn je mehr berichtet
wird, umso mehr wird diskutiert. Dadurch dass die Kommentare unterbunden werden
können, trifft das nicht mehr zu.



„[\ldots] we can actually talk to these people now! And they can talk to us! And
this is great!“ (Reich, 2011, S. 105).

„In addition, media gatekeepers  turned older participation channels into
exclusive spaces: Only those citizens whom the gatekeepers decided were
worth hearing were allowed a public voice. Comment threads, in contrast,
are inclusive spaces; most comments that do not break explicit rules of
participation are included.“ (Reich, 2011, S. 97)


























