\chapter{Positive Aspekte der Kommentarfunktion}

Was macht Kommentare wertvoll und was bringen sie eigentlich? Dieser Frage wird
im folgenden Abschnitt nachgegangen.


\section{Beliebtheit der Kommentare} \label{sec:beliebtheit}
Obwohl nur ein kleiner Teil der Leser Kommentare verfasst und ein ebenso kleiner
Teil diese auch liest \autocite[96ff]{reich}, schlägt diese Art der
Nutzerbeteiligung wie eine Bombe ein. Das merken die Redaktionen, die von nicht
mehr zu bewältigenden Kommentaren erreicht werden. Dies ist dem Umstand
geschuldet, dass es auf einmal (2005 entstehen die ersten Kommentare) die
Möglichkeit gibt, rauszulassen, was einem gerade durch den Kopf geht beim Lesen
eines Artikels.

Es ist aber nicht nur die Gelegenheit, spontan etwas zum eben Gelesenen zu sagen,
was die Kommentare attraktiv macht. Menschen haben selten die Chance, ihren
Unmut kund zu tun oder Zustimmung auszusprechen, weil sie auf das nur Zuhören
beschränkt sind \autocite[S.~99]{reich}. Diese auferlegte Passivität wird
aufgehoben.\\
Für manche Leser werden die Kommentare sogar genauso wichtig, wie die Nachricht
selbst. Und manchen Leser gefällt es einfach, sich selbst veröffentlicht zu
sehen.


\section{Öffentlicher Diskurs}
Eine ganz wichtige Rolle nehmen die Kommentare als ein Platz oder Ort ein, wo
sich jeder zum Diskutieren treffen und wo man auch die Breite und
Verschiedenheit der Ansichten erfahren kann. Für den demokratischen Prozess ist
es wichtig, dass bestimmte Themen besprochen werden. Jede Stimme soll gehört
werden, auch diejenigen, die sonst leicht untergehen, wie Minderheiten oder
Leute, die sich nicht trauen, oder weil es an Unterstützung fehlt oder weil die
Angelegenheit einfach kontrovers ist \autocite[S.~12]{santana:2014}. Außerdem
gibt es erst mal keine Beschränkungen, wer mit sprechen darf und wer nicht.  Zu
einer gesunden Demokratie gehört das einfach dazu: \glqq By allowing citizens to
participate, journalists behave ethically, and hence they democratize journalism
and the web.\grqq\- \autocite[S.~125]{singer}

Dieser neu geschaffene öffentliche Ort soll auch dazu dienen, herauszufinden,
wie die Leute im Moment \glqq ticken\grqq\- (\glqq serving as a gauge of society’s pulse\grqq\-
 \autocite[S.~181]{loke}. Man kann herauslesen, wo die neuralgischen Punkte in
der Gesellschaft liegen und auf diesen Zug aufspringen, wenn nötig.

Auf die Kommentare kann Einspruch erhoben werden, Bedenken können geäußert werden.
Ebenso funktionieren Kommentare als Anregung für weitere Diskussionen oder
greifen korrigierend  in die Redaktionen ein. Auf diesem Weg erhalten die
Journalisten auch direktes Feedback: \glqq Comments can confirm that the website is
doing a good job [\ldots] they can help improve accuracy [\ldots]\grqq\-
\autocite[S.~105]{reich}

Ufern die Diskussionen jedoch ins Unendliche aus oder werden aggressiv und
beleidigend, dann wird aus diesem liberalen Aspekt ein Problem der Kommentare.
\glqq[\ldots] free expression and exposure to differing views can hold deliberative
potential only when participants were respectful toward each other\grqq\-
\autocite[S.~7]{santana:2011}. Das nächste Kapitel wird sich damit beschäftigen.





\section{Leserbindung}
\begin{quote}
\glqq Newspaper websites compete in a marketplace where a rival news source is simply
a click away, so gaining and retaining the attention of readers is more
important than ever. [\ldots] It’s not just getting the eyes on your site
[\ldots] It’s getting them to stay on your site.\grqq\- \autocite[S.~144]{singer}
\end{quote}

Schaffen es die Online-Redaktionen, dass die Leser dabei bleiben, dann erreichen
sie ihre Ziele. Und die Kommentare sind dabei ein ganz wesentlicher Faktor. Die
Teilnahme am Kommentieren kann sich zu einem Wir-Gefühl entwickeln. Das steigert
wiederum die Besuche auf der Website, sowie das Vertrauen und die
Glaubwürdigkeit der Seite \autocite[S.~215]{meyer-carey}. So bekommen die
Medienhäuser den gewünschten Verkehr im Netz und große Klickzahlen. Die Leser
entwickeln eine gewisse Loyalität gegenüber ihrer Zeitung. Das Medienprodukt
wird zur (begehrten) Marke und zieht positive wirtschaftliche Aspekte mit sich.\\
Alle positiven Aspekte können sich durch eine große Masse an Kommentaren jedoch
auch umkehren, z.B. werden die Leser der Online-Zeitung den Rücken zukehren,
wenn sie von den Äußerungen der anderen regelrecht erschlagen werden.

\section{Neue oder zusätzliche Inhalte}
Die Kommentierenden begeistern sich für ihr Thema, lassen sich als Experten
erkennen, liefern Hinweise und sprechen ganz einfach aus, was ihnen wichtig ist.
Dort kann der Journalist ansetzen und solche Beiträge als \glqq journalistisches
Werkzeug\grqq{} \autocite[S.~133]{robinson} benutzen.\\
Kommentare dienen als Informationsquelle oder als Inspiration für Themen.
Manche Journalisten sehen sie als die Möglichkeit, Kontext herzustellen (durch
Hyperlinks) oder um bestimmte Punkte zu klären. Andere Perspektiven tun sich
auf und der Journalist wird sensibilisiert, was die Leser hören wollen
\autocite[S.~12]{santana:2014}.

Dafür müssen die Schreiber jedoch alle Kommentare durchforsten: entweder nur von
ihren eigenen Texten oder von der gesamten Redaktion. Es liegt nahe, dass dies
sehr schnell ausarten kann. \glqq If a journalist were to read all 300 comments, he
wouldn’t be able to do anything else [\ldots]\grqq\- \autocite[S.~84]{domingo}.  So
kommt es, dass nur wenige Journalisten Kommentare überhaupt lesen
\footnote{\glqq Only 35 percent of national journalists and 36 percent of local
  journalists said they have a positive view of citizens posting news content on
  news organizations' websites.  [\ldots] Even fewer said they take the time to
  read the comments and submissions they receive.\grqq\- \autocite[S.
216]{meyer-carey}}
und dass sie eine schlechte Meinung darüber entwickeln.\footnote{The study finds
  that the more comments a news website receives, the more negative attitudes
  the journalists who work at those sites have toward the value and utility of
  comments. \autocite[S.~214]{meyer-carey}}

\section{Secondary gatekeeper} \label{sec:secondary-gate}
Den Nutzern geht es nicht darum, hauptsächlich neue Inhalte zu schaffen. Sie
bedienen sich nur an der Nachrichtenauswahl und verändern dann die Gewichtung.
Sie suchen sich aus, was sie gut finden und teilen dies anderen mit. Das
Material wird ein zweites Mal veröffentlicht und verteilt
\autocite[S.~66]{singer:2014}. Damit übernehmen die Nutzer neue Rollen und zwar
die des \glqq active redistributor\grqq, des \glqq active recipient\grqq\- und in einer gewissen
Weise die des \glqq gatekeepers\grqq\- \autocite[S.~57]{singer:2014}. Dadurch, dass ihnen
aber die Autonomie fehlt, was wirklich zur Veröffentlichung frei gegeben wird,
werden sie immer hinter dem eigentlichen \emph{gatekeeper}, dem Journalisten,
stehen. Auch weil sie sich an Regeln halten müssen (wie diese Regeln konkret
aussehen, wird in den folgenden Abschnitten beschrieben). Diese Regeln
bekräftigen den eigentlichen \emph{gatekeeper}, indem sie ihm eine neues \glqq gate\grqq\-
liefern \autocite[S.~13]{santana:2014}.\\
Der Journalist besitzt zu Recht aufgrund seiner Professionalität die
Entscheidungsgewalt. Aber der Nutzer wird zumindest zu einem \glqq secondary
gatekeeper\grqq\- \autocite[S.~5]{santana:2014}, der Inhalte verändern kann, indem er
sie hervorhebt, markiert, mit anderen teilt, weiter verschickt, zugänglich macht
\autocite[S.~57]{singer:2014}.
Es gibt im Internet unglaublich viele Nachrichten und Informationen. Diese sind
da, aber irgendwo \glqq da draußen\grqq. Deswegen ist das \glqq
Sichtbarmachen\grqq\- von Nachrichten umso wichtiger geworden.

% vim: set ai si et tw=80 sts=2 ts=2 sw=2:
