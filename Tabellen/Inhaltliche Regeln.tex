Kommentar: inhaltliche Regeln und Hinweise

Wie sie sich die Diskussion auf ihren Plattformen vorstellen und wie nicht, beschreiben die Online-Zeitungen in der Netiquette, den AGB oder den Nutzungsbedingungen. Auf manchen Portalen muss der Nutzer bei der Registrierung dazu aktiv zustimmen und damit bestätigen, die Regeln zumindest zu akzeptieren. 

Für die Regeln verwenden die Redaktionen eigene Formulierungen mit unterschiedlichem Umfang. Es geht erst einmal darum, zu erklären, was man unter einem ``guten Ton'' versteht. Dann wird erläutert, was nicht toleriert wird und wann und wo ein Eingreifen seitens der Moderation erfolgt. 

Gerade für Foren ohne Moderation sind inhaltliche Vorgaben besonders wichtig. Da es niemanden gibt, der die Beiträge vorab oder danach liest und gegebenenfalls einschreitet, müssen die Nutzer darüber informiert werden, was erlaubt ist und was nicht. Außerdem werden sie zur Selbstregulierung aufgefordert - zum Beispiel über die Möglichkeit Verstöße zu melden - und brauchen Handlungsvorgaben.

Es gibt inhaltliche Regeln zu den Kommentaren, die auf allen Plattformen zu finden sind (Ausnahmen werden entsprechend gekennzeichnet). 
Das sind zunächst rechtliche Hinweise. Der Nutzer bestätigt der Urheber seiner Beiträge zu sein oder die Urheberrechte zu besitzen, d.h. er muss Gewähr leisten, dass fremde Inhalte zur Verbreitung freigegeben sind. Er trägt somit die volle Verantwortung für die eingestellten Beiträge und stellt sicher, dass keine Rechte Dritter oder Urheberrechte oder Persönlichkeitsrechte oder sonstige Rechte verletzt werden. Außerdem gilt die Rechteeinräumung. Der Nutzer stimmt damit zu, dass die entsprechende Zeitung sein Beiträge ``benutzen'' kann (z.B. vervielfältigen, modifizieren, anpassen, übersetzten, bearbeiten, verbreiten, verwerten, hervorheben, bewerten, archivieren, usw.). 
Es gibt infolgedessen auch einen Haftungsausschluss des Anbieteres. Dieser haftet nicht für den Inhalt von Nutzerbeiträgen. Dasselbe gilt für die Inhalte fremder Seiten durch Verlinkung. Entsteht ein Schaden haftet der Nutzer. 

Was auf keinen Fall geduldet wird ist Werbung in irgendeiner Form. Diskussionsforen dürfen nicht für kommerzielle Zwecke missbraucht werden. Diskussionsforen sind keine Werbefläche für Webseiten oder Dienste (Spamming). 

Auch dem Datenschutz müssen die Nutzer zustimmen. Damit ist z.B. das Überprüfen der Emails oder Abfragen auf Viren gemeint oder das Einhalten von gesetzlichen, behördlichen und technischen Vorschriften. Das Passwort soll geheim gehalten und Vertraulichkeit gewahrt werden. 

Jedes Portal weist ausdrücklich darauf hin, dass Kommentare \bf{themenbezogen} sein sollen. 
Es gibt Inhalte, die kategorisch bei allen Zeitungen nicht erlaubt sind. Dazu gehören Beleidigungen und Beiträge mit sexistischen/sittenwidrigen/pornographischen/obszönen/grob anstößigen Inhalten und Rassismus. 
(Diese eben genannten inhaltlichen Verbote werden in der Tabelle nicht mehr aufgeführt, weil sie in allen Online-Zeitungen genannt werden.)

Auch das Verbot von Diskriminierungen zählen die meisten Nachrichtenportale in irgendeiner Form auf. 

Die Redaktionen wählen unterschiedliche Formulierungen, um unerwünschte Beiträge zu beschreiben. Viele Formulierungen sind sich im Inhalt ähnlich  und unterscheiden sich um Nuancen (z.B. Beleidigung und Beschimpfung). Trotzdem werden sie hier bewusst aufgelistet und nicht zusammengefasst, um deutlich zu machen, was die Zeitungen extra erwähnt haben wollen und was nicht. 



&		% Spalte 1
		Nutzungsbedingungen: allgemeine und besondere (Zustimmung verlangt bei Registrierung); Netiquette\\
		sachlich, höflich bleiben, andere respektieren, nicht dagegen argumentieren, Angriffe versuchen zu ignorieren; wie man selbst behandelt werden möchte, keine unangemessenen Beiträge wie Beschimpfungen/Belästigungen/Drohungen/Diskriminierungen, keine Beiträge mit nicht-themenbezogen/antisemitische Inhalten; keine privaten Angaben\footnote{Angaben von Postadresse und/oder Telefonnummer und/oder Emailadresse) oder Angaben über Dritte verbreiten; keine automatisierte Nutzung; kein Mobbing; keine Links zu Werbung/Chats/Foren; Datensicherung vom Nutzer selbst; keine Trolle; kein Spam
		&
		% Spalte 2
		Nutzungsbedingungen: allgemeine und für Foren (Zustimmung)\\
		faire, sachliche, angenehme, offene, freundschaftliche, respektvolle Diskussion (auch bei Streit), obwohl es sich um verbale Auseinandersetzung handeln soll; keine Beiträge mit strafbaren/inakzeptablen/nicht-themenbezogenen Inhalten; 
		&
		% Spalte 3
		Nutzungsbedingungen: allgemein (Zustimmung); ``wie Sie mit diskutieren''-Button\\
		keine Beiträge mit links- und rechtsradikalen/verleumderischen/ruf-/geschäftsschädigenden Inhalten;  keine falschen/nicht nachprüfbaren Behauptungen; keine Hyperlinks
		&
		% Spalte 4
		AGB (Zustimmung), Netiquette (Zustimmung)\\
		sachliche, freundliche, respektvolle, tolerante Diskussion, keine Fremdtexte; keine privaten Angaben (Adresse, Telefonnummer); keine Links zu Werbung; keine Diskriminierungen jeder Art\footnote{Diskriminierungen aufgrund von Herkunft, Nationalität, Religion, sexueller Orientierung, Alter, Geschlecht, usw.}; keine Beiträge mit  demagogischen Inhalten; keine Schadsoftware verwenden, kein Missbrauch von Daten; Datensicherung vom Nutzer selbst; 
		&
		% Spalte 5
		Nutzungsbedingungen, veraltete Netiquette\\
		fair, höflich, verständlich, kritische Kommentare erwünscht, keine Beschimpfungen/Diskriminierungen/Provokationen/Entwürdigungen/Aufruf zu Demonstrationen oder Gewalt; Gesetze beachten; keine Angaben über Dritte verbreiten; keine Hasspropaganda; kein Hinweis auf Haftungsausschluss
		&
		% Spalte 6
		Nutzungsbedingungen (Zustimmung), Netiquette\\
		keine Beschimpfungen/Kränkungen; das Forum ist kein Veranstaltungskalender/keine Terminankündigungen; Zitate müssen Quellenangabe enthalten; keine Beiträge mit gewaltverherrlichenden/antisemitischen/gesetzeswidrigen Inhalten
		&
		% Spalte 9
		AGB
		keine Beiträge mit nicht-themenbezogen/sinnlosen Inhalten/Inhalten, die die Diskussion stören; sich so verhalten, wie man selbst behandelt werden möchte; keine Anschuldigungen/Tatsachenbehauptungen; keine Beiträge mit strafbaren Inhalten/üble Nachrede/Urheberrechtsverstöße/Drohungen/volksverhetzende Äußerungen/Aufforderung zu Gewalt; Verbot von Inhalten, die dem Ansehen von Verstorbenen und deren Angehörigen schaden könnten/die doppeldeutig sind oder anderweitige Darstellungen, deren Rechtswidrigkeit vermutet wird, aber nicht abschließend festgestellt werden kann; keine unwahren/unsachlichen Beiträge; keine Schadsoftware
		&
		% Spalte 10
		Nutzungshinweise (Zustimmung), Netiquette \\
		mit zynischen/ironischen Äußerungen vorsichtig sein; guter Ton; nicht persönlich werden; keine persönlichen Angriffe/Diskriminierungen jeder Art; keine Beiträge mit verleumderischen/ruf-/geschäftsschädigenden/strafrechtlich relevanten Inhalten; keine Veröffentlichung Daten Dritter; sich bewusst machen, welche eigenen Daten frei zugänglich ins Internet gestellt werden;  ignorieren von Provokationen/Trollen; keine Junkmails/Spam/Scraping/sonstige rechtswidrige Kommunikationsformen; keine Nennung von Produktnamen/Dienstleistern/Marken/Produzenten; kein Hinweis auf Nutzungsrechte von handelsblatt.de
		&
		% Spalte 11
		Nutzungsbedingungen (Zustimmung), Netiquette\\
		faire, sachliche, offene, gehaltvolle Diskussion; keine nicht-belegbaren Behauptungen/Verleumdungen/Diffamierungen/Drohungen/Diskriminierungen aller Art/Hetze/Gewaltverherrlichung/Vulgärausdrücke; keine Beiträge mit ruf-/geschäftsschädigenden Inhalten; keine Veröffentlichung Daten Dritter
		&
		% Spalte 12
		AGB (Zustimmung), Netiquette (besonders ausführlich und erklärend)\\
		Durchlesen vor Abschicken, guter Umgangston, nicht provozieren lassen, mit zynischen/ironischen Äußerungen vorsichtig sein; keine Diskriminierungen aller Art (auch Behinderung/Einkommensverhältnisse)/Diffamierungen/Verleumdungen; nicht prüfbare Unterstellungen/Verdächtigungen; nachvollziehbare Aussagen; keine Beiträge mit ruf-/geschäftsschädigenden Inhalten;  keine Veröffentlichung Daten Dritter; sich bewusst machen, welche persönlichen Daten frei zugänglich werden; keine Schadsoftware
		&
		% Spalte 13
		AGB (Zustimmung), Netiquette\\
		sachliche, niveauvolle, faire, offene Diskussion; freundlich, tolerant sein; guter Umgangston, andere so behandeln, wie man es selber möchte; keine persönlichen Angriffe; keine Beiträge mit vulgärem/hetzerischem/gewaltverherrlichendem Inhalt; keine privaten Daten
		&
		% Spalte 14
		AGB (Zustimmung), Netiquette\\
		engagiert, fair; akzeptable, respektvolle Wortwahl; sachkritisch, seriös; keine Schmähungen/Diskriminierungen/Volksverhetzung/Propaganda, keine Beiträge mit jugendgefährdenden/antisemitischen/strafbaren/verleumderischen/ruf-/geschäftsschädigenden/menschenverachtenden/gegen die guten Sitten verstoßenden Inhalten; keine Trolle; keine Junkmails/Spam/Kettenbriefe; keine privaten Daten; keine Veröffentlichung Daten Dritter;  nur für private Zwecke (keine Vervielfältigung)
		&
		% Spalte 15
		AGB (Zustimmung), Netiquette (Zustimmung)\\
		sachlich, freundlich, verständlicher Umgangston, man soll Spaß haben und sich wohl fühlen,  keine persönlichen Angriffe/Beschimpfungen, andere Meinungen akzeptieren; keine Beiträge mit unwahren/unsachlichen/jugendgefährdenden/verleumderischen/verfassungsfeindlichen/extremistischen/illegalen/ethisch-moralisch-problematischen/ Inhalten; keine Schadsoftware; keine privaten Daten; nur für private Zwecke; keine Junkmails/Spam/Kettenbriefe 
		&
		% Spalte 16
		AGB (Zustimmung), Netiquette (Zustimmung)\\
		 sachlich, freundlich, verständlicher Umgangston, man soll Spaß haben und sich wohl fühlen, keine Beschimpfungen/persönlichen Angriffe; andere Meinungen akzeptieren; keine Beiträge mit unwahren/unsachlichen/jugendgefährdenden/strafbaren/verleumderischen/verfassungsfeindlichen/extremistischen/illegalen/ethisch-moralisch-problematischen Inhalten; keine Schadsoftware; keine privaten Daten; kein Hinweis auf Nutzungsrechte von hna.de, kein Hinweis auf Haftungsausschluss
		&
		% Spalte 17
		\\
		keine
		&
		% Spalte 18
		Netiquette \\
		faire, akzeptable, respektvolle Wortwahl; keine verbalen Angriffe, privaten Details aus dem Leben anderer; mit zynischen/ironischen Äußerungen vorsichtig sein; kein Aufruf zu Straftaten; keine privaten Daten; keine Beiträge mit ehrverletzenden/gewaltverherrlichenden Inhalten/Aufrufen zur Gewalt; Löschen von Beiträgen, die gegen das Gesetz verstoßen; kein Hinweis auf Nutzungsrechte von mainpost.de, kein HInweis auf Haftungsausschluss
		&
		% Spalte 19
		Richtlinien für Community\\
		sachlich, respektvoll; fairer Umgang, angenehme Atmosphäre;  bewusst machen, dass K. öffentlich werden; keine Angaben, die nicht für die Öffentlichkeit sind; keine pauschale/persönliche Herabwürdigung; keine Beiträge mit pietätlosen/menschenverachtenden/gewaltverherrlichenden/verleumderischen Inhalten; kein Geschichtsrevisionismus; keine Trolle; keine Unterstellungen/Kampagnen/Diskriminierungen aller Art (auch aufgrund von Weltanschauung/sozialem Status); Links zu anderen Webinhalten/Kritik an Artikeln mit denselben inhaltlichen Regeln wie K.; keine Mehrfachaccounts; kein Hinweis auf Nutzungsrechte von tagesspiegel.de
		&
		% Spalte 20
		Netiquette \\
		sachlich, fair, freundlich, wie man selbst behandelt werden möchte; keine hetzerischen/gewaltverherrlichenden Töne; keine privaten Adressen; keine Hinweise zu Rechten
		&
	
		\\ \hline
		