\begin{landscape} \small
\begin{tabular}{ccc}

\hline
%Zeile 1: Portale
		&
		%Spalte 1
		bild.de &
		%Spalte 2
		spiegel.de &
		%Spalte 3
		faz.net &
		%Spalte 4
		focus.de 
		% Spalte 5
		diewelt.de 
		% Spalte 6
		derwesten.de
		% Spalte 9
		rp-online
		% Spalte 10
		handelsblatt
		% Spalte 11
		suedkurier
		% Spalte 12
		zeit.de
		% Spalte 13
		badische zeitung
		% Spalte 14
		stuttgarter zeitung
		% Spalte 15
		merkur
		% Spalte 16
		hna
		% Spalte 17
		mopo
		% Spalte 18
		mainpost
		% Spalte 19
		tagesspiegel
		% Spalte 20
		swp
		\\ \hline

% Zeile 2
Kommentare\footnote{werden in der Regel mit Sprechblasen angekündigt; mit Angabe der Anzahl der abgegebenen K.; auch eine Zeitangabe ist üblich entweder mit Datum und Uhrzeit oder ``vor ... Stunden''; K. stehen unter dem Artikel in vielen Fällen nach Werbung (in eigener Sache)  \\
abweichende Angaben und Sonstiges
&		% Spalte 1
		\\
		K. nicht zu allen Artikeln möglich 
		&
		% Spalte 2
		\\
		Startseite: keine Sprechblase sondern Hinweis auf [Forum]; beim Beitrag: Sprechblase mit Ausrufungszeichen\\
		K. werden durchnummeriert
		&
		% Spalte 3
		\\
		zeitliche Begrenzung um K. zu schreiben; Hinweis, dass die K. im Internet recherchierbar sind; K. = Lesermeinung
		&
		% Spalte 4
		 \\
		 auch Videos kommentierbar 
		&
		% Spalte 5
		\\
		über Disqus verwaltet; Schließung nach zwei/drei Tagen oder früher bei Regelverstößen
		&
		% Spalte 6
		\\
		keine Ankündigung der K., keine Sprechblasen; alle Beiträge können kommentiert werden; K. werden durchnummeriert
		&
		% Spalte 9
		\\
		fast jedes Thema kommentierbar; 
		&
		% Spalte 10
		Kommentierzeiten: 7.30 - 21 Uhr, bis zu sieben Tage lang; K. werden (u.U. gekürzt) multimedial verbreitet
		\\
		&
		% Spalte 11
		\\
		K. nicht zu allen Artikeln möglich
		&
		% Spalte 12
		\\
		keine Sprechblase; Hinweis mit  [Anzahl + Kommentare]; K. werden durchnummeriert;  K. nicht zu allen Artikeln möglich; Definition von K.: ``kürzere Textbeiträge, die Sie unter vorhandenen Artikeln, Videos, Fotostrecken oder anderen Multimedia-Inhalten abgeben können''
		&
		% Spalte 13
		&
		% Spalte 14
		für alle Nutzer, K. geben nicht die Meinung der Stuttgarter Zeitung wieder
		&
		% Spalte 15
		über Disqus verwaltet, für alle Nutzer
		&
		% Spalte 16
		über Disqus verwaltet, für alle Nutzer; freie Meinungsäußerung; bei Verstößen Aufruf eine Email zu schreiben 
		&
		% Spalte 17
		über Disqus verwaltet; Kommentierzeiten: 8 - 21 Uhr
		&
		% Spalte 18
		K. unter Artikel und neben Artikel (als neuer Tab!); Problem bei Mainpost.de: begrenzter Zugang, nur bestimmte Zahl an Artikeln im Monat kostenlos lesbar
		&
		% Spalte 19
		keine Sprechblase; Hinweis mit [Anzahl + Kommentare]
		&
		% Spalte 20
		kein Hinweis auf K.; Problem bei swp: begrenzter Zugang, nur bestimmte Zahl an Artikeln im Monat kostenlos lesbar
		&
		
		\\ \hline
	


		
		

		
		
Kommentar: formale Regeln\\
Zeichenbegrenzung\\
Überschrift (Pflichtfeld)\\
Sonstiges 
&		% Spalte 1
		\\
		keine\\
		keine\\
		vorsichtig mit Großbuchstaben, Zitate kennzeichnen	
		&
		% Spalte 2
		\\
		keine\\
		optional\\
		keine Bilder posten, keine langen Kopien von Quellen (Links verwenden)\\
		&
		% Spalte 3
		\\
		1000 Zeichen\\
		ja, 100 Zeichen\\
		\\
		&
		% Spalte 4
		\\
		800 Zeichen\\
		ja\\
		reiner Text ohne besondere Kennzeichnungen (z.B. keine Smilies, Hervorhebungen, Chat-Symbole, nur Kleinschreibung, usw.), korrektes Deutsch, auf Rechtschreibung/Interpunktion achten, Absätze machen
		&
		% Spalte 5
		\\
		keine\\
		keine\\
		Zitate kennzeichnen; keine Fremdsprachen, keine Links zu externen Webseiten (seriöse Ausnahmen möglich)\\
		&
		% Spalte 6
		\\
		keine\\
		keine\\
		&
		% Spalte 9
		\\
		keine\\
		Betreff\\
		deutsche Sprache; Links möglich (keine Links zu Werbung/strafbare Inhalte)
		&
		% Spalte 10
		\\
		2000\\
		keine\\
		Großbuchstaben (Schreien) vermeiden; Absätze machen und strukturieren; Wortwahl überprüfen; auf Rechtschreibung achten; 				Zitate/Quellen kennzeichnen
		&
		% Spalte 11
		\\
		1000 Zeichen\\
		ja\\
		Zitate kennzeichnen mit Quellenangabe\\
		&
		% Spalte 12
		\\
		1500 Zeichen\\
		ja, mindestens 5 Zeichen\\
		Absätze machen; auf Rechtschreibung achten; vorsichtig mit Großbuchstaben; Zitate kennzeichnen, wenig verwenden, Quellenangabe machen, nur als Ergänzung verwenden; Links möglich \\
		&
		% Spalte 13
		\\
		keine\\
		keine\\
		\\
		&% Spalte 14
		\\
		keine\\
		Betreff\\
		Links möglich (keine Links zu Werbung/kommerziellen Angeboten/Chats/Foren/strafbaren Inhalten\\
		&% Spalte 15
		\\
		keine\\
		keine\\
		deutsche Sprache; Links nicht erwünscht (falls doch distanziert sich merkur-online von Inhalten der gelinkten Seiten)\\
		&
		% Spalte 16
		\\
		keine\\
		keine\\
		deutsche Sprache; Beiträge in Fremdsprachen werden gegebenenfalls entfernt, da für größten Teil der Nutzer nicht verständlich; Links nicht erwünscht (falls doch distanziert sich hna.de von Inhalten der gelinkten Seiten)\\
		&
		% Spalte 17
		\\
		keine\\
		keine\\
		\\
		&
		% Spalte 18
		\\
		1000 Zeichen\\
		ja\\
		auf deutsche Rechtschreibung achten; korrekte Interpunktion, Absätze machen; Hervorhebungen möglich: fett, kursiv, unterstrichen; markieren von Links, Zitaten möglich; einfügen von Emoticons möglich(Grinsen, Zwinkern, traurig sein); \\
		&% Spalte 19
		\\
		2000 Zeichen\\
		Titel\\
		auf Rechtschreibung/Grammatik achten; Hervorhebungen möglich: fett, kursiv; markieren von Quellen durch Links; markieren von Links/Zitaten; Zitate als Ergänzung, nicht alleinstehend\\
		&
		% Spalte 20
		\\
		3000 Zeichen\\
		Betreff\\
		\\
		&
		\\ \hline
		
		

		
Funktionen im Kommentar	\\
``Bewerten''\\
``Antworten''\\
Sonstiges\\
&		% Spalte 1
		\\
		positiv\\
		\\
		Ordnen nach beliebteste, älteste, neueste K.
		&
		% Spalte 2
		\\
		\\
		ja: auf was man antwortet wird in Zitate gesetzt\\
		\\
		&
		% Spalte 3
		\\
		positiv\\
		dem Kommentator folgen\\
		&
		% Spalte 4
		\\
		positiv und negativ\\
		ja\\
		\\
		&
		% Spalte 5
		Oberfläche von Disqus\\
		positiv (Klicks werden gezählt), nach Anmeldung oder als ``Gast'' möglich; negativ (nach Anmeldung)\\
		ja\\
		``Teilen''-Button (= diese Diskussion auf Twitter/Facebook teilen, nach Anmeldung dort), ``Teilen''-Button für einzelnen K., 		``Empfehlen''-Button (= diese Diskussion empfehlen); Ordnen nach beste, neueste, älteste Kommentare; ``Abonnieren'', um Mittleiungen dieser Diskussion per Email zu erhalten; Disqus auf der eigenen Seite hinzufügen; Datenschutz\\
		&
		% Spalte 6
		\\
		\\
		ja, mit Anzahl Antworten\\
		\\
		&
		% Spalte 9
		\\
		positiv\\
		\\
		Ordnen nach älteste; Button für ``mehr K.''\\
		% Spalte 10
		\\
		\\
		ja\\
		\\
		&
		% Spalte 11
		\\
		\\
		ja\\
		Ordnen nach älteste, neueste, beste Bewertung; zur Auswahl: ``informiert bleiben'' (bei jedem neuen Beitrag der Diskussion erhält man Benachrichtigung)\\
		&
		% Spalte 12
		\\
		\\
		ja\\
		Reaktionen/Antworten auf diesen K. anzeigen; Ordnen nach neuesten, empfohlenen, allen K.; Empfehlungen aussprechen; Empfehlungen der Redaktion (bedeutet aber nicht, dass die Redaktion der Meinung des Lesers zustimmt)\\
		&
		% Spalte 13
		\\
		\\
		\\
		\\
		&% Spalte 14
		\\
		ja\\
		ja\\
		Ordnen nach älteste, neueste K.; Empfehlen\\
		&% Spalte 15
		Oberfläche von Disqus\\
		positiv (Klicks werden gezählt), nach Anmeldung oder als ``Gast'' möglich; negativ (nach Anmeldung)\\
		ja\\
		Empfehlen; Ordnen nach beste, neueste, älteste K.; ``Abonnieren'', um Mittleiungen dieser Diskussion per Email zu erhalten; Disqus auf der eigenen Seite hinzufügen; Datenschutz \\
		&
		% Spalte 16
		Oberfläche von Disqus\\
		positiv (Klicks werden gezählt), nach Anmeldung oder als ``Gast'' möglich; negativ (nach Anmeldung)\\
		ja\\
		Empfehlen; Ordnen nach beste, neueste, älteste K.; ``Abonnieren'', um Mittleiungen dieser Diskussion per Email zu erhalten; Disqus auf der eigenen Seite hinzufügen; Datenschutz\\
		&
		% Spalte 17
		Oberfläche von Disqus\\
		positiv (Klicks werden gezählt), nach Anmeldung oder als ``Gast'' möglich; negativ (nach Anmeldung)\\
		ja\\
		Empfehlen; Ordnen nach beste, neueste, älteste K.; ``Abonnieren'', um Mittleiungen dieser Diskussion per Email zu erhalten; Disqus auf der eigenen Seite hinzufügen; Datenschutz\\
		&
		% Spalte 18
		\\
		\\
		ja\\
		Ordnen nach älteste, neueste, best bewertete K.\\
		&
		% Spalte 19
		\\
		\\
		ja; zur Auswahl: Antworten anzeigen\\
		Ordnen nach neueste, älteste K., chronologisch\\
		&
		% Spalte 20
		\\
		positiv\\
		ja\\
		\\
		&
		\\ \hline
		
		


		
Sonstiges\\ 
Artikel teilen auf sozialen Netzwerken (Facebook ``teilen'' und/oder ``empfehlen'', Twitter, g+) und durch Versenden (Briefsymbol)\\
Kommentar in der Community/Forum\\
Funktionen beim Artikel\\
Felder auf Startseite\\
Außergewöhnliches
&		% Spalte 1
		\\
		tumbl, Pinterest; K. gleichzeitig auf Facebook veröffentlichen möglich \\
		Profil; Ranglisten der Nutzer; Chronologie der Kommentare der Nutzer; kommentiert letzte 24 h (Top 5)\\
		Korrektur-Button\footnote{Formular zum Versenden an die Redaktion mit Hinweisen auf Fehler oder anderes}\\
		\\
		``Reaktionen'' möglich: Lachen, Weinen, Wut, Staunen, Wow (zur Auswahl, welche Reaktion man zu dem Beitrag empfindet)
		&
		% Spalte 2
		\\
		Xing, LinkedIn, Tumbl, Pinterest, deli.cio.us, Diggy, reddit\\
		meistkommentierte Themen (Top 5); eigene Beiträge anzeigen; Sichtbarmachen der Emailadresse für andere Teilnehmer\\
		Button ``merken'' (auf die Merkliste im Benutzerprofil setzen); Button ``feedback'' (Feedback an die Redaktion über Formular) \\
		\\
		\\
		&
		% Spalte 3
		\\
		\\
		Profilbild möglich; jüngste/älteste Lesermeinung, viel/wenig diskutiert, viel/wenig empfohlen, TOP-Argumente; K. verwalten/von der Veröffentlichung zurückziehen\\
		Beitrag empfehlen, Beitrag merken, Permalink, Drucker\\
		\\
		sämtliche Buttons/Symbole/Funktionen mit Hilfe/Erklärungen
		&
		% Spalte 4
		\\
		Facebook ``gefällt mir'', Xing\\
		Chronologie; aktivste Kommentatoren (des Monats, gesamt, top 50), Kommentar des Tages; Videofavoriten der Leser (meistkommentiert, top 20)\\
		Fehler-Melden; Beitrag ``Bewerten'' mit Sternen (Anzahl Bewertungen)\\
		\\
		Leserbericht schreiben (zusätzlich zum Kommentar mit mehr Zeichen (4000), persönliche Erfahrungen), Kommentare abonnieren\\
		&
		% Spalte 5
		\\
		\\
		Profilfoto möglich, DIE WELT auf Disqus: neueste K.; Top Kommentatoren\\
		\\
		\\
		Hinweis bei Löschung: ``Dieser Kommentar wurde entfernt'' (Antworten darauf aber noch sichtbar)\\
		&
		% Spalte 6
		\\
		\\
		\\
		\\
		\\
		\\
		&
		% Spalte 9
		\\
		\\
		\\
		Beitrag empfehlen, Drucken, Schriftgröße ändern\\
		\\
		Kontakt mit der Zeitung über Email an den Chefredakteur, Newsletter, Leserbrief schreiben (über Formular)
		&
		% Spalte 10
		\\
		Xing, Email schreiben\\
		\\
		Beitrag ``Merken''\\
		\\
		\\
		&
		% Spalte 11
		\\
		\\
		\\
		\\
		``Meistkommentiert''  (Top 3), \\
		Leserreporter-Beitrag schreiben \\
		&
		% Spalte 12
		\\
		keine Symbole/Verweise auf Startseite\\
		Profil anlegen möglich\\
		Drucken, als PDF speichern\\
		``Meistgelesen''/''Meistkommentiert'' (Top 5)\\
		Leserartikel schreiben = ausführliche Meinungsbeiträge und Erfahrungsberichte (meistgelesene/meistkommentierte Leserartikel, Top 3 auf Leserartikel-Seite), Debattenkultur: ``Aus den Kommentaren'' (Höhepunkte aus den Leserdebatten mit neuer Fragestellung), Kommentarkultur: ``Bitte weichen Sie vom Thema ab'' (Experiment: Kommentieren ohne Artikel), Empfehlungen bei Facebook (aktuelle Empfehlungen aus Facebook-Freundeskreis), Tweets von ZEIT ONLINE Politik\\
		&
		% Spalte 13
		\\
		keine Symbole/Verweise auf Startseite) kein g+; Versenden (kein Briefsymbol), Verlinken\\
		Profilbild (optional)\\
		Drucken, Vorlesen, Fehler-Melden\\
		``Meistkommentiert'' (Top 5)/''zuletzt kommentiert''\\
		Nutzer registriert seit [...] + Anzahl der bereits geschriebenen K. vom Nutzer; Vorschau möglich: man kann K. sehen, wie er online aussehen wird\\
		&
		% Spalte 14
		\\
		\\
		\\
		\\
		\\
		\\
		&
		% Spalte 15
		\\
		Youtube; Briefsymbol bedeutet Feedback geben (kein Versenden)\\
		Profilbild (optional; angemessen, Verbot von rassistischen, pornografischen, menschenverachtenden, beleidigenden oder gegen die guten Sitten verstoßenden Abbildungen); Merkur Online Community auf Disqus: neueste K.; Top Kommentatoren\\
		\\
		\\
		\\
		&
		% Spalte 16
		\\
		\\
		HNA Community auf Disqus: neueste K.; Top Kommentatoren; Profilbild (optional; Urheberrecht beachten/Recht auf eigenes Bild; keine Werbung/Logos; kein beleidigendes, verletzendes Motiv; Verbot von rassistischen, pornografischen, menschenverachtenden, beleidigenden oder gegen die guten Sitten verstoßenden Abbildungen)\\
		\\
		\\
		\\
		&
		% Spalte 17
		\\
		\\
		mopo.de Community auf Disqus: neueste K.; Top Kommentatoren\\
		\\
		\\
		\\
		&
		% Spalte 18
		\\
		\\
		Auswahl auf Profilseite: niemandem/allen/nur Mitgliedern zeigen\\
		zur Auswahl: ``ich möchte bei neuen K. per Email benachrichtigt werden'' \\
		``aktuelle Leserkommentare'', ``kommentiert'' (Top 5), Teilen auf Youtube\\
		Anzahl der bereits geschriebenen K. vom Nutzer, Kontakt mit Redaktion, Sicherheitsfrage\\
		&
		% Spalte 19
		\\
		\\
		Profil anlegen möglich (Profilbild optional/angemessen), Nutzer-Statistik\\
		Newsletter abonnieren, Drucken, Lesezeichen setzen\\
		\\
		Community-Statistik\\
		&
		% Spalte 20
		\\
		\\
		\\
		\\
		\\
		\\
		&
		
		\\ \hline
		
		


\end{tabular}
\end{landscape}
