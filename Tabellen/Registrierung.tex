Kommentare können immer gelesen werden, um selbst welche zu schreiben bedarf es jedoch einer Anmeldung, darin sind sich alle Anbieter einig. Dies ist die erste Maßnahme, um Missbrauch bei der Kommentarfunktion zu vermeiden. Man kann nicht einfach drauf los schreiben, sondern muss den Prozess einer Registrierung durch gehen. In der Regel bieten die Online-Zeitungen eine eigene Registrierung an. Man ist dann auf dem Portal angemeldet und sieht es auch (außer Hamburger Morgenpost und Südwestpresse). Manche Zeitungen ermöglichen auch eine Anmeldung über Dritte Anbieter. Das ist dann eine Kompromisslösung für die Nutzer. Eine Anmeldefunktion wird nämlich auch als Hemmschwelle zum Schreiben diskutiert. Kann man sich über ein Konto anmelden, das man eh schon hat, dann gilt dieses Argument nicht mehr. 

Durch eine Anmeldung wird sicher gestellt, dass die Person, die einen Kommentar verfasst, existiert und gegebenenfalls zur Verantwortung gezogen werden kann. Natürlich ist auch hier Missbrauch nicht ausgeschlossen. Der Nutzer wird auf wahrheitsgemäße Angaben hin gewiesen. 

Interessant ist, welche Angaben die Zeitungen zur Anmeldung fordern. Unabdingbar sind die Eingabe einer Emailadresse sowie ein Passwort. Dann kommt es wieder zu unterschiedlichen Handhabungen und unterschiedlichen Lösungen. Manche Zeitungen verlangen die Nennung von Klarnamen und Benutzernamen, manche nur Klarnamen, manche nur Benutzernamen. Einige Portale geben zusätzliche Hinweise zur Registrierung. 

Zeile 1: Anmeldemöglichkeiten (außer über die Portale selbst): Anmelden mit/über ... \\
Zeile 2: was zusätzlich zur obligatorischen Angabe von Emailadresse und Passwort verlangt wird: Klarname = KN
Zeile 3: Benutzername = BN
Zeile 4: Sonstiges
	
 &		% Spalte 1bei BILD.de-Community über 
		Anmeldedienst ``mypass'', Facebook\\
		KN\\
		BN\\
		Volljährigkeit bzw. Einverständnis der Erziehungsberechtigten bei Minderjährigen
		&
		% Spalte 2 bei ''mein spiegel'' als Abonnent oder Nicht-Abonnent, 
		Facebook\\
		KN\\
		BN (angezeigt)\\
		\\
		&
		% Spalte 3 bei ''Mein FAZ.NET'' mit
		\\
		KN
		\\
		\\
		&
		% Spalte 4 bei FOCUS online, 
		Facebook\\
		KN (Hinweis: im gesamten Internet recherchierbar)
		\\
		\\
		&
		% Spalte 5 bei WELT DIGITAL über 
		Anmeldedienst ``mypass'', Disqus, Facebook, Twitter, Google\\
		KN\\
		BN\\
		als Gast schreiben
		&
		% Spalte 6 auf derwesten.de mit
		\\
		KN\\
		BN(angezeigt)\\
		jeder ist zugangs- und teilnahmeberechtigt
		&
		% Spalte 9 bei mein RP ONLINE mi
		\\
		\\
		BN\\ 
		auch Minderjährige, wenn sie sich über Nutzung bewusst sind bzw. mit Einverständnis der Erziehungsberechtigten
		&
		% Spalte 10 auf handelsblatt.de mit
		\\
		KN\\
		\\
		Volljährigkeit
		&
		% Spalte 11auf suedkurier.de mit
		 \\
		KN\\ 
		BN\\
		Anrede, Land, Adresse (Angaben werden auf Richtigkeit geprüft (teilweise auch telefonisch); Einverständnis nach sechs Monaten Inaktivität wird Registrierung/Benutzername gesperrt/freigegeben
		&
		% Spalte 12 auf zeit.de mit
		\\
		\\
		BN\\
		\\
		&
		% Spalte 13 bei Meine BZ mit
		\\
		KN
		\\
		Anrede, Löschen des Accounts bei Wegwerf-Emailadresse
		&
		% Spalte 14 auf stuttgarterzeitung.de
		Facebook \\
		KN\\
		\\
		\\
		&
		% Spalte 15 bei Merkur-Online
		Disqus, Facebook, Twitter, Google\\
		\\
		BN (erwünscht, angemessen)\\
		als Gast schreiben
		&
		% Spalte 16 mit HNA-Login
		Disqus\\
		\\
		BN(angemessen, nicht beleidigend/anstößig), 
		\\
		&
		% Spalte 17
		Disqus, Facebook, Twitter, Google
		\\
		\\
		\\
		&
		% Spalte 18 auf Mainpost.de
		\\
		KN
		BN (nicht beleidigend/ehrverletzend/hetzerisch)
		Adresse
		&
		% Spalte 19 auf tagesspiegel.de
		\\
		\\
		BN (angemessen); 
		\\
		&
		% Spalte 20
		\\
		KN\\ 
		BN\\
		\\
		&
		\\ \hline
		