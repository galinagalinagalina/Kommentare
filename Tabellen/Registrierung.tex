

\section{Kategorie Registrierung}

Kommentare können immer gelesen werden, um selbst welche zu schreiben bedarf es
jedoch einer Anmeldung, darin sind sich alle Anbieter einig. Dies ist die erste
Maßnahme, um Missbrauch bei der Kommentarfunktion zu vermeiden. Man kann nicht
einfach drauf los schreiben, sondern muss den Prozess einer Registrierung durch
gehen. In der Regel bieten die Online-Zeitungen eine eigene Registrierung an.
Man ist dann auf dem Portal angemeldet und sieht es auch (außer Hamburger
Morgenpost und Südwestpresse). Manche Zeitungen ermöglichen auch eine Anmeldung
über Dritte Anbieter. Das ist dann eine Kompromisslösung für die Nutzer. Eine
Anmeldefunktion wird nämlich auch als Hemmschwelle zum Schreiben diskutiert.
Kann man sich über ein Konto anmelden, das man eh schon hat, dann gilt dieses
Argument nicht mehr. Der Trend, den Singer (2014, S. 69)
beschreibt\footnote{\glqq Some newspapers also had begun requiring users to
comment through Facebook, an intriguing step that removes many of the problems
created by anonymous postings while also helping generate social network
traffic. Such trends richly deserve the attention that journalism scholars have
begun to afford them.\grqq} bestätigt sich also.

Durch eine Anmeldung wird sicher gestellt, dass die Person, die einen Kommentar
verfasst, existiert und gegebenenfalls zur Verantwortung gezogen werden kann.
Natürlich ist auch hier Missbrauch nicht ausgeschlossen. Der Nutzer wird auf
wahrheitsgemäße Angaben hin gewiesen. 

Interessant ist, welche Angaben die Zeitungen zur Anmeldung fordern. Unabdingbar
sind die Eingabe einer Emailadresse sowie ein Passwort. Dann kommt es wieder zu
unterschiedlichen Handhabungen und unterschiedlichen Lösungen. Manche Zeitungen
verlangen die Nennung von Klarnamen und Benutzernamen, manche nur Klarnamen,
manche nur Benutzernamen. Einige Portale geben zusätzliche Hinweise zur
Registrierung. 


\begin{landscape}\footnotesize
  \begin{longtable}{l|p{28mm}p{20mm}p{20mm}p{90mm}}
  \caption{Kategorie Registrierung}%\footnote{Die Angabe einer Emailadresse und eines Passworts wird immer verlangt. In der Tabelle sind weitere obligatorische Angaben, Unterkategorien, aufgeführt.}
  \\
\bfseries Portal & \bfseries alternative \mbox{Anmeldung} & \bfseries Klarname & \bfseries Benutzer\-name & \bfseries Sonstiges \\\hline
\endfirsthead

\bfseries Portal & \bfseries alternative \mbox{Anmeldung} & \bfseries Klarname & \bfseries Benutzer\-name & \bfseries Sonstiges \\ \hline
\endhead

\hline \multicolumn{5}{r}{\emph{Fortsetzung auf der nächsten Seite}}
\endfoot

\hline
\endlastfoot

  bild.de &
    mypass, Facebook & ja & ja &
    Volljährigkeit bzw. Einverständnis der Erziehungsberechtigten bei
    Minderjährigen \\\hline

  spiegel.de & % Spalte 2 bei ''mein spiegel'' als Abonnent oder Nicht-Abonnent
    Facebook & ja & ja\footnote{Es wird darauf hingewiesen, dass der Benutzername angezeigt wird.\label{foot:angezeigt}} &
    \\\hline

  faz.net & % Spalte 3 bei ''Mein FAZ.NET'' mit
    keine & ja & nein &
    \\\hline

  focus.de & % Spalte 4 bei FOCUS online
    Facebook & ja\footnote{Es wird darauf hingewiesen, dass der Name im gesamten Internet recherchierbar ist.} & nein &
    \\\hline

  diewelt.de & % Spalte 5 bei WELT DIGITAL über
    mypass, Disqus, Facebook, Twitter, Google & ja & ja &
    als Gast schreiben \\\hline

  derwesten.de & % Spalte 6 auf derwesten.de mit
    keine & ja & ja\footref{foot:angezeigt} &
    jeder ist zugangs- und teilnahmeberechtigt \\\hline

  rp-online.de & % Spalte 9 bei mein RP ONLINE mit
    keine & nein & ja &
    auch Minderjährige, wenn sie sich über Nutzung bewusst sind bzw. mit
    Einverständnis der Erziehungsberechtigten \\\hline

  handelsblatt.de & % Spalte 10 auf handelsblatt.de mit
    keine & ja & nein &
    Volljährigkeit \\\hline

  suedkurier.de & % Spalte 11auf suedkurier.de mit
    keine & ja & ja &
    Anrede, Land, Adresse (Angaben werden auf Richtigkeit geprüft (teilweise
    auch telefonisch); Einverständnis nach sechs Monaten Inaktivität wird
    Registrierung/Benutzername gesperrt/freigegeben \\\hline

  zeit.de & % Spalte 12 auf zeit.de mit
    keine & nein & ja &
    \\\hline

  badische-zeitung.de & % Spalte 13 bei Meine BZ mit
    keine & ja & nein &
    Anrede, Löschen des Accounts bei Wegwerf-Emailadresse
    \\\hline

  stuttgarter-zeitung.de & % Spalte 14 auf stuttgarterzeitung.de
    Facebook & ja & nein &
    \\\hline

  merkur.de & % Spalte 15 bei Merkur-Online
    Disqus, Facebook, Twitter, Google & nein & ja\footnote{Ein Benutzername ist erwünscht und soll angemessen sein.} &
    als Gast schreiben \\\hline

  hna.de & % Spalte 16 mit HNA-Login
    Disqus & nein & ja\footnote{Der Benutzername soll angemessen und nicht beleidigend/anstößig sein.} &
    \\\hline

  mopo.de & % Spalte 17
    Disqus, Facebook, Twitter, Google & nein & nein &
    \\\hline

  mainpost.de & % Spalte 18 auf Mainpost.de
    keine & ja & ja\footnote{Der Benutzername soll nicht beleidigend, ehrverletzend, hetzerisch sein.} &
    Adresse \\\hline

  tagesspiegel.de & % Spalte 19 auf tagesspiegel.de
    keine & nein & ja\footnote{Der Benutzername soll angemessen sein.} &
    \\\hline

  swp.de & % Spalte 20
    keine & ja & ja &
    \\\hline

  augsburger-allgemeine.de &%spalte 21
    keine & ja & ja & Wohnort, Ausschluss bei Mehrfach-/Scheinregistrierungen oder unter dem Namen anderer, bei Minderjährigen Einverständnis der Erziehungsberechtigten
    \\\hline

\end{longtable}
\end{landscape}

% vim: set ai si et tw=80 sts=2 ts=2 sw=2:
