\chapter{Tabellarische Darstellung der Kategorien}
%1. Art der Moderation
%2. Registrierung
%3. Inhaltliche Regeln
%4. Zusammenfassung

\section{Einführung und Allgemeines}

Sprechblasen bei den einzelnen Artikeln der Online-Zeitungen weisen in der Regel darauf hin, dass kommentiert wurde
(mit Angabe der Anzahl der abgegebenen Kommentare) und/oder ein Kommentar
geschrieben werden kann.  Beim Kommentar selbst ist eine Zeitangabe üblich
entweder mit Datum und Uhrzeit oder \glqq vor \ldots Stunden\grqq.  Kommentare
stehen unter dem Artikel in vielen Fällen nach Werbung (in eigener Sache).\\
Beobachtung bei den deutschen Nachrichtenportalen:\\
Auch auf formaler Ebene werden Kommentare beschränkt. Es gibt ganz
unterschiedliche Handhabungen. Manche Zeitungen bieten Kommentierzeiten an. Nur
dann gibt es die Möglichkeit zu schreiben. Außerhalb dieser Zeiten ist es nicht
möglich. Andere Zeitungen schließen die Kommentarfunktion nach einer bestimmten
Zeit. Dann können die verfassten Kommentare zwar noch gelesen werden, aber keine
neuen mehr geschrieben. Bei manchen Zeitungen ist nicht jeder Artikel
kommentierbar. Es gibt Zeitungen, die nur volljährigen Nutzern einen Zugang zum
Kommentieren erlauben.  Die Sprechblase, die üblicherweise dafür verwendet wird
Kommentare anzukündigen oder auf bereits geschriebenen Kommentare hinzuweisen,
wird nicht immer verwendet.

\begin{landscape} \footnotesize
\begin{longtable}{lp{170mm}}

  \caption{Allgemeines zur Kommentarfunktion auf Nachrichtenportalen} \\ \\
  \toprule
  \bfseries Portal & \bfseries Kommentarfunktion \\\midrule[\heavyrulewidth]
  \endfirsthead

  \toprule
  \bfseries Portal & \bfseries Kommentarfunktion \\\midrule[\heavyrulewidth]
  \endfirsthead

  \bottomrule
  \multicolumn{2}{r}{\emph{Fortsetzung auf der nächsten Seite}}
  \endfoot

  \bottomrule
  \endlastfoot

%Spalte 1		 % Spalte 1
bild.de			& K. nicht zu allen Artikeln möglich \\\hline
%Spalte 2		 % Spalte 2
spiegel.de		  	& keine Sprechblase sondern Hinweis auf [Forum]; beim Beitrag: Sprechblase mit Ausrufungszeichen; K. durchnummeriert \\\hline
%Spalte 3		 % Spalte 3
faz.net			& zeitliche Begrenzung um K. zu schreiben; Hinweis, dass die K. im Internet recherchierbar sind\\\hline
%Spalte 4		 % Spalte 4
focus.de			& auch Videos kommentierbar \\\hline
% Spalte 5		 % Spalte 5
diewelt.de			& über Disqus verwaltet; Schließung nach zwei/drei Tagen oder früher bei Regelverstößen \\\hline
% Spalte 6		 % Spalte 6
derwesten.de		& kein Link auf K., keine Sprechblasen; alle Beiträge kommentierbar; K. durchnummeriert \\\hline
% Spalte 9		 % Spalte 9
rp-online.de		& fast jedes Thema kommentierbar \\\hline
% Spalte 10		 % Spalte 10
handelsblatt.de		& Kommentierzeiten: 7.30 - 21 Uhr, bis zu sieben Tage lang; K. werden (u.U. gekürzt) multimedial verbreitet \\\hline
% Spalte 11		 % Spalte 11
suedkurier.de		& K. nicht zu allen Artikeln möglich \\\hline
% Spalte 12		 % Spalte 12
zeit.de			& keine Sprechblase aber [Anzahl + Kommentare]; K. werden durchnummeriert;  K. nicht zu allen Artikeln möglich; Definition von K.: ``kürzere Textbeiträge, die Sie unter vorhandenen Artikeln, Videos, Fotostrecken oder anderen Multimedia-Inhalten abgeben können'' \\\hline
% Spalte 13		 % Spalte 13: TODO leer?
badische-zeitung.de	& \\\hline
% Spalte 14		 % Spalte 14
stuttgarter-zeitung.de	& für alle Nutzer, K. geben nicht die Meinung der Stuttgarter Zeitung wieder\\\hline
% Spalte 15		 % Spalte 15
merkur.de			& über Disqus verwaltet, für alle Nutzer\\\hline
% Spalte 16		% Spalte 16
hna.de			& über Disqus verwaltet, für alle Nutzer; freie Meinungsäußerung; bei Verstößen Aufruf eine Email zu schreiben \\\hline
% Spalte 17		 % Spalte 17
mopo.de			& über Disqus verwaltet; Kommentierzeiten: 8 - 21 Uhr\\\hline
% Spalte 18		% Spalte 18
mainpost.de		& K. unter Artikel und neben Artikel (als neuer Tab!); begrenzter Zugang\footnote{Das heißt, dass nur eine bestimmte Zahl an Artikeln im Monat kostenlos lesbar sind.}\\\hline
% Spalte 19		 % Spalte 19
tagesspiegel.de		& keine Sprechblase aber [Anzahl + Kommentare]\\\hline
% Spalte 20		% Spalte 20
swp.de			& kein Link auf K.; begrenzter Zugang\\\hline
%Spalte 21		
augsburger-allgemeine.de	& begrenzter Zugang; K. durchnummeriert

\end{longtable}
\end{landscape}

% vim: set ai si et tw=80 sts=2 ts=2 sw=2:
