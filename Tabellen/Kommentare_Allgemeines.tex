
\begin{landscape} \footnotesize
%\begingroup \footnotesize
	\begin{longtable}{l|p{170mm}}

  \caption{Allgemeines zur Kommentarfunktion auf Nachrichtenportalen.} \\

& \bfseries Kommentare\footnote{Kommentare werden in der Regel mit Sprechblasen
  angekündigt; mit Angabe der Anzahl der abgegebenen K.; auch eine Zeitangabe
  ist üblich entweder mit Datum und Uhrzeit oder \glqq vor \ldots Stunden''; K.
  stehen unter dem Artikel in vielen Fällen nach Werbung (in eigener Sache)}
  \\\hline
\endfirsthead

& \bfseries Kommentare \\\hline
\endhead

%abweichende Angaben und Sonstiges
%  Sonstiges \\\hline

\hline \multicolumn{2}{r}{\emph{Fortsetzung auf der nächsten Seite}}
\endfoot

\hline
\endlastfoot

\hline
%Zeile 1: Portale	 % Zeile 2
%Spalte 1		 % Spalte 1
bild.de			& K. nicht zu allen Artikeln möglich \\\hline
%Spalte 2		 % Spalte 2
spiegel.de		& Startseite: keine Sprechblase sondern Hinweis auf [Forum]; beim Beitrag: Sprechblase mit Ausrufungszeichen; K. werden durchnummeriert \\\hline
%Spalte 3		 % Spalte 3
faz.net			& zeitliche Begrenzung um K. zu schreiben; Hinweis, dass die K. im Internet recherchierbar sind; K. = Lesermeinung \\\hline
%Spalte 4		 % Spalte 4
focus.de		& auch Videos kommentierbar \\\hline
% Spalte 5		 % Spalte 5
diewelt.de		& über Disqus verwaltet; Schließung nach zwei/drei Tagen oder früher bei Regelverstößen \\\hline
% Spalte 6		 % Spalte 6
derwesten.de		& keine Ankündigung der K., keine Sprechblasen; alle Beiträge können kommentiert werden; K. werden durchnummeriert \\\hline
% Spalte 9		 % Spalte 9
rp-online.de		& fast jedes Thema kommentierbar \\\hline
% Spalte 10		 % Spalte 10
handelsblatt.de		& Kommentierzeiten: 7.30 - 21 Uhr, bis zu sieben Tage lang; K. werden (u.U. gekürzt) multimedial verbreitet \\\hline
% Spalte 11		 % Spalte 11
suedkurier.de		& K. nicht zu allen Artikeln möglich \\\hline
% Spalte 12		 % Spalte 12
zeit.de			& keine Sprechblase; Hinweis mit  [Anzahl + Kommentare]; K. werden durchnummeriert;  K. nicht zu allen Artikeln möglich; Definition von K.: ``kürzere Textbeiträge, die Sie unter vorhandenen Artikeln, Videos, Fotostrecken oder anderen Multimedia-Inhalten abgeben können'' \\\hline
% Spalte 13		 % Spalte 13: TODO leer?
badische-zeitung.de	& \\\hline
% Spalte 14		 % Spalte 14
stuttgarter-zeitung.de	& für alle Nutzer, K. geben nicht die Meinung der Stuttgarter Zeitung wieder\\\hline
% Spalte 15		 % Spalte 15
merkur.de			& über Disqus verwaltet, für alle Nutzer\\\hline
% Spalte 16		% Spalte 16
hna.de			& über Disqus verwaltet, für alle Nutzer; freie Meinungsäußerung; bei Verstößen Aufruf eine Email zu schreiben \\\hline
% Spalte 17		 % Spalte 17
mopo.de			& über Disqus verwaltet; Kommentierzeiten: 8 - 21 Uhr\\\hline
% Spalte 18		% Spalte 18
mainpost.de		& K. unter Artikel und neben Artikel (als neuer Tab!); Problem bei Mainpost.de: begrenzter Zugang, nur bestimmte Zahl an Artikeln im Monat kostenlos lesbar\\\hline
% Spalte 19		 % Spalte 19
tagesspiegel.de		& keine Sprechblase; Hinweis mit [Anzahl + Kommentare]\\\hline
% Spalte 20		% Spalte 20
swp.de			& kein Hinweis auf K.; Problem bei swp.de: begrenzter Zugang, nur bestimmte Zahl an Artikeln im Monat kostenlos lesbar\\\hline
%Spalte 21		
augsburger-allgemeine.de	& begrenzter Zugang, nur bestimmte Zahl an Artikeln im Monat kostenlos lesbar; K.durchnummeriert
\\ \hline               

\end{longtable}
\end{landscape}

Beobachtung:
Auch auf formaler Ebene werden Kommentare beschränkt. Es gibt ganz unterschiedliche Handhabungen. Manche Zeitungen bieten Kommentierzeiten an. Nur dann gibt es die Möglichkeit zu schreiben. Außerhalb dieser Zeiten ist es nicht möglich. Andere Zeitungen schließen die Kommentarfunktion nach einer bestimmten Zeit. Dann können die verfassten Kommentare zwar noch gelesen werden, aber keine neuen mehr geschrieben. Bei manchen Zeitungen ist nicht jeder Artikel kommentierbar. Es gibt Zeitungen, die nur volljährigen Nutzer einen Zugang zum Kommentieren erlauben. 


(Die Sprechblase, die üblicherweise dafür verwendet wird, Kommentare anzukündigen oder auf bereits geschriebenen Kommentare hinzuweisen, wird nicht immer verwendet.)

