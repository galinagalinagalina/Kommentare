\begin{landscape}
\footnotesize
\begin{longtable}{l*{5}{p{32mm}}}

  \caption{Sonstiges} \\ \\
  \toprule
  \bfseries Portal & \bfseries Teilen & \bfseries Community & \bfseries Funktionen bem Artikel & \bfseries Felder auf Startseite & \bfseries Außergewöhnliches\\
  \midrule[\heavyrulewidth]
  \endfirsthead

  \toprule
  \bfseries Portal & \bfseries Teilen & \bfseries Community & \bfseries Funktionen bem Artikel & \bfseries Felder auf Startseite & \bfseries Außergewöhnliches\\
  \midrule[\heavyrulewidth]
  \endhead

  \multicolumn{6}{r}{\emph{Fortsetzung auf der nächsten Seite}}
  \endfoot

  \bottomrule
  \endlastfoot

  % Bemerkungen: Spaltennamen waren ursprünglich:
  % - Artikel teilen auf sozialen Netzwerken (Facebook ``teilen'' und/oder ``empfehlen'', Twitter, g+) und durch Versenden (Briefsymbol)
  % - Kommentar in der Community/Forum
  % - Funktionen beim Artikel
  % - Felder auf Startseite
  % - Außergewöhnliches


%Zeile 1
bild.de
& tumblr, Pinterest; K. gleichzeitig auf Facebook veröffentlichen möglich
& Profil; Ranglisten der Nutzer; Chronologie der Kommentare der Nutzer;
  kommentiert letzte 24 h (Top 5)
& Korrektur-Button\footnote{Formular zum Versenden an die Redaktion mit
  Hinweisen auf Fehler oder anderes}
&
& ``Reaktionen'' möglich: Lachen, Weinen, Wut, Staunen, Wow (zur Auswahl,
  welche Reaktion man zu dem Beitrag empfindet)
\\\midrule

%Zeile 2
spiegel.de
& Xing, LinkedIn, Tumbl, Pinterest, deli.cio.us, Diggy, reddit
& meistkommentierte Themen (Top 5); eigene Beiträge anzeigen; Sichtbarmachen
  der Emailadresse für andere Teilnehmer
& Button ``merken'' (auf die Merkliste im Benutzerprofil setzen); Button
  ``feedback'' (Feedback an die Redaktion über Formular)
&
&
\\\midrule

%Zeile 3
faz.net
&
& Profilbild möglich; jüngste/älteste Lesermeinung, viel/wenig diskutiert,
  viel/wenig empfohlen, TOP-Argumente; K. verwalten/von der Veröffentlichung
  zurückziehen
& Beitrag empfehlen, Beitrag merken, Permalink, Drucker
&
& sämtliche Buttons/Symbole/Funktionen mit Hilfe/Erklärungen
\\\midrule

%Zeile 4
focus.de
& Facebook ``gefällt mir'', Xing
& Chronologie; aktivste Kommentatoren (des Monats, gesamt, top 50), Kommentar
  des Tages; Videofavoriten der Leser (meistkommentiert, top 20)
& Fehler-Melden; Beitrag ``Bewerten'' mit Sternen (Anzahl Bewertungen)
&
& Leserbericht schreiben (zusätzlich zum Kommentar mit mehr Zeichen (4000),
  persönliche Erfahrungen), Kommentare abonnieren
\\\midrule

% Zeile 5
diewelt.de
&
& Profilfoto möglich, DIE WELT auf Disqus: neueste K.; Top Kommentatoren
&
&
& Hinweis bei Löschung: ``Dieser Kommentar wurde entfernt'' (Antworten darauf
  aber noch sichtbar)
\\\midrule

% Zeile 6
derwesten.de
&
&
&
&
&
\\\midrule

% Zeile 9
rp-online
&
&
& Beitrag empfehlen, Drucken, Schriftgröße ändern
&
& Kontakt mit der Zeitung über Email an den Chefredakteur, Newsletter,
  Leserbrief schreiben (über Formular)
\\\midrule

% Zeile 10
handelsblatt
& Xing, Email schreiben
&
& Beitrag ``Merken''
&
&
\\\midrule

% Zeile 11
suedkurier
&
&
&
& ``Meistkommentiert''  (Top 3)
& Leserreporter-Beitrag schreiben
\\\midrule

% Zeile 12
zeit.de
& keine Symbole/Verweise auf Startseite
& Profil anlegen möglich
& Drucken, als PDF speichern
& ``Meistgelesen''/''Meistkommentiert'' (Top 5)
& Leserartikel schreiben = ausführliche Meinungsbeiträge und Erfahrungsberichte
  (meistgelesene/meistkommentierte Leserartikel, Top 3 auf Leserartikel-Seite),
  Debattenkultur: ``Aus den Kommentaren'' (Höhepunkte aus den Leserdebatten mit
  neuer Fragestellung), Kommentarkultur: ``Bitte weichen Sie vom Thema ab''
  (Experiment: Kommentieren ohne Artikel), Empfehlungen bei Facebook (aktuelle
  Empfehlungen aus Facebook-Freundeskreis), Tweets von ZEIT ONLINE Politik
\\\midrule

% Zeile 13
badische zeitung
& keine Symbole/Verweise auf Startseite; kein g+; Versenden (kein Briefsymbol),
  Verlinken
& Profilbild (optional)
& Drucken, Vorlesen, Fehler-Melden
& ``Meistkommentiert'' (Top 5)/''zuletzt kommentiert''
& Nutzer registriert seit [...] + Anzahl der bereits geschriebenen K. vom
  Nutzer; Vorschau möglich: man kann K. sehen, wie er online aussehen wird
\\\midrule

% Zeile 14
stuttgarter zeitung
&
&
&
&
&
\\\midrule

% Zeile 15
merkur
& Youtube; Briefsymbol bedeutet Feedback geben (kein Versenden)
& Profilbild (optional; angemessen, Verbot von rassistischen, pornografischen,
  menschenverachtenden, beleidigenden oder gegen die guten Sitten verstoßenden
  Abbildungen); Merkur Online Community auf Disqus: neueste K.; Top
  Kommentatoren
&
&
&
\\\midrule

% Zeile 16
hna
&
& HNA Community auf Disqus: neueste K.; Top Kommentatoren; Profilbild
  (optional; Urheberrecht beachten/Recht auf eigenes Bild; keine Werbung/Logos;
  kein beleidigendes, verletzendes Motiv; Verbot von rassistischen,
  pornografischen, menschenverachtenden, beleidigenden oder gegen die guten
  Sitten verstoßenden Abbildungen)
&
&
&
\\\midrule

% Zeile 17
mopo
&
& mopo.de Community auf Disqus: neueste K.; Top Kommentatoren
&
&
&
\\\midrule

% Zeile 18
mainpost
&
& Auswahl auf Profilseite: niemandem/allen/nur Mitgliedern zeigen
& zur Auswahl: ``ich möchte bei neuen K. per Email benachrichtigt werden''
& ``aktuelle Leserkommentare'', ``kommentiert'' (Top 5), Teilen auf Youtube
& Anzahl der bereits geschriebenen K. vom Nutzer, Kontakt mit Redaktion,
  Sicherheitsfrage
\\\midrule

% Zeile 19
tagesspiegel
&
& Profil anlegen möglich (Profilbild optional/angemessen), Nutzer-Statistik
& Newsletter abonnieren, Drucken, Lesezeichen setzen
&
& Community-Statistik
\\\midrule

% Zeile 20
swp
&
&
&
&
&
\\\midrule

%Zeile 21
Augsburger Allgemeine
&
& Profilbild, registriert seit, Beiträge (Anzahl), Bewertungen zum Kommentator
  \glqq Sehr erfahrenes/äußerst erfahrenes/erffahrenes Mitglied\grqq, Redaktion
  ist Ehrenmitglied und macht mit dem Artikel den ersten Beitrag
&
& andere Orte zum Diskutieren: Plauderecke, Diskussionen (mit Ordnern mit
  verschiedenen Themen; Anzeige \glqq letzer Beitrag\grqq, Anzahl der Themen
  und Beiträge) und Forum (Auswahl nach \glqq neuesten/aktuellsten Themen\grqq,
  Nutzerliste, Suche, Regeln, Hilfe). Suche nach Nutzern möglich: viele
  Übersichten, was es für K. gibt, wie viele, über welche Themen, usw.
&

\end{longtable}
\end{landscape}

% vim: set ai et tw=80 sts=2 ts=2 sw=2:
