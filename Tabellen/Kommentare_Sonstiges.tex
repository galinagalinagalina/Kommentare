\section{Kategorie: \glqq Sonstiges\grqq} 

mit den Unterkategorien:
\begin{itemize}
\item\glqq{\bfseries Teilen}\grqq:\\
Neben der Möglichkeit eigene Beiträge zu verfassen, ist es den Nutzern auch wichtig, Inhalte mit anderen teilen zu können (siehe 3.5). Es gibt bei den Artikeln überall die Funktion, ihn auf dem sozialen Netzwerk Facebook zu \glqq teilen\grqq\ und/oder zu \glqq empfehlen\grqq. Ebenso ist \glqq Teilen\grqq\ auf Twitter und Google+ möglich. Man kann einen Artikel auch überall versenden (wird mit dem Briefsymbol gekennzeichnet). Ausnahmen werden erwähnt. 



\item\glqq{\bfseries Community}\grqq: \\
Hier wird angegeben, ob die Online-Zeitung eine Community anbietet, wo der Nutzer ein Profil anlegen und weitere Funktionen einer Community nutzen kann. 


\item\glqq{\bfseries Funktionen beim Artikel}\grqq:\\
Was bietet die Online-Zeitung für Funktionen beim Artikel an, die im Zusammenhang mit dem Verfassen eines Kommentars stehen?


\item\glqq{\bfseries Außergewöhnliches}\grqq:\\
Was bietet die Online-Zeitung an, das in Zusammenhang mit dem Verfassen eines Kommentars steht?

\end{itemize}


Profil: Man kann ein Profil anlegen; Profilbild: Man kann ein Bild hochladen.\\
Disqus: Es gibt eine Community der entsprechenden Online-Zeitung auf Disqus. Disqus bzw. die Zeitung bietet dort eine Rangliste nach \glqq neuesten Kommentaren\grqq und  \glqq Top Kommentatoren\grqq\ an. \\
Ranglisten sind z.B. neueste/älteste Lesermeinung, viel/wenig diskutiert, viel/wenig empfohlen, TOP-Argumente meistkommentierte Themen, usw. \\
Chronologie = Chronologie der eingegangen Kommentare\\
Empfehlen =\\
Nutzer-Statistik = Nutzer registriert seit [...] und Anzahl der vom Nutzer verfassten Beiträge





\begin{landscape}
\footnotesize
\begin{longtable}{l*{4}{p{32mm}}}

  \caption{Sonstiges} \\ \\
  \toprule
  \bfseries Portal & \bfseries Teilen & \bfseries Community & \bfseries Funktionen beim Artikel  & \bfseries Außergewöhnliches\\
  \midrule[\heavyrulewidth]
  \endfirsthead

  \toprule
  \bfseries Portal & \bfseries Teilen & \bfseries Community & \bfseries Funktionen beim Artikel & \bfseries Außergewöhnliches\\
  \midrule[\heavyrulewidth]
  \endhead

  \multicolumn{5}{r}{\emph{Fortsetzung auf der nächsten Seite}}
  \endfoot

  \bottomrule
  \endlastfoot

  


%Zeile 1
bild.de
& tumblr, Pinterest; K. gleichzeitig auf Facebook veröffentlichen möglich
& Profil; Ranglisten; Chronologie
  %Chronologie der K. der Nutzer;
  %kommentiert letzte 24 h (Top 5)
& Korrektur\footnote{Formular zum Versenden an die Redaktion mit
  Hinweisen auf Fehler oder anderes}
& \glqq Reaktionen\grqq\ zum Anklicken\footnote{\emph{Lachen}, \emph{Weinen}, \emph{Wut}, \emph{Staunen}, \emph{Wow} stehen zur Auswahl,
  welche Reaktion man zu dem Beitrag empfindet}
\\\midrule

%Zeile 2
spiegel.de
& Xing, LinkedIn, tumblr, Pinterest, deli.cio.us, Diggy, reddit
& \glqq mein Spiegel\grqq: Ranglisten; Sichtbarmachen  der Emailadresse für andere Teilnehmer
%meistkommentierte Themen (Top 5); 
& Merken\footnote{Artikel auf eine Merkliste setzen}; 
  Feedback\footnote{Feedback an die Redaktion über Formular}
&
\\\midrule

%Zeile 3
faz.net
&
& Profilbild; Ranglisten; K. verwalten/von der Veröffentlichung  zurückziehen
%jüngste/älteste Lesermeinung, viel/wenig diskutiert, viel/wenig empfohlen, 
%TOP-Argumente; 

& Empfehlen, Merken, Permalink, Drucker
& sämtliche Buttons/Symbole/Funktionen mit Hilfe/Erklärungen
\\\midrule

%Zeile 4
focus.de
& Facebook ``gefällt mir'', Xing
& Chronologie; Ranglisten 
%aktivste Kommentatoren (des Monats, gesamt, top 50), K. des Tages; Videofavoriten der Leser (meistkommentiert, top 20)
& Korrektur %mit Sternen (Anzahl Bewertungen)
& Leserbericht schreiben\footnote{Es kann ein Text mit mehr Zeichen (4000) verfasst werden, in dem 
  persönliche Erfahrungen beschrieben werden}; Kommentare abonnieren
\\\midrule

% Zeile 5
diewelt.de
&
& Disqus mit Profilbild\footnote{Beim Profilbild das Urheberrecht beachten/Recht auf eigenes Bild; keine Werbung/Logos;  kein beleidigendes, verletzendes Motiv; Verbot von rassistischen, pornografischen, menschenverachtenden, beleidigenden oder gegen die guten Sitten verstoßenden Abbildungen}
&
& Hinweis bei Löschung: \glqq Dieser Kommentar wurde entfernt\grqq\footnote{Antworten darauf
  sind aber noch sichtbar.}
\\\midrule

% Zeile 6
derwesten.de
&
&
&
&
\\\midrule

% Zeile 9
rp-online.de
&
&
& Empfehlen, Drucken, Schriftgröße ändern
& Hinweis zu Kontakt mit der Zeitung\footnote{Email an den Chefredakteur, Leserbrief schreiben}
\\\midrule

% Zeile 10
handelsblatt.de
& Xing, Email schreiben
&
& Merken
&
\\\midrule

% Zeile 11
suedkurier.de
&
&
&
& Leserreporter-Beitrag schreiben; \glqq Meistkommentiert \grqq
\\\midrule

% Zeile 12
zeit.de
& keine Symbole/Verweise auf Startseite
& Profil
& Drucken, als PDF speichern
& Leserbericht schreiben %= ausführliche Meinungsbeiträge und Erfahrungsberichte
  %(meistgelesene/meistkommentierte Leserartikel, Top 3 auf Leserartikel-Seite)
  mit Rankings;
 \glqq Aus den Kommentaren\grqq\footnote{Höhepunkte aus den Leserdebatten mit
  neuer Fragestellung};
\glqq Bitte weichen Sie vom Thema ab\grqq\footnote{Dabei handelt es sich um ein Experiment, bei dem die Nutzer kommentieren können, ohne sich auf einen Artikel beziehen zu müssen};
Empfehlungen bei Facebook\footnote{aktuelle
  Empfehlungen aus Facebook-Freundeskreis};
   Tweets von ZEIT ONLINE Politik; \glqq Meistkommentiert/-gelesen\grqq
\\\midrule

% Zeile 13
badische-zeitung.de
& keine Symbole/Verweise auf Startseite; kein g+; Versenden (kein Briefsymbol),
  Verlinken
& Profilbild 
& Drucken, Vorlesen, Fehler-Melden
& Nutzer-Statistik; Vorschau; \glqq Meist-/Zuletztkommentiert\grqq %Nutzer registriert seit [...] + Anzahl Beiträge; %der bereits geschriebenen K. vom Nutzer;   %man kann K. sehen, wie er online aussehen wird
\\\midrule

% Zeile 14
stuttgarter-zeitung.de
&
&
&
&
\\\midrule

% Zeile 15
merkur.de
& Youtube; Briefsymbol bedeutet Feedback geben (kein Versenden)
& Disqus mit Profilbild\footnote{Profilbild soll angemessen sein; es gilt ein Verbot von rassistischen, pornografischen,
  menschenverachtenden, beleidigenden oder gegen die guten Sitten verstoßenden
  Abbildungen}
&
&
\\\midrule

% Zeile 16
hna.de
&
& Disqus mit Profilbild
&
&
\\\midrule

% Zeile 17
mopo.de
&
& Disqus mit Profilbild
&
&
\\\midrule

% Zeile 18
mainpost.de
&
& Profil
& Benachrichtigung bei neuem Kommentar %zur Auswahl: ``ich möchte bei neuen K. per Email benachrichtigt werden''
& Anzahl Beiträge, Kontakt mit Redaktion;  \glqq aktuelle Leserkommentare\grqq, \glqq kommentiert\grqq, Teilen auf Youtube
\\\midrule

% Zeile 19
tagesspiegel.de
&
& Profil; Profilbild (angemessen), Nutzer-Statistik
& Newsletter abonnieren, Drucken, Lesezeichen setzen
& Nutzer-Statistik\footnote{Nutzer war seit ... Tagen dabei, das letzte Mal aktiv vor ... Minuten, ... hat .... Kommentare geschrieben und das Profil wurde ... Mal angesehen}
\\\midrule

% Zeile 20
swp.de
&
&
&
&
\\\midrule

%Zeile 21
augsburger-allgemeine.de
&
& Profil (umfangreich, besonders!); Profilbild 
&
& Nutzer-Statistik; Charakterisierung Kommentator\footnote{Sehr erfahrenes/äußerst erfahrenes/erfahrenes Mitglied; Redaktion
  ist Ehrenmitglied und macht mit dem Artikel den ersten Beitrag}; andere Diskussionen (mit Rankings, Statistiken, usw)
  %andere Orte zum Diskutieren\footnote{Plauderecke, Diskussionen %(mit Ordnern mit verschiedenen Themen; Anzeige \glqq letzer Beitrag\grqq, Anzahl der Themen und Beiträge) 
  %und Forum %(Auswahl nach \glqq neuesten/aktuellsten Themen\grqq, Nutzerliste, Suche, Regeln, Hilfe). 
  %Suche nach Nutzern möglich: viele  Übersichten, was es für K. gibt, wie viele, über welche Themen, usw.

\end{longtable}
\end{landscape}


%Felder auf Startseite :
%südkurier.de:  ``Meistkommentiert''  (Top 3)

%zeit.de: ``Meistgelesen''/''Meistkommentiert'' (Top 5)

%badjsche-zeitunge.de:  ``Meistkommentiert'' (Top 5)/''zuletzt kommentiert''

%mainpost.de: ``aktuelle Leserkommentare'', ``kommentiert'' (Top 5), Teilen auf Youtube

%augsburger-allgemeine.de: andere Orte zum Diskutieren: Plauderecke, Diskussionen (mit Ordnern mit verschiedenen Themen; Anzeige \glqq letzer Beitrag\grqq, Anzahl der Themen und Beiträge) und Forum (Auswahl nach \glqq neuesten/aktuellsten Themen\grqq, Nutzerliste, Suche, Regeln, Hilfe). Suche nach Nutzern möglich: viele  Übersichten, was es für K. gibt, wie viele, über welche Themen, usw.



% vim: set ai et tw=80 sts=2 ts=2 sw=2:
