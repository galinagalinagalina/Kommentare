\section{Kategorie: \glqq Sonstiges\grqq} 

mit den Unterkategorien:
\begin{itemize}[noitemsep]
  \item\glqq{\bfseries Teilen}\grqq:\\
    Neben der Möglichkeit eigene Beiträge zu verfassen, ist es den Nutzern auch
    wichtig, Inhalte mit anderen teilen zu können (siehe 3.5). So verfügt jeder
    Artikel, der auch kommentiert werden kann, über einen Link zu Facebook über
    \glqq teilen\grqq\ und/oder \glqq empfehlen\grqq.  Ebenso ist \glqq
    Teilen\grqq\ auf Twitter und Google+ möglich. Man kann einen Artikel auch
    überall versenden (wird mit dem Briefsymbol gekennzeichnet). Ausnahmen
    werden erwähnt.

  \item\glqq{\bfseries Community}\grqq:\\
    Hier wird angegeben, ob die Online-Zeitung eine Community anbietet, wo der
    Nutzer ein Profil anlegen und weitere Funktionen einer Community nutzen
    kann.
    \begin{itemize}[noitemsep]
      \item Profil: Man kann ein Profil anlegen.
      \item Profilbild: Man kann ein Bild hochladen.
      \item Disqus: Es gibt eine Community der entsprechenden Online-Zeitung auf
        {\slshape Disqus}. {\slshape Disqus} bzw.  die Zeitung bietet dort eine Rangliste nach \glqq
        neuesten Kommentaren\grqq{} und  \glqq Top Kommentatoren\grqq{} an.
      \item Ranglisten: z.B. neueste/älteste Lesermeinung, viel/\-we\-nig
        diskutiert, viel/we\-nig empfohlen, TOP-Argumente, meistkommentierte
        Themen, usw.
      \item Nutzer-Statistik: Nutzer registriert seit [...] und
        Anzahl der vom Nutzer verfassten Beiträge
    \end{itemize}

  \item\glqq{\bfseries Funktionen beim Artikel}\grqq:\\
    Was bietet die Online-Zeitung für Funktionen beim Artikel an, die im
    Zusammenhang mit dem Verfassen eines Kommentars stehen?

  \item\glqq{\bfseries Außergewöhnliches}\grqq:\\
    Was bietet die Online-Zeitung an, das in Zusammenhang mit dem Verfassen
    eines Kommentars steht?
\end{itemize}


\begin{landscape}
\footnotesize
\begin{longtable}{l*{4}{p{32mm}}}

  \caption{Kategorie \glqq Sonstiges\grqq} \\ \\
  \toprule
  \bfseries Portal & \bfseries Teilen & \bfseries Community & \bfseries Funktionen beim Artikel  & \bfseries Außergewöhnliches\\
  \midrule[\heavyrulewidth]
  \endfirsthead

  \toprule
  \bfseries Portal & \bfseries Teilen & \bfseries Community & \bfseries Funktionen beim Artikel & \bfseries Außergewöhnliches\\
  \midrule[\heavyrulewidth]
  \endhead

  \multicolumn{5}{r}{\emph{Fortsetzung auf der nächsten Seite}}
  \endfoot

  \bottomrule
  \endlastfoot

  


%Zeile 1
bild.de
& tumblr, Pinterest; K. gleichzeitig auf Facebook veröffentl. mögl.
& Profil; Chronologie\footnote{Es gibt eine Chronologie der Kommentare bestimmter Kommentatoren.}
  %Chronologie der K. der Nutzer;
  %kommentiert letzte 24 h (Top 5)
& Korrektur\footnote{Formular zum Versenden an die Redaktion mit Hinweisen auf Fehler oder anderes}
& \glqq Reaktionen\grqq\ zum Anklicken\footnote{\emph{Lachen}, \emph{Weinen}, \emph{Wut}, \emph{Staunen}, \emph{Wow} stehen zur Auswahl,
  welche Reaktion man zu dem Beitrag empfindet}; Rangliste Artikel\footnote{Übersicht der am meisten kommentierten Leserartikel}\label{foot:Rangliste} 
\\\midrule

%Zeile 2
spiegel.de
& Xing, LinkedIn, tumblr, Pinterest, deli.cio.us, Diggy, reddit
& \glqq MEIN SPIEGEL\grqq; Rangl.; Email f. and. Nutzer sichtb. machen 
%meistkommentierte Themen (Top 5); 
& Merken\footnote{Artikel auf eine Merkliste setzen}; 
  Feedback\footnote{Feedback an die Redaktion über ein Formular}
&%Rangliste Artikel\footref{foot:Rangliste}
\\\midrule

%Zeile 3
faz.net
&
&  \glqq Mein FAZ.NET\grqq; Profilbild; Rangl.; K. verwalten/nicht veröffentlichen
%jüngste/älteste Lesermeinung, viel/wenig diskutiert, viel/wenig empfohlen, 
%TOP-Argumente; 

& Empfehlen\footnote{Man kann den Artikel mit Sternchen bewerten und empfehlen}, Merken, Permalink, Drucker
& sämtliche Buttons, Symbole, Funk\-tio\-nen mit Hilfe
\\\midrule

%Zeile 4
focus.de
& Facebook \glqq gefällt mir\grqq, Xing
& Chronologie; Ranglisten\footnote{aktivste Kommentatoren (des Monats, gesamt, top 50), K. des Tages; Videofavoriten der Leser (meistkommentiert, top 20)}
& Korrektur %mit Sternen (Anzahl Bewertungen)
& Leserbericht schreiben\footnote{Es kann ein Text mit mehr Zeichen  verfasst werden, in dem 
  persönliche Erfahrungen beschrieben werden\label{foot:Leserbericht}}; K. abonnieren
\\\midrule

% Zeile 5
welt.de
&
& Disqus mit Profilbild\footnote{Die Nutzungsbedingungen (siehe Tabelle 6.3 \glqq inhaltliche Regeln\grqq) gelten auch für das Profilbild.}

%\footnote{Beim Profilbild das Urheberrecht beachten/Recht auf eigenes Bild; keine Werbung/Logos;  kein beleidigendes, verletzendes Motiv; Verbot von rassistischen, pornografischen, menschenverachtenden, beleidigenden oder gegen die guten Sitten verstoßenden Abbildungen}
&
& \glqq Dieser K. wurde entfernt\grqq\footnote{Hinweis bei Löschung des Kommentars, die Antworten darauf
  sind aber noch sichtbar}; Live-Chats mit Autoren
\\\midrule

% Zeile 6
derwesten.de
&
&
&
&
\\\midrule

% Zeile 9
rp-online.de
&
&
& Empfehlen, Drucken, Schriftgröße ändern
& Hinweis zu Kontakt mit der Zeitung\footnote{Email an den Chefredakteur, Leserbrief schreiben}
\\\midrule

% Zeile 10
handelsblatt.com
& Xing, Email schreiben
&
& Merken
&
\\\midrule

% Zeile 11
suedkurier.de
&
&
&
& Leserreporter-Beitrag; Rangliste Artikel\footref{foot:Rangliste}   %Seite \glqq Meistkommentiert \grqq
\\\midrule

% Zeile 12
zeit.de
& kein Teilen auf Startseite\footnote{Beim Artikel ist die Funktion \glqq Teilen\grqq\- vorhanden}
& Profil
& Drucken, als PDF speichern
& Leserbericht\footref{foot:Leserbericht}; %= ausführliche Meinungsbeiträge und Erfahrungsberichte
  %(meistgelesene/meistkommentierte Leserartikel, Top 3 auf Leserartikel-Seite)
  Rangl. Art.\footnote{Übersicht der am meisten gelesenen oder kommentierten Artikel und Top 3 der Artikel}
 \glqq Aus den Kommentaren\grqq\footnote{Höhepunkte aus den Leserdebatten mit
  neuer Fragestellung};
\glqq Bitte weichen Sie vom Thema ab\grqq\footnote{Dabei handelt es sich um ein Experiment, bei dem die Nutzer kommentieren können, ohne sich auf einen Artikel beziehen zu müssen};
Empfehl. bei Facebook\footnote{aktuelle
  Empfehlungen aus dem Facebook-Freundeskreis};
   Tweets v. ZEIT ONLINE Politik %Seite \glqq Meistkommentiert/-gelesen\grqq
\\\midrule

% Zeile 13
badische-zeitung.de
& Versenden\footnote{Wird jedoch nicht mit Briefsymbol gekennzeichnet},
  Verlinken; kein g+; kein Teilen auf Startseite
& Kommentarprofil mit Profilbild möglich (keine Community) 
& Drucken, Vorlesen, Fehler-Melden
& Nutzer-Statistik; Vorschau; Rangliste Artikel\footnote{Übersicht der am meisten und zuletzt kommentierten Artikel}%Seite \glqq Meist-/Zuletztkommentiert\grqq %Nutzer registriert seit [...] + Anzahl Beiträge; %der bereits geschriebenen K. vom Nutzer;   %man kann K. sehen, wie er online aussehen wird
\\\midrule

% Zeile 14
stuttgarter-zeitung.de
&
&
&
&
\\\midrule

% Zeile 15
merkur.de
& Youtube; Briefsymbol = Feedback (kein Versenden)
& Disqus mit Profilbild\footnote{Profilbild soll angemessen sein; es gilt ein Verbot von rassistischen, pornografischen,
  menschenverachtenden, beleidigenden oder gegen die guten Sitten verstoßenden
  Abbildungen}
&
&
\\\midrule

% Zeile 16
hna.de
&
& Disqus mit Profilbild
&
&
\\\midrule

% Zeile 17
mopo.de
&
& Disqus mit Profilbild
&
&
\\\midrule

% Zeile 18
mainpost.de
& Youtube 
& Profil
& Benachrichtigung bei neuem Kommentar %zur Auswahl: ``ich möchte bei neuen K. per Email benachrichtigt werden''
& Kontakt zu Redaktion; Rangliste\footref{foot:Rangliste}, \glqq aktuelle Leserkommentare\grqq
\\\midrule

% Zeile 19
tagesspiegel.de
&
& Profil; Profilbild (angemessen), Nutzer-Statistik\footnote{Nutzer war seit ... Tagen dabei, das letzte Mal aktiv vor ... Minuten, ... hat .... Kommentare geschrieben und das Profil wurde ... Mal angesehen}
& Newsletter abonnieren, Drucken, Lesezeichen setzen
& 
\\\midrule

% Zeile 20
swp.de
&
&
&
&
\\\midrule

%Zeile 21
augsburger-allgemeine.de
&
& Profil (umfangreich, besonders!); Profilbild; Nutzer/Community-Statistik\footnote{mit Anzahl: Themen, Beiträge, angemeldete Nutzer, aktive Nutzer, Gäste, usw.}
&
& Charakterisierung Kommentator\footnote{Sehr erfahrenes/äußerst erfahrenes/erfahrenes Mitglied; Redaktion
  ist Ehrenmitglied und macht mit dem Artikel den ersten Beitrag}; andere Diskussionen und Foren\footnote{in Ordnern mit verschiedenen Themen; \glqq letzter Beitrag\grqq\- 
  wird angezeigt;
  Auswahl nach \glqq neuesten/aktuellsten Themen\grqq\- möglich; Nutzerliste; Suche; Regeln; Hilfe; Suche nach Nutzern möglich; Übersichten der K., wie viele, über welche Themen, usw.}
  %andere Orte zum Diskutieren\footnote{Plauderecke, Diskussionen %(mit Ordnern mit verschiedenen Themen; Anzeige \glqq letzer Beitrag\grqq, Anzahl der Themen und Beiträge) 
  %und Forum %(Auswahl nach \glqq neuesten/aktuellsten Themen\grqq, Nutzerliste, Suche, Regeln, Hilfe). 
  %Suche nach Nutzern möglich: viele  Übersichten, was es für K. gibt, wie viele, über welche Themen, usw.

\end{longtable}
\end{landscape}


%Felder auf Startseite :
%südkurier.de:  ``Meistkommentiert''  (Top 3)

%zeit.de: ``Meistgelesen''/''Meistkommentiert'' (Top 5)

%badjsche-zeitunge.de:  ``Meistkommentiert'' (Top 5)/''zuletzt kommentiert''

%mainpost.de: ``aktuelle Leserkommentare'', ``kommentiert'' (Top 5), Teilen auf Youtube

%augsburger-allgemeine.de: andere Orte zum Diskutieren: Plauderecke, Diskussionen (mit Ordnern mit verschiedenen Themen; Anzeige \glqq letzer Beitrag\grqq, Anzahl der Themen und Beiträge) und Forum (Auswahl nach \glqq neuesten/aktuellsten Themen\grqq, Nutzerliste, Suche, Regeln, Hilfe). Suche nach Nutzern möglich: viele  Übersichten, was es für K. gibt, wie viele, über welche Themen, usw.



% vim: set ai et tw=80 sts=2 ts=2 sw=2:
