\begin{landscape} \footnotesize
\begin{longtable}{l|llll}
\caption{Zusammenfassung der Regeln auf Nachrichtenportalen}\\
\bfseries Portal &\bfseries Moderation &\bfseries Bedingungen &\bfseries Registrierung &\bfseries Community\\ \hline
\endfirsthead
\bfseries Portal &\bfseries Moderation &\bfseries Bedingungen &\bfseries Registrierung &\bfseries Community\\ \hline
\endhead
\hline \multicolumn{5}{r}{\emph{Forsetzung auf der nächsten Seite}}
\endfoot
\hline
\endlastfoot


%Spalte 1		Spalte 1		% Spalte 1			% Spalte 1bei BILD.de-Community über			% Spalte 1
bild.de			& keine			& NB (all/bes) (ZS), Net	&KN, BN, mypass, Fb					& ja \tabularnewline \hline
%Spalte 2               Spalte 2		% Spalte 2			% Spalte 2 bei 'mein spiegel' als Abo. oder Nicht-Abo., % Spalte 2
spiegel.de		& Prä-M.		& NB (all/bes) (ZS)		& KN, BN, Fb 						& ja \tabularnewline \hline
%Spalte 3                % Spalte 3		% Spalte 3			% Spalte 3 bei ''Mein FAZ.NET'' mit			% Spalte 3
faz.net			& Prä-M.		& NB (all) (ZS)			& KN 							& ja \tabularnewline \hline
%Spalte 4                % Spalte 4		% Spalte 4			% Spalte 4 bei FOCUS online, 				% Spalte 4
focus.de		& eing. Prä-M.		& AGB (ZS), Net (ZS)		& KN, Fb						& ja \tabularnewline \hline
% Spalte 5               % Spalte 5		% Spalte 5			% Spalte 5 bei WELT DIGITAL über 			% Spalte 5
diewelt.de		& Prä-M.		& NB, veraltete Net		& KN, BN, mypass, Dis, Fb, Tw, G 			& ja \tabularnewline \hline
% Spalte 6               % Spalte 6		% Spalte 6			% Spalte 6 auf derwesten.de mit				% Spalte 6
derwesten.de		& keine			& NB (ZS), Net 			& KN, BN (angezeigt) 					& nein \tabularnewline \hline
% Spalte 9               % Spalte 9		% Spalte 9			% Spalte 9 bei mein RP ONLINE mi			% Spalte 9
rp-online.de		& keine			& AGB				& BN 							& nein \tabularnewline \hline
% Spalte 10              % Spalte 10		% Spalte 10			% Spalte 10 auf handelsblatt.de mit			% Spalte 10
handelsblatt		& eing. Post-M.		& NB (ZS), Net			& KN 							& nein \tabularnewline \hline
% Spalte 11              % Spalte 11		% Spalte 11			% Spalte 11 auf suedkurier.de mit			% Spalte 11
suedkurier.de		& eing. Post-M.		& NB (ZS), Net			& KN, BN 						& nein \tabularnewline \hline
% Spalte 12              % Spalte 12		% Spalte 12			% Spalte 12 auf zeit.de mit				% Spalte 12
zeit.de			& Prä-M.		& AGB (ZS), Net			& BN 							& nein \tabularnewline \hline
% Spalte 13              % Spalte 13		% Spalte 13			% Spalte 13 bei Meine BZ mit				% Spalte 13
badische-zeitung.de	& eing. Post-M.		& AGB (ZS), Net			& KN 							& nein \tabularnewline \hline
% Spalte 14              % Spalte 14		% Spalte 14			% Spalte 14 auf stuttgarterzeitung.de			% Spalte 14
stuttgarter-zeitung.de	& Prä-M.		& AGB (ZS), Net			& KN, Fb 						& nein \tabularnewline \hline
% Spalte 15              % Spalte 15		% Spalte 15			% Spalte 15 bei Merkur-Online				% Spalte 15
merkur.de			& keine			& AGB (ZS), Net (ZS)		& BN, Dis, Fb, Tw, G, 				& ja \tabularnewline \hline
% Spalte 16              % Spalte 16		% Spalte 16			% Spalte 16 mit HNA-Login				% Spalte 16
hna.de			& keine			& AGB (ZS), Net (ZS)		& BN, Dis 						& ja \tabularnewline \hline
% Spalte 17              % Spalte 17		% Spalte 17			% Spalte 17						% Spalte 17
mopo.de			& keine			&				& Dis, Fb, Tw, G 					& ja \tabularnewline \hline
% Spalte 18              % Spalte 18		% Spalte 18			% Spalte 18 auf Mainpost.de				% Spalte 18
mainpost.de		& Prä-M.		& Net				& KN, BN 						& ja \tabularnewline \hline
% Spalte 19              % Spalte 19		% Spalte 19			% Spalte 19 auf tagesspiegel.de				% Spalte 19
tagesspiegel.de		& Prä-M.		& NB				& BN 							& nein \tabularnewline \hline
% Spalte 20              % Spalte 20		% Spalte 20			% Spalte 20						% Spalte 20
swp.de			& keine/eing. Post-M.	& Net				& KN, BN 						& nein \tabularnewline \hline

% REGISTRIERUNG (mypass = Anmeldung über ``mypass'', Fb = Facebook, Tw = Twitter, G = Google, Dis = Disqus)

\end{longtable}
\end{landscape}

Legende: 
Moderation: was für eine Moderation wird gemacht
Prä-M. = Prä-Moderation
eing. Post-M. = eingeschränkte Post-Moderation

Bedingungen: wo stehen die Regeln und Hinweise zum Kommentieren
NB (all/bes) = Nutzungsbedinungen (allgemein/besonders)
Net = Netiquette
ZS = Zustimmung durch Klick wird verlangt

Registrierung: welche Angaben sind nötig zu einer Registrierung
KN = Klarnamen
BN = Benutzername
alternative Anmeldemöglichkeiten: Fb = Facebook, Dis = Disqus, Tw = Twitter, G = Google



