\section{Kategorie: \glqq formale Regeln\grqq}

mit  den Unterkategorien
\begin{itemize}
\item\glqq {\bf Zeichenbegrenzung}\grqq: Gibt es eine Zeichenbegrenzung oder kann der Nutzer schreiben so viel er möchte?


\item\glqq{\bfÜberschrift}\grqq: Wird die Eingabe einer Überschrift verlangt?


\item\glqq{\bf Sonstiges}\grqq: Welche formalen Hinweise geben die Online-Zeitungen zusätzlich?
\end{itemize}


Beobachtungen bei den Nachrichtenportalen:\\
Die Zeitungen geben auch Hinweise zum formalen Umgang mit Kommentaren. Die Hälfte gibt eine Zeichenbegrenzung vor. Diese reicht von 800 bis 3000 Zeichen. 
Ebenso fordert ungefähr die Hälfte der Redaktionen die Eingabe eines Betreffs. Die sonstigen formalen Regeln haben unterschiedlichen Inhalt. Interessant ist die Möglichkeit bei mainpost.de, tagesspiegel.de und augsburger-allgemeine.de den eingegebenen Text sogar hervorzuheben. Diese Funktion geht über die übliche Art von Kommentaren hinaus. Bei augsburger-allgemeine.de kann der Kommentar vollständig formatiert werden.


\begin{landscape} \footnotesize
  \begin{longtable}{l|llp{100mm}}

  & \multicolumn{3}{c}{\bfseries formale Regeln} \\
  & Zeichenbegrenzung & Überschrift (Pflichtfeld) & Sonstiges \\\hline
  \endfirsthead

  & \multicolumn{3}{c}{\bfseries formale Regeln} \\
  & Zeichenbegrenzung & Überschrift (Pflichtfeld) & Sonstiges \\\hline
  \endhead

  \hline \multicolumn{4}{r}{\emph{Fortsetzung auf der nächsten Seite}}
  \endfoot

  \hline
  \endlastfoot

%Kommentar: formale Regeln\\
%  & Zeichenbegrenzung & Überschrift (Pflichtfeld) & Sonstiges
%
%Zeile 1: Portale	 %
%Spalte 1		
bild.de			& keine & keine & vorsichtig mit Großbuchstaben, Zitate kennzeichnen \\\hline
%Spalte 2	
spiegel.de			& keine & optional & keine Bilder posten, keine langen Zitate (Links verwenden) \\\hline
%Spalte 3		
faz.net			& 1000 Zeichen & ja, 100 Zeichen & \\\hline
%Spalte 4		
focus.de			& 800 Zeichen & ja & reiner Text ohne besondere Kennzeichnungen (z.B. keine Smilies, Hervorhebungen, Chat-Symbole, nur Kleinschreibung, usw.), korrektes Deutsch, auf Rechtschreibung/Interpunktion achten, Absätze machen \\\hline
% Spalte 5		
welt.de			& keine & keine & Zitate kennzeichnen; keine Fremdsprachen, keine Links zu externen Webseiten (seriöse Ausnahmen möglich) \\\hline
% Spalte 6	
derwesten.de		& keine & keine & Zitate nur mit Quellenangabe\\\hline
% Spalte 9		
rp-online.de		& keine & ja & deutsche Sprache; Links möglich (keine Links zu Werbung/strafbare Inhalte) \\\hline
% Spalte 10	
handelsblatt.com		& 2000 & keine & Großbuchstaben (Schreien) vermeiden; Absätze machen und strukturieren; Wortwahl überprüfen; auf Rechtschreibung achten; 	Zitate/Quellen kennzeichnen \\\hline
% Spalte 11		
suedkurier.de		& 1000 Zeichen & ja & Zitate kennzeichnen mit Quellenangabe \\\hline
% Spalte 12		
zeit.de			& 1500 Zeichen & ja, mindestens 5 Zeichen & Absätze machen; auf Rechtschreibung achten; vorsichtig mit Großbuchstaben; Zitate kennzeichnen, wenig verwenden, mit Quellenangabe, nur als Ergänzung verwenden; Links möglich \\\hline
% Spalte 13	
badische-zeitung.de	& keine & keine & \\\hline
% Spalte 14	
stuttgarter-zeitung.de	& keine & ja & Links möglich (keine Links zu Werbung/kommerziellen Angeboten/Chats/Foren/strafbaren Inhalten) \\\hline
% Spalte 15	
merkur.de			& keine & keine & deutsche Sprache; Links nicht erwünscht (falls doch distanziert sich merkur-online von Inhalten der gelinkten Seiten) \\\hline
% Spalte 16	
hna.de			& keine & keine & deutsche Sprache; Beiträge in Fremdsprachen werden gegebenenfalls entfernt, da für größten Teil der Nutzer nicht verständlich; Links nicht erwünscht (falls doch distanziert sich hna.de von Inhalten der gelinkten Seiten) \\\hline
% Spalte 17		
mopo.de			& keine & keine & \\\hline
% Spalte 18		
mainpost.de		& 1000 Zeichen & ja & auf deutsche Rechtschreibung achten; korrekte Interpunktion, Absätze machen; Hervorhebungen möglich: fett, kursiv, unterstrichen; markieren von Links, Zitaten möglich; einfügen von Emojis möglich (Grinsen, Zwinkern, traurig sein); \\\hline
% Spalte 19	
tagesspiegel.de		& 2000 Zeichen & ja & auf Rechtschreibung/Grammatik achten; Hervorhebungen möglich: fett, kursiv; markieren von Quellen durch Links; markieren von Links/Zitaten; keine Links zu anderen Webinhalten; Zitate als Ergänzung, nicht alleinstehend \\\hline
% Spalte 20		
swp.de			& 3000 Zeichen & ja & \\ \hline
%Spalte 21		
augsburger-allgemeine.de	& keine & optional & sämtliche Formatierungsmöglichkeiten, auch Emojis; Vorschau möglich!; Button zum Zitieren; Anhängen von Daten möglich (begrenzte Größe)\\ \hline

\end{longtable}
\end{landscape}


% vim: set ai si et tw=80 sts=2 ts=2 sw=2:
