
Bei einer Moderation geht es vor allem darum, Kommentare auf ihren Inhalt hin zu
prüfen. Verstößt dieser Inhalt gegen den guten Ton und die Regeln, dann greift
der Moderator ein und löscht die Beiträge (Entfernen). Wenn der Nutzer
wiederholt negativ auffällt, dann wird er unter Umständen vom Kommentieren
ausgeschlossen und sein Account wird gesperrt (Sperren). Das kann zeitweise sein
oder auch dauerhaft.

Die Aufforderung, Verstöße zu melden (Melden), ist an die Nutzer gerichtet. Dies
ist umso wichtiger, wenn gar keine Moderation statt findet. Dann sind die Nutzer
selbst diejenigen, die kontrollieren. Aber auch mit Moderation ist eine
Melden-Funktion angebracht. Alle  untersuchten Nachrichtenportale (bis auf
tagesspiegel.de) haben einen entsprechenden Melden-Button.

Stand Anfang 2015: Für welche Art der Moderation haben sich die meistgelesenen
Online-Zeitungen entschieden? Wie moderieren sie: greifen sie vor einer
Veröffentlichung ein oder bearbeiten sie einen Kommentar erst wenn er bereits
erschienen ist? Oder überlassen sie das Feld ganz den Nutzern allein?

Die meisten Zeitungen wollen wissen, was veröffentlicht wird und arbeiten mit
einer Prä-Moderation.

Ein Drittel der Portale hat sich für gar keine Moderation entschieden. Das
bedeutet jedoch nicht, dass alles online stehen bleibt, was von den Nutzern
geschrieben wird. Wird der Redaktion Missbrauch gemeldet, dann werden die
entsprechenden Kommentare gelöscht.

Eine Post-Moderation machen die wenigsten, und wenn, dann mit Einschränkungen,
das heißt, sie prüfen nicht alles, sondern machen Stichproben. Das wird hier mit
\glqq eingeschränkte Post-Moderation\grqq\ bezeichnet. Inwieweit sie aktiv prüfen,
kann nicht herausgefunden werden. Reagieren sie nur auf Melden der Nutzer, dann
ist das eigentlich wie \glqq keine Moderation\grqq\ einzustufen.

Inwieweit Computerprogramme die Moderation übernehmen, kann auch nicht
herausgefunden werden.

Zeile1: Angabe, welche Art der Moderation angewendet wird
Zeile2: gibt es einen Melden-Button?

\begin{landscape} \footnotesize
\begin{longtable}{l|p{110mm}p{50mm}}

  \multicolumn{1}{c}{} &
  \multicolumn{1}{c}{Art der Moderation} &
  \multicolumn{1}{c}{Melde-Button} \\\hline\hline
  \endhead

  \hline \multicolumn{3}{r}{\emph{Fortsetzung auf der nächsten Seite}}
  \endfoot

  \hline
  \endlastfoot

%Zeile 1
bild.de &
  keine; Melden; Entfernen; Sperren &
  ja, mit Angabe von vier Möglichkeiten (Spam, Copyright, beleidigend, anderer
  Grund), kurze Begründung möglich \\\hline

%Zeile 2
spiegel.de &
  Prä-Moderation (zeitliche Verzögerungen); Forumsmoderator mit Namen
  sysop/forum@spiegel.de; Entfernen/keine Veröffentlichung; Redaktion kann
  bearbeiten, verschieben, Diskussionen schließen; keine Benachrichtigung bei
  Nicht-Erscheinen; Melden &
  ja \\\hline

%Zeile 3
faz.net &
  Prä-Moderation (zeitliche Verzögerungen) nach Prüfung den Richtlinien
  entsprechend; Sperren; K. werden evtl. gekürzt/verändert &
  ja \\\hline

%Zeile 4
focus.de &
  eingeschränkte Prä-Moderation: keine umfassende Prüfung, aber Stichproben vom
  Digitalvermarkter; Moderator: TOMORROW FOCUS Media GmbH/TOMORROW FOCUS NEWS+
  GmbH/Beauftragte; Ändern/Entfernen; Sperren &
  ja, besonders hervorgehoben \\\hline

% Zeile 5
diewelt.de &
  Prä-Moderation (zeitliche Verzögerungen); Moderator = Team von
  ``Welt''-Mitarbeitern; Kritik an Moderationsweise per Email; bestimmte
  Moderationszeiten, Moderation gemäß Nutzungsregeln; bei Verstößen (vom
  Beiträgen/Benutzernamen/Profilfoto) Ändern der Beiträge/keine
  Veröffentlichung; Entfernen von Beleidigungen/Beschimpfungen; Sperren
  (zeitweise/dauerhaft); Whitelist: Nutzer können ohne Moderation kommentieren
  (Redaktion und Community-Mitglieder, die auffallend positiv kommentieren) &
  Fähnchen-Button (nicht immer sichtbar) \\\hline

% Zeile 6
derwesten.de &
  keine, automatische Veröffentlichung; Entfernen (zeitweise/ganz) durch
  Community Management; Hinweis, dass Beiträge falsche Tatsachen enthalten,
  Rechte Dritter verletzen, in die Irre führen, täuschen können &
  ja \\\hline

% Zeile 9
rp-online &
  keine; bei Melden Entfernen/Sperren (auch die Antworten dazu); Abbruch der
  K.-Funktion bei Verstößen/nicht themenbezogen (bis dahin veröffentlichte
  Beiträge bleiben); Entfernen/Bearbeiten nach eigenem Ermessen &
  ja (mit Begründung mit Name/Email, auch ohne Registrierung möglich) \\\hline

% Zeile 10
handelsblatt &
  eingeschränkte Post-Moderation: keine umfassende Prüfung, aber Eingriff bei
  Verstößen;  bei Melden Löschen/Sperren; man setzt sich mit dem Nutzer in
  Verbindung; nicht eindeutige Sachverhalte müssen abgeklärt werden; verstoßen
  einzelnen Abschnitte gegen Regeln, werden diese entfernt und der Eingriff
  kenntlich gemacht; wird Kommentar komplett entfernt wird dies auch kenntlich
  gemacht; verstoßen viele Kommentare gegen die Regeln wird die Funktion
  abgeschaltet &
  ja (auch Email oder telefonisch möglich) \\\hline

% Zeile 11
suedkurier &
  eingeschränkte Post-Moderation: keine umfassende Prüfung;  Bearbeiten/Löschen
  (wenn nicht themenbezogen; von Trollen); Nutzer, die regelmäßig gegen Regeln
  verstoßen werden per Email ermahnt; schwere/wiederholte Verstöße führen zum
  Ausschluss der Community; bei gehäuften Verstößen wird Funktion abgeschaltet;
  keine systematische Prüfung externer Links; keine Verantwortung von südkurier
  für externe Links; Löschen externer Links bei Verstößen oder bei Links, die
  auf rechtswidrige Inhalte weiterleiten &
  ja (mit Name, Emailadresse, Grund an Community Manager) \\\hline

% Zeile 12
zeit.de &
  Prä-Moderation (wie bei Leserbriefen); Entfernen/Kürzen (mit Begründung),
  warum eingeschritten wurde mit Anmerkungen und Kennzeichnungen; Sperren bei
  schweren/wiederholten Verstöße; Abbruch der K.-Funktion bei gehäuften
  Verstößen/nicht themenbezogen; keine Verantwortung von zeit.de für verlinkte
  Inhalte; Löschen externer Links bei Verstößen &
  ja \\\hline

% Zeile 13
badische zeitung &
  eingeschränkte Post-Moderation: keine umfassende Prüfung;
  Bearbeiten/Entfernen; schwere/wiederholte Verstöße führen zum Ausschluss der
  Community; keine Crosspostings; Löschen externer Links bei Verstößen oder bei
  Links, die auf rechtswidrige Inhalte weiterleiten &
  ja \\\hline

% Zeile 14
stuttgarter zeitung &
  Prä-Moderation; Bearbeiten/Entfernen/Sperren (wenn nötig, zeitweise, ganz) &
  positiv und negativ \\\hline

% Zeile 15
merkur &
  keine; Entfernen (kein Bearbeiten) von nicht-themenbezogenen
  Beiträgen/privaten Chats/Sperren des Nutzers &
  Fähnchen-Button (nicht immer sichtbar) \\\hline

% Zeile 16
hna &
  keine; kommentarloses Entfernen /Sperren des Nutzers &
  Fähnchen-Button (nicht immer sichtbar) \\\hline

% Zeile 17
mopo &
  keine & Fähnchen-Button (nicht immer sichtbar) \\\hline

% Zeile 18
mainpost &
  Prä-Moderation &
  ja, mit Begründung mit Name/Email wegen Rückfragen \\\hline

% Zeile 19
tagesspiegel &
  Prä-Moderation: Rahmen für sachlichen Austausch von Argumenten schaffen &
  kein Melden \\\hline

% Zeile 20
swp &
  keine/eingeschränkte Post-Moderation; Entfernen; Versuch unerwünschte
  Beiträge fernzuhalten; keine umfassende Überprüfung möglich; &
  ja \\\hline

%Zeile 21
Augsburger Allgemeine &
  Post-Moderation: zurückhaltendes Moderieren, Überwachen der
  Nutzungsbedingungen; Entfernen entsprechender Passagen oder ganz wenn nicht
  themenbezogen, bei Provokationen, gegen Nutzungsregeln; Abbruch der
  K.-Funktion bei Verstößen, nicht themenbezogen nach Ankündigung oder sofort;
  Sperren  des Nutzers (zeitweise/dauerhaft) &
  ja \\

\end{longtable}
\end{landscape}

% vim: set et ai si tw=80 ts=2 sts=2 sw=2:
