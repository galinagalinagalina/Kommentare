\section{Kategorien: \glqq Moderation\grqq und \glqq Melden\grqq}

Bei einer Moderation geht es vor allem darum, Kommentare auf ihren Inhalt hin zu
prüfen. Verstößt dieser Inhalt gegen den guten Ton und die Regeln, dann greift
der Moderator ein und löscht die Beiträge (im Folgenden als \glqq
Entfernen\grqq\ bezeichnet). Wenn der Nutzer wiederholt negativ auffällt, dann
wird er unter Umständen vom Kommentieren ausgeschlossen und sein Account wird
gesperrt (im Folgenden als  \glqq Sperren\grqq\ bezeichnet). Das kann zeitweise
sein oder auch dauerhaft.

Die Aufforderung, Verstöße zu melden (im Folgenden als \glqq Melden\grqq\
bezeichnet), ist an die Nutzer gerichtet. Dies ist umso wichtiger, wenn gar
keine Moderation statt findet. Dann sind die Nutzer selbst diejenigen, die
kontrollieren. Aber auch mit Moderation ist eine Melden-Funktion angebracht.
Alle  untersuchten Nachrichtenportale (bis auf tagesspiegel.de und stuttgarter-zeitung.de) haben einen
entsprechenden Melden-Button. Ein \glqq Melden\grqq\- gehört also 
mittlerweile fast zum Standard einer Kommentarfunktion. 

Stand Anfang 2015: Für welche Art der Moderation haben sich die meistgelesenen
Online-Zeitungen entschieden? Wie moderieren sie: greifen sie vor einer
Veröffentlichung ein oder bearbeiten sie einen Kommentar erst wenn er bereits
erschienen ist? Oder überlassen sie das Feld ganz den Nutzern allein?

Die meisten Zeitungen wollen wissen, was veröffentlicht wird und arbeiten mit
einer Prä-Moderation.  Ein Drittel der Portale hat sich für gar keine Moderation
entschieden. Das sind die regional geprägten Online-Zeitungen, die sich auch für
den Anbieter {\slshape Disqus} entschieden haben. Das bedeutet jedoch nicht,
dass alles online stehen bleibt, was von den Nutzern geschrieben wird.  Wird der
Redaktion Missbrauch gemeldet, dann werden die entsprechenden Kommentare
gelöscht.

Eine Post-Moderation machen die wenigsten, und wenn, dann mit Einschränkungen,
das heißt, sie prüfen nicht alles, sondern machen Stichproben. Das wird hier mit
\glqq stichprobenartige Post-Moderation\grqq{} bezeichnet. Inwieweit sie aktiv
prüfen, kann nicht herausgefunden werden. Reagieren sie nur auf Melden der
Nutzer, dann ist das eigentlich wie \glqq keine Moderation\grqq{} einzustufen.
Inwieweit Computerprogramme die Moderation übernehmen, kann auch nicht
herausgefunden werden.

Die überregionalen Zeitungen entscheiden sich alle für eine Prä-Moderation
(focus.de mit Einschränkungen). bild.de überlässt die Kommentare ganz dem Volk
und macht keine Moderation, misst dem \glqq Melden\grqq{} aber größere Bedeutung
zu. Denn dann übernehmen die Nutzer mit dieser Funktion in gewisser Weise die
Moderation. Auch bei \glqq stichprobenartiger Prä-/Post-Moderation\grqq{} ist
\glqq Melden\grqq{} umfangreicher als bei anderen Moderationen.

%Spalte1: Angabe, welche Art der Moderation angewendet wird
%Spalte2: gibt es einen Melden-Button?

%\begin{landscape}
\begingroup
  \footnotesize
  \begin{longtable}{p{24mm}p{98mm}p{11mm}}
  \caption{Kategorie \glqq Moderation\grqq} \\ \\

  \toprule
  \bfseries Portal &
  \multicolumn{1}{c}{\bfseries Art der Moderation} &
  \bfseries Melde-Button \\
  \midrule[\heavyrulewidth]
  \endfirsthead

  \toprule
  \bfseries Portal &
  \multicolumn{1}{c}{\bfseries Art der Moderation} &
  \bfseries Melde-Button \\
  \midrule[\heavyrulewidth]
  \endhead

  \multicolumn{3}{r}{\emph{Fortsetzung auf der nächsten Seite}}
  \endfoot

  \bottomrule
  \endlastfoot

%Zeile 1
bild.de
& {\bfseries keine}: Entfernen; Sperren (bei Melden!)
& \centerline{ja\footnote{Mit Angabe von vier Möglichkeiten: Spam, Copyright, beleidigend,
  anderer Grund; kurze Begründung möglich}}
\\\midrule

%Zeile 2
spiegel.de
& {\bfseries Prä-Moderation} (zeitliche Verzögerungen); Forumsmoderator mit
  Namen sysop/forum@spiegel.de; Entfernen/keine Veröffentlichung; Redaktion kann
  bearbeiten, verschieben, Diskussionen schließen; keine Benachrichtigung bei
  Nicht-Erscheinen
  & \centerline{ja}
\\\midrule

%Zeile 3
faz.net
& {\bfseries Prä-Moderation}: (zeitliche Verzögerungen) nach Prüfung den
  Richtlinien entsprechend; Sperren; K. werden evtl. gekürzt/verändert
  & \centerline{ja}
\\\midrule

%Zeile 4
focus.de
& {\bfseries stichprobenartige Prä-Moderation}: keine umfassende Prüfung, aber
  Stichproben vom Digitalvermarkter; übertragen der Moderatin an TOMORROW FOCUS Media
  GmbH/TOMORROW FOCUS NEWS+ GmbH/Beauftragte; Ändern/Entfernen; Sperren
  & \centerline{ja\footnote{besonders hervorgehoben}}
\\\midrule

% Zeile 5
welt.de
& {\bfseries Prä-Moderation} (zeitliche Verzögerungen): Moderator = Team von
  \glqq Welt\grqq-Mitarbeitern; Kritik an Moderationsweise per Email; bestimmte
  Moderationszeiten, Moderation gemäß Nutzungsregeln; bei Verstößen (vom
  Beiträgen/Benutzernamen/Profilfoto) Ändern der Beiträge/keine
  Veröffentlichung; Entfernen von Beleidigungen/Beschimpfungen; Sperren
  (zeitweise/dauerhaft); Whitelist: Nutzer können ohne Moderation kommentieren
  (Redaktion und Community-Mitglieder, die auffallend positiv kommentieren)

  & \centerline{ja\footnote{Symbol \glqq Fähnchen\grqq{} mit
  Hovereffekt\label{foot:fahne}}} \\\midrule

% Zeile 6
derwesten.de
& {\bfseries keine}: automatische Veröffentlichung; Entfernen (zeitweise/ganz)
  durch Community Management; Hinweis, dass Beiträge falsche Tatsachen
  enthalten, Rechte Dritter verletzen, in die Irre führen, täuschen können
  & \centerline{ja}
\\\midrule

% Zeile 9
rp-online.de
& {\bfseries keine}: bei Melden Entfernen/Sperren (auch die Antworten dazu);
  Abbruch der K.-Funktion bei Verstößen/nicht themenbezogen (bis dahin
  veröffentlichte Beiträge bleiben); Entfernen/Bearbeiten nach eigenem Ermessen
  & \centerline{ja\footnote{mit Begründung mit Name/Email, auch ohne
  Registrierung möglich}}
\\\midrule

% Zeile 10
handelsblatt.com
& {\bfseries stichprobenartige Post-Moderation}: keine umfassende Prüfung, aber
  Eingriff bei Verstößen;  bei Melden Löschen/Sperren; man setzt sich mit dem
  Nutzer in Verbindung; nicht eindeutige Sachverhalte müssen abgeklärt werden;
  verstoßen einzelnen Abschnitte gegen Regeln, werden diese entfernt und der
  Eingriff kenntlich gemacht; wird Kommentar komplett entfernt wird dies auch
  kenntlich gemacht; verstoßen viele Kommentare gegen die Regeln wird die
  Funktion abgeschaltet
  & \centerline{ja\footnote{auch Email oder telefonisch möglich}}
\\\midrule

% Zeile 11
suedkurier.de
& {\bfseries stichprobenartige Post-Moderation}: keine umfassende Prüfung;
  Bearbeiten/Löschen (wenn nicht themenbezogen; von Trollen); Nutzer, die
  regelmäßig gegen Regeln verstoßen werden per Email ermahnt;
  schwere/wiederholte Verstöße führen zum Ausschluss der Community; bei
  gehäuften Verstößen wird Funktion abgeschaltet; externe Links: keine systematische Prüfung,
 keine Verantwortung von suedkurier.de, Löschen bei Verstößen oder bei Links, die auf rechtswidrige Inhalte
  weiterleiten
  & \centerline{ja\footnote{mit Name, Emailadresse, Grund an Community Manager}}
\\\midrule

% Zeile 12
zeit.de
& {\bfseries Prä-Moderation} (wie bei Leserbriefen): Entfernen/Kürzen (mit
  Begründung), warum eingeschritten wurde mit Anmerkungen und Kennzeichnungen;
  Sperren bei schweren/wiederholten Verstößen; Abbruch der K.-Funktion bei
  gehäuften Verstößen/nicht themenbezogen; externe Links: keine Verantwortung von zeit.de, 
  Löschen bei Verstößen
  & \centerline{ja}
\\\midrule

% Zeile 13
badische-zeitung.de
& {\bfseries stichprobenartige Post-Moderation}: keine umfassende Prüfung;
  Bearbeiten/Entfernen; schwere/wiederholte Verstöße führen zum Ausschluss der
  Community; keine Crosspostings; Löschen externer Links bei Verstößen oder bei
  Links, die auf rechtswidrige Inhalte weiterleiten
  & \centerline{ja}
\\\midrule

% Zeile 14
stuttgarter-zeitung.de
& {\bfseries Prä-Moderation}: Bearbeiten/Entfernen/Sperren (wenn nötig,
  zeitweise, ganz)
  & \centerline{nein}
\\\midrule

% Zeile 15
merkur.de
& {\bfseries keine}: Entfernen (kein Bearbeiten) von nicht-themenbezogenen
  Beiträgen/privaten Chats/Sperren des Nutzers
  & \centerline{ja\footref{foot:fahne}}
\\\midrule

% Zeile 16
hna.de
& {\bfseries keine}: kommentarloses Entfernen/Sperren des Nutzers
& \centerline{ja\footref{foot:fahne}}
\\\midrule

% Zeile 17
mopo.de
& {\bfseries keine}
& \centerline{ja\footref{foot:fahne}}
\\\midrule

% Zeile 18
mainpost.de
& {\bfseries Prä-Moderation}: Löschen von Beiträgen, die gegen das Gesetz verstoßen
& \centerline{ja\footnote{Mit Begründung, mit Name/Email wegen Rückfragen}}
\\\midrule

% Zeile 19
tagesspiegel.de
& {\bfseries Prä-Moderation}: Rahmen für sachlichen Austausch von Argumenten
  schaffen
  & \centerline{nein}
\\\midrule

% Zeile 20
swp.de
& {\bfseries keine/stichprobenartige Post-Moderation}: Entfernen; Versuch
  unerwünschte Beiträge fernzuhalten; keine umfassende Überprüfung möglich
  & \centerline{ja}
\\\midrule

%Zeile 21
augsburger-allgemeine.de

& {\bfseries Post-Moderation}: zurückhaltendes Moderieren, Überwachen der
  Nutzungsbedingungen; Entfernen entsprechender Passagen oder ganz wenn nicht
  themenbezogen, bei Provokationen, gegen Nutzungsregeln; Abbruch der
  K.-Funktion bei Verstößen, nicht themenbezogen nach Ankündigung oder sofort;
  Sperren  des Nutzers (zeitweise/dauerhaft)
  & \centerline{ja}
\\
\end{longtable}

%\end{landscape}
\endgroup

% vim: set et ai tw=80 ts=2 sts=2 sw=2:
