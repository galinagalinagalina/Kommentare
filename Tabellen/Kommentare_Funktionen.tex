\section{Kategorie Funktionen beim Kommentar}

\begin{landscape} \footnotesize
  \begin{longtable}{l|p{40mm}p{40mm}p{80mm}}

  & \multicolumn{3}{c}{\bfseries Kategorie Funktionen beim Kommentar} \\
  & ``Bewerten'' & ``Antworten'' & Sonstiges \\\hline
  \endfirsthead

  & \multicolumn{3}{c}{\bfseries Kommentare: Funktionen} \\
  & ``Bewerten'' & ``Antworten'' & Sonstiges \\\hline
  \endhead

  \hline \multicolumn{4}{r}{\emph{Fortsetzung auf der nächsten Seite}}
  \endfoot

  \hline
  \endlastfoot

% Funktionen im Kommentar	\\
% ``Bewerten''\\ ``Antworten''\\ Sonstiges\\

\hline

  % Zeile 1
  bild.de &
    positiv & & Ordnen nach beliebteste, älteste, neueste K.\\
    & \multicolumn{3}{c}{}\\\hline

  % Zeile 2
  spiegel.de &
  & ja: auf was man antwortet wird in Zitate gesetzt & \\
  & \multicolumn{3}{c}{}\\\hline

  % Zeile 3
  faz.net & positiv & & dem Kommentator folgen \\
          & \multicolumn{3}{c}{}\\\hline

  % Zeile 4
  focus.de & positiv und negativ & ja & \\
           & \multicolumn{3}{c}{}\\\hline

  % Spalte 5
  diewelt.de &
    positiv\footnote{Klicks werden gezählt, nach Anmeldung oder als ``Gast'' möglich}
    und negativ\footnote{nach Anmeldung} &
    ja &
    ``Teilen''-Button (= diese Diskussion auf Twitter/Facebook teilen, nach
    Anmeldung dort), ``Teilen''-Button für einzelnen K., ``Empfehlen''-Button (=
    diese Diskussion empfehlen); Ordnen nach beste, neueste, älteste Kommentare;
    ``Abonnieren'', um Mittleiungen dieser Diskussion per Email zu erhalten;
    Disqus auf der eigenen Seite hinzufügen; Datenschutz\\
    & \multicolumn{3}{c}{Oberfläche von Disqus}\\\hline

  % Spalte 6
  derwesten.de & & ja, mit Anzahl Antworten & \\
               & \multicolumn{3}{c}{}\\\hline

  % Spalte 9
  rp-online.de & positiv & & Ordnen nach älteste; Button für ``mehr K.''\\
            & \multicolumn{3}{c}{}\\\hline

  % Spalte 10
  handelsblatt.de & & ja & \\
               & \multicolumn{3}{c}{}\\\hline

  % Spalte 11
  suedkurier.de & & ja &
    Ordnen nach älteste, neueste, beste Bewertung; zur Auswahl: ``informiert
    bleiben'' (bei jedem neuen Beitrag der Diskussion erhält man
    Benachrichtigung)\\
    & \multicolumn{3}{c}{}\\\hline

  % Spalte 12
  zeit.de & & ja &
    Reaktionen/Antworten auf diesen K. anzeigen; Ordnen nach neuesten,
    empfohlenen, allen K.; Empfehlungen aussprechen; Empfehlungen der Redaktion
    (bedeutet aber nicht, dass die Redaktion der Meinung des Lesers zustimmt)\\
    & \multicolumn{3}{c}{}\\\hline

  % Spalte 13
  badische-zeitung.de & & & \\
                   & \multicolumn{3}{c}{} \\\hline

  % Spalte 14
  stuttgarter-zeitung.de & & ja & Ordnen nach älteste, neueste K.; Empfehlen\\
                      & \multicolumn{3}{c}{}\\\hline

  % Spalte 15
  merkur.de &  & & \\
         & \multicolumn{3}{c}{Oberfläche von Disqus: siehe die Welt.de}\\\hline


  % Spalte 16
  hna.de & & & \\
      & \multicolumn{3}{c}{Oberfläche von Disqus: siehe die Welt.de}\\\hline

  % Spalte 17
  mopo.de & & & \\
         & \multicolumn{3}{c}{Oberfläche von Disqus: siehe die Welt.de}\\\hline

  % Spalte 18
  mainpost.de & & ja & Ordnen nach älteste, neueste, best bewertete K.\\
           & \multicolumn{3}{c}{}\\\hline

  % Spalte 19
  tagesspiegel.de & & ja; zur Auswahl: Antworten anzeigen &
    Ordnen nach neueste, älteste K., chronologisch\\
    & \multicolumn{3}{c}{}\\\hline

  % Spalte 20
    swp.de & positiv & ja & \\
        & \multicolumn{3}{c}{}\\\hline \\

  %Spalte 21
    augsburger-allgemeine.de &
      & ja; Anhänge möglich, Vorschau vorhanden, Speichern möglich & \\
      & \multicolumn{3}{c}{}\\\hline

\end{longtable}
\end{landscape}

Beobachtungen: Die meisten Oberflächen geben die Möglichkeit auf einen Kommentar
zu antworten. Weniger als die Hälfte aller Zeitungen geben die Möglichkeit eine
Bewertung des Kommentars abzugeben. Wenn eine Bewertung gemacht werden kann,
dann eher nur als positive Bewertung. Bei sonstigen Funktionen wird häufig das
Ordnen nach bestimmten Kriterien angeboten. 

%
%Funktionen im Kommentar	\\
%``Bewerten''\\ ``Antworten''\\ Sonstiges\\
%
%% Spalte 1
%\\
%  positiv\\ \\ Ordnen nach beliebteste, älteste, neueste K. &
%% Spalte 2
%\\
%  \\ ja: auf was man antwortet wird in Zitate gesetzt\\ \\ &
%% Spalte 3
%\\
%  positiv\\ \\ dem Kommentator folgen\\ &
%% Spalte 4
%\\
%  positiv und negativ\\ ja\\ \\ &
%% Spalte 5
%Oberfläche von Disqus\\
%  positiv (Klicks werden gezählt), nach Anmeldung oder als ``Gast'' möglich; negativ (nach Anmeldung)\\ ja\\ ``Teilen''-Button (= diese Diskussion auf Twitter/Facebook teilen, nach Anmeldung dort), ``Teilen''-Button für einzelnen K., ``Empfehlen''-Button (= diese Diskussion empfehlen); Ordnen nach beste, neueste, älteste Kommentare; ``Abonnieren'', um Mittleiungen dieser Diskussion per Email zu erhalten; Disqus auf der eigenen Seite hinzufügen; Datenschutz\\ &
%% Spalte 6
%\\
%  \\ ja, mit Anzahl Antworten\\ \\ &
%% Spalte 9
%\\
%  positiv\\ \\ Ordnen nach älteste; Button für ``mehr K.''\\ &
%% Spalte 10
%\\
%  \\ ja\\ \\ &
%% Spalte 11
%\\
%  \\ ja \\ Ordnen nach älteste, neueste, beste Bewertung; zur Auswahl: ``informiert bleiben'' (bei jedem neuen Beitrag der Diskussion erhält man Benachrichtigung)\\ &
%% Spalte 12
%\\
%  \\ ja\\ Reaktionen/Antworten auf diesen K. anzeigen; Ordnen nach neuesten, empfohlenen, allen K.; Empfehlungen aussprechen; Empfehlungen der Redaktion (bedeutet aber nicht, dass die Redaktion der Meinung des Lesers zustimmt)\\ &
%% Spalte 13
%\\
%  \\ \\ \\ &
%% Spalte 14
%\\
%  \\ ja\\ Ordnen nach älteste, neueste K.; Empfehlen\\ &
%% Spalte 15
%Oberfläche von Disqus: siehe die Welt.de\\
%  \\ \\ \\ &
%% Spalte 16
%Oberfläche von Disqus: siehe die Welt.de\\
%  \\ \\ \\ &
%% Spalte 17
%Oberfläche von Disqus: siehe die Welt.de\\
%  \\ \\ \\ &
%% Spalte 18
%\\
%  \\ ja\\ Ordnen nach älteste, neueste, best bewertete K.\\ &
%% Spalte 19
%\\
%  \\ ja; zur Auswahl: Antworten anzeigen\\ Ordnen nach neueste, älteste K., chronologisch\\ &
%% Spalte 20
%\\
%  positiv\\ ja\\ \\ &
%% Spalte 21
%\\
%  \\ ja; Anhänge möglich, Vorschau vorhanden, Speichern möglich\\ \\ &
%
%%%%

% vim: set ai si et tw=80 sts=2 ts=2 sw=2:
