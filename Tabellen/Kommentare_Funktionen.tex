\section{Kategorie: \glqq Funktionen beim Kommentar\grqq}

mit den Unterkategorien:
\begin{itemize}
\item\glqq{\bfseries Bewerten}\grqq:\\
Gibt es die Möglichkeit für einen bereits geschriebenen und veröffentlichten Kommentar eine Bewertung abzugeben, ob man ihn gut (oder schlecht) findet?

\item\glqq{\bfseries Antworten}\grqq: \\
Gibt es die Möglichkeit auf einen bereits geschriebenen und veröffentlichten Kommentar zu antworten?


\item\glqq{\bfseries Sonstiges}\grqq:\\
Was gibt es für weitere Funktionen beim Kommentar selbst? Was kann sich der Nutzer anzeigen lassen? 


\end{itemize}

Beobachtungen bei den Nachrichtenportalen:\\
Weniger als die Hälfte aller Zeitungen geben die Möglichkeit eine
Bewertung des Kommentars abzugeben. Wenn eine Bewertung gemacht werden kann,
dann eher nur als positive Bewertung.\\
Die meisten Oberflächen geben die Möglichkeit auf einen Kommentar
zu antworten. \\
Bei sonstigen Funktionen wird häufig das
Ordnen nach bestimmten Kriterien angeboten. 


\begin{landscape} \footnotesize
\begin{longtable}{lccp{100mm}}

  \caption{Funktionen beim Kommentar} \\ \\
  \toprule
  \bfseries Portal & \bfseries Bewerten & \bfseries Antworten & \bfseries Sonstiges \\
  \midrule[\heavyrulewidth]
  \endfirsthead

  \toprule
  \bfseries Portal & \bfseries Bewerten & \bfseries Antworten & \bfseries Sonstiges \\
  \midrule[\heavyrulewidth]
  \endhead

  \multicolumn{4}{r}{\emph{Fortsetzung auf der nächsten Seite}}
  \endfoot

  \bottomrule
  \endlastfoot

% Zeile 1
bild.de
& positiv
&
& Ordnen nach beliebteste, älteste, neueste K.
\\\midrule

% Zeile 2
spiegel.de
&
& ja\footnote{auf was man antwortet wird in Zitate gesetzt}
&
\\\midrule

% Zeile 3
faz.net
& positiv
&
& dem Kommentator folgen
\\\midrule

% Zeile 4
focus.de
& positiv und negativ
& ja
&
\\\midrule

% Spalte 5
diewelt.de & \multicolumn{3}{l}{\hspace{2cm}\em Oberfläche von Disqus}
\\\cmidrule(lr){2-4}

& positiv\footnote{Klicks werden gezählt; eine Bewertung ist nach Anmeldung oder
  als ``Gast'' möglich.} und negativ\footnote{Bewertung ist nach Anmeldung
  möglich.}
& ja
& ``Teilen''-Button (= diese Diskussion auf Twitter/Facebook teilen, nach
  Anmeldung dort), ``Teilen''-Button für einzelnen K., ``Empfehlen''-Button (=
  diese Diskussion empfehlen); Ordnen nach beste, neueste, älteste Kommentare;
  ``Abonnieren'', um Mittleiungen dieser Diskussion per Email zu erhalten;
  Disqus auf der eigenen Seite hinzufügen; Datenschutz
\\\midrule

% Spalte 6
derwesten.de
&
& ja\footnote{Anzahl Antworten wird angegeben.}
&
\\\midrule

% Spalte 9
rp-online.de
& positiv
&
& Ordnen nach älteste; Button für ``mehr K.''
\\\midrule

% Spalte 10
handelsblatt.de
&
& ja
&
\\\midrule

% Spalte 11
suedkurier.de
&
& ja
& Ordnen nach älteste, neueste, beste Bewertung; zur Auswahl: ``informiert
  bleiben'' (bei jedem neuen Beitrag der Diskussion erhält man
  Benachrichtigung)
\\\midrule

% Spalte 12
zeit.de
&
& ja

& Reaktionen/Antworten auf diesen K. anzeigen; Ordnen nach neuesten,
  empfohlenen, allen K.; Empfehlungen aussprechen; Empfehlungen der Redaktion
  (bedeutet aber nicht, dass die Redaktion der Meinung des Lesers
  zustimmt)
\\\midrule

% Spalte 13
badische-zeitung.de
&
&
&
\\\midrule

% Spalte 14
stuttgarter-zeitung.de
&
& ja
& Ordnen nach älteste, neueste K.; Empfehlen
\\\midrule

% Spalte 15
merkur.de & \multicolumn{3}{l}{\hspace{2cm}\em Oberfläche von Disqus: siehe diewelt.de}
\\\midrule

% Spalte 16
hna.de & \multicolumn{3}{l}{\hspace{2cm}\em Oberfläche von Disqus: siehe diewelt.de}
\\\midrule

% Spalte 17
mopo.de & \multicolumn{3}{l}{\hspace{2cm}\em Oberfläche von Disqus: siehe diewelt.de}
\\\midrule

% Spalte 18
mainpost.de
&
& ja
& Ordnen nach älteste, neueste, best bewertete K.
\\\midrule

% Spalte 19
tagesspiegel.de
&
& ja\footnote{Man kann wählen, ob man die Antworten sehen möchte.}
& Ordnen nach neueste, älteste K., chronologisch
\\\midrule

% Spalte 20
swp.de
& positiv
& ja
&
\\\midrule

%Spalte 21
augsburger-allgemeine.de
&
& ja\footnote{Anhänge möglich, Vorschau vorhanden, Speichern möglich}
&

\end{longtable}
\end{landscape}

% vim: set ai si et tw=80 sts=2 ts=2 sw=2:
