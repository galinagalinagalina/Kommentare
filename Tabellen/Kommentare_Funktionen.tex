\section{Kategorie: \glqq Funktionen beim Kommentar\grqq}

mit den Unterkategorien:
\begin{itemize}
\item\glqq{\bfseries Bewerten}\grqq:\\
Gibt es die Möglichkeit für einen bereits geschriebenen und veröffentlichten Kommentar eine Bewertung abzugeben, ob man ihn gut (oder schlecht) findet?

\item\glqq{\bfseries Antworten}\grqq: \\
Gibt es die Möglichkeit auf einen bereits geschriebenen und veröffentlichten Kommentar zu antworten?


\item\glqq{\bfseries Sonstiges}\grqq:\\
Was gibt es für weitere Funktionen beim Kommentar selbst? Was kann sich der Nutzer anzeigen lassen? 


\end{itemize}

Beobachtungen bei den Nachrichtenportalen:\\
Die Funktion \glqq Bewerten\grqq \- soll helfen,
die besten Kommentare zu filtern und unterstützt die Selbstregulierung (siehe 5.1). 
Wenn eine Bewertung gemacht werden kann,
dann als positive Bewertung, außer wenn  {\slshape Disqus} verwendet wird. 
Dort können die Kommentare auch negativ bewertet werden. Ebenso bei focus.de.
Drei Viertel der Online-Zeitungen benutzt die Funktion \glqq Bewerten\grqq. \\
Die Funktion \glqq Antworten\grqq\- bieten bis auf 
rp-online.de und badische-zeitung.de alle Online-Zeitungen an. 
Sie dient hauptsächlich der Übersichtlichkeit für
das Kommentarmanagement und ist überaus benutzerfreundlich. Man kann direkt auf einen
bereits geschriebenen Kommentar antworten und sieht die Diskussion, die daraus entsteht.
Wird die Funktion nicht angeboten, muss man sich mühsam durch die Kommentare lesen 
und schreiben, auf wen oder was man sich bezieht. 
Man kommt jedoch dabei vom  Ursprungsgedanken der Kommentarfunktion, nämlich spontan auf
einen Artikel zu antworten, weg. Dafür werden mehr Diskussionen angeregt.
\\
Bei \glqq sonstigen Funktionen\grqq\- wird häufig das
Ordnen nach bestimmten Kriterien angeboten. Auch hier will man dem Nutzer mit verschiedenen
Optionen entgegenkommen.


\begin{landscape} \footnotesize
\begin{longtable}{lccp{100mm}}

  \caption{Kategorie \glqq Funktionen beim Kommentar\grqq} \\ \\
  \toprule
  \bfseries Portal & \bfseries Bewerten & \bfseries Antworten & \bfseries Sonstiges \\
  \midrule[\heavyrulewidth]
  \endfirsthead

  \toprule
  \bfseries Portal & \bfseries Bewerten & \bfseries Antworten & \bfseries Sonstiges \\
  \midrule[\heavyrulewidth]
  \endhead

  \multicolumn{4}{r}{\emph{Fortsetzung auf der nächsten Seite}}
  \endfoot

  \bottomrule
  \endlastfoot

% Zeile 1
bild.de
& positiv
& ja
& Ordnen (beliebteste, älteste, neueste K.)
\\\midrule

% Zeile 2
spiegel.de
&
& ja\footnote{Auf was man antwortet wird in Zitate gesetzt}
&
\\\midrule

% Zeile 3
faz.net
& positiv
&ja
& dem Kommentator folgen
\\\midrule

% Zeile 4
focus.de
& positiv und negativ\footnote{Positive und negative Bewertungen sind nur nach Anmeldung möglich}
& ja
&
\\\midrule

% Spalte 5
welt.de & \multicolumn{3}{l}{\hspace{2cm}\em Oberfläche von Disqus}
\\\cmidrule(lr){2-4}

& positiv und negativ\footnote{Eine positive Bewertung ist nach Anmeldung oder
  als \glqq Gast\grqq\- möglich, eine negative Bewertung ist nur nach Anmeldung
  möglich}
& ja
& Ordnen (beste, älteste, neueste  K.); die Diskussion auf Twitter/Facebook teilen; die D. empfehlen; 
  Mitteilungen dieser D. per Email erhalten
\\\midrule

% Spalte 6
derwesten.de
&
& ja\footnote{Anzahl der Antworten wird angegeben}
&
\\\midrule

% Spalte 9
rp-online.de
& positiv
& nein
& Ordnen (älteste K.); Button für \glqq mehr Kommentare\grqq
\\\midrule

% Spalte 10
handelsblatt.com
&
& ja
&
\\\midrule

% Spalte 11
suedkurier.de
& positiv\footnote{Bewerten nach Anmeldung möglich\label{foot:Anmeldung}}
& ja
& Ordnen (älteste, neueste, beste Bewertung); \glqq informiert
  bleiben\grqq: bei jedem neuen Beitrag der Diskussion erhält man eine
  Benachrichtigung
\\\midrule

% Spalte 12
zeit.de
& empfehlen
& ja

& Ordnen (neueste,  empfohlene, alle K.); 
	Reaktionen/Antworten auf diesen K. anzeigen; K. empfehlen; es gibt auch Empfehlungen der Redaktion
	(muss nicht bedeuten, dass die Redaktion der Meinung des Lesers zustimmt)
\\\midrule

% Spalte 13
badische-zeitung.de
&
&
&
\\\midrule

% Spalte 14
stuttgarter-zeitung.de
& 
& ja
& Ordnen (älteste, neueste K.)
\\\midrule

% Spalte 15
merkur.de & \multicolumn{3}{l}{\hspace{2cm}\em Oberfläche von Disqus: siehe welt.de}
\\\midrule

% Spalte 16
hna.de & \multicolumn{3}{l}{\hspace{2cm}\em Oberfläche von Disqus: siehe welt.de}
\\\midrule

% Spalte 17
mopo.de & \multicolumn{3}{l}{\hspace{2cm}\em Oberfläche von Disqus: siehe welt.de}
\\\midrule

% Spalte 18
mainpost.de
& positiv\footref{foot:Anmeldung}
& ja
& Ordnen (älteste, neueste, am besten bewertete K.)
\\\midrule

% Spalte 19
tagesspiegel.de
&
& ja\footnote{Man kann wählen, ob man die Antworten sehen möchte}
& Ordnen (älteste, neueste K., chronologisch)
\\\midrule

% Spalte 20
swp.de
& positiv\footref{foot:Anmeldung}
& ja
&
\\\midrule

%Spalte 21
augsburger-allgemeine.de
&
& ja\footnote{Beim Antworten sind Anhänge möglich, es gibt eine Vorschau der Antwort und auch Speichern der Antwort ist möglich}
&

\end{longtable}
\end{landscape}

% vim: set ai si et tw=80 sts=2 ts=2 sw=2:
