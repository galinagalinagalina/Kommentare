\chapter{Ausblick}

FF1: Wie werden die Nutzerkommentare auf deutschen Nachrichtenportalen
gehandhabt?\\
FF2: An welche Vorgaben müssen sich die Nutzer halten?

Ziel und Beantwortung der Forschungsfragen ist, einen Überblick zu verschaffen und darzustellen, wie das Kommentarmanagement bei deutschen Online-Zeitungen aussieht. Kommentarmanagement heißt, welche Bedingungen die Zeitungen verlangen, um überhaupt schreiben zu können, was es für Regeln gibt und welche Funktionen zusätzlich vorhanden sind. Dieser Überblick ist anhand der Tabellen dargestellt. 
Es ist infolgedessen auch eine Beschreibung von Kommentaren im Allgemeinen entstanden.



Beobachtungen, in welche Richtung das Kommentarmanagement gerade geht:

Kommentarfunktion wird einfach von einem externen Anbieter (hier: Disqus) übernommen. Die Kommentarregeln können natürlich selbst bestimmt werden. \\
Vorteil: der Nutzer kennt sich sofort auf der Oberfläche aus; er benötigt nur den Zugang vom Anbieter\\
Nachteil: Individualität geht verloren


Aufbau von Communities, d.h. der Nutzer kann u.a. ein Profil anlegen, persönliche Rankings erstellen oder sich in Rankings wiederfinden.
Vorteil für Kommentarmanagement: Selbstregulierung durch aktive und vertrauenswürdige Nutzer; man kann eher konstruktive Beiträge von den Mitgliedern erwarten. \\
Es entstehen keine Nachteile, da man das Profil auch ignorieren kann.\\
Die Online-Ausgabe der Augsburger Allgemeinen sticht mit ihrer Community und den Nutzerbeiträgen dort besonders hervor. Neben einem großen Profil lassen gibt es ein Textprogramm zur Formatierung. Außerdem kann man Beiträge im Forum nach Themen und Aktualität suchen. Es gibt Aufstellungen über Themen, Beiträge, Antworten, Aufrufe. Dieses äußerst umfangreiche Forum ist interessant, schießt aber unter Umständen über das Ziel hinaus. 
 

Anmelden über andere soziale Netzwerke, allen voran Facebook und Twitter. Das ist besonders nutzerfreundlich, da man den Zugang nehmen kann, den man unter Umständen eh schon hat. 




Weitere neue Ideen und Modelle für das Kommentarmanagement:
Bei dem Online-Magazin ``Tablet'' muss man fürs Kommentar-Schreiben bezahlen (http://www.sueddeutsche.de/\-me\-dien/onlinemagazin-mit-kostenpflichter-kommentarfunktion-trolle-zur-kasse-bitte-1.2351918).
Damit soll versucht werden, den Missbrauch zu umgehen. Ob die Nutzer bereits sind, Geld fürs Kommentieren zu bezahlen, wird sich zeigen. Genauso, ob sich Störenfriede von Geld abhalten lassen.


Auslagern der Kommentarfunktion, d.h. dass man nicht mehr auf der Seite der Online-Zeitung kommentieren kann. Bei süddeutsche.de wird man auf ``Rivva'' (Rivva filtert das Social Web nach meist empfohlenen Artikeln und debattierten Themen) weitergeleitet, wo man die Diskussion verfolgen und über die entsprechenden sozialen Netzwerke auch selber kommentieren kann. Süddeutsche. de versucht auf diesem Weg die Schwächen des Kommentarmanagements zu umgehen. 

