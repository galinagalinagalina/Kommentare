\chapter{Ausblick}\label{Ausblick}



Ziel und Beantwortung der Forschungsfragen\footnote{FF1: Wie werden die Nutzerkommentare auf deutschen Nachrichtenportalen
gehandhabt?\\
FF2: An welche Vorgaben müssen sich die Nutzer halten?} ist es, einen Überblick zu schaffen und darzustellen, wie das Kommentarmanagement bei deutschen Online-Zeitungen aussieht. Infolgedessen ist auch eine Beschreibung von Kommentaren im Allgemeinen entstanden.

Kommentarmanagement heißt, welche Bedingungen die Zeitungen verlangen, um überhaupt schreiben zu können, was es für Regeln gibt und welche Funktionen zusätzlich vorhanden sind. Im Sinne einer Dokumentenanalyse wurde Kategorien erarbeitet, die das Kommentarmanagement beschreiben. Diese Kategorien sind in Tabellen zusammengefasst und dargestellt. Bei den jeweiligen Tabellen stehen die entsprechenden Beobachtungen zu den Kategorien.
%Dieser Überblick mit Kategorien ist anhand der Tabellen dargestellt. 
Es gibt noch weitere Beobachtungen in welche Richtung das Kommentarmanagement gerade geht.

\begin{itemize}
\item Die Technik der Kommentarfunktion wird von einem externen Unternehmen zur Verfügung gestellt. Diese externe 
Kommentarfunktion nennt sich \glqq{\bfseries Disqus}\grqq\- und wird von mehreren Online-Zeitungen genutzt. Die Kommentarregeln können selbst bestimmt werden. Für den Nutzer liegt der Vorteil darin, dass er sich sofort auf der Oberfläche auskennt. Außerdem kann er sich über Disqus anmelden und muss sich nicht mehr auf dem Portal registrieren. Dabei geht aber ein Stück Individualität einer Zeitung verloren. Andererseits kann man sich fragen, ob eine individuelle Kommentarfunktion überhaupt nötig ist. Schließlich wollen die Nutzer nur kurz ihre spontanen Gedanken loswerden. Das führt zu einer anderen Beobachtung, die wiederum mit Individualisierung zu tun hat. 
%Vorteil: der Nutzer kennt sich sofort auf der Oberfläche aus; er benötigt nur den Zugang vom Anbieter\\
%Nachteil: Individualität geht verloren
\item Kommentare lassen sich {\bfseries formatieren}. Gerade die Online-Ausgabe der Augsburger Allgemeinen bietet in der Kommentarfunktion ein volles Textprogramm zur Formatierung an.  Auch hier ist die Frage, ob das beim Kommentieren nötig ist. Oder hofft die Zeitungen damit auf bessere Beiträge? Wenn ja, würden auch ein paar wenige Formatierungsfunktionen ausreichen? 

Disqus-Kommentierungsfunktion

Die Disqus-Kommentierungsfunktion wird von der Big Head Labs, Inc., San Francisco/USA (im Folgenden als "disqus.com" bezeichnet), als Dienstleistung zur Verfügung gestellt. Disqus ist ein interaktives Kommentarsystem, das es dem Nutzer ermöglicht, mit nur einer Anmeldung auf allen Internetangeboten, die Disqus als Kommentarsystem verwenden, zu kommentieren. Außerdem können sich die Nutzer über bestehende Accounts bei Facebook (über Facebook Connect), Twitter, Yahoo und OpenID anmelden. Nähere Informationen zu Disqus und seinen Funktionen finden Sie unter www.disqus.com.



\item Eine andere Tendenz ist der Aufbau von {\bfseries Communities}, d.h. der Nutzer kann u.a. ein Profil anlegen, persönliche Rankings erstellen oder sich in Rankings wiederfinden. Auch hier ist das Ziel des Kommentarmanagements, dass gute Kommentare verfasst werden. Außerdem findet eine Selbstregulierung durch aktive und vertrauenswürdige Nutzer statt. Man kann eher konstruktive Beiträge von den Mitgliedern erwarten. Es entstehen keine Nachteile, da man das Profil auch ignorieren kann.
Die Online-Ausgabe der Augsburger Allgemeinen sticht mit ihrer Community und den Nutzerbeiträgen auch hier besonders hervor. Neben einem großen Profil gibt es ein Textprogramm zur Formatierung (siehe oben). Außerdem kann man Beiträge im Forum nach Themen und Aktualität suchen. Es gibt Aufstellungen über Themen, Beiträge, Antworten, Aufrufe. Dieses äußerst umfangreiche Forum ist interessant, schießt aber unter Umständen über das Ziel hinaus. 
Schließlich stellt sich noch die Frage, ob das Vorhandensein einer Community die Kommentare beeinflusst und ob es dadurch zu einem besseren Umgangston kommt?

\item Was immer stärker von den Nachrichtenportalen genutzt wird ist das
{\bfseries Anmelden über andere} soziale Netzwerke, allen voran Facebook und Twitter. Das ist besonders nutzerfreundlich, da man den Zugang nehmen kann, den man unter Umständen eh schon hat. Die Zeitungen schlagen hier zwei Fliegen mit einer Klappe. Sie machen es dem Nutzer leicht, da die Registrierung wegfällt und stellen eine Verbindung in die genannten Netzwerke her. 

\item Inwieweit die Zahl der Kommentare einer Zeitung insgesamt Einfluss auf die Art des Kommentarmanagements hat und umgekehrt, wurde nicht
untersucht. Die regionalen Zeitungen aus dem Süden (z.B. suedkurier.de und swp.de) haben nämlich extrem wenig Kommentare auf ihren Seiten zu verzeichnen. 
Daraus ergeben sich neue Fragen am Rande. Sind die Nutzer einfach nur zu träge zum Kommentieren oder
ist die Kommentarfunktion nicht attraktiv genug? Ist es von den Zeitungen gewollt, dass wenig kommentiert wird? 

\end{itemize}




Es gibt immer wieder neue Ideen und Modelle für das Kommentarmanagement. Hier sind zwei Beispiele.
\begin{itemize}
\item Bei dem Online-Magazin \glqq Tablet\grqq\- muss man fürs Kommentar-Schreiben {\bfseries bezahlen} (http://www.sueddeutsche.de/\-me\-dien/onlinemagazin-mit-kostenpflichter-kommentarfunktion-trolle-zur-kasse-bitte-1.2351918).
Damit soll versucht werden, den Missbrauch zu umgehen. Ob die Nutzer bereits sind, Geld fürs Kommentieren zu bezahlen, wird sich zeigen. Genauso, ob sich Störenfriede von Geld abhalten lassen.


\item Oder die Kommentarfunktion wird komplett {\bfseries ausgelagert}, d.h. dass man nicht mehr auf der Seite der Online-Zeitung kommentieren kann. Bei sueddeutsche.de wird man auf \glqq Rivva\grqq\- (Rivva filtert das Social Web nach am meisten empfohlenen Artikeln und debattierten Themen) weitergeleitet, wo man die Diskussion verfolgen und über die entsprechenden sozialen Netzwerke auch selber kommentieren kann. Sueddeutsche.de versucht auf diesem Weg die Schwächen des Kommentarmanagements zu umgehen. 

\end{itemize}