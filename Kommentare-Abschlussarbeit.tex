% !TEX TS-program = pdflatex
% !TEX encoding = UTF-8 Unicode

% This is a simple template for a LaTeX document using the "article" class.
% See "book", "report", "letter" for other types of document.

% \documentclass[12pt]{memoir}
\documentclass[12pt,parskip=full,ngerman,draft]{scrreprt} % use larger type; default would be 10pt

\usepackage{showkeys}
\usepackage{fixme}
\fxusetheme{color}

\usepackage[onehalfspacing]{setspace}
\AfterTOCHead{\singlespacing}

\KOMAoptions{DIV=last}

\usepackage[utf8]{inputenc} % set input encoding (not needed with XeLaTeX)
\usepackage{babel}

% For biblatex
\usepackage{csquotes}% Recommended
\usepackage[backend=biber,sorting=nyt,style=apa]{biblatex}
\DeclareLanguageMapping{ngerman}{ngerman-apa}

\addbibresource{Literatur.bib}


%%% Examples of Article customizations
% These packages are optional, depending whether you want the features they provide.
% See the LaTeX Companion or other references for full information.

%%% PAGE DIMENSIONS
\usepackage{geometry} % to change the page dimensions
\geometry{a4paper} % or letterpaper (US) or a5paper or....
% \geometry{margin=2in} % for example, change the margins to 2 inches all round
% \geometry{landscape} % set up the page for landscape
%   read geometry.pdf for detailed page layout information

\usepackage{graphicx} % support the \includegraphics command and options

% \usepackage[parfill]{parskip} % Activate to begin paragraphs with an empty line rather than an indent

%%% PACKAGES
\usepackage{booktabs} % for much better looking tables
\usepackage{array} % for better arrays (eg matrices) in maths
\usepackage{paralist} % very flexible & customisable lists (eg. enumerate/itemize, etc.)
\usepackage{verbatim} % adds environment for commenting out blocks of text & for better verbatim
\usepackage{subfig} % make it possible to include more than one captioned figure/table in a single float
% These hpackages are all incorporated in the memoir class to one degree or another...
\usepackage{enumitem}
\setlist{noitemsep}


\usepackage{pdflscape}
\usepackage{longtable}

%%% HEADERS & FOOTERS
\usepackage{fancyhdr} % This should be set AFTER setting up the page geometry
\pagestyle{fancy} % options: empty , plain , fancy
\renewcommand{\headrulewidth}{0pt} % customise the layout...
\lhead{}\chead{}\rhead{}
\lfoot{}\cfoot{\thepage}\rfoot{}

%%% SECTION TITLE APPEARANCE
\usepackage{sectsty}
\allsectionsfont{\sffamily\mdseries\upshape} % (See the fntguide.pdf for font help)
% (This matches ConTeXt defaults)

%%% ToC (table of contents) APPEARANCE
%\usepackage[nottoc,notlof,notlot]{tocbibind} % Put the bibliography in the ToC
%\usepackage[titles,subfigure]{tocloft} % Alter the style of the Table of Contents

% \renewcommand{\cftsecfont}{\rmfamily\mdseries\upshape}
% \renewcommand{\cftsecpagefont}{\rmfamily\mdseries\upshape} % No bold!

%%% END Article customizations
%\usepackage{natbib}

%%% The "real" document content comes below...

% My font adaptions:
% Euler for math | Palatino for rm | Helvetica for ss | Courier for tt
%\renewcommand{\rmdefault}{ppl} % rm
%\linespread{1.05}        % Palatino needs more leading
%\usepackage[scaled]{helvet} % ss
%\usepackage{courier} % tt
%\usepackage{euler} % math
%\usepackage{eulervm} % a better implementation of the euler package (not in gwTeX)
%\normalfont

\usepackage[osf,proportional]{libertine}
\usepackage[T1]{fontenc}
\usepackage[final,kerning]{microtype}

% Maybe colorize links?
\usepackage[%
  pdftitle={Kommentarmanagement auf deutschen Nachrichtenseiten},
  pdfauthor={Nadine Ambrosch},
  bookmarksnumbered,
  draft=false
]{hyperref}
\hypersetup{colorlinks}

%\setkomafont{date}{\large}
%\setkomafont{author}{\large}
\title{Kommentarmanagement auf deutschen Nachrichtenseiten}
\subtitle{Abschlussarbeit im Aufbaustudiengang Journalistik}
\author{Nadine Ambrosch}

%\date{} % Activate to display a given date or no date (if empty),
         % otherwise the current date is printed 

\hyphenation{Zei-chen-be-gren-zung}

\begin{document}

\begin{spacing}{1}
  \maketitle
\end{spacing}

\tableofcontents
\listoftables


\chapter{Einleitung}

\section{Relevanz des Themas}

\glqq Here comes everybody\grqq, so lautet der Satz von Clay Shirky, um die neue
Medienwelt zu beschreiben: Jeder kommt also dran! Jeder kann sich heute schnell
und direkt in den Kommunikationsprozess einklinken.  Und niemand muss mehr
hoffnungsvoll warten, ob es der Leserbrief über die Hürde der Redaktionen
schafft und veröffentlicht wird.  Die Kommentarfunktion ist eine Möglichkeit des
sich \glqq Einschaltens\grqq. Man kann dabei ohne größeren Aufwand zu einem
bestimmten Thema etwas sagen und loswerden, was einem beim Lesen eines Artikels
durch den Kopf schießt.

Es ist also geschafft, was sich sowohl Leser als auch Medienschaffende gewünscht
haben. \glqq Journalists’ desire to maintain communication with readers has been
historically, and remains today one of the basic tenets of journalism\grqq{}
\autocite[S.~67]{santana:2011}.%(Santana, 2011, S. 67).

Der Leser beteiligt sich an der journalistischen Arbeit, er nimmt aktiv am
demokratischen Prozess teil.  Aber dadurch, dass man sich jederzeit einschalten
kann, kommt es zu einer Flut von Äußerungen, deren Qualität nicht garantiert
ist. Die Redakteure sind hin und her gerissen zwischen Nutzen und Nutzlosigkeit.
Sie sehen auch, dass Kommentare auf ihren Plattformen schädlich für das Ansehen
der Zeitung sein können.  Ein einziger \glqq schlechter\grqq\- Kommentar macht
50 gute wertlos, berichtet ein Redakteur in
\textcite[S.~131]{participatory}.

Eine Regulierung ist unumgänglich. Darüber tauchen viele Fragen und Diskussionen
unter den Journalisten zur Organisation von Kommentaren auf.

Das größte Problem stellt der Missbrauch seitens der Kommentierenden dar, die
unter dem Schutz der Anonymität beleidigen und beschimpfen. Das geht
mittlerweile soweit, dass sogar der deutsche Presserat Handlungsbedarf sieht, da
sich Beschwerden mittlerweile zu 60\% auf online erschienene Texte beziehen:

\begin{quote}
„Aber die Zunahme von Beschwerden, unter anderem zu Leserkommentaren und
Online-Archiven, zeigt, dass wir die Publizistischen Grundsätze an einigen
Stellen ergänzen sollten, um den digitalen Ver\-öffentlichungs\-for\-men besser
gerecht zu werden.“ Jahrespressekonferenz, 19.02.2014, Presserat.
\end{quote}

Süddeutsche.de hat sich Mitte des Jahres 2014 für Einschnitte entschieden
\autocite{mark}, um dann die Kommentare ganz weg zu lassen \autocite{fromm},
genauso wie \url{reuters.de} \autocite{standard}.


Wie sieht es bei den anderen deutschen Online-Zeitungen aus? Wie gehen die
Redaktionen mit den Kommentaren um? Lassen sie den Nutzern komplett freie Hand?
Ist eine Anmeldung erforderlich (mit oder ohne Klarnamenpflicht) oder bleiben
die Kommentare anonym? Werden die Kommentare moderiert (vorher oder danach)?
Gibt es Verhaltensrichtlinien für die Nutzer? Wo ist Kommentieren möglich? Gibt
es Verweise auf soziale Medien?


\section{Forschungsziel- und fragen}

Mit dieser Arbeit soll der Frage nachgegangen werden, wie das
Kommentarmanagement der größten Nachrichtenportale jeweils aussieht. Es wird
eine Bestandsaufnahme gemacht, wie die Redaktionen Anfang des Jahres 2015 die
Kommentare handhaben und welche Richtlinien dort vorgegeben werden. In diesem
Zuge wird ein Überblick des aktuellen Kommentarmanagements entstehen.

\begin{itemize} \em
  \item FF1: Wie werden die Nutzerkommentare auf deutschen Nachrichtenportalen
    gehandhabt?
  \item FF2: An welche Vorgaben müssen sich die Nutzer halten?
\end{itemize}


\section{Methodische Umsetzung und Kategorien}

Da es sich beim Analysematerial um schriftliche Dokumente handelt, wird zur
methodischen Umsetzung dieses Überblicks eine Dokumentenanalyse herangezogen.
Das heißt auch, dass eine qualitative Inhaltsanalyse gemacht wird, in Abgrenzung
zur quantitativen Inhaltsanalyse, die eigentlich zur Analyse großer Mengen an
Medieninhalten entwickelt wurde. Nur von qualitativer Inhaltsanalyse zu sprechen
lehnt \textcite{kunzler} ab, da es zu kurz greift bzw.~\textcite{mayring} diesen
„Begriff (bereits) für drei von ihm entwickelte Auswertungsverfahren
[Zusammenfassung, Strukturierung bzw.~Kategorisierung, Explikation] geprägt hat“.

Hier liefern die jeweiligen Nachrichtenseiten das Material. Dieses Material wird
bestimmten Kategorien, welche auf Grundlage des theoretischen Vorwissens
angelegt wurden, zugeordnet. Die Kategorien müssen vorher genau definiert werden
(siehe \textcite{mayring}, deduktive Kategorienanwendung).

Es wird aber auch mit Kategorien gearbeitet, die nicht empirisch gehaltvoll
sind, die also zunächst \glqq leer\grqq{} sind (\glqq heuristische
Rahmenkonzepte\grqq{} oder \glqq sensitizing concepts\grqq{}).  Sie dienen als
\glqq Platzhalter\grqq{}, um sich erst später in empirisch gehaltvolle
Kategorien zu entwickeln. Sie dienen als \glqq Starthilfe", da die Dokumente
erst entdeckt und verstanden werden müssen. Somit wird das deduktive Verfahren
mit dem induktiven kombiniert.

Bei solchen heuristischen Rahmenkonzepten gibt es also eine Kombination von
theoretischem \glqq Fachwissen des Forschers\grqq{} \autocite[S.~65]{strauss}
und empirisch generiertem Wissen. Es ist eine Zusammensetzung von zwei Typen von
Kategorien \autocite[S.~64f]{strauss}, den soziologisch konstruierten und den
natürlichen Kodes. Die soziologisch konstruierten Kodes entstehen erst im Laufe
der Beschäftigung mit dem Material, die anderen werden vorausgesetzt.


\section{Auswahl der Nachrichtenportale}

Im Sinne der Forschungsfrage und der Dokumentenanalyse wird nicht zufällig auf
irgendwelche Portale zugegriffen. Nicht die statistische Repräsentativität ist
das Ziel der Analyse, sondern das kriteriengesteuerte Sammeln von Dokumenten
\autocite{kunzler}.\\
Es kommen die \glqq größten\grqq{} Online-Zeitungen in die Auswahl, d.h. die
Portale mit den meisten Besuchern. Auf diesen Portalen gibt es, meistens am Ende
von Artikeln, das Angebot, Kommentare zu verfassen. Die Rahmenbedingungen und
Funktionalitäten, um dort seine Meinung wieder zu geben, sind das Material für
die Dokumentenanalyse (Quellenbeschreibung).\\
Die Quellenbeschreibung ist Teil der Quellenkritik, die wiederum ein
Gütekriterium der Dokumentanalyse ist. Weitere Bestandteile der Quellenkritik
sind: Die Nachrichtenportale sind im Internet frei verfügbar und für jeden
zugänglich (Textsicherung). Die Nachrichtenportale werden in einem bestimmten
Zeitraum betrachtet (äußere Kritik).\\
Für den vergleichenden Überblick werden deutsche Online-Zeitungen mit den
meisten Besuchern ausgewählt. Diese Zeitungen stellen den Nutzern jeweils eine
Kommentarfunktion für ihre Beiträge zur Verfügung.  Diese Funktionen werden dann
analysiert.\\
Die IVW listet sämtliche Online-Angebote nach der Gesamt-Besucherzahl auf.
Laut den Messungen 1/2015 sind die 21 meistgelesenen Online-Zeitungen folgende:

\href{http://www.bild.de}{Bild.de},
\href{http://www.spiegel.de}{SPIEGEL ONLINE},
\href{http://www.faz.net}{FAZ.NET},
\href{http://www.focus.de}{FOCUS ONLINE},
\href{http://www.welt.de}{Die Welt}, 
\href{http://www.derwesten.de}{Der Westen},
\href{http://www.stern.de}{Stern},
\href{http://www.rp-online.de}{RP ONLINE},
\href{http://www.handelsblatt.com}{Handelsblatt.com},
\href{http://www.suedkurier.de}{SÜDKURIER Online},
\href{http://www.zeit.de}{ZEIT ONLINE}, 
\href{http://www.badische-Zeitung.de}{Badische Zeitung},
\href{http://www.stuttgarter-zeitung.de}{Stuttgarter Zeitung},
\href{http://www.merkur.de}{Merkur},
\href{http://www.hna.de}{HNA},
\href{http://www.mopo.de}{Hamburger Morgenpost},
\href{http://www.mainpost.de}{Mainpost},
\href{http://www.tagesspiegel.de}{Tagesspiegel},
\href{http://www.swp.de}{Südwestpresse},
\href{http://www.augsburger-allgemeine.de}{Augsburger Allgemeine},
\href{http://www.abendzeitung-muenchen.de}{Abendzeitung Online},
\href{http://www.fr-online.de}{Frankfurter Rundschau}

% vim: set ai si et tw=80 sts=2 ts=2 sw=2:

\chapter{Nutzerbeteiligung im Journalismus: Kommentare als Nutzerbeteiligung}
\label{kap:nutzerbeteiligung}

\section{Vom Leser zum Nutzer zur Nutzerbeteiligung}

Mit der Digitalisierung ändert sich der Kommunikationsprozess. Der Leser kommt
weg von seiner Rolle als reiner Zuhörer. Er muss nicht mehr warten, bis ihm
etwas angeboten wird. Der Leser wird zum Nutzer der Medienangebote, denen er
sich auch mitteilen kann. Er kann sie nicht nur nutzen, sondern auch benutzen.
Er kann sich richtig in den Kommunikationsprozess einschalten und zwar direkt
und unmittelbar.

\begin{quote}
\glqq A great many other people also contribute content, representing their own
interests, ideas, observations and opinions. That content comes in a steadily
expanding volume and variety of forms and formats – words, images and sounds,
alone or in combination.\grqq{} \autocite[S.~1]{participatory}
\end{quote}

Im Englischen werden u.a. die Ausdrücke „user generated content“, „citizen
journalism“ oder „participatory journalism“ \autocite[S.~2]{participatory} dafür
verwendet. Sie alle beinhalten, dass Medienschaffende und Nutzer miteinander
kommunizieren und die Nutzer ihren Teil zur Bildung von Nachrichten und
Gemeinschaft dazu tun.

Ausdrucksformen dieser Nutzerbeteiligung sind vor allem Beiträge in
(Diskussions-)Foren und sozialen Netzwerken, Blogs, Berichte, Beurteilungen,
Bewertungen, Kommentare, hochgeladene Fotos und Videos und noch viele mehr.
„Indeed, new participatory formats appear all the time.“ \autocite[S.~2 und
S.~17]{participatory}


\section{Kommentare als Nutzerbeteiligung}

In dieser Arbeit stehen die Nutzerkommentare im Mittelpunkt. Als Kommentare
bezeichnet man „views on a story or other online item, which users typically
submit by filling in a form on the bottom of the item.“
\autocite[S.~17]{participatory}.  Sie markieren eine neue Stufe in der
Nutzerbeteiligung und sie sind überaus beliebt, was auch immer wieder in Studien
bestätigt wird \autocite[S.~97, siehe 3.1]{reich}.

Weil die Funktion bei Onlinezeitungen stark genutzt wird, kommt es zu einer Flut
von Material. Mit diesem Material haben die Redaktionen zu kämpfen. Sie sehen
zwar die Vorteile, die die Kommentarfunktion bringt. In der täglichen Arbeit
führt es jedoch immer wieder zu schlechten Erfahrungen.

Als nächstes wird aufgezeigt, was die Kommentare besonders macht, warum sie so
populär sind, was die positiven Aspekte sind und wie sie den
Kommunikationsprozess bereichern. Danach werden die Probleme und die Eingriffe,
das Kommentarmanagement, erörtert.

\begin{quote}
User-generated posts attached to a published item, typically an article or blog
entry, on a media website. Most news organizations moderate or screen user
comments, either before or after publication;
\autocite[S.~204, Glossar]{participatory}
\end{quote}

% vim: set ai si et tw=80 sts=2 ts=2 sw=2:

\chapter{Positive Aspekte der Kommentarfunktion}

Was macht Kommentare wertvoll und was bringen sie eigentlich? Dieser Frage wird
im folgenden Abschnitt nachgegangen.


\section{Beliebtheit der Kommentare} \label{sec:beliebtheit}
Obwohl nur ein kleiner Teil der Leser Kommentare verfasst und ein ebenso kleiner
Teil diese auch liest \autocite[96ff]{reich}, schlägt diese Art der
Nutzerbeteiligung wie eine Bombe ein. Das merken die Redaktionen, die von nicht
mehr zu bewältigenden Kommentaren erreicht werden. Dies ist dem Umstand
geschuldet, dass es auf einmal (2005 entstehen die ersten Kommentare) die
Möglichkeit gibt, rauszulassen, was einem gerade durch den Kopf geht beim Lesen
eines Artikels.

Es ist aber nicht nur die Gelegenheit spontan etwas zum eben Gelesenen zu sagen,
was die Kommentare attraktiv macht. Menschen haben selten die Chance, ihren
Unmut kund zu tun oder Zustimmung auszusprechen, weil sie auf das nur Zuhören
beschränkt sind \autocite[S.~99]{reich}. Diese auferlegte Passivität wird
aufgehoben.\\
Für manche Leser werden die Kommentare sogar genauso wichtig, wie die Nachricht
selbst. Und manchen Leser gefällt es einfach, sich selbst veröffentlicht zu
sehen.


\section{Öffentlicher Diskurs}
Eine ganz wichtige Rolle nehmen die Kommentare als ein Platz oder Ort ein, wo
sich jeder zum Diskutieren treffen und wo man auch die Breite und
Verschiedenheit der Ansichten erfahren kann. Für den demokratischen Prozess ist
es wichtig, dass bestimmte Themen besprochen werden. Jede Stimme soll gehört
werden, auch diejenigen, die sonst leicht untergehen, wie Minderheiten oder
Leute, die sich nicht trauen, oder weil es an Unterstützung fehlt oder weil die
Angelegenheit einfach kontrovers ist \autocite[S.~12]{santana:2014}. Außerdem
gibt es erst mal keine Beschränkungen, wer mit sprechen darf und wer nicht.  Zu
einer gesunden Demokratie gehört das einfach dazu: \glqq By allowing citizens to
participate, journalists behave ethically, and hence they democratize journalism
and the web.\grqq\- \autocite[S.~125]{singer}

Dieser neu geschaffene öffentliche Ort soll auch dazu dienen, herauszufinden,
wie die Leute im Moment \glqq ticken\grqq\- (\glqq serving as a gauge of society’s pulse\grqq\-
 \autocite[S.~181]{loke}. Man kann herauslesen, wo die neuralgischen Punkte in
der Gesellschaft liegen und auf diesen Zug ausspringen, wenn nötig.

Über die Kommentare können Einspruch erhoben oder Bedenken geäußert werden.
Ebenso funktionieren Kommentare als Anregung für weitere Diskussionen oder
greifen korrigierend  in die Redaktionen ein. Auf diesem Weg erhalten die
Journalisten auch direktes Feedback: \glqq Comments can confirm that the website is
doing a good job [\ldots] they can help improve accuracy [\ldots]\grqq\-
\autocite[S.~105]{reich}

Ufern die Diskussionen jedoch ins Unendliche aus oder werden aggressiv und
beleidigend, dann wird aus diesem liberalen Aspekt ein Problem der Kommentare.
\glqq[\ldots] free expression and exposure to differing views can hold deliberative
potential only when participants were respectful toward each other\grqq\-
\autocite[S.~7]{santana:2011}. Das nächste Kapitel wird sich damit beschäftigen.





\section{Leserbindung}
\begin{quote}
\glqq Newspaper websites compete in a marketplace where a rival news source is simply
a click away, so gaining and retaining the attention of readers is more
important than ever. [\ldots] It’s not just getting the eyes on your site
[\ldots] It’s getting them to stay on your site.\grqq\- \autocite[S.~144]{singer}
\end{quote}

Schaffen es die Online-Redaktionen, dass die Leser dabei bleiben, dann erreichen
sie ihre Ziele. Und die Kommentare sind dabei ein ganz wesentlicher Faktor. Die
Teilnahme am Kommentieren kann sich zu einem Wir-Gefühl entwickeln. Das steigert
wiederum die Besuche auf der Website, sowie das Vertrauen und die
Glaubwürdigkeit der Seite \autocite[S.~215]{meyer-carey}. So bekommen die
Medienhäuser den gewünschten Verkehr im Netz und große Klickzahlen. Die Leser
entwickeln eine gewisse Loyalität gegenüber ihrer Zeitung. Das Medienprodukt
wird zur (begehrten) Marke und zieht positive wirtschaftliche Aspekte mit sich.\\
Alle positiven Aspekte können sich durch eine große Masse an Kommentaren jedoch
auch umkehren, z.B. werden die Leser der Online-Zeitung den Rücken zukehren,
wenn sie von den Äußerungen der anderen regelrecht erschlagen werden.

\section{Neue oder zusätzliche Inhalte}
Die Kommentierenden begeistern sich für ihr Thema, lassen sich als Experten
erkennen, liefern Hinweise und sprechen ganz einfach aus, was ihnen wichtig ist.
Dort kann der Journalist ansetzen und solche Beiträge als \glqq journalistisches
Werkzeug\grqq{} \autocite[S.~133]{robinson} benutzen.\\
Kommentare dienen als Informationsquelle oder als Inspiration für Themen.
Manche Journalisten sehen sie als die Möglichkeit, Kontext herzustellen (durch
Hyperlinks) oder um bestimmte Punkte zu klären. Andere Perspektiven tun sich
auf und der Journalist wird sensibilisiert, was die Leser hören wollen
\autocite[S.~12]{santana:2014}.

Dafür müssen die Schreiber jedoch alle Kommentare durchforsten: entweder nur von
ihren eigenen Texten oder von der gesamten Redaktion. Es liegt nahe, dass dies
sehr schnell ausarten kann. \glqq If a journalist were to read all 300 comments, he
wouldn’t be able to do anything else [\ldots]\grqq\- \autocite[S.~84]{domingo}.  So
kommt es, dass nur wenige Journalisten Kommentare überhaupt lesen
\footnote{\glqq Only 35 percent of national journalists and 36 percent of local
  journalists said they have a positive view of citizens posting news content on
  news organizations' websites.  [\ldots] Even fewer said they take the time to
  read the comments and submissions they receive.\grqq\- \autocite[S.
216]{meyer-carey}}
und dass sie eine schlechte Meinung darüber entwickeln.\footnote{The study finds
  that the more comments a news website receives, the more negative attitudes
  the journalists who work at those sites have toward the value and utility of
  comments. \autocite[S.~214]{meyer-carey}}

\section{Secondary gatekeeper} \label{sec:secondary-gate}
Den Nutzern geht es nicht darum, hauptsächlich neue Inhalte zu schaffen. Sie
bedienen sich nur an der Nachrichtenauswahl und verändern dann die Gewichtung.
Sie suchen sich aus, was sie für gut finden und teilen dies anderen mit. Das
Material wird ein zweites Mal veröffentlicht und verteilt
\autocite[S.~66]{singer:2014}. Damit übernehmen die Nutzer neue Rollen und zwar
die des \glqq active redistributor\grqq, des \glqq active recipient\grqq\- und in einer gewissen
Weise die des \glqq gatekeepers\grqq\- \autocite[S.~57]{singer:2014}. Dadurch, dass ihnen
aber die Autonomie fehlt, was wirklich zur Veröffentlichung frei gegeben wird,
werden sie immer hinter dem eigentlichen \emph{gatekeeper}, dem Journalisten,
stehen. Auch weil sie sich an Regeln halten müssen (wie diese Regeln konkret
aussehen, wird in den folgenden Abschnitten beschrieben). Diese Regeln
bekräftigen den eigentlichen \emph{gatekeeper}, indem sie ihm eine neues \glqq gate\grqq\-
liefern \autocite[S.~13]{santana:2014}.\\
Der Journalist besitzt zu Recht aufgrund seiner Professionalität die
Entscheidungsgewalt. Aber der Nutzer wird zumindest zu einem \glqq secondary
gatekeeper\grqq\- \autocite[S.~5]{santana:2014}, der Inhalte verändern kann, indem er
sie hervorhebt, markiert, mit anderen teilt, weiter verschickt, zugänglich macht
\autocite[S.~57]{singer:2014}.\\
Es gibt im Internet unglaublich viele Nachrichten und Informationen. Diese sind
da, aber irgendwo \glqq da draußen\grqq. Deswegen ist das \glqq
Sichtbarmachen\grqq\- von Nachrichten umso wichtiger geworden.

% vim: set ai si et tw=80 sts=2 ts=2 sw=2:

\chapter{Probleme mit Kommentaren}

Die Nutzerbeteiligung „Kommentare“ führt allerdings nicht nur zu einer
Bereicherung innerhalb der Medien. So hat süddeutsche.de die Kommentarfunktion
auf ihren Internetseiten entfernt bzw. ausgelagert. Dabei entsteht der Eindruck,
dass die Nachteile die Vorteile überwiegen. Die Entscheidung ist in der
Redaktion sicherlich nicht ohne heftige Diskussion gefallen. Am Ende wurden doch
die nachteiligen Erscheinungen der Kommentare für schwerwiegender empfunden. Was
jedoch wiederum für die Kommentare spricht, ist, dass sie nicht ganz gelöscht,
sondern eben nur auf andere Plattformen verschoben wurden. Ausgewählte
Kommentare schaffen es sogar auf die Nachrichtenseiten. Außerdem kann den
Nachteilen mit entsprechenden Maßnahmen entgegen gewirkt werden. Wie diese
Maßnahmen aussehen, wird dann im darauffolgenden Kapitel erklärt

Was die Probleme genau sind, soll nun betrachtet werden.

\section{Kommentarmanagement ist mit Zeitaufwand verbunden}

Es wurde bereits erwähnt, dass die Kommentare Überhand nehmen und nicht mehr
bewältigt werden können. Manchmal schafft es der Autor nicht, die Kommentare zu
seinem eigenen Artikel zu lesen, geschweige denn, die seiner Mitstreiter. Es
werden in manchen Redaktionen externe Mitarbeiter angestellt, die sich
ausschließlich mit den Kommentaren beschäftigen. Das belastet die Redaktionen
zusätzlich: ``[\ldots] the overall volume of user input demanding to be filtered
or factchecked was a burden.'' \autocite[S.~172]{quandt}


\section{Schlechte Qualität der Kommentare und „incivilty“}

Es ist nicht nur die reine Menge der Kommentare, sondern auch deren mindere
Qualität, die bei den Journalisten schlecht ankommt und Arbeit verursacht. Die
Kommentare müssen gesichtet und sortiert werden. ``Very few of them make
intelligent comments or have intelligent things to say. It's very deceiving
[\ldots]'' \autocite[S.~ 103]{reich}. Und dabei schreiben und lesen sowieso nur
sehr wenige Nutzer Kommentare, wie in Kapitel \ref{kap:nutzerbeteiligung}
bemerkt!

Wenn die Beiträge einfach nur schlecht sind, dann hinterlässt das sicherlich
keinen guten Eindruck und die Qualität der Zeitung leidet insgesamt. Aber
Kommentare arten teilweise aus und es kommt zu beleidigenden oder rassistischen
Ausdrücken und irrationalen Argumenten. Die Kommentierenden erniedrigen und
schuldigen an \autocite[S.~103]{reich}.

Diese Verstöße gegen einen angemessenen Diskussionston (\glqq discursive
civility\grqq{} versus \glqq discursive incivility\grqq\footnote{Hwang defines
  ``discursive civility'' as arguing the justice of one's own view while
  admitting and respecting the justice of others' views. Conversely,
  ``discursive incivility'' is defined as expressing disagreement that denies
and disrespects the justice of others' views \autocite{hwang}
\autocite[S.~6/7]{santana:2014}} werden durch die Anonymität des Internets
gefördert. Die Hemmschwelle etwas zu sagen, das unter die Gürtellinie zielt,
sinkt ganz einfach, wenn man sich nicht zu erkennen geben muss. Andererseits ist
gerade diese Anonymität der Schlüsselfaktor, der zu mehr Beteiligung am Diskurs
führt, siehe Kapitel \ref{kap:nutzerbeteiligung}.

Selten wird nicht nur gegen den guten Ton verstoßen, sondern auch gegen das
Gesetz. Hassrede zum Beispiel ist ein bekanntes Problem und in Deutschland
verboten.

Es gibt bestimmte Themen, die sehr kontrovers sind und die Gemüter erhitzen,
weil verhärtete entgegengesetzte Meinungen aufeinandertreffen. ``Certain news
stories do not receive even a single comment, but those that do [\ldots] could
provide a peek into `the community’s heartbeat'.'' \autocite[S.~181]{loke} Die
Zeitungen beobachten Themengebiete, bei denen es immer wieder zu „discursive
incivility” der Kommentierenden kommt. Dazu gehören vor allem sämtliche
Angelegenheiten über Religion\footnote{Ein aktuelles Beispiel ist der {\slshape
shitstorm}, mit dem sich Christiane Florin auseindersetzen musste, weil sie eine
Anzeige in einem christlichen Blatt ablehnte. Es ging „nur“ um eine Anzeige und
die Journalistin hat die Absage auch begründet. Trotzdem erreichten sie
hasserfüllte Leserbriefe, die sie schließlich öffentlich gemacht hat
\url{http://www.christundwelt.de/detail/artikel/sie-kotzen-mich-an}} und Rasse,
Einwanderung und soziale Themen, obwohl gerade in diesen Bereichen ein
öffentlicher Diskurs und verschiedene Meinungen wichtig sind! Sie sind
notwendig, um sämtliche Aspekte abzubilden. Sonst greift auch hier die
Schweigespirale.\footnote{„[\ldots] would-be commenters might feel remiss to
register their opinion if they perceive themselves to hold a minority opinion
[\ldots]“ \autocite[S.~12]{santana:2014}}

Verschiedene Redaktionen reagieren darauf, indem sie zu Themen, mit denen sie
schlechte Erfahrungen gemacht haben, keine Kommentarfunktion mehr zu lassen oder
sie schließen, wenn die Reaktionen zu hitzig werden
\autocite[S.~4]{santana:2014}. Dem gegenüber gibt es Beobachtungen, wie die von
\textcite{loke}\footnote{``Loke found that online discussions of culturally
sensitive issues, such as race, have allowed newsreaders the opportunity to
amplify socially regressive views where they would not have previously. She
argues that despite the hateful dialogue that some sensitive topics attract, the
forums still serve a useful function in allowing the public, as well as
journalists, to get a glimpse into the consciousness of the community, however
unappealing.'' (Santana, 2014, \autocite[S.~12]{santana:2014}}, dass auch bei
kontroversen Themen die Leser aus den Diskussionen etwas über die Gegner
mitnehmen und ihren Blick auf die andere Seite öffnen. Loke tritt dafür ein,
dass alle Kommentare veröffentlicht werden, weil alle in gewisser Weise etwas
aussagen.

In Deutschland jedenfalls wünschen sich die Journalisten und Nutzer einen
angemessenen Umgangston, was sich in den Beschwerden beim deutschen Presserat
widerspiegelt. Aus diesem Grund werden Maßnahmen zur Regulierung von Kommentaren
ergriffen. Diese Maßnahmen werden im nächsten Punkt erläutert.

% vim: set ai si et tw=80 sts=2 ts=2 sw=2:

\chapter{Kommentarmanagement}

\section{Was beinhaltet das Kommentarmanagement?}
Im vorherigen Kapitel wurde dargestellt, warum es irgendeiner Form bedarf, die
Kommentare zu regeln. Es kommt eben vor, dass sich Kommentierende nicht an einen
angemessenen Diskussionston halten. Diese Nutzer schaden sowohl den anderen
Nutzern, als auch der Zeitung. Die Nutzerschaft will ja eine angenehme Umgebung
vorfinden \autocite[S.~217]{meyer-carey} und alle wollen \glqq gute\grqq{}
Kommentare lesen, obwohl insgesamt das Diskussionsniveau im Internet gesunken
ist.\footnote{"The Internet has lowered the discourse in general – the brevity,
the speed, this sense of ‘why should I make an effort?' [\ldots] It is
problematic for the society, problematic for the democracy, problematic in every
sense." \autocite[S.~130]{singer}} Die Nutzer sollten aus diesem Grund auf
{\bfseries Empfehlungen oder Richtlinien zum Kommentieren} hingewiesen werden
(\glqq Netiquette\grqq).


\begin{quote}
"[\ldots] a `good' comment entailed one that obeyed the policy rules (i.e. no
ranting or trash talk), stayed on topic (introduced and framed by the news
article), praised the original journalism, confirmed the facts in that story,
and informed other readers by adding new information."
\autocite[S.~134]{robinson}
\end{quote}

Im schlimmsten Fall kann es sogar zu Gesetzesverstößen kommen. Es gibt noch
andere Gründe, die zu einer Reglementierung von Kommentaren führen. Es können
sowohl strategische Faktoren sein, als auch das politische Klima, oder die
Themen selbst. Natürlich spielt auch das journalistische Verständnis und die
Vorgaben innerhalb der Redaktion eine Rolle \autocite[S.~106f]{reich}. Es geht
jedoch immer um die grundsätzliche Frage, ob ein Kommentar \glqq rein\grqq{}
darf oder ob er besser \glqq draußen\grqq{}  bleibt.

Diese kategorische Entscheidung hat natürlich eine negative Komponente.
Kommentare werden so zuerst nach einer Regelverletzung bewertet und nicht nach
ihrem eigentlichen Wert. Außerdem geht diese Art der Bewertung weg von
journalistischen Gesichtspunkten hin zu wirtschaftlichen, wie zum Beispiel das
Erreichen großer Klickzahlen.

Im folgenden Abschnitt werden die Möglichkeiten des Kommentarmanagements
skizziert. Es ist vorab noch zu bemerken, dass jede Zeitung irgendeine Form des
Kommentarmanagements macht und dass die meisten Kommentare zugelassen werden.
Die geschätzten Angaben der interviewten Journalisten aus aller Welt in
\textcite[S.~106]{singer} reichen von 40 bis 90 Prozent der frei gegebenen
Kommentare.

Die zwei Säulen des Kommentarmanagements sind Moderieren und Registrieren. Es
gibt jeweils verschieden Arten der Moderation und verschiedene Möglichkeiten der
Registrierung. Zusätzlich gibt es immer wieder neue Funktionen, die ganz neue
Möglichkeiten schaffen.  Die Nutzer wollen ja vor allem Inhalte teilen und
sichtbar machen (siehe Abschnitt~\ref{sec:secondary-gate}).
Dazu braucht es die technischen Voraussetzungen.

Eine {\bfseries Moderation} ist vor oder nach einer Freischaltung möglich (\glqq
pre-mo\-de\-ra\-tion\grqq{} vs. \glqq post-moderation\grqq) und wird entweder vom
Journalisten durchgeführt oder jemand anderem des Medienbetriebs. Sie kann ganz
ausgegliedert werden. Oder der Nutzer übernimmt die Moderation, in manchen
Redaktionen zusammen mit dem Journalisten, als \glqq super-user\grqq{}
\autocite[S.~112]{reich}. Diese Zusammenarbeit (\glqq collaborative
moderation\grqq{} \autocite[S.~109]{reich} geht in Richtung einer
Selbstregulierung von Kommentaren, was als erstrebenswert angesehen werden
kann.\footnote{``We need to tend towards this self-regulation of users by other
users because it is the more logical thing to do. It is them, in the end, who
know what do they want, what information is more useful and what is less, and
what bothers them.'' \autocite[S.~112]{reich}} Auch hier kommt der Nutzer als
\emph{secondary gatekeeper} zum Einsatz.

Derjenige, der die Moderation übernimmt, ist der \glqq comment moderator\grqq{}
\autocite[S.~68]{paulussen}. Er sichtet und sortiert aus. Wichtig für einen
{\slshape comment moderator} ist eine klare Vorgabe, was erlaubt ist und was
nicht. An diese {\bfseries Richtlinien} sollen sich einheitlich die Moderatoren
halten, denn jeder hat unterschiedliche Ansichten über eine mögliche
Grenzziehung.\footnote{Es gibt nicht nur unterschiedliche Ansichten zu der
Qualität, sondern zu Kommentaren überhaupt. \textcite{robinson} unterscheidet
zwischen \glqq traditionalist\grqq{} und \glqq converger\grqq{}. Die ersten
sind eher älter und länger im Unternehmen und sehen sich als Autorität. Sie
wollen eine gewisse Verantwortung des Medienbetriebs auch online aufrecht
erhalten. Die anderen sind eher jünger und kürzer dabei, legen weniger Wert
auf Registrierung und wollen vor allem mit den Nutzern interagieren.} Es gibt
aber auch Überlegungen, ob jeder Journalist für seinen Beitrag selbst die
Kommentierregeln bestimmt \autocite[S.~127]{singer}.

\begin{itemize}
  \item[-] {\bfseries Prä-Moderation}
    bedeutet, dass alles gesichtet wird, bevor es online geht (mit Hilfe
    entsprechender Software). Das entspricht einem klassischen journalistischen
    Verständnis, ist aber mit hohen finanziellen und zeitlichen Kosten
    verbunden.  Diese \glqq proaktive\grqq{} Herangehensweise
    \autocite[S.~108]{reich} kann unter Umständen die Diskussion verzerren. Die
    Journalisten sind damit jedoch auf der sicheren (legalen) Seite. Denn
    unangemessene Kommentare treten auch bei Artikeln auf, von denen es man
    nicht gedacht hätte. Obendrein können \glqq Trolle\grqq{} die Moderation
    stören oder umgehen. Für die Nutzer ist eher demotivierend, denn die meisten
    wollen ihren Beitrag veröffentlicht sehen.\footnote{``You can't really beat
    hitting 'submit' and seeing your comment there before you go away. It
    encourages you to come back. You feel you've engaged.''
    \autocite[S.~109]{reich}}

  \item[-] {\bfseries Post-Moderation}
    findet statt, nachdem der Kommentar bereits veröffentlich wurde, und es
    irgendeinen Grund gibt, einzugreifen. Dieser Umgang mit Kommentaren ist viel
    offener und entspannter und in diesem Fall \glqq reaktiv\grqq. Einer
    Post-Moderation geht jedoch fast immer eine Registrierung voraus. Bei
    heiklen Themen (siehe Abschnitt~\ref{sec:schlecht})
    wird die Post-Moderation allerdings problematisch. Hier kann mit einem
    Umschwenken auf Prä-Moderation dagegen gesteuert werden.
\end{itemize}

Eine {\bfseries Registrierung} ist meistens die Voraussetzung, um kommentieren
zu können (vor allem bei Post-Moderation). Dabei muss man seine persönlichen
Daten (be\-stä\-tig\-te Emailadresse und Klarnamen) angeben, um einen Zugang zu
erhalten. Das verhindert natürlich, dass die Nutzer unkontrolliert drauf los
schreiben. Aber es verhindert ebenso manchen Kommentar, der eigentlich
geschrieben werden möchte. Registrierungen können unterschiedlich gehandhabt
werden. Oft genügt die Angabe einer Emailadresse. Und auch wenn mehr als das
verlangt wird, also die Angabe von Namen, dann kann die tatsächliche Identität
einer Person damit nicht bestätigt werden. Der Name kann erfunden sein (auch bei
Klarnamenpflicht) oder die Nutzer verwenden Spitz- oder Fantasienamen. Ebenfalls
besteht die Befürchtung, dass Kommentare allein deswegen nicht geschrieben
werden, weil man mehr als die Emailadresse angeben muss.  {\bfseries Klarnamen}
schaffen zwar Vertrauen (z. B. die wertvollste Rezension bei Amazon stammt
meistens von Nutzern mit Angabe von vollständigen Namen), machen aber auch Angst
(z.B. vor Ärger, den die Meinungsäußerung bei Nachbarn oder beim Chef
möglicherweise mit sich bringt). Mittlerweile sind die Nutzer jedoch daran
gewöhnt, ihre Meinung zu sagen und ihren Namen zu nennen, den jeder sehen kann,
oft sogar mit einem Foto verbunden (wie bei Facebook oder Twitter).


Es gibt {\bfseries zusätzliche  Funktionen} für das Management von Kommentaren.

\begin{itemize}

  \item[-] Dazu gehört zum Beispiel der  \glqq{\bfseries report abuse button
    (Melden-Button)}\grqq{} \autocite[S.~110f]{reich}. Andere Nutzer können auf
    diese Weise melden, wenn sie unangebrachte Kommentare lesen. In der
    aktuellen Untersuchung von \textcite[S.~63]{singer:2014} verwenden drei
    viertel der Online-Zeitungen diese Funktion.

  \item[-] Nutzer, die bereits negativ aufgefallen sind, weil sie das System
    ausnutzen, werden markiert. Deren Kommentare müssen dann vorher kontrolliert
    oder ganz gestoppt werden, entweder endgültig oder nur für eine bestimmte
    Zeit.

  \item[-] Man bietet eine  {\bfseries freiwillige Registrierung} an, und wer
    das tut, bekommt zusätzliche Privilegien (z.B. mehr Platz zum Kommentieren,
    Kommentare ohne Moderation, Möglichkeit für oder gegen andere Kommentare
    abzustimmen).

  \item[-] Die Option des {\bfseries \glqq Bewerten\grqq} ist überhaupt eine
    gute Lösung, die besten Kommentare zu filtern und die Selbstregulierung zu
    unterstützen. Nutzer können so Kommentare oder andere Kommentatoren bewerten
    und/oder weiter empfehlen \autocite[S.~63f]{singer:2014}.

  \item[-] Hilfreich sind dabei die {\bfseries neuen technischen Möglichkeiten}
    des Kommentarmanagements. Damit können automatisch Nutzerprofile erstellt
    und Kommentare in soziale Netzwerke gestellt oder auf diese verwiesen
    werden. Die Beliebtheit bei anderen Nutzern kann dargestellt werden.
    {\slshape Social Bookmarking} wird möglich. Man kann Artikel {\bfseries
    weiter schicken}. Welche dieser Optionen nutzen die Online-Zeitungen?
    Benutzen die Leser diese {\slshape features}? Erfinden die Redaktionen neue?

   % \fxnote*[author=VP]{Absatz Teil vom letzen Punkt oder besser außerhalb der
    %Auflistung?}{%
    

\section{Umsetzung der Dokumentenanalyse}
    
    Gerade wurde beschrieben, wie ein Kommentarmanagement ablaufen
    kann. Daraus ergeben sich auch die Kategorien (siehe oben fettgedruckt),
    die das Kommentarmanagement ausmachen.
   Der nächste Schritt ist die tatsächliche Analyse des Kommentarmanagements im 
   Internet. 
    
    Dazu wurden die entsprechenden Nachrichtenportale aufgerufen und die
    Kommentarfunktionen aufgesucht. Ich habe mich mit der Online-Zeitung vertraut
    gemacht und mit der Funktion, Kommentare zu schreiben. 
    Dann habe ich versucht, Antworten auf die vorgefertigten Kategorien zu finden,
    entweder durch Beobachten (z.B. Welche Funktionen gibt es rund um den Bereich,
    wo man Kommentare verfassen kann?) oder Ausprobieren (z.B. sich selber
    Registrieren, sich in der Community umschauen) oder Suchen.
    
    Fragen nach Moderation und inhaltlichen Richtlinien konnte ich ausschließlich 
    über die Allgemeinen Geschäftsbedingungen, Nutzungsbedingungen, Richtlinien,
    und/oder Netiquette herausfinden. Diese sind oft bei der Registrierung hervorgehoben,
    und/oder direkt verlinkt. Teilweise muss man den Bedingungen auch zustimmen bei einer
    Registrierung. Einige Online-Zeitungen haben ihre Bedingungen aber auch im 
    \glqq Kleingedruckten\grqq\- stehen.
    
    Ich habe die Informationen benutzt, die von den Nachrichtenportalen veröffentlicht
    worden sind. Bei der Kategorie \glqq Moderation\grqq\- wäre es eigentlich noch besser, 
    die Redaktionen direkt nach ihrer Vorgehensweise zu befragen. Ich finde die 
    verfügbaren Antworten beim Thema Moderation ungenügend,
    gerade was die \glqq stichprobenartige\grqq\- Moderation betrifft: Was heißt stichprobenartig genau?
    Wie viele Stichproben werden gemacht? Sitzt da jetzt jemand, ein Moderator, oder macht übernimmt  
    ein Computer die Aufgabe?
    
    Bei der Beschäftigung mit dem Kommentarmanagement auf den entsprechenden Internetseiten
    bin ich auch auf Funktionen gestoßen, die nicht bei Punkt 5.1
    besprochen worden sind, z.B.  \glqq alternative Anmeldung\grqq\- anstelle von Registrierung.
    Dies ist im Sinne der Dokumentenanalyse, die offen für neue Kategorien ist.
    
    Nachdem die ganzen Informationen zum Kommentarmanagement gesammelt waren, 
    habe ich Tabellen für die Kategorien angelegt und die Ergebnisse eingetragen.
    Während des Eintragens und auch danach habe ich versucht, die Ergebnisse zu 
    vereinheitlichen, damit ein Vergleich und Überblick der Online-Zeitungen möglich ist. 
    
    Zu Beginn der Arbeit war es ein Ziel, alle Kategorien in einer Tabelle darzustellen.
    Dies ist jedoch durch den begrenzten Platz nicht möglich. Es sind also mehrere
    Tabellen entstanden. Am Schluss gibt es eine gekürzte Zusammenfassung von ausgewählten
    Kategorien.


    
   % Die Ergebnisse werden tabellarisch wieder gegeben und entsprechend
    %kommentiert.
    
    
    
    
    
    

\end{itemize}


% vim: set ai si et tw=80 sts=2 ts=2 sw=2:




\chapter{Kategoriensystem}

%1. Art der Moderation
%2. Registrierung
%3. Inhaltliche Regeln
%4. Zusammenfassung

\section{Einführung}

Sprechblasen bei den einzelnen Artikeln der Online-Zeitungen weisen in der Regel darauf hin, dass kommentiert wurde
(mit Angabe der Anzahl der abgegebenen Kommentare) und/oder ein Kommentar
geschrieben werden kann. Keine Sprechblasen machen zeit.de und tagesspiegel.de. Sie verlinken mit dem Wort
\glqq Kommentar\grqq. Swp.de verwendet weder Sprechblasen noch benutzt diese Zeitung einen anderen 
Link direkt zu Kommentaren\footnote{Ist das der Grund, warum extrem wenig kommentiert wird? Die Nutzer 
müssen selbst bis zum Ende
des Artikels scrollen, um zur Kommentarfunktion zu gelangen. Darüberhinaus hat swp.de einen begrenzten Zugang. 
Den Nutzern wird es schwer gemacht einen Kommentar abzugeben.}.
Beim Kommentar selbst ist eine Zeitangabe üblich
entweder mit Datum und Uhrzeit oder \glqq vor \ldots Stunden\grqq.  Kommentare
stehen unter dem Artikel in vielen Fällen nach Werbung (in eigener Sache).

Auf den Nachrichtenportalen wurden folgende Besonderheiten beobachtet:
Auch auf formaler Ebene werden Kommentare beschränkt. Es gibt ganz
unterschiedliche Handhabungen. Handelsblatt, Hamburger Morgenpost und Die Welt bieten Kommentierzeiten an. 
Nur dann gibt es die Möglichkeit zu schreiben. Außerhalb dieser Zeiten ist es nicht
möglich (handelsblatt.com: 7.30 - 21 Uhr (bis zu sieben Tage lang kann man Kommentare schreiben), mopo.de: 8 - 21 Uhr und 
welt.de: werktags von 6 - 23 Uhr, samstags, sonntags und feiertags von 7 - 23 Uhr).

Faz.net und welt.de\footnote{Welt.de schließt die Kommentarfunktion nach zwei/drei Tagen oder früher bei
  Regelverstößen.} schließen die Kommentarfunktion nach einer bestimmten
Zeit. Dann können die verfassten Kommentare zwar noch gelesen werden, aber keine
neuen mehr geschrieben. 

Bei  bild.de, suedkurier.de und zeit.de wird in den Richtlinien erwähnt, dass nicht jeder Artikel
kommentiert werden kann.

Mainpost.de, swp.de und augsburger-allgemeine.de haben das Problem eines begrenzten
Zugangs, d.h. es kann nur eine bestimmte Anzahl an Artikeln im Monat kostenlos gelesen werden. 

%Faz.net und handelsblatt.de weisen darauf hin, dass Kommentare im Internet recherchierbar sind und frei zugänglich (steht jetzt bei inhaltliche Regeln).
Stuttgarter-zeitung.de, hna.de und abendzeitung-muenchen.de betonen, dass es sich bei den Kommentaren um eine freie 
Meinungsäußerung der Nutzer handelt. 

Der Anbieter {\slshape Disqus} stellt eine Oberfläche bzw. Technik zur Verwaltung von Kommentaren zur Verfügung. 
Die Nutzungsbedingungen können vom Kunden selbst gewählt werden, die Form bleibt jedoch 
bei allen Nutzern von {\slshape Disqus} gleich. Für diese Art der Kommentarfunktion haben sich diese
Zeitungen entschieden: welt.de, merkur.de, hna.de, mopo.de


%Zeile 1
%bild.de &
  %K. nicht zu allen Artikeln möglich (themenunabhängig) \\\midrule

%Zeile 2
%spiegel.de &
 % keine Sprechblase sondern Hinweis auf [Forum] auf Startseite; beim Artikel: Sprechblase mit
%  Ausrufungszeichen; K. durchnummeriert \\\midrule

%Zeile 3
%faz.net &
  %K.-Funktion kann irgendwann eingestellt werden; %zeitliche Begrenzung um K. zu schreiben; 
 % Hinweis, dass die K. im Internet recherchierbar sind\\\midrule

%Zeile 4
%focus.de &
  %auch Videos kommentierbar \\\midrule

% Zeile 5
%diewelt.de &
  %über Disqus verwaltet; %Schließung nach zwei/drei Tagen oder früher bei Regelverstößen \\\midrule

% Zeile 6
%derwesten.de &
 % alle Artikel kommentierbar %aber kein direkter Link/keine Sprechblasen; K.
%  durchnummeriert \\\midrule

% Zeile 9
%rp-online.de &
%  (fast) alle Artikel kommentierbar%fast jedes Thema kommentierbar
  %\\\midrule

% Zeile 10
%handelsblatt.de &
%  Kommentierzeiten: 7.30 - 21 Uhr, bis zu sieben Tage lang; 
%K.  werden (u.U.  gekürzt) multimedial verbreitet \\\midrule

% Zeile 11
%suedkurier.de &
 % K. nicht zu allen Artikeln möglich (themenunabhängig) \\\midrule

% Zeile 12
%zeit.de &
 % keine Sprechblase aber [Anzahl + Kommentare]; K. werden durchnummeriert;  
%  K. nicht zu allen Artikeln möglich; %Definition von K.: ``kürzere Textbeiträge,
 % die Sie unter vorhandenen Artikeln, Videos, Fotostrecken oder anderen
 % Multimedia-Inhalten abgeben können'' \\\midrule

% Zeile 13
%badische-zeitung.de & 
%K. geben nicht unbedingt die Meinung der Zeitung wieder
  %\\\midrule

% Zeile 14
%stuttgarter-zeitung.de &
 % für alle Nutzer, %K. geben nicht die Meinung der Stuttgarter Zeitung
 % wieder\\\midrule

% Zeile 15
%merkur.de &
%  über Disqus verwaltet, 
 % für alle Nutzer
  %\\\midrule

% Zeile 16
%hna.de &
 % über Disqus verwaltet,
 % für alle Nutzer; %freie Meinungsäußerung; %bei Verstößen
  %Aufruf eine Email zu schreiben \\\midrule

% Zeile 17
%mopo.de &
 % über Disqus verwaltet; %Kommentierzeiten: 8 - 21 Uhr\\\midrule

% Zeile 18
%mainpost.de &
  %K. unter Artikel und neben Artikel (als neuer Tab!); 
  %begrenzter Zugang\footnote{Das heißt, dass nur eine bestimmte Zahl an Artikeln im Monat
  %kostenlos lesbar sind.}\\\midrule

% Zeile 19
%tagesspiegel.de &
%  keine Sprechblase aber [Anzahl + Kommentare]\\\midrule

% Zeile 20
%swp.de &
  %kein Link auf K.; begrenzter Zugang\\\midrule

%Zeile 21
%augsburger-allgemeine.de &
  %begrenzter Zugang; %K. durchnummeriert

\section{Kategorien: \glqq Moderation\grqq\- und \glqq Melden\grqq}

Bei einer Moderation geht es vor allem darum, Kommentare auf ihren Inhalt hin zu
prüfen. Verstößt dieser Inhalt gegen den guten Ton und die Regeln, dann greift
der Moderator ein und löscht die Beiträge (im Folgenden als \glqq
Entfernen\grqq\ bezeichnet). Wenn der Nutzer wiederholt negativ auffällt, dann
wird er unter Umständen vom Kommentieren ausgeschlossen und sein Account wird
gesperrt (im Folgenden als  \glqq Sperren\grqq\ bezeichnet). Das kann zeitweise
sein oder auch dauerhaft.

Die Aufforderung, Verstöße zu melden (im Folgenden als \glqq Melden\grqq\
bezeichnet), ist an die Nutzer gerichtet. Dies ist umso wichtiger, wenn gar
keine Moderation statt findet. Dann sind die Nutzer selbst diejenigen, die
kontrollieren. Aber auch mit Moderation ist eine Melden-Funktion angebracht.
Alle  untersuchten Nachrichtenportale (bis auf tagesspiegel.de und stuttgarter-zeitung.de) haben einen
entsprechenden Melden-Button. Ein \glqq Melden\grqq\- gehört also 
mittlerweile fast zum Standard einer Kommentarfunktion. 

Stand Anfang 2015: Für welche Art der Moderation haben sich die meistgelesenen
Online-Zeitungen entschieden? Wie moderieren sie: greifen sie vor einer
Veröffentlichung ein oder bearbeiten sie einen Kommentar erst wenn er bereits
erschienen ist? Oder überlassen sie das Feld ganz den Nutzern allein?

Die meisten Zeitungen wollen wissen, was veröffentlicht wird und arbeiten mit
einer Prä-Moderation.  Ein Drittel der Portale hat sich für gar keine Moderation
entschieden. Das sind eher die regional geprägten Online-Zeitungen, die sich auch für
den Anbieter {\slshape Disqus} entschieden haben. Das bedeutet jedoch nicht,
dass alles online stehen bleibt, was von den Nutzern geschrieben wird.  Wird der
Redaktion Missbrauch gemeldet, dann werden die entsprechenden Kommentare
gelöscht.

Eine Post-Moderation machen die wenigsten, und wenn, dann mit Einschränkungen,
das heißt, sie prüfen nicht alles, sondern machen Stichproben. Das wird hier mit
\glqq stichprobenartige Post-Moderation\grqq{} bezeichnet. Inwieweit sie aktiv
prüfen, kann nicht herausgefunden werden. Reagieren sie nur auf Melden der
Nutzer, dann ist das eigentlich wie \glqq keine Moderation\grqq{} einzustufen.
Inwieweit Computerprogramme die Moderation übernehmen, kann auch nicht
herausgefunden werden.

Die überregionalen Zeitungen entscheiden sich alle für eine Prä-Moderation
(focus.de mit Einschränkungen). Bild.de überlässt die Kommentare ganz dem Volk
und macht keine Moderation, misst dem \glqq Melden\grqq{} aber größere Bedeutung
zu. Denn dann übernehmen die Nutzer mit dieser Funktion in gewisser Weise die
Moderation. Auch bei \glqq stichprobenartiger Prä-/Post-Moderation\grqq{} ist
\glqq Melden\grqq{} umfangreicher als bei anderen Moderationen.

%Spalte1: Angabe, welche Art der Moderation angewendet wird
%Spalte2: gibt es einen Melden-Button?

%\begin{landscape}
\begingroup
  \footnotesize
  \begin{longtable}{p{24mm}p{98mm}p{11mm}}
  \caption{Kategorie \glqq Moderation\grqq} \\ \\

  \toprule
  \bfseries Portal &
  \multicolumn{1}{c}{\bfseries Art der Moderation} &
  \bfseries Melde-Button \\
  \midrule[\heavyrulewidth]
  \endfirsthead

  \toprule
  \bfseries Portal &
  \multicolumn{1}{c}{\bfseries Art der Moderation} &
  \bfseries Melde-Button \\
  \midrule[\heavyrulewidth]
  \endhead

  \multicolumn{3}{r}{\emph{Fortsetzung auf der nächsten Seite}}
  \endfoot

  \bottomrule
  \endlastfoot

%Zeile 1
bild.de
& {\bfseries keine}: Entfernen; Sperren (bei Melden!)
& \centerline{ja\footnote{Mit Angabe von vier Möglichkeiten: Spam, Copyright, beleidigend,
  anderer Grund; kurze Begründung möglich}}
\\\midrule

%Zeile 2
spiegel.de
& {\bfseries Prä-Moderation} (zeitliche Verzögerungen); Forumsmoderator mit
  Namen sysop/forum@spiegel.de; Entfernen/keine Veröffentlichung; Redaktion kann
  bearbeiten/verschieben/Diskussionen schließen; keine Benachrichtigung bei
  Nicht-Erscheinen
  & \centerline{ja}
\\\midrule

%Zeile 3
faz.net
& {\bfseries Prä-Moderation}: (zeitliche Verzögerungen) nach Prüfung den
  Richtlinien entsprechend; Sperren; K. werden evtl. gekürzt/verändert
  & \centerline{ja}
\\\midrule

%Zeile 4
focus.de
& {\bfseries stichprobenartige Prä-Moderation}: keine umfassende Prüfung, aber
  Stichproben vom Digitalvermarkter; übertragen der Moderatin an TOMORROW FOCUS Media
  GmbH/TOMORROW FOCUS NEWS+ GmbH/Beauftragte; Ändern/Entfernen; Sperren
  & \centerline{ja\footnote{besonders hervorgehoben}}
\\\midrule

% Zeile 5
welt.de
& {\bfseries Prä-Moderation} (zeitliche Verzögerungen): Moderator = Team von
  \glqq Welt\grqq-Mitarbeitern; Kritik an Moderationsweise per Email; bestimmte
  Moderationszeiten, Moderation gemäß Nutzungsregeln; bei Verstößen (von
  Beiträgen/Benutzernamen/Profilfoto): Ändern der Beiträge/keine
  Veröffentlichung; Entfernen v. Beleidigungen/Beschimpfungen; Sperren
  (zeitweise/dauerhaft); Whitelist: Nutzer können ohne Moderation kommentieren
  (Redaktion und Community-Mitglieder, die auffallend positiv kommentieren)
& \centerline{ja\footnote{Symbol \glqq Fähnchen\grqq{} mit
  Hovereffekt\label{foot:fahne}}} \\\midrule

% Zeile 6
derwesten.de
& {\bfseries keine}: automatische Veröffentlichung; Entfernen (zeitweise/ganz)
  durch Community Management; Hinweis, dass Beiträge falsche Tatsachen
  enthalten/Rechte Dritter verletzen/in die Irre führen/täuschen können
  & \centerline{ja}
\\\midrule

% Zeile 9
rp-online.de
& {\bfseries keine}: bei Melden Entfernen/Sperren (auch Antworten dazu);
  Abbruch der K.-Funktion bei Verstößen/nicht themenbezogen (bis dahin
  veröffentlichte Beiträge bleiben); Entfernen/Bearbeiten nach eigenem Ermessen
  & \centerline{ja\footnote{mit Begründung mit Name/Email, auch ohne
  Registrierung möglich}}
\\\midrule

% Zeile 10
handelsblatt.com
& {\bfseries stichprobenartige Post-Moderation}: keine umfass. Prüfung, aber
  Eingriff b. Verstößen;  bei Melden Löschen/Sperren; man setzt sich mit
  Nutzer in Verbindung; nicht eindeutige Sachverhalte müssen abgeklärt werden;
  verstoßen einzelne Abschnitte gg Regeln, werden diese entfernt u. der
  Eingriff kenntlich gemacht; wird K. komplett entfernt, wird dies auch
  kenntl. gemacht; Abbruch der Funktion bei gehäuften Verstößen
  %verstoßen viele K. gg Regeln, wird die Funktion abgeschaltet
  & \centerline{ja\footnote{auch Email oder telefonisch möglich}}
\\\midrule

% Zeile 11
suedkurier.de
& {\bfseries stichprobenartige Post-Moderation}: keine umfass. Prüfung;
  Bearbeiten/Löschen (wenn nicht themenbezogen/von Trollen); Nutzer, die
  regelmäßig gg Regeln verstoßen werden per Email ermahnt;
  schwere/wiederholte Verstöße führen z. Ausschluss der Community; 
  Abbruch d. Funktion b. gehäuften Verstößen;
  %bei  gehäuften Verstößen wird Funktion abgeschaltet; 
  externe Links: keine systematische Prüfung/keine Verantwortung von suedkurier.de, 
  Löschen b. Verstößen/Weiterleiten auf rechtswidrige Inhalte
  %bei Links, die auf rechtswidrige Inhalte weiterleiten
  & \centerline{ja\footnote{mit Name, Emailadresse, Grund an Community Manager}}
\\\midrule

% Zeile 12
zeit.de
& {\bfseries Prä-Moderation} (wie bei Leserbriefen): Entfernen/Kürzen (mit
  Begründung), warum eingeschritten wurde mit Anmerkungen und Kennzeichnungen;
  Sperren bei schweren/wiederholten Verstößen; Abbruch der Funktion bei
  gehäuften Verstößen/nicht themenbezogen; 
  externe Links: keine Verantwortung von zeit.de, Löschen bei Verstößen
  & \centerline{ja}
\\\midrule

% Zeile 13
badische-zeitung.de
& {\bfseries stichprobenartige Post-Moderation}: keine umfass. Prüfung;
  Bearbeiten/Entfernen; schwere/wiederholte Verstöße führen zum Ausschluss d.
  Community; keine Crosspostings; externe Links: Löschen bei Verstößen/Weiterleiten auf rechtswidrige Inhalte
  %bei Links, die auf rechtswidrige Inhalte weiterleiten
  & \centerline{ja}
\\\midrule

% Zeile 14
stuttgarter-zeitung.de
& {\bfseries Prä-Moderation}: Bearbeiten/Entfernen/Sperren (wenn nötig,
  zeitweise, ganz)
  & \centerline{nein}
\\\midrule

% Zeile 15
merkur.de
& {\bfseries keine}: Entfernen (kein Bearbeiten) von nicht-themenbezogenen
  Beiträgen/privaten Chats/Sperren des Nutzers
  & \centerline{ja\footref{foot:fahne}}
\\*\midrule

% Zeile 16
hna.de
& {\bfseries keine}: kommentarloses Entfernen/Sperren des Nutzers
& \centerline{ja\footref{foot:fahne}}
\\\midrule

% Zeile 17
mopo.de
& {\bfseries keine}
& \centerline{ja\footref{foot:fahne}}
\\\midrule

% Zeile 18
mainpost.de
& {\bfseries Prä-Moderation}: Löschen von Beiträgen, die gegen das Gesetz verstoßen
& \centerline{ja\footnote{Mit Begründung, mit Name/Email wegen Rückfragen}}
\\\midrule

% Zeile 19
tagesspiegel.de
& {\bfseries Prä-Moderation}: Rahmen für sachlichen Austausch von Argumenten
  schaffen
  & \centerline{nein}
\\\midrule

% Zeile 20
swp.de
& {\bfseries keine/stichprobenartige Post-Moderation}: Entfernen; Versuch
  unerwünschte Beiträge fernzuhalten; keine umfass. Prüfung
  & \centerline{ja}
\\\midrule

%Zeile 21
augsburger-allgemeine.de

& {\bfseries Post-Moderation}: zurückhaltendes Moderieren, Überwachen der
  Nutzungsbedingungen; Entfernen entsprechender Passagen o. ganz, wenn nicht
  themenbezogen/bei Provokationen/gg Nutzungsregeln; Abbruch der
  Funktion bei Verstößen/nicht themenbezogen nach Ankündigung o. sofort;
  Sperren  des Nutzers (zeitweise/dauerhaft)
  & \centerline{ja}
\\\midrule

%Zeile 22
abendzeitung-muenchen.de

& {\bfseries keine/Post-Moderation}: zu den Moderationszeiten montags - freitags 9 - 19 Uhr
wird moderiert, ansonsten keine M.; Entfernen v. Beleidigungen/gesetzl. Verstößen
  & \centerline{nein}


\end{longtable}

%\end{landscape}
\endgroup

% vim: set et ai tw=80 ts=2 sts=2 sw=2:



\section{Kategorie: \glqq Registrierung\grqq}

Kommentare können immer gelesen werden, um selbst welche zu schreiben, bedarf es
jedoch einer Anmeldung. Darin sind sich alle Anbieter einig (Ausnahme ist die Abendzeitung
Online, bei der Kommentieren ohne Registrierung möglich ist). Dies ist die erste
Maßnahme, um Missbrauch bei der Kommentarfunktion zu vermeiden. Man kann nicht
einfach drauf los schreiben, sondern muss den Prozess einer Registrierung durch
gehen. In der Regel bieten die Online-Zeitungen eine eigene Registrierung an.
Man ist dann auf dem Portal angemeldet und sieht es auch (außer Hamburger
Morgenpost und Südwest Presse). Manche Zeitungen ermöglichen auch eine Anmeldung
über Dritte Anbieter. Das ist dann eine Kompromisslösung für die Nutzer. Eine
Anmeldefunktion wird nämlich auch als Hemmschwelle zum Schreiben diskutiert.
Kann man sich über ein Konto anmelden, das man bereits hat, dann gilt dieses
Argument nicht mehr. Der Trend, den \textcite[S.~69]{singer:2014} beschreibt
bestätigt sich also: \glqq Some newspapers also had begun requiring users to
comment through Facebook, an intriguing step that removes many of the problems
created by anonymous postings while also helping generate social network
traffic. Such trends richly deserve the attention that journalism scholars have
begun to afford them.\grqq

Durch eine Anmeldung wird sicher gestellt, dass die Person existiert und
gegebenenfalls zur Verantwortung gezogen werden kann.  Natürlich ist auch hier
Missbrauch nicht ausgeschlossen. Der Nutzer wird darauf hingewiesen,
wahrheitsgemäße Angaben zu machen. 

Interessant ist, welche Angaben die Zeitungen zur Anmeldung fordern. Unabdingbar
sind die Eingabe einer Emailadresse sowie ein Passwort. Dann kommt es wieder zu
unterschiedlichen Handhabungen und unterschiedlichen Lösungen. Manche Zeitungen
verlangen die Nennung von Klarnamen und Benutzernamen, manche nur Klarnamen,
manche nur Benutzernamen. Einige Portale geben zusätzliche Hinweise zur
Registrierung. Bei manchen Zeitungen kann man sich über ein anderes Portale oder
soziales Netzwerk anmelden.

Daraus ergeben sich folgende Unterkategorien zur Registrierung:

\begin{itemize}
  \item \glqq{\bfseries alternative Anmeldung}\grqq:\\ Kann man sich über ein anderes
    Portal anmelden, um dann eigene Inhalten einzustellen? Welches Portal ist
    das?

  \item\glqq{\bfseries Klarname}\grqq{} (KN): \\ Ist die Angabe von Zu- und Nachnamen
    notwendig?

  \item\glqq{\bfseries Benutzername}\grqq{} (BN):\\ Ist die Angabe eines
    Benutzernamens, d.h.  eines frei gewählten Namens oder Spitzname möglich?

  \item\glqq{\bfseries Sonstiges}\grqq:\\ Welche Angaben werden von der Online-Zeitung
    zusätzlich verlangt zur Registrierung?
\end{itemize}

%\begin{landscape}
\begingroup
  \footnotesize
  \begin{longtable}{p{24mm}p{20mm}p{10mm}p{10mm}p{60mm}}

  \caption{Kategorie \glqq Registrierung\grqq}
  % FIXME Footnote einbauen!
  %\footnote{Die Angabe einer Emailadresse und eines Passworts wird immer
  %verlangt. In der Tabelle sind weitere obligatorische Angaben, Unterkategorien,
  %aufgeführt.}
  \\
  \toprule
  \bfseries Portal & \bfseries alt. Anmeld. &
  \centerline{\bfseries KN} & \centerline{\bfseries BN} & \bfseries Sonstiges\\
  \midrule[\heavyrulewidth]
  \endfirsthead

  \toprule
  \bfseries Portal & \bfseries alt. Anmeld. & \centerline{\bfseries KN}
  & \centerline{\bfseries BN} & \bfseries Sonstiges\\
  \midrule[\heavyrulewidth]
  \endhead

  \multicolumn{5}{r}{\emph{Fortsetzung auf der nächsten Seite}}
  \endfoot

  \bottomrule
  \endlastfoot

bild.de
& mypass, Facebook
& \centerline{ja}
& \centerline{ja}
& Volljährigkeit bzw. Einverständnis der Erziehungsberechtigten bei
  Minderjährigen
\\*\midrule

spiegel.de % Spalte 2 bei ''mein spiegel'' als Abonnent oder Nicht-Abonnent
& Facebook
& \centerline{ja}
& \centerline{ja\footnote{Es wird darauf hingewiesen, dass der Benutzername
  angezeigt wird.\label{foot:angezeigt}}}
&
\\*\midrule

faz.net % Spalte 3 bei ''Mein FAZ.NET'' mit
& keine
& \centerline{ja}
& \centerline{nein}
&
\\*\midrule

focus.de % Spalte 4 bei FOCUS online
& Facebook
& \centerline{ja\footnote{Es wird darauf hingewiesen, dass der Name im gesamten
  Internet recherchierbar ist.}}
& \centerline{nein}
&
\\*\midrule

welt.de % Spalte 5 bei WELT DIGITAL über
& mypass/Disqus\footnote{Disqus bietet an, sich auch über Facebook, Twitter, Google anzumelden\label{foot:Disqus}}
& \centerline{ja}
& \centerline{ja\footnote{Die Nutzungsbedingungen (siehe Tabelle 6.3 \glqq inhaltliche Regeln\grqq) gelten auch für den Benutzernamen.}}
& als Gast schreiben möglich
\\*\midrule

derwesten.de % Spalte 6 auf derwesten.de mit
& keine
& \centerline{ja}
& \centerline{ja\footref{foot:angezeigt}}
& jeder ist zugangs- und teilnahmeberechtigt
\\*\midrule

rp-online.de % Spalte 9 bei mein RP ONLINE mit
& keine
& \centerline{nein}
& \centerline{ja}
& auch Minderjährige, wenn sie sich über Nutzung bewusst sind bzw. mit
  Einverständnis der Erziehungsberechtigten
\\*\midrule

handelsblatt.com % Spalte 10 auf handelsblatt.de mit
& keine
& \centerline{ja}
& \centerline{nein}
& Volljährigkeit
\\*\midrule

suedkurier.de % Spalte 11auf suedkurier.de mit
& keine
& \centerline{ja}
& \centerline{ja}
& Anrede, Land, Adresse (Angaben werden auf Richtigkeit geprüft (teilweise auch
  telefonisch); Einverständnis nach sechs Monaten Inaktivität wird
  Registrierung/Benutzername gesperrt/freigegeben
\\*\midrule

zeit.de % Spalte 12 auf zeit.de mit
& keine
& \centerline{nein}
& \centerline{ja}
&
\\*\midrule

badische-zeitung.de % Spalte 13 bei Meine BZ mit
& keine
& \centerline{ja}
& \centerline{nein}
& Anrede, Löschen des Accounts bei Wegwerf-Emailadresse
\\*\midrule

stuttgarter-zeitung.de % Spalte 14 auf stuttgarterzeitung.de
& Facebook
& \centerline{ja}
& \centerline{nein}
& für alle Nutzer
\\*\midrule

merkur.de % Spalte 15 bei Merkur-Online
& Disqus\footref{foot:Disqus}
& \centerline{nein}
& \centerline{ja\footnote{Ein Benutzername ist erwünscht und soll angemessen
  sein.}}
& für alle Nutzer; als Gast schreiben
\\*\midrule

hna.de % Spalte 16 mit HNA-Login
& Disqus\footref{foot:Disqus}
& \centerline{nein}
& \centerline{ja\footnote{Der Benutzername soll angemessen und nicht
  beleidigend/anstößig sein.}}
& für alle Nutzer
\\*\midrule

mopo.de % Spalte 17
& Disqus\footref{foot:Disqus}
& \centerline{nein}
& \centerline{nein}
&
\\*\midrule

mainpost.de % Spalte 18 auf Mainpost.de
& keine
& \centerline{ja}
& \centerline{ja\footnote{Der Benutzername soll nicht beleidigend,
  ehrverletzend, hetzerisch sein.}}
& Adresse
\\*\midrule

tagesspiegel.de % Spalte 19 auf tagesspiegel.de
& keine
& \centerline{nein}
& \centerline{ja\footnote{Der Benutzername soll angemessen sein.}}
&
\\*\midrule

swp.de % Spalte 20
& keine
& \centerline{ja}
& \centerline{ja}
&
\\*\midrule

augsburger-allgemeine.de %spalte 21
& keine
& \centerline{ja}
& \centerline{ja}
& Wohnort, Ausschluss bei Mehrfach-/Scheinregistrierungen oder unter dem Namen
  anderer, bei Minderjährigen Einverständnis der Erziehungsberechtigten
\\*\midrule

abendzeitung-muenchen.de %spalte 22
& keine
& \centerline{nein}
& \centerline{ja}
& für alle Nutzer
\\*\midrule

fr-online.de %spalte 23
& Disqus\footref{foot:Disqus}
& \centerline{ja}
& \centerline{ja}
& 


\end{longtable}
%\end{landscape}
\endgroup

% vim: set ai si et tw=80 sts=2 ts=2 sw=2:


\section{Kategorie: \glqq inhaltliche Regeln und Hinweise\grqq}

Wie sie sich die Diskussion auf ihren Plattformen vorstellen und wie nicht,
beschreiben die Online-Zeitungen in der Netiquette, den AGB oder den
Nutzungsbedingungen. Auf manchen Portalen muss der Nutzer bei der Registrierung
dazu aktiv zustimmen und damit bestätigen, die Regeln zumindest zu akzeptieren.

Für die Regeln verwenden die Redaktionen eigene Formulierungen mit
unterschiedlichem Umfang. Es geht erst einmal darum, zu erklären, was man unter
einem \glqq guten Ton\grqq\ versteht. Dann wird erläutert, was nicht toleriert
wird und wann und wo ein Eingreifen seitens der Moderation erfolgt.

Gerade für Foren ohne Moderation sind inhaltliche Vorgaben besonders wichtig. Da
es niemanden gibt, der die Beiträge vorab oder danach liest und gegebenenfalls
einschreitet, müssen die Nutzer darüber informiert werden, was erlaubt ist und
was nicht. Außerdem werden sie zur Selbstregulierung aufgefordert - zum Beispiel
über die Möglichkeit Verstöße zu melden - und brauchen Handlungsvorgaben.

Es gibt inhaltliche Regeln zu den Kommentaren, die auf allen Plattformen zu
finden sind (Ausnahmen werden entsprechend gekennzeichnet).  Das sind zunächst
rechtliche Hinweise. Der Nutzer bestätigt der Urheber seiner Beiträge zu sein
oder die Urheberrechte zu besitzen, d.h. er muss Gewähr leisten, dass fremde
Inhalte zur Verbreitung freigegeben sind. Er trägt somit die volle Verantwortung
für die eingestellten Beiträge und stellt sicher, dass keine Rechte Dritter oder
Urheberrechte oder Persönlichkeitsrechte oder sonstige Rechte verletzt werden.
Außerdem gilt die Rechteeinräumung. Der Nutzer stimmt damit zu, dass die
entsprechende Zeitung sein Beiträge \glqq benutzen\grqq\ kann (z.B. vervielfältigen,
modifizieren, anpassen, übersetzten, bearbeiten, verbreiten, verwerten,
hervorheben, bewerten, archivieren, usw.).

Es gibt infolgedessen auch einen Haftungsausschluss des Anbieteres. Dieser
haftet nicht für den Inhalt von Nutzerbeiträgen. Dasselbe gilt für die Inhalte
fremder Seiten durch Verlinkung. Entsteht ein Schaden haftet der Nutzer.

Was auf keinen Fall geduldet wird ist Werbung in irgendeiner Form.
Diskussionsforen dürfen nicht für kommerzielle Zwecke missbraucht werden.
Diskussionsforen sind keine Werbefläche für Webseiten oder Dienste (Spamming).
Auch dem Datenschutz müssen die Nutzer zustimmen. Damit ist z.B. das
Über\-prü\-fen der Emails oder Abfragen auf Viren gemeint oder das Einhalten von
gesetzlichen, behördlichen und technischen Vorschriften. Das Passwort soll
geheim gehalten und Vertraulichkeit gewahrt werden.

Jedes Portal weist ausdrücklich darauf hin, dass Kommentare {\bfseries
themenbezogen} sein sollen.  Es gibt Inhalte, die kategorisch bei allen
Zeitungen nicht erlaubt sind. Dazu gehören Beleidigungen und Beiträge mit
se\-xi\-sti\-schen/sit\-ten\-wi\-dri\-gen/por\-no\-gra\-phi\-schen/obs\-zö\-nen/grob
anstößigen Inhalten und Rassismus. (Diese eben genannten inhaltlichen Verbote
werden in der Tabelle nicht mehr aufgeführt, weil sie in allen Online-Zeitungen
genannt werden.)

Auch das Verbot von Diskriminierungen zählen die meisten Nachrichtenportale in
irgendeiner Form auf.

Die Redaktionen wählen unterschiedliche Formulierungen, um unerwünschte Beiträge
zu beschreiben. Viele Formulierungen sind sich im Inhalt ähnlich  und
unterscheiden sich um Nuancen (z.B. Beleidigung und Beschimpfung). Trotzdem
werden sie hier bewusst aufgelistet und nicht zusammengefasst, um deutlich zu
machen, was die Zeitungen extra erwähnt haben wollen und was nicht.

Eine lange Liste von Richtlinien kann als besonders sorgfältige Beschäftigung
mit den Inhalten von Nutzern gesehen werden. Inwieweit die Nutzer sich damit
auseinandersetzen, weiß man nicht. Sie müssen zwar bei den meisten Zeitungen den
Richtlinien zustimmen. Ob sie diese auch tatsächlich lesen sei dahingestellt.
Auf alle Fälle wird bei Zustimmung direkt auf diese Regeln verwiesen oder
verlinkt, was dem Nutzer entgegenkommt.

In der ersten Zeile (schräggedruckt) steht, wo diese Regeln zu finden sind und
wo eine Zustimmung mit Anklicken gefordert wird.

\pagebreak

\begingroup
\footnotesize
\begin{longtable}{p{28mm}p{110mm}}
  \caption{Kategorie \glqq inhaltliche Regeln\grqq} \label{inhaltliche Regeln} \\ \\

  \toprule
  \bfseries Portal & \multicolumn{1}{c}{\textbf{Inhaltliche Regeln}}\\\midrule[\heavyrulewidth]
  \endfirsthead

  \toprule
  \bfseries Portal & \multicolumn{1}{c}{\textbf{Inhaltliche Regeln}}\\\midrule[\heavyrulewidth]
  \endhead

  \multicolumn{2}{r}{\emph{Fortsetzung auf der nächsten Seite}}
  \endfoot

  \bottomrule
  \endlastfoot


bild.de & \emph{Nutzungsbedingungen: allgemeine und besondere (Zustimmung
  verlangt bei Registrierung); Netiquette}

  Diskussion: sachlich, höflich, respektvoll, nicht dagegen argumentieren,
  Angriffe ignorieren, wie man selbst behandelt werden möchte

  keine: Beschimpfungen\footnote{Hier und im Folgenden: Unter Beschimpfungen fallen auch
  Diskriminierungen, Erniedrigungen, Verleumdungen, Ruf- und
  Geschäftsschädigungen}, Belästigungen, Drohungen; privaten
  Angaben\footnote{Hier und im Folgenden: Private Angaben sind Angaben von Postadresse und/oder
  Telefonnummer und/oder Emailadresse} oder Angaben über Dritte; automatisierte
  Nutzung; Links\footnote{Hier und im Folgenden: keine Links, die auf Werbung, Chats, Foren, strafbare Inhalte, u.ä.
  weiterleiten.}; Trolle; Schadsoftware;

  kein Mobbing
  \\*\midrule

spiegel.de & \emph{Nutzungsbedingungen: allgemeine und für Foren (Zustimmung)}

  Diskussion: fair, sachlich, angenehm, offen, freundschaftlich, respektvoll

  keine: Beiträge mit strafbaren, inakzeptablen  Inhalten
  \\*\midrule

faz.net & \emph{Nutzungsbedingungen: allgemein (Zustimmung); \glqq wie Sie
  mitdiskutieren\grqq\--Button}

  Diskussion: sich bewusst machen, dass die Beiträge oder persönlichen Daten
  frei zugänglich ins Internet gestellt werden und dort auch recherchierbar sind
  (= bewusst machen)

  keine: Beschimpfungen; Beiträge mit links- und rechtsradikalen Inhalten;
  falsche, nicht nachprüfbare Behauptungen; Links
  \\*\midrule

focus.de & \emph{AGB (Zustimmung), Netiquette (Zustimmung)}

  Diskussion: sachlich, freundlich, respektvoll, tolerant

  keine: Fremdtexte; privaten Angaben; Links; Diskriminierungen jeder
  Art\footnote{Hier und im Folgenden: Unter Diskriminierungen jeder Art versteht man Diskriminierungen
  aufgrund von Herkunft, Nationalität, Religion, sexueller Orientierung, Alter,
  Geschlecht, usw.}; Beiträge mit  demagogischen Inhalten; Schadsoftware

  kein Missbrauch von Daten
  \\*\midrule

welt.de & \emph{Nutzungsbedingungen, veraltete Netiquette}

  Diskussion: fair, höflich, verständlich, kritisch, zivil, kontrovers, gehaltvoll

  keine Beschimpfungen, Provokationen, Entwürdigungen, Aufruf zu Demonstrationen
  oder Gewalt; Angaben über Dritte; Hasspropaganda

  Gesetze beachten; auf Haftungsausschluss wird nicht hingewiesen
  \\*\midrule

derwesten.de & \emph{Nutzungsbedingungen (Zustimmung), Netiquette}

  keine: Beschimpfungen, Beleidigungen; Beiträge mit gewaltverherrlichenden,
  antisemitischen, gesetzeswidrigen Inhalten

  das Forum ist kein Veranstaltungskalender, keine Terminankündigungen
  \\*\midrule


% Spalte 9
rp-online.de & \emph{AGB}

  Diskussion: wie man selbst behandelt werden möchte

  keine: Anschuldigungen, Tatsachenbehauptungen; Beschimpfungen, unwahren,
  sinnlosen, störenden Beiträge oder Beiträge mit strafbaren Inhalten,
  Urheberrechtsverstöße, Drohungen, volksverhetzende Äußerungen, Aufforderung zu
  Gewalt; Schadsoftware

  Verbot von Inhalten, die dem Ansehen von Verstorbenen und deren Angehörigen
  schaden könnten, die doppeldeutig sind oder anderweitige Darstellungen, deren
  Rechtswidrigkeit vermutet wird, aber nicht abschließend festgestellt werden
  kann \\*\midrule

% Spalte 10
handelsblatt.com & \emph{Nutzungshinweise (Zustimmung), Netiquette}

  Diskussion: vorsichtig mit zynischen, ironischen Äußerungen, guter Ton, nicht
  persönlich werden, Provokationen ignorieren, sich bewusst machen

  keine: Diskriminierungen jeder Art;  Beschimpfungen, Beiträge mit
  strafrechtlich relevanten Inhalten; Angaben über Dritte;
  %sich bewusst machen, welche eigenen Daten frei zugänglich ins Internet
  %gestellt werden (steht jetzt bei Allgemeines);
  Trolle; Schadsoftware; Nennung von Produktnamen, Dienstleistern, Marken,
  Produzenten

  auf Nutzungsrechte von handelsblatt.de wird nicht
  hingewiesen\\*\midrule

% Spalte 11
suedkurier.de & \emph{Nutzungsbedingungen (Zustimmung), Netiquette}

  Diskussion: fair, sachlich, offen, gehaltvoll

  keine: nicht-belegbaren Behauptungen, Beschimpfungen, Diffamierungen,
  Drohungen, Diskriminierungen aller Art, Hetze, Gewaltverherrlichung,
  Vulgärausdrücke;  Angaben über Dritte
  \\*\midrule

% Spalte 12
zeit.de & \emph{AGB (Zustimmung), Netiquette (besonders ausführlich und erklärend)}

  Diskussion: vorsichtig mit zynischen, ironischen Äußerungen, guter Ton,
  Provokationen ignorieren, nachvollziehbar argumentieren; sich bewusst machen
  und Durchlesen vor Abschicken

  keine: Beschimpfungen, Diskriminierungen aller Art (auch Behinderung,
  Einkommensverhältnisse), Diffamierungen; Verdächtigungen; Angaben über Dritte;
  Schadsoftware
  \\*\midrule

% Spalte 13
badische-zeitung.de & \emph{AGB (Zustimmung), Netiquette}

  Diskussion: sachlich, niveauvoll, fair, offen, freundlich, tolerant, wie man
  es selber möchte, nicht persönlich werden

  keine: Beiträge mit hetzerischem, gewaltverherrlichendem Inhalt; privaten
  Daten
  \\*\midrule

% Spalte 14
stuttgarter-zeitung.de & \emph{AGB (Zustimmung), Kommentarregeln = Netiquette}

  Diskussion: engagiert, fair, respektvoll, sachkritisch, seriös

  eine: Beschimpfungen, Schmähungen, Volksverhetzung, Propaganda, Beiträge mit
  jugendgefährdenden, antisemitischen, strafbaren, menschenverachtenden, gegen
  die guten Sitten verstoßenden Inhalten; Trolle; Schadsoftware; privaten Daten;
  Angaben über Dritte
  \\*\midrule

% Spalte 15
merkur.de & \emph{AGB (Zustimmung), Netiquette (Zustimmung)}

  Diskussion: sachlich, freundlich, verständlich, nicht persönlich werden, man
  soll Spaß haben und sich wohl fühlen, andere Meinungen akzeptieren

  keine: Beiträge mit unwahren, unsachlichen, jugendgefährdenden,
  verleumderischen, verfassungsfeindlichen, extremistischen, illegalen,
  ethisch-moralisch-problematischen Inhalten; Schadsoftware; privaten
  Daten
  \\*\midrule

% Spalte 16
hna.de & \emph{AGB (Zustimmung), Netiquette (Zustimmung)}

  Diskussion: sachlich, freundlich, verständlich, nicht persönlich werden, man
  soll Spaß haben und sich wohl fühlen, andere Meinungen akzeptieren

  keine: Beiträge mit unwahren, unsachlichen, jugendgefährdenden,
  verleumderischen, verfassungsfeindlichen, extremistischen, illegalen,
  ethisch-moralisch-problematischen Inhalten; Schadsoftware; keine privaten
  Daten

  auf Nutzungsrechte und Haftungsausschluss von hna.de wird nicht
  hingewiesen
  \\*\midrule

% Spalte 17
mopo.de &  \emph{keine}\tabularnewline\midrule

% Spalte 18
mainpost.de& \emph{Netiquette}

  Diskussion: fair, respektvoll, nicht persönlich werden, vorsichtig mit
  zynischen, ironischen Äußerungen

  keine: Angaben über Dritte; privaten Daten; Beiträge mit ehrverletzenden,
  gewaltverherrlichenden Inhalten oder Aufrufen zur Gewalt

  kein Aufruf zu Straftaten; auf Nutzungsrechte und Haftungsausschluss von
  mainpost.de wird nicht hingewiesen
  \\*\midrule

% Spalte 19
tagesspiegel.de & \emph{Richtlinien für Community}

  Diskussion: sachlich, respektvoll, fair, angenehme Atmosphäre; bewusst machen

  keine: Beschimpfungen, Diskriminierungen aller Art (auch aufgrund von
  Weltanschauung, sozialem Status), Verdächtigungen, Beiträge mit pietätlosen,
  menschenverachtenden, gewaltverherrlichenden, Inhalten; Trolle;
  %keine Links zu anderen Webinhalten, (steht jetzt bei formale Regeln)
  kein Geschichtsrevisionismus; auf Nutzungsrechte von tagesspiegel.de wird
  nicht hingewiesen
  \\*\midrule

% Spalte 20
swp.de & \emph{Netiquette}

  Diskussion: sachlich, fair, freundlich, wie man selbst behandelt werden möchte

  keine: hetzerischen, gewaltverherrlichenden Töne; privaten Daten

  auf Rechte wird nicht hingewiesen
  \\*\midrule

% Spalte 21
augsburger-allgemeine.de &
  \emph{AGB, Nutzungsbedingungen der Community (Zustimmung) (äußerst
  umfangreiche Erklärung, was erwünscht ist und was nicht, was er für Fälle
  geben kann und wie damit umgegangen werden soll}

  keine: Beschimpfungen; Beiträge mit menschenverachtenden, abscheulichen,
  bedrohlichen, belästigenden, strafbaren, jugendgefährdenden,
  verfassungsfeindlichen, extremistischen Inhalten (Anzeige gegen den Nutzer);
  Inhalte von verfassungsfeindlichen Gruppierungen; Inhalte, die dem Ansehen des
  Forums Schaden zufügen oder stören können;  Angaben über Dritte; Schadsoftware

  das Folgen von Tipps, Ratschlägen folgt auf eigene Gefahr
  \\*\midrule

% Spalte 22
abendzeitung-muenchen.de &
  \emph{Nutzungsbedingungen, Kommentarregeln}
	
	Diskussion: sachkritisch, seriös
  keine: Beschimpfungen, Drohungen, Diskriminierungen, strafbare Äußerungen, sinnlose 
  Inhalte, Provokationen; Links
  
  Verbot: von Schadsoftware, den freien Zugang für Nutzer einzuschränken/zu unterbinden

\end{longtable}
\endgroup

% vim: set ai si et tw=80 sts=2 ts=2 sw=2:

\section{Kategorie: \glqq formale Regeln\grqq}

mit  den Unterkategorien
\begin{itemize}
\item\glqq {\bf Zeichenbegrenzung}\grqq: Gibt es eine Zeichenbegrenzung oder kann der Nutzer schreiben so viel er möchte?


\item\glqq{\bfÜberschrift}\grqq: Wird die Eingabe einer Überschrift verlangt?


\item\glqq{\bf Sonstiges}\grqq: Welche formalen Hinweise geben die Online-Zeitungen zusätzlich?
\end{itemize}


Beobachtungen bei den Nachrichtenportalen:\\
Die Zeitungen geben auch Hinweise zum formalen Umgang mit Kommentaren. Die Hälfte gibt eine Zeichenbegrenzung vor. Diese reicht von 800 bis 3000 Zeichen. 
Ebenso fordert ungefähr die Hälfte der Redaktionen die Eingabe eines Betreffs. Die sonstigen formalen Regeln haben unterschiedlichen Inhalt. Interessant ist die Möglichkeit bei mainpost.de, tagesspiegel.de und augsburger-allgemeine.de den eingegebenen Text sogar hervorzuheben. Diese Funktion geht über die übliche Art von Kommentaren hinaus. Bei augsburger-allgemeine.de kann der Kommentar vollständig formatiert werden.


\begin{landscape} \footnotesize
  \begin{longtable}{l|llp{100mm}}

  & \multicolumn{3}{c}{\bfseries formale Regeln} \\
  & Zeichenbegrenzung & Überschrift (Pflichtfeld) & Sonstiges \\\hline
  \endfirsthead

  & \multicolumn{3}{c}{\bfseries formale Regeln} \\
  & Zeichenbegrenzung & Überschrift (Pflichtfeld) & Sonstiges \\\hline
  \endhead

  \hline \multicolumn{4}{r}{\emph{Fortsetzung auf der nächsten Seite}}
  \endfoot

  \hline
  \endlastfoot

%Kommentar: formale Regeln\\
%  & Zeichenbegrenzung & Überschrift (Pflichtfeld) & Sonstiges
%
%Zeile 1: Portale	 %
%Spalte 1		
bild.de			& keine & keine & vorsichtig mit Großbuchstaben, Zitate kennzeichnen \\\hline
%Spalte 2	
spiegel.de			& keine & optional & keine Bilder posten, keine langen Zitate (Links verwenden) \\\hline
%Spalte 3		
faz.net			& 1000 Zeichen & ja, 100 Zeichen & \\\hline
%Spalte 4		
focus.de			& 800 Zeichen & ja & reiner Text ohne besondere Kennzeichnungen (z.B. keine Smilies, Hervorhebungen, Chat-Symbole, nur Kleinschreibung, usw.), korrektes Deutsch, auf Rechtschreibung/Interpunktion achten, Absätze machen \\\hline
% Spalte 5		
welt.de			& keine & keine & Zitate kennzeichnen; keine Fremdsprachen, keine Links zu externen Webseiten (seriöse Ausnahmen möglich) \\\hline
% Spalte 6	
derwesten.de		& keine & keine & Zitate nur mit Quellenangabe\\\hline
% Spalte 9		
rp-online.de		& keine & ja & deutsche Sprache; Links möglich (keine Links zu Werbung/strafbare Inhalte) \\\hline
% Spalte 10	
handelsblatt.com		& 2000 & keine & Großbuchstaben (Schreien) vermeiden; Absätze machen und strukturieren; Wortwahl überprüfen; auf Rechtschreibung achten; 	Zitate/Quellen kennzeichnen \\\hline
% Spalte 11		
suedkurier.de		& 1000 Zeichen & ja & Zitate kennzeichnen mit Quellenangabe \\\hline
% Spalte 12		
zeit.de			& 1500 Zeichen & ja, mindestens 5 Zeichen & Absätze machen; auf Rechtschreibung achten; vorsichtig mit Großbuchstaben; Zitate kennzeichnen, wenig verwenden, mit Quellenangabe, nur als Ergänzung verwenden; Links möglich \\\hline
% Spalte 13	
badische-zeitung.de	& keine & keine & \\\hline
% Spalte 14	
stuttgarter-zeitung.de	& keine & ja & Links möglich (keine Links zu Werbung/kommerziellen Angeboten/Chats/Foren/strafbaren Inhalten) \\\hline
% Spalte 15	
merkur.de			& keine & keine & deutsche Sprache; Links nicht erwünscht (falls doch distanziert sich merkur-online von Inhalten der gelinkten Seiten) \\\hline
% Spalte 16	
hna.de			& keine & keine & deutsche Sprache; Beiträge in Fremdsprachen werden gegebenenfalls entfernt, da für größten Teil der Nutzer nicht verständlich; Links nicht erwünscht (falls doch distanziert sich hna.de von Inhalten der gelinkten Seiten) \\\hline
% Spalte 17		
mopo.de			& keine & keine & \\\hline
% Spalte 18		
mainpost.de		& 1000 Zeichen & ja & auf deutsche Rechtschreibung achten; korrekte Interpunktion, Absätze machen; Hervorhebungen möglich: fett, kursiv, unterstrichen; markieren von Links, Zitaten möglich; einfügen von Emojis möglich (Grinsen, Zwinkern, traurig sein); \\\hline
% Spalte 19	
tagesspiegel.de		& 2000 Zeichen & ja & auf Rechtschreibung/Grammatik achten; Hervorhebungen möglich: fett, kursiv; markieren von Quellen durch Links; markieren von Links/Zitaten; keine Links zu anderen Webinhalten; Zitate als Ergänzung, nicht alleinstehend \\\hline
% Spalte 20		
swp.de			& 3000 Zeichen & ja & \\ \hline
%Spalte 21		
augsburger-allgemeine.de	& keine & optional & sämtliche Formatierungsmöglichkeiten, auch Emojis; Vorschau möglich!; Button zum Zitieren; Anhängen von Daten möglich (begrenzte Größe)\\ \hline

\end{longtable}
\end{landscape}


% vim: set ai si et tw=80 sts=2 ts=2 sw=2:

\section{Kategorie: \glqq Funktionen beim Kommentar\grqq}

mit den Unterkategorien:
\begin{itemize}
\item\glqq{\bfseries Bewerten}\grqq:\\
Gibt es die Möglichkeit für einen bereits geschriebenen und veröffentlichten Kommentar eine Bewertung abzugeben, ob man ihn gut (oder schlecht) findet?

\item\glqq{\bfseries Antworten}\grqq: \\
Gibt es die Möglichkeit auf einen bereits geschriebenen und veröffentlichten Kommentar zu antworten?


\item\glqq{\bfseries Sonstiges}\grqq:\\
Was gibt es für weitere Funktionen beim Kommentar selbst? Was kann sich der Nutzer anzeigen lassen? 


\end{itemize}

\paragraph{Beobachtungen bei den Nachrichtenportalen:}

Die Funktion \glqq Bewerten\grqq \- soll helfen, die besten Kommentare zu
filtern und unterstützt die Selbstregulierung (siehe
\ref{abschnitt:kommentarmanagement}).  Wenn eine Bewertung gemacht werden kann,
dann als positive Bewertung, außer wenn  {\slshape Disqus} verwendet wird. Dort
können die Kommentare auch negativ bewertet werden. Ebenso bei focus.de.  Drei
Viertel der Online-Zeitungen benutzt die Funktion \glqq Bewerten\grqq. \\ Die
Funktion \glqq Antworten\grqq{} bieten bis auf rp-online.de und
badische-zeitung.de alle Online-Zeitungen an.  Sie dient hauptsächlich der
Übersichtlichkeit für das Kommentarmanagement und ist überaus
benutzerfreundlich. Man kann direkt auf einen bereits geschriebenen Kommentar
antworten und sieht die Diskussion, die daraus entsteht.  Wird die Funktion
nicht angeboten, muss man sich mühsam durch die Kommentare lesen und schreiben,
auf wen oder was man sich bezieht.  Man kommt jedoch dabei vom Ursprungsgedanken
der Kommentarfunktion, nämlich spontan auf einen Artikel zu antworten, weg.
Dafür werden mehr Diskussionen angeregt.

Bei \glqq sonstigen Funktionen\grqq{} wird häufig das
Ordnen nach bestimmten Kriterien angeboten. Auch hier will man dem Nutzer mit verschiedenen
Optionen entgegenkommen.


\begin{landscape} \footnotesize
\begin{longtable}{lccp{100mm}}

  \caption{Kategorie \glqq Funktionen beim Kommentar\grqq} \\ \\
  \toprule
  \bfseries Portal & \bfseries Bewerten & \bfseries Antworten & \bfseries Sonstiges \\
  \midrule[\heavyrulewidth]
  \endfirsthead

  \toprule
  \bfseries Portal & \bfseries Bewerten & \bfseries Antworten & \bfseries Sonstiges \\
  \midrule[\heavyrulewidth]
  \endhead

  \multicolumn{4}{r}{\emph{Fortsetzung auf der nächsten Seite}}
  \endfoot

  \bottomrule
  \endlastfoot

% Zeile 1
bild.de
& positiv
& ja
& Ordnen (beliebteste, älteste, neueste K.)
\\\midrule

% Zeile 2
spiegel.de
&
& ja\footnote{Auf was man antwortet wird in Zitate gesetzt}
&
\\\midrule

% Zeile 3
faz.net
& positiv
&ja
& dem Kommentator folgen
\\\midrule

% Zeile 4
focus.de
& positiv und negativ\footnote{Positive und negative Bewertungen sind nur nach Anmeldung möglich}
& ja
&
\\\midrule

% Spalte 5
welt.de & \multicolumn{3}{l}{\hspace{2cm}\em Oberfläche von Disqus}
\\\cmidrule(lr){2-4}

& positiv und negativ\footnote{Eine positive Bewertung ist nach Anmeldung oder
  als \glqq Gast\grqq{} möglich, eine negative Bewertung ist nur nach Anmeldung
  möglich}
& ja
& Ordnen (beste, älteste, neueste  K.); die Diskussion auf Twitter/Facebook teilen; die D. empfehlen; 
  Mitteilungen dieser D. per Email erhalten
\\\midrule

% Spalte 6
derwesten.de
&
& ja\footnote{Anzahl der Antworten wird angegeben}
&
\\\midrule

% Spalte 9
rp-online.de
& positiv
& nein
& Ordnen (älteste K.); Button für \glqq mehr Kommentare\grqq
\\\midrule

% Spalte 10
handelsblatt.com
&
& ja
&
\\\midrule

% Spalte 11
suedkurier.de
& positiv\footnote{Bewerten nach Anmeldung möglich\label{foot:Anmeldung}}
& ja
& Ordnen (älteste, neueste, beste Bewertung); \glqq informiert
  bleiben\grqq: bei jedem neuen Beitrag der Diskussion erhält man eine
  Benachrichtigung
\\\midrule

% Spalte 12
zeit.de
& empfehlen
& ja

& Ordnen (neueste,  empfohlene, alle K.); 
	Reaktionen/Antworten auf diesen K. anzeigen; K. empfehlen; es gibt auch Empfehlungen der Redaktion
	(muss nicht bedeuten, dass die Redaktion der Meinung des Lesers zustimmt)
\\\midrule

% Spalte 13
badische-zeitung.de
&
&
&
\\\midrule

% Spalte 14
stuttgarter-zeitung.de
& 
& ja
& Ordnen (älteste, neueste K.)
\\\midrule

% Spalte 15
merkur.de & \multicolumn{3}{l}{\hspace{2cm}\em Oberfläche von Disqus: siehe welt.de}
\\\midrule

% Spalte 16
hna.de & \multicolumn{3}{l}{\hspace{2cm}\em Oberfläche von Disqus: siehe welt.de}
\\\midrule

% Spalte 17
mopo.de & \multicolumn{3}{l}{\hspace{2cm}\em Oberfläche von Disqus: siehe welt.de}
\\\midrule

% Spalte 18
mainpost.de
& positiv\footref{foot:Anmeldung}
& ja
& Ordnen (älteste, neueste, am besten bewertete K.)
\\\midrule

% Spalte 19
tagesspiegel.de
&
& ja\footnote{Man kann wählen, ob man die Antworten sehen möchte}
& Ordnen (älteste, neueste K., chronologisch)
\\\midrule

% Spalte 20
swp.de
& positiv\footref{foot:Anmeldung}
& ja
&
\\\midrule

%Spalte 21
augsburger-allgemeine.de
&
& ja\footnote{Beim Antworten sind Anhänge möglich, es gibt eine Vorschau der Antwort und auch Speichern der Antwort ist möglich}
&

\end{longtable}
\end{landscape}

% vim: set ai si et tw=80 sts=2 ts=2 sw=2:

\section{Kategorie: \glqq Sonstiges\grqq} 

mit den Unterkategorien:
\begin{itemize}[noitemsep]
  \item\glqq{\bfseries Teilen}\grqq:\\
    Neben der Möglichkeit eigene Beiträge zu verfassen, ist es den Nutzern auch
    wichtig, Inhalte mit anderen teilen zu können (siehe 3.5). So verfügt jeder
    Artikel, der auch kommentiert werden kann, über einen Link zu Facebook über
    \glqq teilen\grqq\ und/oder \glqq empfehlen\grqq.  Ebenso ist \glqq
    Teilen\grqq\ auf Twitter und Google+ möglich. Man kann einen Artikel auch
    überall versenden (wird mit dem Briefsymbol gekennzeichnet). Ausnahmen
    werden erwähnt.\\
    Beobachtung: 
    
    Alle Zeitungen stellen eine Verbindung zu Facebook/Twitter/Google+
    her, aber nur ein paar wenige nutzen auch alternative soziale Netzwerke. Es lässt 
    sich kein Trend erkennen. 

  \item\glqq{\bfseries Community}\grqq:\\
    Hier wird angegeben, ob die Online-Zeitung eine Community anbietet, wo der
    Nutzer ein Profil anlegen und weitere Funktionen einer Community nutzen
    kann (siehe auch \ref{Ausblick} \glqq Ausblick\grqq).
    \begin{itemize}[noitemsep]
      \item Profil: Man kann ein Profil anlegen.
      \item Profilbild: Man kann ein Bild hochladen.
      \item Disqus: Es gibt eine Community der entsprechenden Online-Zeitung auf
        {\slshape Disqus}. {\slshape Disqus} bzw.  die Zeitung bietet dort eine Rangliste nach \glqq
        neuesten Kommentaren\grqq{} und  \glqq Top Kommentatoren\grqq{} an.
      \item Ranglisten: z.B. neueste/älteste Lesermeinung, viel/\-we\-nig
        diskutiert, viel/we\-nig empfohlen, TOP-Argumente, meistkommentierte
        Themen, usw.
      \item Nutzer-Statistik: Nutzer registriert seit [...] und
        Anzahl der vom Nutzer verfassten Beiträge
    \end{itemize}

  \item\glqq{\bfseries Funktionen beim Artikel}\grqq:\\
    Was bietet die Online-Zeitung für Funktionen beim Artikel an, die im
    Zusammenhang mit dem Verfassen eines Kommentars stehen?
    
    Die \glqq Funktionen beim Artikel\grqq\- dienen ausschließlich der Benutzerfreundlichkeit. 
 Man geht hier auf den Nutzer zu und gibt ihm die Möglichkeit, das zu tun, was er 
 möglicherweise gerne tun möchte beim oder nach dem Lesen eines Artikels.

  \item\glqq{\bfseries Außergewöhnliches}\grqq:\\
    Was bietet die Online-Zeitung an, das in Zusammenhang mit dem Verfassen
    eines Kommentars steht?
    
    In dieser Unterkategorie ist zusammengefasst, was die Redaktionen der 
    jeweiligen Online-Zeitungen zusätzlich und außerhalb der Reihe anbieten. 
    Hervorzuheben sind focus.de, suedkurier.de und zeit.de, welche  die Nutzer ermuntern einen 
    längeren Kommentar, Leserbericht, zu verfassen. 
 
 
 
    
\end{itemize}


\begin{landscape}
\footnotesize
\begin{longtable}{p{28mm}*{2}{p{36mm}}p{25mm}p{64mm}}

  \caption{Kategorie \glqq Sonstiges\grqq} \\
  \toprule
  \bfseries Portal & \bfseries Teilen & \bfseries Community & \bfseries Funktionen beim Artikel  & \bfseries Außergewöhnliches\\
  \midrule[\heavyrulewidth]
  \endfirsthead

  \toprule
  \bfseries Portal & \bfseries Teilen & \bfseries Community & \bfseries Funktionen beim Artikel & \bfseries Außergewöhnliches\\
  \midrule[\heavyrulewidth]
  \endhead

  \multicolumn{5}{r}{\emph{Fortsetzung auf der nächsten Seite}}
  \endfoot

  \bottomrule
  \endlastfoot


%Zeile 1
bild.de
& tumblr, Pinterest; K. gleichzeitig auf Facebook veröffentl. mögl.
& Profil; Chronologie\footnote{Es gibt eine Chronologie der Kommentare bestimmter Kommentatoren.}
  %Chronologie der K. der Nutzer;
  %kommentiert letzte 24 h (Top 5)
& Korrektur\footnote{Formular zum Versenden an die Redaktion mit Hinweisen auf Fehler oder anderes}
& \glqq Reaktionen\grqq\ zum Anklicken\footnote{\emph{Lachen}, \emph{Weinen}, \emph{Wut}, \emph{Staunen}, \emph{Wow} stehen zur Auswahl,
  welche Reaktion man zu dem Beitrag empfindet}; Rangliste Artikel\footnote{Übersicht der am meisten kommentierten Leserartikel}\label{foot:Rangliste} 
\\*\midrule

%Zeile 2
spiegel.de
& Xing, LinkedIn, tumblr, Pinterest, deli.cio.us, Diggy, reddit
& \glqq MEIN SPIEGEL\grqq; Email für Nutzer sichtbar machen; Ranglisten
%meistkommentierte Themen (Top 5); 
& Merken\footnote{Artikel auf eine Merkliste setzen}; 
  Feedback\footnote{Feedback an die Redaktion über ein Formular}
&%Rangliste Artikel\footref{foot:Rangliste}
\\*\midrule

%Zeile 3
faz.net
&
&  \glqq Mein FAZ.NET\grqq; Profilbild; Ranglisten; K. verwalten, nicht mehr veröffentl.
%jüngste/älteste Lesermeinung, viel/wenig diskutiert, viel/wenig empfohlen, 
%TOP-Argumente; 

& Empfehlen\footnote{Man kann den Artikel mit Sternchen bewerten und empfehlen}, Merken, Permalink, Drucken
& sämtliche Buttons, Symbole, Funk\-tio\-nen mit Hilfe
\\*\midrule

%Zeile 4
focus.de
& Facebook \glqq gefällt mir\grqq, Xing
& Chronologie; Ranglisten\footnote{aktivste Kommentatoren (des Monats, gesamt, top 50), K. des Tages; Videofavoriten der Leser (meistkommentiert, top 20)}
& Korrektur %mit Sternen (Anzahl Bewertungen)
& Leserbericht schreiben\footnote{Es kann ein Text mit mehr Zeichen  verfasst werden, in dem 
  persönliche Erfahrungen beschrieben werden\label{foot:Leserbericht}}; K. abonnieren
\\*\midrule

% Zeile 5
welt.de
&
& Disqus mit Profilbild\footnote{Beim Profilbild das Urheberrecht beachten/Recht auf eigenes Bild; keine Werbung/Logos;  kein beleidigendes, verletzendes Motiv; Verbot von rassistischen, pornografischen, menschenverachtenden, beleidigenden oder gegen die guten Sitten verstoßenden Abbildungen}
&
& \glqq Dieser Kommentar wurde entfernt\grqq\footnote{Hinweis bei Löschung des Kommentars, die Antworten darauf
  sind aber noch sichtbar}; Live-Chats mit Autoren
\\*\midrule

% Zeile 6
derwesten.de
&
&
&
&
\\*\midrule

% Zeile 9
rp-online.de
&
&
& Empfehlen, Drucken, Schriftgröße ändern
& Hinweis zu Kontakt mit der Zeitung\footnote{Email an den Chefredakteur, Leserbrief schreiben}
\\*\midrule

% Zeile 10
handelsblatt.com
& Xing, Email schreiben
&
& Merken
&
\\*\midrule

% Zeile 11
suedkurier.de
&
&
&
& Leserreporter-Beitrag; Rangliste Artikel\footref{foot:Rangliste}   %Seite \glqq Meistkommentiert \grqq
\\*\midrule

% Zeile 12
zeit.de
& kein Teilen auf Startseite\footnote{Beim Artikel ist die Funktion Teilen vorhanden}
& Profil
& Drucken, als PDF speichern
& Leserbericht\footref{foot:Leserbericht}; %= ausführliche Meinungsbeiträge und Erfahrungsberichte
  %(meistgelesene/meistkommentierte Leserartikel, Top 3 auf Leserartikel-Seite)
  Rangliste Art.\footnote{Übersicht der am meisten gelesenen oder kommentierten Leserartikel und Top 3 der Leserartikel}
 \glqq Aus den Kommentaren\grqq\footnote{Höhepunkte aus den Leserdebatten mit
  neuer Fragestellung};
\glqq Bitte weichen Sie vom Thema ab\grqq\footnote{Dabei handelt es sich um ein Experiment, bei dem die Nutzer kommentieren können, ohne sich auf einen Artikel beziehen zu müssen};
Empfehl. bei Facebook\footnote{aktuelle
  Empfehlungen aus dem Facebook-Freundeskreis};
   Tweets von ZEIT ONLINE Politik %Seite \glqq Meistkommentiert/-gelesen\grqq
\\*\midrule

% Zeile 13
badische-zeitung.de
& kein Teilen auf Startseite; kein g+; Versenden (kein Briefsymbol),
  Verlinken
&  \glqq MEINE BZ\grqq; Profilbild 
& Drucken, Vorlesen, Fehler-Melden
& Nutzer-Statistik; Vorschau; Rangliste Artikel\footnote{Übersicht der am meisten und zuletzt kommentierten Artikel}%Seite \glqq Meist-/Zuletztkommentiert\grqq %Nutzer registriert seit [...] + Anzahl Beiträge; %der bereits geschriebenen K. vom Nutzer;   %man kann K. sehen, wie er online aussehen wird
\\*\midrule

% Zeile 14
stuttgarter-zeitung.de
&
&
&
&
\\*\midrule

% Zeile 15
merkur.de
& Youtube; Briefsymbol bedeutet Feedback (kein Versenden)
& Disqus mit Profilbild\footnote{Profilbild soll angemessen sein; es gilt ein Verbot von rassistischen, pornografischen,
  menschenverachtenden, beleidigenden oder gegen die guten Sitten verstoßenden
  Abbildungen}
&
&
\\*\midrule

% Zeile 16
hna.de
&
& Disqus mit Profilbild
&
&
\\*\midrule

% Zeile 17
mopo.de
&
& Disqus mit Profilbild
&
&
\\*\midrule

% Zeile 18
mainpost.de
& Youtube 
& Profil
& Benachrichtigung bei neuem Kommentar %zur Auswahl: ``ich möchte bei neuen K. per Email benachrichtigt werden''
& Kontakt zu Redaktion; Rangliste\footref{foot:Rangliste}, \glqq aktuelle Leserkommentare\grqq
\\*\midrule

% Zeile 19
tagesspiegel.de
&
& Profil; Profilbild (angemessen), Nutzer-Statistik\footnote{Nutzer war seit ... Tagen dabei, das letzte Mal aktiv vor ... Minuten, ... hat .... Kommentare geschrieben und das Profil wurde ... Mal angesehen}
& Newsletter abonnieren, Drucken, Lesezeichen setzen
& 
\\*\midrule

% Zeile 20
swp.de
&
&
&
&
\\*\midrule

%Zeile 21
augsburger-allgemeine.de
&
& Profil (umfangreich, besonders!); Profilbild; Nutzer/Community-Statistik\footnote{mit Anzahl: Themen, Beiträge, angemeldete Nutzer, aktive Nutzer, Gäste, usw.}
&
& Charakterisierung Kommentator\footnote{Sehr erfahrenes/äußerst erfahrenes/erfahrenes Mitglied; Redaktion
  ist Ehrenmitglied und macht mit dem Artikel den ersten Beitrag}; andere Diskussionen und Foren\footnote{in Ordnern mit verschiedenen Themen; \glqq letzter Beitrag\grqq\- 
  wird angezeigt;
  Auswahl nach \glqq neuesten/aktuellsten Themen\grqq\- möglich; Nutzerliste; Suche; Regeln; Hilfe; Suche nach Nutzern möglich; Übersichten der K., wie viele, über welche Themen, usw.}
  %andere Orte zum Diskutieren\footnote{Plauderecke, Diskussionen %(mit Ordnern mit verschiedenen Themen; Anzeige \glqq letzer Beitrag\grqq, Anzahl der Themen und Beiträge) 
  %und Forum %(Auswahl nach \glqq neuesten/aktuellsten Themen\grqq, Nutzerliste, Suche, Regeln, Hilfe). 
  %Suche nach Nutzern möglich: viele  Übersichten, was es für K. gibt, wie viele, über welche Themen, usw.

\\*\midrule

%Zeile 22
abendzeitung-muenchen.de
& kein Teilen auf Startseite, kein Versenden
& 
& Drucken
& Veröffentl. v. ausgewählt. K. in Print-Ausgabe wenn Name, Adres., Email angegeb. %(Adresse wird nicht veröffentlicht)
\end{longtable}
\end{landscape}


%Felder auf Startseite :
%südkurier.de:  ``Meistkommentiert''  (Top 3)

%zeit.de: ``Meistgelesen''/''Meistkommentiert'' (Top 5)

%badjsche-zeitunge.de:  ``Meistkommentiert'' (Top 5)/''zuletzt kommentiert''

%mainpost.de: ``aktuelle Leserkommentare'', ``kommentiert'' (Top 5), Teilen auf Youtube

%augsburger-allgemeine.de: andere Orte zum Diskutieren: Plauderecke, Diskussionen (mit Ordnern mit verschiedenen Themen; Anzeige \glqq letzer Beitrag\grqq, Anzahl der Themen und Beiträge) und Forum (Auswahl nach \glqq neuesten/aktuellsten Themen\grqq, Nutzerliste, Suche, Regeln, Hilfe). Suche nach Nutzern möglich: viele  Übersichten, was es für K. gibt, wie viele, über welche Themen, usw.



% vim: set ai et tw=80 sts=2 ts=2 sw=2:

\section{Zusammenfassung}

Im Folgenden werden in Tabelle \ref{tab:zus} nochmals ausgewählte Kategorien kompakt dargestellt.

Die Spalte {\bfseries Moderation} sagt, was für eine Moderation gemacht wird.
Die jeweiligen Moderationsarten werden wie folgt abgekürzt:
\begin{itemize}
  \item Prä-M. = Prä-Moderation bzw. st.~Prä-M. = stichprobenartige Prä-Moderation
  \item Post-M. = Post-Moderation bzw. st.~Post-M. = stichprobenartige Post-Moderation
\end{itemize}

Die Spalte {\bfseries Registrierung} sagt, welche Angaben nötig sind zu einer Registrierung:

\begin{itemize}
  \item KN = Klarnamen,
  \item BN = Benutzername oder
  \item alt/Disqus = alternative Anmeldemöglichkeiten oder Disqus, wenn man nur über Disqus kommentieren kann.
%alternative Anmeldemöglichkeiten: Fb = Facebook, Dis = Disqus, Tw = Twitter, G = Google
\end{itemize}

In der Spalte {\bfseries Bewerten} bzw. {\bfseries Antworten} geht es darum, ob der Nutzer die Option hat, den Kommentar
zu bewerten bzw. auf einen Kommentar zu antworten.

Bei {\bfseries Community} geht es darum, ob eine solche vorhanden ist. \emph{dis} gibt an, dass es sich um die
Disqus-Community handelt.


%\begin{landscape}
\begin{table}
  \footnotesize
  \caption{Zusammenfassung\label{tab:zus}}
  \begin{tabular}{p{28mm}*{5}{l}}

    \\\toprule
    \bfseries Portal & \bfseries Moderation & \bfseries Registr. & \bfseries Bewerten & \bfseries Antwort & \bfseries Community\\*\toprule

bild.de
& keine
& KN, BN, alt
& ja
& ja
& ja
\\*\midrule

spiegel.de
& Prä-M.
& KN, BN, alt
& nein
& ja
& ja
\\*\midrule

faz.net
& Prä-M.
& KN
& ja
& ja
& ja
\\*\midrule

focus.de
& st. Prä-M.
& KN, alt
& ja
& ja
& ja
\\*\midrule

welt.de
& Prä-M.
& KN, BN, alt
& ja
& ja
& dis
\\*\midrule

derwesten.de
& keine
& KN, BN
& nein
& ja
& nein
\\*\midrule

rp-online.de
& keine
& BN
& ja
& nein
& nein
\\*\midrule

handelsblatt.com
& st. Post-M.
& KN
& nein
& ja
& nein
\\*\midrule

suedkurier.de
& st. Post-M.
& KN, BN
& ja
& ja
& nein
\\*\midrule

zeit.de
& Prä-M.
& BN
& ja
& ja
& nein
\\*\midrule

badische-zeitung.de
& st. Post-M.
& KN
&nein
&nein
& nein
\\*\midrule

stuttgarter-zeitung.de
& Prä-M.
& KN, alt
& nein
& ja
& nein
\\*\midrule

merkur.de
& keine
& BN, alt
& ja
& ja
& dis
\\*\midrule

hna.de
& keine
& BN, alt
& ja
& ja
& dis
\\*\midrule

mopo.de
& keine
& Disqus
& ja
& ja
& ja
\\*\midrule

mainpost.de
& Prä-M.
& KN, BN
& ja
& ja
& ja
\\*\midrule

tagesspiegel.de
& Prä-M.
& BN
& nein
& ja
& nein
\\*\midrule

swp.de
& keine/st. Post-M.
& KN, BN
& ja
& ja
& nein
\\*\midrule

augsburger-allgemeine.de
& Post-M.
& KN, BN
& nein
& ja
& ja
\\*\midrule

abendzeitung-muenchen.de
& keine/Post-M.
& keine
& ja
& ja
& nein

\\*\midrule

fr-online.de
& keine/Prä-M.
& Disqus
& ja
& ja
& dis

\\*\bottomrule
% FIXME Legende?
% REGISTRIERUNG (mypass = Anmeldung über ``mypass'', Fb = Facebook, Tw = Twitter, G = Google, Dis = Disqus)

  \end{tabular}
\end{table}
%\end{landscape}

% vim: set ai si et tw=120 sts=2 ts=2 sw=2:

\chapter{Ausblick}\label{Ausblick}



Ziel und Beantwortung der Forschungsfragen\footnote{FF1: Wie werden die Nutzerkommentare auf
deutschen Nachrichtenportalen gehandhabt?\hfill\\
FF2: An welche Vorgaben müssen sich die Nutzer halten?} ist es, einen Überblick zu schaffen und
darzustellen, wie das Kommentarmanagement bei deutschen Online-Zeitungen aussieht. Infolgedessen ist
auch eine Beschreibung von Kommentaren im Allgemeinen entstanden.

Kommentarmanagement heißt, welche Bedingungen die Zeitungen verlangen, um überhaupt schreiben zu
können, was es für Regeln gibt und welche Funktionen zusätzlich vorhanden sind. Im Sinne einer
Dokumentenanalyse wurde Kategorien erarbeitet, die das Kommentarmanagement beschreiben. Diese
Kategorien sind in Tabellen zusammengefasst und dargestellt. Bei den jeweiligen Tabellen stehen die
entsprechenden Beobachtungen zu den Kategorien.  Es gibt noch weitere Beobachtungen in welche
Richtung das Kommentarmanagement gerade geht.

\begin{itemize}
  \item Kommentare lassen sich {\bfseries formatieren}. Gerade die Online-Ausgabe der Augsburger
    Allgemeinen bietet in der Kommentarfunktion ein volles Textprogramm zur Formatierung an.  Hier
    ist die Frage, ob diese Individualisierung für das Schreiben von  Kommentaren nötig ist. Oder
    hoffen die Zeitungen damit auf bessere Beiträge? Wenn ja, würden auch ein paar wenige
    Formatierungsfunktionen ausreichen?

  \item Die Technik der Kommentarfunktion wird von einem externen Unternehmen zur Verfügung
    gestellt. Dieses externe interaktive Kommentarsystem nennt sich \glqq{\bfseries
    Disqus}\grqq\footnote{Die Disqus-Kommentierungsfunktion wird von der Big Head Labs, Inc., San
    Francisco/USA als Dienstleistung zur Verfügung gestellt.} und wird von mehreren Online-Zeitungen
    genutzt. Es wäre interessant zu untersuchen, ob sich diese Tendenz bestätigt und das
    Kommentarsystem von weiteren Portalen genutzt wird.

    Die Kommentarregeln können bei Disqus selbst bestimmt werden. Disqus moderiert ausdrücklich
    nicht und verwaltet auch nicht die Communities. Es gibt ein paar \emph{Basic Rules}, die
    allgemein für Disqus gelten. Für den Nutzer liegt der Vorteil bei der Verwendung von Disqus
    darin, dass er sich sofort auf der Oberfläche auskennt. Außerdem kann er sich über Disqus
    anmelden und muss sich nicht mehr auf dem Portal registrieren. Die Nutzer können sich auch über
    bestehende Accounts bei Facebook, Twitter und Google anmelden. Dabei geht aber ein Stück
    Individualität einer Zeitung verloren.  Andererseits kann man sich fragen, ob eine individuelle
    Kommentarfunktion überhaupt nötig ist.  Schließlich wollen die Nutzer nur kurz ihre spontanen
    Gedanken loswerden.

%Vorteil: der Nutzer kennt sich sofort auf der Oberfläche aus; er benötigt nur den Zugang vom Anbieter\\
%Nachteil: Individualität geht verloren

  \item Eine andere Tendenz ist der Aufbau von {\bfseries Communities}, d.h.  der Nutzer kann u.a.
    ein Profil anlegen, persönliche Rankings erstellen oder sich in Rankings wiederfinden (Disqus,
    siehe oben,  bietet ebenfalls eine Community an). Auch hier ist das Ziel des
    Kommentarmanagements, dass gute Kommentare verfasst werden. Außerdem findet eine
    Selbstregulierung durch aktive und vertrauenswürdige Nutzer statt. Man kann eher konstruktive
    Beiträge von den Mitgliedern erwarten.  Es entstehen keine Nachteile, da man das Profil auch
    ignorieren kann.  Die Online-Ausgabe der Augsburger Allgemeinen sticht mit ihrer Community und
    den Nutzerbeiträgen auch hier besonders hervor. Neben einem großen Profil gibt es ein
    Textprogramm zur Formatierung (siehe oben). Außerdem kann man Beiträge im Forum nach Themen und
    Aktualität suchen. Es gibt Aufstellungen über Themen, Beiträge, Antworten, Aufrufe. Dieses
    äußerst umfangreiche Forum ist interessant, schießt aber unter Umständen über das Ziel hinaus.
    Schließlich stellt sich noch die Frage, ob das Vorhandensein einer Community die Kommentare
    beeinflusst und ob es dadurch zu einem besseren Umgangston kommt?

  \item Was immer stärker von den Nachrichtenportalen genutzt wird ist das {\bfseries Anmelden über
    andere} soziale Netzwerke, allen voran Facebook und Twitter (Auch bei Disqus ist das Standard,
    siehe oben.). Das ist besonders nutzerfreundlich, da man den Zugang nehmen kann, den man unter
    Umständen eh schon hat. Die Zeitungen schlagen hier zwei Fliegen mit einer Klappe. Sie machen es
    dem Nutzer leicht, da die Registrierung wegfällt, und stellen eine Verbindung in die genannten
    Netzwerke her. 

  \item Inwieweit die Zahl der Kommentare einer Zeitung insgesamt Einfluss auf die Art des
    Kommentarmanagements hat und umgekehrt, wurde nicht untersucht. Die regionalen Zeitungen aus dem
    Süden (z.B. suedkurier.de und swp.de) haben nämlich extrem wenig Kommentare auf ihren Seiten zu
    verzeichnen.  Daraus ergeben sich neue Fragen am Rande. Sind die Nutzer einfach nur zu träge zum
    Kommentieren oder ist die Kommentarfunktion nicht attraktiv genug? Ist es von den Zeitungen
    gewollt, dass wenig kommentiert wird? 
\end{itemize}

Es gibt immer wieder neue Ideen und Modelle für das Kommentarmanagement. Hier sind zwei Beispiele.
\begin{itemize}
	\item Bei dem Online-Magazin \glqq Tablet\grqq{} muss man fürs Kommentar-Schreiben {\bfseries bezahlen}
	\autocite{schenz}.
Damit soll versucht werden, den Missbrauch zu umgehen. Ob die Nutzer bereits sind, Geld fürs Kommentieren zu bezahlen, wird sich zeigen. Genauso, ob sich Störenfriede von Geld abhalten lassen.


\item Oder die Kommentarfunktion wird komplett {\bfseries ausgelagert}, d.h. dass man nicht mehr auf der Seite der
	Online-Zeitung kommentieren kann. Bei sueddeutsche.de wird man auf \glqq Rivva\grqq{} (Rivva filtert das Social Web nach am meisten empfohlenen Artikeln und debattierten Themen) weitergeleitet, wo man die Diskussion verfolgen und über die entsprechenden sozialen Netzwerke auch selber kommentieren kann. Sueddeutsche.de versucht auf diesem Weg die Schwächen des Kommentarmanagements zu umgehen. 

\end{itemize}

% vim: set ai si et tw=100 sts=2 ts=2 sw=2:


\printbibliography
\input{Erklärung}
\end{document}

% vim: set ai si et tw=80 sts=2 ts=2 sw=2:
