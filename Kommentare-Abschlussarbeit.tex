% !TEX TS-program = pdflatex
% !TEX encoding = UTF-8 Unicode

% This is a simple template for a LaTeX document using the "article" class.
% See "book", "report", "letter" for other types of document.

% \documentclass[12pt]{memoir}
\documentclass[12pt,parskip=full]{scrreprt} % use larger type; default would be 10pt

\usepackage[onehalfspacing]{setspace}
\AfterTOCHead{\singlespacing}

\KOMAoptions{DIV=last}

\usepackage[utf8]{inputenc} % set input encoding (not needed with XeLaTeX)
\usepackage[ngerman]{babel}

%%% Examples of Article customizations
% These packages are optional, depending whether you want the features they provide.
% See the LaTeX Companion or other references for full information.

%%% PAGE DIMENSIONS
\usepackage{geometry} % to change the page dimensions
\geometry{a4paper} % or letterpaper (US) or a5paper or....
% \geometry{margin=2in} % for example, change the margins to 2 inches all round
% \geometry{landscape} % set up the page for landscape
%   read geometry.pdf for detailed page layout information

\usepackage{graphicx} % support the \includegraphics command and options

% \usepackage[parfill]{parskip} % Activate to begin paragraphs with an empty line rather than an indent

%%% PACKAGES
\usepackage{booktabs} % for much better looking tables
% \usepackage{array} % for better arrays (eg matrices) in maths
\usepackage{paralist} % very flexible & customisable lists (eg. enumerate/itemize, etc.)
\usepackage{verbatim} % adds environment for commenting out blocks of text & for better verbatim
\usepackage{subfig} % make it possible to include more than one captioned figure/table in a single float
% These packages are all incorporated in the memoir class to one degree or another...

\usepackage{pdflscape}
\usepackage{longtable}

%%% HEADERS & FOOTERS
\usepackage{fancyhdr} % This should be set AFTER setting up the page geometry
\pagestyle{fancy} % options: empty , plain , fancy
\renewcommand{\headrulewidth}{0pt} % customise the layout...
\lhead{}\chead{}\rhead{}
\lfoot{}\cfoot{\thepage}\rfoot{}

%%% SECTION TITLE APPEARANCE
\usepackage{sectsty}
\allsectionsfont{\sffamily\mdseries\upshape} % (See the fntguide.pdf for font help)
% (This matches ConTeXt defaults)

%%% ToC (table of contents) APPEARANCE
\usepackage[nottoc,notlof,notlot]{tocbibind} % Put the bibliography in the ToC
\usepackage[titles,subfigure]{tocloft} % Alter the style of the Table of Contents

% \renewcommand{\cftsecfont}{\rmfamily\mdseries\upshape}
% \renewcommand{\cftsecpagefont}{\rmfamily\mdseries\upshape} % No bold!

%%% END Article customizations

%%% The "real" document content comes below...

% My font adaptions:
% Euler for math | Palatino for rm | Helvetica for ss | Courier for tt
%\renewcommand{\rmdefault}{ppl} % rm
%\linespread{1.05}        % Palatino needs more leading
%\usepackage[scaled]{helvet} % ss
%\usepackage{courier} % tt
%\usepackage{euler} % math
%\usepackage{eulervm} % a better implementation of the euler package (not in gwTeX)
%\normalfont
%\usepackage[T1]{fontenc}

\usepackage{libertine}
\usepackage[T1]{fontenc}

% Maybe colorize links?
\usepackage[%
  pdftitle={Kommentarmanagement auf deutschen Nachrichtenseiten},
  pdfauthor={Nadine Ambrosch},
  bookmarksnumbered,
]{hyperref}

%\setkomafont{date}{\large}
%\setkomafont{author}{\large}
\title{Kommentarmanagement auf deutschen Nachrichtenseiten}
\subtitle{Abschlussarbeit im Aufbaustudiengang Journalistik}
\author{Nadine Ambrosch}

%\date{} % Activate to display a given date or no date (if empty),
         % otherwise the current date is printed 

\begin{document}

\begin{spacing}{1}
  \maketitle
\end{spacing}


\chapter{Einleitung}

\section{Relevanz des Themas}

\glqq Here comes everybody\grqq, so lautet der Satz von Clay Shirky, um die neue
Medienwelt zu beschreiben: Jeder kommt also dran! Jeder kann sich heute schnell
und direkt in den Kommunikationsprozess einklinken.  Und niemand muss mehr
hoffnungsvoll warten, ob es der Leserbrief über die Hürde der Redaktionen
schafft und veröffentlicht wird.  Die Kommentarfunktion ist eine Möglichkeit des
sich \glqq Einschaltens\grqq. Man kann dabei ohne größeren Aufwand zu einem
bestimmten Thema etwas sagen und loswerden, was einem beim Lesen eines Artikels
durch den Kopf schießt.

Es ist also geschafft, was sich sowohl Leser als auch Medienschaffende gewünscht
haben. \glqq Journalists’ desire to maintain communication with readers has been
historically, and remains today one of the basic tenets of journalism\grqq{}
\autocite[S.~67]{santana:2011}.%(Santana, 2011, S. 67).

Der Leser beteiligt sich an der journalistischen Arbeit, er nimmt aktiv am
demokratischen Prozess teil.  Aber dadurch, dass man sich jederzeit einschalten
kann, kommt es zu einer Flut von Äußerungen, deren Qualität nicht garantiert
ist. Die Redakteure sind hin und her gerissen zwischen Nutzen und Nutzlosigkeit.
Sie sehen auch, dass Kommentare auf ihren Plattformen schädlich für das Ansehen
der Zeitung sein können.  Ein einziger \glqq schlechter\grqq\- Kommentar macht
50 gute wertlos, berichtet ein Redakteur in
\textcite[S.~131]{participatory}.

Eine Regulierung ist unumgänglich. Darüber tauchen viele Fragen und Diskussionen
unter den Journalisten zur Organisation von Kommentaren auf.

Das größte Problem stellt der Missbrauch seitens der Kommentierenden dar, die
unter dem Schutz der Anonymität beleidigen und beschimpfen. Das geht
mittlerweile soweit, dass sogar der deutsche Presserat Handlungsbedarf sieht, da
sich Beschwerden mittlerweile zu 60\% auf online erschienene Texte beziehen:

\begin{quote}
„Aber die Zunahme von Beschwerden, unter anderem zu Leserkommentaren und
Online-Archiven, zeigt, dass wir die Publizistischen Grundsätze an einigen
Stellen ergänzen sollten, um den digitalen Ver\-öffentlichungs\-for\-men besser
gerecht zu werden.“ Jahrespressekonferenz, 19.02.2014, Presserat.
\end{quote}

Süddeutsche.de hat sich Mitte des Jahres 2014 für Einschnitte entschieden
\autocite{mark}, um dann die Kommentare ganz weg zu lassen \autocite{fromm},
genauso wie \url{reuters.de} \autocite{standard}.


Wie sieht es bei den anderen deutschen Online-Zeitungen aus? Wie gehen die
Redaktionen mit den Kommentaren um? Lassen sie den Nutzern komplett freie Hand?
Ist eine Anmeldung erforderlich (mit oder ohne Klarnamenpflicht) oder bleiben
die Kommentare anonym? Werden die Kommentare moderiert (vorher oder danach)?
Gibt es Verhaltensrichtlinien für die Nutzer? Wo ist Kommentieren möglich? Gibt
es Verweise auf soziale Medien?


\section{Forschungsziel- und fragen}

Mit dieser Arbeit soll der Frage nachgegangen werden, wie das
Kommentarmanagement der größten Nachrichtenportale jeweils aussieht. Es wird
eine Bestandsaufnahme gemacht, wie die Redaktionen Anfang des Jahres 2015 die
Kommentare handhaben und welche Richtlinien dort vorgegeben werden. In diesem
Zuge wird ein Überblick des aktuellen Kommentarmanagements entstehen.

\begin{itemize} \em
  \item FF1: Wie werden die Nutzerkommentare auf deutschen Nachrichtenportalen
    gehandhabt?
  \item FF2: An welche Vorgaben müssen sich die Nutzer halten?
\end{itemize}


\section{Methodische Umsetzung und Kategorien}

Da es sich beim Analysematerial um schriftliche Dokumente handelt, wird zur
methodischen Umsetzung dieses Überblicks eine Dokumentenanalyse herangezogen.
Das heißt auch, dass eine qualitative Inhaltsanalyse gemacht wird, in Abgrenzung
zur quantitativen Inhaltsanalyse, die eigentlich zur Analyse großer Mengen an
Medieninhalten entwickelt wurde. Nur von qualitativer Inhaltsanalyse zu sprechen
lehnt \textcite{kunzler} ab, da es zu kurz greift bzw.~\textcite{mayring} diesen
„Begriff (bereits) für drei von ihm entwickelte Auswertungsverfahren
[Zusammenfassung, Strukturierung bzw.~Kategorisierung, Explikation] geprägt hat“.

Hier liefern die jeweiligen Nachrichtenseiten das Material. Dieses Material wird
bestimmten Kategorien, welche auf Grundlage des theoretischen Vorwissens
angelegt wurden, zugeordnet. Die Kategorien müssen vorher genau definiert werden
(siehe \textcite{mayring}, deduktive Kategorienanwendung).

Es wird aber auch mit Kategorien gearbeitet, die nicht empirisch gehaltvoll
sind, die also zunächst \glqq leer\grqq{} sind (\glqq heuristische
Rahmenkonzepte\grqq{} oder \glqq sensitizing concepts\grqq{}).  Sie dienen als
\glqq Platzhalter\grqq{}, um sich erst später in empirisch gehaltvolle
Kategorien zu entwickeln. Sie dienen als \glqq Starthilfe", da die Dokumente
erst entdeckt und verstanden werden müssen. Somit wird das deduktive Verfahren
mit dem induktiven kombiniert.

Bei solchen heuristischen Rahmenkonzepten gibt es also eine Kombination von
theoretischem \glqq Fachwissen des Forschers\grqq{} \autocite[S.~65]{strauss}
und empirisch generiertem Wissen. Es ist eine Zusammensetzung von zwei Typen von
Kategorien \autocite[S.~64f]{strauss}, den soziologisch konstruierten und den
natürlichen Kodes. Die soziologisch konstruierten Kodes entstehen erst im Laufe
der Beschäftigung mit dem Material, die anderen werden vorausgesetzt.


\section{Auswahl der Nachrichtenportale}

Im Sinne der Forschungsfrage und der Dokumentenanalyse wird nicht zufällig auf
irgendwelche Portale zugegriffen. Nicht die statistische Repräsentativität ist
das Ziel der Analyse, sondern das kriteriengesteuerte Sammeln von Dokumenten
\autocite{kunzler}.\\
Es kommen die \glqq größten\grqq{} Online-Zeitungen in die Auswahl, d.h. die
Portale mit den meisten Besuchern. Auf diesen Portalen gibt es, meistens am Ende
von Artikeln, das Angebot, Kommentare zu verfassen. Die Rahmenbedingungen und
Funktionalitäten, um dort seine Meinung wieder zu geben, sind das Material für
die Dokumentenanalyse (Quellenbeschreibung).\\
Die Quellenbeschreibung ist Teil der Quellenkritik, die wiederum ein
Gütekriterium der Dokumentanalyse ist. Weitere Bestandteile der Quellenkritik
sind: Die Nachrichtenportale sind im Internet frei verfügbar und für jeden
zugänglich (Textsicherung). Die Nachrichtenportale werden in einem bestimmten
Zeitraum betrachtet (äußere Kritik).\\
Für den vergleichenden Überblick werden deutsche Online-Zeitungen mit den
meisten Besuchern ausgewählt. Diese Zeitungen stellen den Nutzern jeweils eine
Kommentarfunktion für ihre Beiträge zur Verfügung.  Diese Funktionen werden dann
analysiert.\\
Die IVW listet sämtliche Online-Angebote nach der Gesamt-Besucherzahl auf.
Laut den Messungen 1/2015 sind die 21 meistgelesenen Online-Zeitungen folgende:

\href{http://www.bild.de}{Bild.de},
\href{http://www.spiegel.de}{SPIEGEL ONLINE},
\href{http://www.faz.net}{FAZ.NET},
\href{http://www.focus.de}{FOCUS ONLINE},
\href{http://www.welt.de}{Die Welt}, 
\href{http://www.derwesten.de}{Der Westen},
\href{http://www.stern.de}{Stern},
\href{http://www.rp-online.de}{RP ONLINE},
\href{http://www.handelsblatt.com}{Handelsblatt.com},
\href{http://www.suedkurier.de}{SÜDKURIER Online},
\href{http://www.zeit.de}{ZEIT ONLINE}, 
\href{http://www.badische-Zeitung.de}{Badische Zeitung},
\href{http://www.stuttgarter-zeitung.de}{Stuttgarter Zeitung},
\href{http://www.merkur.de}{Merkur},
\href{http://www.hna.de}{HNA},
\href{http://www.mopo.de}{Hamburger Morgenpost},
\href{http://www.mainpost.de}{Mainpost},
\href{http://www.tagesspiegel.de}{Tagesspiegel},
\href{http://www.swp.de}{Südwestpresse},
\href{http://www.augsburger-allgemeine.de}{Augsburger Allgemeine},
\href{http://www.abendzeitung-muenchen.de}{Abendzeitung Online},
\href{http://www.fr-online.de}{Frankfurter Rundschau}

% vim: set ai si et tw=80 sts=2 ts=2 sw=2:

\chapter{Nutzerbeteiligung im Journalismus: Kommentare als Nutzerbeteiligung}
\label{kap:nutzerbeteiligung}

\section{Vom Leser zum Nutzer zur Nutzerbeteiligung}

Mit der Digitalisierung ändert sich der Kommunikationsprozess. Der Leser kommt
weg von seiner Rolle als reiner Zuhörer. Er muss nicht mehr warten, bis ihm
etwas angeboten wird. Der Leser wird zum Nutzer der Medienangebote, denen er
sich auch mitteilen kann. Er kann sie nicht nur nutzen, sondern auch benutzen.
Er kann sich richtig in den Kommunikationsprozess einschalten und zwar direkt
und unmittelbar.

\begin{quote}
\glqq A great many other people also contribute content, representing their own
interests, ideas, observations and opinions. That content comes in a steadily
expanding volume and variety of forms and formats – words, images and sounds,
alone or in combination.\grqq{} \autocite[S.~1]{participatory}
\end{quote}

Im Englischen werden u.a. die Ausdrücke „user generated content“, „citizen
journalism“ oder „participatory journalism“ \autocite[S.~2]{participatory} dafür
verwendet. Sie alle beinhalten, dass Medienschaffende und Nutzer miteinander
kommunizieren und die Nutzer ihren Teil zur Bildung von Nachrichten und
Gemeinschaft dazu tun.

Ausdrucksformen dieser Nutzerbeteiligung sind vor allem Beiträge in
(Diskussions-)Foren und sozialen Netzwerken, Blogs, Berichte, Beurteilungen,
Bewertungen, Kommentare, hochgeladene Fotos und Videos und noch viele mehr.
„Indeed, new participatory formats appear all the time.“ \autocite[S.~2 und
S.~17]{participatory}


\section{Kommentare als Nutzerbeteiligung}

In dieser Arbeit stehen die Nutzerkommentare im Mittelpunkt. Als Kommentare
bezeichnet man „views on a story or other online item, which users typically
submit by filling in a form on the bottom of the item.“
\autocite[S.~17]{participatory}.  Sie markieren eine neue Stufe in der
Nutzerbeteiligung und sie sind überaus beliebt, was auch immer wieder in Studien
bestätigt wird \autocite[S.~97, siehe 3.1]{reich}.

Weil die Funktion bei Onlinezeitungen stark genutzt wird, kommt es zu einer Flut
von Material. Mit diesem Material haben die Redaktionen zu kämpfen. Sie sehen
zwar die Vorteile, die die Kommentarfunktion bringt. In der täglichen Arbeit
führt es jedoch immer wieder zu schlechten Erfahrungen.

Als nächstes wird aufgezeigt, was die Kommentare besonders macht, warum sie so
populär sind, was die positiven Aspekte sind und wie sie den
Kommunikationsprozess bereichern. Danach werden die Probleme und die Eingriffe,
das Kommentarmanagement, erörtert.

\begin{quote}
User-generated posts attached to a published item, typically an article or blog
entry, on a media website. Most news organizations moderate or screen user
comments, either before or after publication;
\autocite[S.~204, Glossar]{participatory}
\end{quote}

% vim: set ai si et tw=80 sts=2 ts=2 sw=2:

\chapter{Positive Aspekte der Kommentarfunktion}

Was macht Kommentare wertvoll und was bringen sie eigentlich? Dieser Frage wird
im folgenden Abschnitt nachgegangen.


\section{Beliebtheit der Kommentare} \label{sec:beliebtheit}
Obwohl nur ein kleiner Teil der Leser Kommentare verfasst und ein ebenso kleiner
Teil diese auch liest \autocite[96ff]{reich}, schlägt diese Art der
Nutzerbeteiligung wie eine Bombe ein. Das merken die Redaktionen, die von nicht
mehr zu bewältigenden Kommentaren erreicht werden. Dies ist dem Umstand
geschuldet, dass es auf einmal (2005 entstehen die ersten Kommentare) die
Möglichkeit gibt, rauszulassen, was einem gerade durch den Kopf geht beim Lesen
eines Artikels.

Es ist aber nicht nur die Gelegenheit spontan etwas zum eben Gelesenen zu sagen,
was die Kommentare attraktiv macht. Menschen haben selten die Chance, ihren
Unmut kund zu tun oder Zustimmung auszusprechen, weil sie auf das nur Zuhören
beschränkt sind \autocite[S.~99]{reich}. Diese auferlegte Passivität wird
aufgehoben.\\
Für manche Leser werden die Kommentare sogar genauso wichtig, wie die Nachricht
selbst. Und manchen Leser gefällt es einfach, sich selbst veröffentlicht zu
sehen.


\section{Öffentlicher Diskurs}
Eine ganz wichtige Rolle nehmen die Kommentare als ein Platz oder Ort ein, wo
sich jeder zum Diskutieren treffen und wo man auch die Breite und
Verschiedenheit der Ansichten erfahren kann. Für den demokratischen Prozess ist
es wichtig, dass bestimmte Themen besprochen werden. Jede Stimme soll gehört
werden, auch diejenigen, die sonst leicht untergehen, wie Minderheiten oder
Leute, die sich nicht trauen, oder weil es an Unterstützung fehlt oder weil die
Angelegenheit einfach kontrovers ist \autocite[S.~12]{santana:2014}. Außerdem
gibt es erst mal keine Beschränkungen, wer mit sprechen darf und wer nicht.  Zu
einer gesunden Demokratie gehört das einfach dazu: \glqq By allowing citizens to
participate, journalists behave ethically, and hence they democratize journalism
and the web.\grqq\- \autocite[S.~125]{singer}

Dieser neu geschaffene öffentliche Ort soll auch dazu dienen, herauszufinden,
wie die Leute im Moment \glqq ticken\grqq\- (\glqq serving as a gauge of society’s pulse\grqq\-
 \autocite[S.~181]{loke}. Man kann herauslesen, wo die neuralgischen Punkte in
der Gesellschaft liegen und auf diesen Zug ausspringen, wenn nötig.

Über die Kommentare können Einspruch erhoben oder Bedenken geäußert werden.
Ebenso funktionieren Kommentare als Anregung für weitere Diskussionen oder
greifen korrigierend  in die Redaktionen ein. Auf diesem Weg erhalten die
Journalisten auch direktes Feedback: \glqq Comments can confirm that the website is
doing a good job [\ldots] they can help improve accuracy [\ldots]\grqq\-
\autocite[S.~105]{reich}

Ufern die Diskussionen jedoch ins Unendliche aus oder werden aggressiv und
beleidigend, dann wird aus diesem liberalen Aspekt ein Problem der Kommentare.
\glqq[\ldots] free expression and exposure to differing views can hold deliberative
potential only when participants were respectful toward each other\grqq\-
\autocite[S.~7]{santana:2011}. Das nächste Kapitel wird sich damit beschäftigen.





\section{Leserbindung}
\begin{quote}
\glqq Newspaper websites compete in a marketplace where a rival news source is simply
a click away, so gaining and retaining the attention of readers is more
important than ever. [\ldots] It’s not just getting the eyes on your site
[\ldots] It’s getting them to stay on your site.\grqq\- \autocite[S.~144]{singer}
\end{quote}

Schaffen es die Online-Redaktionen, dass die Leser dabei bleiben, dann erreichen
sie ihre Ziele. Und die Kommentare sind dabei ein ganz wesentlicher Faktor. Die
Teilnahme am Kommentieren kann sich zu einem Wir-Gefühl entwickeln. Das steigert
wiederum die Besuche auf der Website, sowie das Vertrauen und die
Glaubwürdigkeit der Seite \autocite[S.~215]{meyer-carey}. So bekommen die
Medienhäuser den gewünschten Verkehr im Netz und große Klickzahlen. Die Leser
entwickeln eine gewisse Loyalität gegenüber ihrer Zeitung. Das Medienprodukt
wird zur (begehrten) Marke und zieht positive wirtschaftliche Aspekte mit sich.\\
Alle positiven Aspekte können sich durch eine große Masse an Kommentaren jedoch
auch umkehren, z.B. werden die Leser der Online-Zeitung den Rücken zukehren,
wenn sie von den Äußerungen der anderen regelrecht erschlagen werden.

\section{Neue oder zusätzliche Inhalte}
Die Kommentierenden begeistern sich für ihr Thema, lassen sich als Experten
erkennen, liefern Hinweise und sprechen ganz einfach aus, was ihnen wichtig ist.
Dort kann der Journalist ansetzen und solche Beiträge als \glqq journalistisches
Werkzeug\grqq{} \autocite[S.~133]{robinson} benutzen.\\
Kommentare dienen als Informationsquelle oder als Inspiration für Themen.
Manche Journalisten sehen sie als die Möglichkeit, Kontext herzustellen (durch
Hyperlinks) oder um bestimmte Punkte zu klären. Andere Perspektiven tun sich
auf und der Journalist wird sensibilisiert, was die Leser hören wollen
\autocite[S.~12]{santana:2014}.

Dafür müssen die Schreiber jedoch alle Kommentare durchforsten: entweder nur von
ihren eigenen Texten oder von der gesamten Redaktion. Es liegt nahe, dass dies
sehr schnell ausarten kann. \glqq If a journalist were to read all 300 comments, he
wouldn’t be able to do anything else [\ldots]\grqq\- \autocite[S.~84]{domingo}.  So
kommt es, dass nur wenige Journalisten Kommentare überhaupt lesen
\footnote{\glqq Only 35 percent of national journalists and 36 percent of local
  journalists said they have a positive view of citizens posting news content on
  news organizations' websites.  [\ldots] Even fewer said they take the time to
  read the comments and submissions they receive.\grqq\- \autocite[S.
216]{meyer-carey}}
und dass sie eine schlechte Meinung darüber entwickeln.\footnote{The study finds
  that the more comments a news website receives, the more negative attitudes
  the journalists who work at those sites have toward the value and utility of
  comments. \autocite[S.~214]{meyer-carey}}

\section{Secondary gatekeeper} \label{sec:secondary-gate}
Den Nutzern geht es nicht darum, hauptsächlich neue Inhalte zu schaffen. Sie
bedienen sich nur an der Nachrichtenauswahl und verändern dann die Gewichtung.
Sie suchen sich aus, was sie für gut finden und teilen dies anderen mit. Das
Material wird ein zweites Mal veröffentlicht und verteilt
\autocite[S.~66]{singer:2014}. Damit übernehmen die Nutzer neue Rollen und zwar
die des \glqq active redistributor\grqq, des \glqq active recipient\grqq\- und in einer gewissen
Weise die des \glqq gatekeepers\grqq\- \autocite[S.~57]{singer:2014}. Dadurch, dass ihnen
aber die Autonomie fehlt, was wirklich zur Veröffentlichung frei gegeben wird,
werden sie immer hinter dem eigentlichen \emph{gatekeeper}, dem Journalisten,
stehen. Auch weil sie sich an Regeln halten müssen (wie diese Regeln konkret
aussehen, wird in den folgenden Abschnitten beschrieben). Diese Regeln
bekräftigen den eigentlichen \emph{gatekeeper}, indem sie ihm eine neues \glqq gate\grqq\-
liefern \autocite[S.~13]{santana:2014}.\\
Der Journalist besitzt zu Recht aufgrund seiner Professionalität die
Entscheidungsgewalt. Aber der Nutzer wird zumindest zu einem \glqq secondary
gatekeeper\grqq\- \autocite[S.~5]{santana:2014}, der Inhalte verändern kann, indem er
sie hervorhebt, markiert, mit anderen teilt, weiter verschickt, zugänglich macht
\autocite[S.~57]{singer:2014}.\\
Es gibt im Internet unglaublich viele Nachrichten und Informationen. Diese sind
da, aber irgendwo \glqq da draußen\grqq. Deswegen ist das \glqq
Sichtbarmachen\grqq\- von Nachrichten umso wichtiger geworden.

% vim: set ai si et tw=80 sts=2 ts=2 sw=2:

\chapter{Probleme mit Kommentaren}

Die Nutzerbeteiligung „Kommentare“ führt allerdings nicht nur zu einer
Bereicherung innerhalb der Medien. So hat süddeutsche.de die Kommentarfunktion
auf ihren Internetseiten entfernt bzw. ausgelagert. Dabei entsteht der Eindruck,
dass die Nachteile die Vorteile überwiegen. Die Entscheidung ist in der
Redaktion sicherlich nicht ohne heftige Diskussion gefallen. Am Ende wurden doch
die nachteiligen Erscheinungen der Kommentare für schwerwiegender empfunden. Was
jedoch wiederum für die Kommentare spricht, ist, dass sie nicht ganz gelöscht,
sondern eben nur auf andere Plattformen verschoben wurden. Ausgewählte
Kommentare schaffen es sogar auf die Nachrichtenseiten. Außerdem kann den
Nachteilen mit entsprechenden Maßnahmen entgegen gewirkt werden. Wie diese
Maßnahmen aussehen, wird dann im darauffolgenden Kapitel erklärt

Was die Probleme genau sind, soll nun betrachtet werden.

\section{Kommentarmanagement ist mit Zeitaufwand verbunden}

Es wurde bereits erwähnt, dass die Kommentare Überhand nehmen und nicht mehr
bewältigt werden können. Manchmal schafft es der Autor nicht, die Kommentare zu
seinem eigenen Artikel zu lesen, geschweige denn, die seiner Mitstreiter. Es
werden in manchen Redaktionen externe Mitarbeiter angestellt, die sich
ausschließlich mit den Kommentaren beschäftigen. Das belastet die Redaktionen
zusätzlich: ``[\ldots] the overall volume of user input demanding to be filtered
or factchecked was a burden.'' \autocite[S.~172]{quandt}


\section{Schlechte Qualität der Kommentare und „incivilty“}

Es ist nicht nur die reine Menge der Kommentare, sondern auch deren mindere
Qualität, die bei den Journalisten schlecht ankommt und Arbeit verursacht. Die
Kommentare müssen gesichtet und sortiert werden. ``Very few of them make
intelligent comments or have intelligent things to say. It's very deceiving
[\ldots]'' \autocite[S.~ 103]{reich}. Und dabei schreiben und lesen sowieso nur
sehr wenige Nutzer Kommentare, wie in Kapitel \ref{kap:nutzerbeteiligung}
bemerkt!

Wenn die Beiträge einfach nur schlecht sind, dann hinterlässt das sicherlich
keinen guten Eindruck und die Qualität der Zeitung leidet insgesamt. Aber
Kommentare arten teilweise aus und es kommt zu beleidigenden oder rassistischen
Ausdrücken und irrationalen Argumenten. Die Kommentierenden erniedrigen und
schuldigen an \autocite[S.~103]{reich}.

Diese Verstöße gegen einen angemessenen Diskussionston (\glqq discursive
civility\grqq{} versus \glqq discursive incivility\grqq\footnote{Hwang defines
  ``discursive civility'' as arguing the justice of one's own view while
  admitting and respecting the justice of others' views. Conversely,
  ``discursive incivility'' is defined as expressing disagreement that denies
and disrespects the justice of others' views \autocite{hwang}
\autocite[S.~6/7]{santana:2014}} werden durch die Anonymität des Internets
gefördert. Die Hemmschwelle etwas zu sagen, das unter die Gürtellinie zielt,
sinkt ganz einfach, wenn man sich nicht zu erkennen geben muss. Andererseits ist
gerade diese Anonymität der Schlüsselfaktor, der zu mehr Beteiligung am Diskurs
führt, siehe Kapitel \ref{kap:nutzerbeteiligung}.

Selten wird nicht nur gegen den guten Ton verstoßen, sondern auch gegen das
Gesetz. Hassrede zum Beispiel ist ein bekanntes Problem und in Deutschland
verboten.

Es gibt bestimmte Themen, die sehr kontrovers sind und die Gemüter erhitzen,
weil verhärtete entgegengesetzte Meinungen aufeinandertreffen. ``Certain news
stories do not receive even a single comment, but those that do [\ldots] could
provide a peek into `the community’s heartbeat'.'' \autocite[S.~181]{loke} Die
Zeitungen beobachten Themengebiete, bei denen es immer wieder zu „discursive
incivility” der Kommentierenden kommt. Dazu gehören vor allem sämtliche
Angelegenheiten über Religion\footnote{Ein aktuelles Beispiel ist der {\slshape
shitstorm}, mit dem sich Christiane Florin auseindersetzen musste, weil sie eine
Anzeige in einem christlichen Blatt ablehnte. Es ging „nur“ um eine Anzeige und
die Journalistin hat die Absage auch begründet. Trotzdem erreichten sie
hasserfüllte Leserbriefe, die sie schließlich öffentlich gemacht hat
\url{http://www.christundwelt.de/detail/artikel/sie-kotzen-mich-an}} und Rasse,
Einwanderung und soziale Themen, obwohl gerade in diesen Bereichen ein
öffentlicher Diskurs und verschiedene Meinungen wichtig sind! Sie sind
notwendig, um sämtliche Aspekte abzubilden. Sonst greift auch hier die
Schweigespirale.\footnote{„[\ldots] would-be commenters might feel remiss to
register their opinion if they perceive themselves to hold a minority opinion
[\ldots]“ \autocite[S.~12]{santana:2014}}

Verschiedene Redaktionen reagieren darauf, indem sie zu Themen, mit denen sie
schlechte Erfahrungen gemacht haben, keine Kommentarfunktion mehr zu lassen oder
sie schließen, wenn die Reaktionen zu hitzig werden
\autocite[S.~4]{santana:2014}. Dem gegenüber gibt es Beobachtungen, wie die von
\textcite{loke}\footnote{``Loke found that online discussions of culturally
sensitive issues, such as race, have allowed newsreaders the opportunity to
amplify socially regressive views where they would not have previously. She
argues that despite the hateful dialogue that some sensitive topics attract, the
forums still serve a useful function in allowing the public, as well as
journalists, to get a glimpse into the consciousness of the community, however
unappealing.'' (Santana, 2014, \autocite[S.~12]{santana:2014}}, dass auch bei
kontroversen Themen die Leser aus den Diskussionen etwas über die Gegner
mitnehmen und ihren Blick auf die andere Seite öffnen. Loke tritt dafür ein,
dass alle Kommentare veröffentlicht werden, weil alle in gewisser Weise etwas
aussagen.

In Deutschland jedenfalls wünschen sich die Journalisten und Nutzer einen
angemessenen Umgangston, was sich in den Beschwerden beim deutschen Presserat
widerspiegelt. Aus diesem Grund werden Maßnahmen zur Regulierung von Kommentaren
ergriffen. Diese Maßnahmen werden im nächsten Punkt erläutert.

% vim: set ai si et tw=80 sts=2 ts=2 sw=2:

\chapter{Kommentarmanagement}

\section{Was beinhaltet das Kommentarmanagement?}
Im vorherigen Kapitel wurde dargestellt, warum es irgendeiner Form bedarf, die
Kommentare zu regeln. Es kommt eben vor, dass sich Kommentierende nicht an einen
angemessenen Diskussionston halten. Diese Nutzer schaden sowohl den anderen
Nutzern, als auch der Zeitung. Die Nutzerschaft will ja eine angenehme Umgebung
vorfinden \autocite[S.~217]{meyer-carey} und alle wollen \glqq gute\grqq{}
Kommentare lesen, obwohl insgesamt das Diskussionsniveau im Internet gesunken
ist.\footnote{"The Internet has lowered the discourse in general – the brevity,
the speed, this sense of ‘why should I make an effort?' [\ldots] It is
problematic for the society, problematic for the democracy, problematic in every
sense." \autocite[S.~130]{singer}} Die Nutzer sollten aus diesem Grund auf
{\bfseries Empfehlungen oder Richtlinien zum Kommentieren} hingewiesen werden
(\glqq Netiquette\grqq).


\begin{quote}
"[\ldots] a `good' comment entailed one that obeyed the policy rules (i.e. no
ranting or trash talk), stayed on topic (introduced and framed by the news
article), praised the original journalism, confirmed the facts in that story,
and informed other readers by adding new information."
\autocite[S.~134]{robinson}
\end{quote}

Im schlimmsten Fall kann es sogar zu Gesetzesverstößen kommen. Es gibt noch
andere Gründe, die zu einer Reglementierung von Kommentaren führen. Es können
sowohl strategische Faktoren sein, als auch das politische Klima, oder die
Themen selbst. Natürlich spielt auch das journalistische Verständnis und die
Vorgaben innerhalb der Redaktion eine Rolle \autocite[S.~106f]{reich}. Es geht
jedoch immer um die grundsätzliche Frage, ob ein Kommentar \glqq rein\grqq{}
darf oder ob er besser \glqq draußen\grqq{}  bleibt.

Diese kategorische Entscheidung hat natürlich eine negative Komponente.
Kommentare werden so zuerst nach einer Regelverletzung bewertet und nicht nach
ihrem eigentlichen Wert. Außerdem geht diese Art der Bewertung weg von
journalistischen Gesichtspunkten hin zu wirtschaftlichen, wie zum Beispiel das
Erreichen großer Klickzahlen.

Im folgenden Abschnitt werden die Möglichkeiten des Kommentarmanagements
skizziert. Es ist vorab noch zu bemerken, dass jede Zeitung irgendeine Form des
Kommentarmanagements macht und dass die meisten Kommentare zugelassen werden.
Die geschätzten Angaben der interviewten Journalisten aus aller Welt in
\textcite[S.~106]{singer} reichen von 40 bis 90 Prozent der frei gegebenen
Kommentare.

Die zwei Säulen des Kommentarmanagements sind Moderieren und Registrieren. Es
gibt jeweils verschieden Arten der Moderation und verschiedene Möglichkeiten der
Registrierung. Zusätzlich gibt es immer wieder neue Funktionen, die ganz neue
Möglichkeiten schaffen.  Die Nutzer wollen ja vor allem Inhalte teilen und
sichtbar machen (siehe Abschnitt~\ref{sec:secondary-gate}).
Dazu braucht es die technischen Voraussetzungen.

Eine {\bfseries Moderation} ist vor oder nach einer Freischaltung möglich (\glqq
pre-mo\-de\-ra\-tion\grqq{} vs. \glqq post-moderation\grqq) und wird entweder vom
Journalisten durchgeführt oder jemand anderem des Medienbetriebs. Sie kann ganz
ausgegliedert werden. Oder der Nutzer übernimmt die Moderation, in manchen
Redaktionen zusammen mit dem Journalisten, als \glqq super-user\grqq{}
\autocite[S.~112]{reich}. Diese Zusammenarbeit (\glqq collaborative
moderation\grqq{} \autocite[S.~109]{reich} geht in Richtung einer
Selbstregulierung von Kommentaren, was als erstrebenswert angesehen werden
kann.\footnote{``We need to tend towards this self-regulation of users by other
users because it is the more logical thing to do. It is them, in the end, who
know what do they want, what information is more useful and what is less, and
what bothers them.'' \autocite[S.~112]{reich}} Auch hier kommt der Nutzer als
\emph{secondary gatekeeper} zum Einsatz.

Derjenige, der die Moderation übernimmt, ist der \glqq comment moderator\grqq{}
\autocite[S.~68]{paulussen}. Er sichtet und sortiert aus. Wichtig für einen
{\slshape comment moderator} ist eine klare Vorgabe, was erlaubt ist und was
nicht. An diese {\bfseries Richtlinien} sollen sich einheitlich die Moderatoren
halten, denn jeder hat unterschiedliche Ansichten über eine mögliche
Grenzziehung.\footnote{Es gibt nicht nur unterschiedliche Ansichten zu der
Qualität, sondern zu Kommentaren überhaupt. \textcite{robinson} unterscheidet
zwischen \glqq traditionalist\grqq{} und \glqq converger\grqq{}. Die ersten
sind eher älter und länger im Unternehmen und sehen sich als Autorität. Sie
wollen eine gewisse Verantwortung des Medienbetriebs auch online aufrecht
erhalten. Die anderen sind eher jünger und kürzer dabei, legen weniger Wert
auf Registrierung und wollen vor allem mit den Nutzern interagieren.} Es gibt
aber auch Überlegungen, ob jeder Journalist für seinen Beitrag selbst die
Kommentierregeln bestimmt \autocite[S.~127]{singer}.

\begin{itemize}
  \item[-] {\bfseries Prä-Moderation}
    bedeutet, dass alles gesichtet wird, bevor es online geht (mit Hilfe
    entsprechender Software). Das entspricht einem klassischen journalistischen
    Verständnis, ist aber mit hohen finanziellen und zeitlichen Kosten
    verbunden.  Diese \glqq proaktive\grqq{} Herangehensweise
    \autocite[S.~108]{reich} kann unter Umständen die Diskussion verzerren. Die
    Journalisten sind damit jedoch auf der sicheren (legalen) Seite. Denn
    unangemessene Kommentare treten auch bei Artikeln auf, von denen es man
    nicht gedacht hätte. Obendrein können \glqq Trolle\grqq{} die Moderation
    stören oder umgehen. Für die Nutzer ist eher demotivierend, denn die meisten
    wollen ihren Beitrag veröffentlicht sehen.\footnote{``You can't really beat
    hitting 'submit' and seeing your comment there before you go away. It
    encourages you to come back. You feel you've engaged.''
    \autocite[S.~109]{reich}}

  \item[-] {\bfseries Post-Moderation}
    findet statt, nachdem der Kommentar bereits veröffentlich wurde, und es
    irgendeinen Grund gibt, einzugreifen. Dieser Umgang mit Kommentaren ist viel
    offener und entspannter und in diesem Fall \glqq reaktiv\grqq. Einer
    Post-Moderation geht jedoch fast immer eine Registrierung voraus. Bei
    heiklen Themen (siehe Abschnitt~\ref{sec:schlecht})
    wird die Post-Moderation allerdings problematisch. Hier kann mit einem
    Umschwenken auf Prä-Moderation dagegen gesteuert werden.
\end{itemize}

Eine {\bfseries Registrierung} ist meistens die Voraussetzung, um kommentieren
zu können (vor allem bei Post-Moderation). Dabei muss man seine persönlichen
Daten (be\-stä\-tig\-te Emailadresse und Klarnamen) angeben, um einen Zugang zu
erhalten. Das verhindert natürlich, dass die Nutzer unkontrolliert drauf los
schreiben. Aber es verhindert ebenso manchen Kommentar, der eigentlich
geschrieben werden möchte. Registrierungen können unterschiedlich gehandhabt
werden. Oft genügt die Angabe einer Emailadresse. Und auch wenn mehr als das
verlangt wird, also die Angabe von Namen, dann kann die tatsächliche Identität
einer Person damit nicht bestätigt werden. Der Name kann erfunden sein (auch bei
Klarnamenpflicht) oder die Nutzer verwenden Spitz- oder Fantasienamen. Ebenfalls
besteht die Befürchtung, dass Kommentare allein deswegen nicht geschrieben
werden, weil man mehr als die Emailadresse angeben muss.  {\bfseries Klarnamen}
schaffen zwar Vertrauen (z. B. die wertvollste Rezension bei Amazon stammt
meistens von Nutzern mit Angabe von vollständigen Namen), machen aber auch Angst
(z.B. vor Ärger, den die Meinungsäußerung bei Nachbarn oder beim Chef
möglicherweise mit sich bringt). Mittlerweile sind die Nutzer jedoch daran
gewöhnt, ihre Meinung zu sagen und ihren Namen zu nennen, den jeder sehen kann,
oft sogar mit einem Foto verbunden (wie bei Facebook oder Twitter).


Es gibt {\bfseries zusätzliche  Funktionen} für das Management von Kommentaren.

\begin{itemize}

  \item[-] Dazu gehört zum Beispiel der  \glqq{\bfseries report abuse button
    (Melden-Button)}\grqq{} \autocite[S.~110f]{reich}. Andere Nutzer können auf
    diese Weise melden, wenn sie unangebrachte Kommentare lesen. In der
    aktuellen Untersuchung von \textcite[S.~63]{singer:2014} verwenden drei
    viertel der Online-Zeitungen diese Funktion.

  \item[-] Nutzer, die bereits negativ aufgefallen sind, weil sie das System
    ausnutzen, werden markiert. Deren Kommentare müssen dann vorher kontrolliert
    oder ganz gestoppt werden, entweder endgültig oder nur für eine bestimmte
    Zeit.

  \item[-] Man bietet eine  {\bfseries freiwillige Registrierung} an, und wer
    das tut, bekommt zusätzliche Privilegien (z.B. mehr Platz zum Kommentieren,
    Kommentare ohne Moderation, Möglichkeit für oder gegen andere Kommentare
    abzustimmen).

  \item[-] Die Option des {\bfseries \glqq Bewerten\grqq} ist überhaupt eine
    gute Lösung, die besten Kommentare zu filtern und die Selbstregulierung zu
    unterstützen. Nutzer können so Kommentare oder andere Kommentatoren bewerten
    und/oder weiter empfehlen \autocite[S.~63f]{singer:2014}.

  \item[-] Hilfreich sind dabei die {\bfseries neuen technischen Möglichkeiten}
    des Kommentarmanagements. Damit können automatisch Nutzerprofile erstellt
    und Kommentare in soziale Netzwerke gestellt oder auf diese verwiesen
    werden. Die Beliebtheit bei anderen Nutzern kann dargestellt werden.
    {\slshape Social Bookmarking} wird möglich. Man kann Artikel {\bfseries
    weiter schicken}. Welche dieser Optionen nutzen die Online-Zeitungen?
    Benutzen die Leser diese {\slshape features}? Erfinden die Redaktionen neue?

   % \fxnote*[author=VP]{Absatz Teil vom letzen Punkt oder besser außerhalb der
    %Auflistung?}{%
    

\section{Umsetzung der Dokumentenanalyse}
    
    Gerade wurde beschrieben, wie ein Kommentarmanagement ablaufen
    kann. Daraus ergeben sich auch die Kategorien (siehe oben fettgedruckt),
    die das Kommentarmanagement ausmachen.
   Der nächste Schritt ist die tatsächliche Analyse des Kommentarmanagements im 
   Internet. 
    
    Dazu wurden die entsprechenden Nachrichtenportale aufgerufen und die
    Kommentarfunktionen aufgesucht. Ich habe mich mit der Online-Zeitung vertraut
    gemacht und mit der Funktion, Kommentare zu schreiben. 
    Dann habe ich versucht, Antworten auf die vorgefertigten Kategorien zu finden,
    entweder durch Beobachten (z.B. Welche Funktionen gibt es rund um den Bereich,
    wo man Kommentare verfassen kann?) oder Ausprobieren (z.B. sich selber
    Registrieren, sich in der Community umschauen) oder Suchen.
    
    Fragen nach Moderation und inhaltlichen Richtlinien konnte ich ausschließlich 
    über die Allgemeinen Geschäftsbedingungen, Nutzungsbedingungen, Richtlinien,
    und/oder Netiquette herausfinden. Diese sind oft bei der Registrierung hervorgehoben,
    und/oder direkt verlinkt. Teilweise muss man den Bedingungen auch zustimmen bei einer
    Registrierung. Einige Online-Zeitungen haben ihre Bedingungen aber auch im 
    \glqq Kleingedruckten\grqq\- stehen.
    
    Ich habe die Informationen benutzt, die von den Nachrichtenportalen veröffentlicht
    worden sind. Bei der Kategorie \glqq Moderation\grqq\- wäre es eigentlich noch besser, 
    die Redaktionen direkt nach ihrer Vorgehensweise zu befragen. Ich finde die 
    verfügbaren Antworten beim Thema Moderation ungenügend,
    gerade was die \glqq stichprobenartige\grqq\- Moderation betrifft: Was heißt stichprobenartig genau?
    Wie viele Stichproben werden gemacht? Sitzt da jetzt jemand, ein Moderator, oder macht übernimmt  
    ein Computer die Aufgabe?
    
    Bei der Beschäftigung mit dem Kommentarmanagement auf den entsprechenden Internetseiten
    bin ich auch auf Funktionen gestoßen, die nicht bei Punkt 5.1
    besprochen worden sind, z.B.  \glqq alternative Anmeldung\grqq\- anstelle von Registrierung.
    Dies ist im Sinne der Dokumentenanalyse, die offen für neue Kategorien ist.
    
    Nachdem die ganzen Informationen zum Kommentarmanagement gesammelt waren, 
    habe ich Tabellen für die Kategorien angelegt und die Ergebnisse eingetragen.
    Während des Eintragens und auch danach habe ich versucht, die Ergebnisse zu 
    vereinheitlichen, damit ein Vergleich und Überblick der Online-Zeitungen möglich ist. 
    
    Zu Beginn der Arbeit war es ein Ziel, alle Kategorien in einer Tabelle darzustellen.
    Dies ist jedoch durch den begrenzten Platz nicht möglich. Es sind also mehrere
    Tabellen entstanden. Am Schluss gibt es eine gekürzte Zusammenfassung von ausgewählten
    Kategorien.


    
   % Die Ergebnisse werden tabellarisch wieder gegeben und entsprechend
    %kommentiert.
    
    
    
    
    
    

\end{itemize}


% vim: set ai si et tw=80 sts=2 ts=2 sw=2:

\chapter{Tabellarische Darstellung der Kategorien}
%1. Art der Moderation
%2. Registrierung
%3. Inhaltliche Regeln
%4. Zusammenfassung

\section{Einführung und Allgemeines zu Kommentaren}

Beim Artikel weisen Sprechblasen in der Regel darauf hin, dass kommentiert wurde
(mit Angabe der Anzahl der abgegebenen Kommentare) oder ein Kommentar
geschrieben werden kann.  Beim Kommentar selbst ist eine Zeitangabe üblich
entweder mit Datum und Uhrzeit oder \glqq vor \ldots Stunden\grqq.  Kommentare
stehen unter dem Artikel in vielen Fällen nach Werbung (in eigener Sache).

Beobachtung bei den Nachrichtenportalen:\\
Auch auf formaler Ebene werden Kommentare beschränkt. Es gibt ganz
unterschiedliche Handhabungen. Manche Zeitungen bieten Kommentierzeiten an. Nur
dann gibt es die Möglichkeit zu schreiben. Außerhalb dieser Zeiten ist es nicht
möglich. Andere Zeitungen schließen die Kommentarfunktion nach einer bestimmten
Zeit. Dann können die verfassten Kommentare zwar noch gelesen werden, aber keine
neuen mehr geschrieben. Bei manchen Zeitungen ist nicht jeder Artikel
kommentierbar. Es gibt Zeitungen, die nur volljährigen Nutzern einen Zugang zum
Kommentieren erlauben.  Die Sprechblase, die üblicherweise dafür verwendet wird
Kommentare anzukündigen oder auf bereits geschriebenen Kommentare hinzuweisen,
wird nicht immer verwendet.

\begin{landscape} \footnotesize
\begin{longtable}{lp{170mm}}

  \caption{Allgemeines zur Kommentarfunktion auf Nachrichtenportalen} \\ \\
  \toprule
  \bfseries Portal & \bfseries Kommentare \\\midrule[\heavyrulewidth]
  \endfirsthead

  \toprule
  \bfseries Portal & \bfseries Kommentare \\\midrule[\heavyrulewidth]
  \endfirsthead

  \bottomrule
  \multicolumn{2}{r}{\emph{Fortsetzung auf der nächsten Seite}}
  \endfoot

  \bottomrule
  \endlastfoot

%Spalte 1		 % Spalte 1
bild.de			& K. nicht zu allen Artikeln möglich \\\hline
%Spalte 2		 % Spalte 2
spiegel.de		  	& keine Sprechblase sondern Hinweis auf [Forum]; beim Beitrag: Sprechblase mit Ausrufungszeichen; K. durchnummeriert \\\hline
%Spalte 3		 % Spalte 3
faz.net			& zeitliche Begrenzung um K. zu schreiben; Hinweis, dass die K. im Internet recherchierbar sind\\\hline
%Spalte 4		 % Spalte 4
focus.de			& auch Videos kommentierbar \\\hline
% Spalte 5		 % Spalte 5
diewelt.de			& über Disqus verwaltet; Schließung nach zwei/drei Tagen oder früher bei Regelverstößen \\\hline
% Spalte 6		 % Spalte 6
derwesten.de		& kein Link auf K., keine Sprechblasen; alle Beiträge kommentierbar; K. durchnummeriert \\\hline
% Spalte 9		 % Spalte 9
rp-online.de		& fast jedes Thema kommentierbar \\\hline
% Spalte 10		 % Spalte 10
handelsblatt.de		& Kommentierzeiten: 7.30 - 21 Uhr, bis zu sieben Tage lang; K. werden (u.U. gekürzt) multimedial verbreitet \\\hline
% Spalte 11		 % Spalte 11
suedkurier.de		& K. nicht zu allen Artikeln möglich \\\hline
% Spalte 12		 % Spalte 12
zeit.de			& keine Sprechblase aber [Anzahl + Kommentare]; K. werden durchnummeriert;  K. nicht zu allen Artikeln möglich; Definition von K.: ``kürzere Textbeiträge, die Sie unter vorhandenen Artikeln, Videos, Fotostrecken oder anderen Multimedia-Inhalten abgeben können'' \\\hline
% Spalte 13		 % Spalte 13: TODO leer?
badische-zeitung.de	& \\\hline
% Spalte 14		 % Spalte 14
stuttgarter-zeitung.de	& für alle Nutzer, K. geben nicht die Meinung der Stuttgarter Zeitung wieder\\\hline
% Spalte 15		 % Spalte 15
merkur.de			& über Disqus verwaltet, für alle Nutzer\\\hline
% Spalte 16		% Spalte 16
hna.de			& über Disqus verwaltet, für alle Nutzer; freie Meinungsäußerung; bei Verstößen Aufruf eine Email zu schreiben \\\hline
% Spalte 17		 % Spalte 17
mopo.de			& über Disqus verwaltet; Kommentierzeiten: 8 - 21 Uhr\\\hline
% Spalte 18		% Spalte 18
mainpost.de		& K. unter Artikel und neben Artikel (als neuer Tab!); begrenzter Zugang\footnote{Das heißt, dass nur eine bestimmte Zahl an Artikeln im Monat kostenlos lesbar sind.}\\\hline
% Spalte 19		 % Spalte 19
tagesspiegel.de		& keine Sprechblase aber [Anzahl + Kommentare]\\\hline
% Spalte 20		% Spalte 20
swp.de			& kein Link auf K.; begrenzter Zugang\\\hline
%Spalte 21		
augsburger-allgemeine.de	& begrenzter Zugang; K. durchnummeriert

\end{longtable}
\end{landscape}

% vim: set ai si et tw=80 sts=2 ts=2 sw=2:

\chapter{Tabellen}
1. Art der Moderation
2. Registrierung
3. Inhaltliche Regeln
4. Zusammenfassung


\section{Registrierung auf Nachrichtenportalen}

Kommentare können immer gelesen werden, um selbst welche zu schreiben bedarf es
jedoch einer Anmeldung, darin sind sich alle Anbieter einig. Dies ist die erste
Maßnahme, um Missbrauch bei der Kommentarfunktion zu vermeiden. Man kann nicht
einfach drauf los schreiben, sondern muss den Prozess einer Registrierung durch
gehen. In der Regel bieten die Online-Zeitungen eine eigene Registrierung an.
Man ist dann auf dem Portal angemeldet und sieht es auch (außer Hamburger
Morgenpost und Südwestpresse). Manche Zeitungen ermöglichen auch eine Anmeldung
über Dritte Anbieter. Das ist dann eine Kompromisslösung für die Nutzer. Eine
Anmeldefunktion wird nämlich auch als Hemmschwelle zum Schreiben diskutiert.
Kann man sich über ein Konto anmelden, das man eh schon hat, dann gilt dieses
Argument nicht mehr. Der Trend, den Singer (2014, S. 69)
beschreibt\footnote{\glqq Some newspapers also had begun requiring users to
comment through Facebook, an intriguing step that removes many of the problems
created by anonymous postings while also helping generate social network
traffic. Such trends richly deserve the attention that journalism scholars have
begun to afford them.\grqq} bestätigt sich also.

Durch eine Anmeldung wird sicher gestellt, dass die Person, die einen Kommentar
verfasst, existiert und gegebenenfalls zur Verantwortung gezogen werden kann.
Natürlich ist auch hier Missbrauch nicht ausgeschlossen. Der Nutzer wird auf
wahrheitsgemäße Angaben hin gewiesen. 

Interessant ist, welche Angaben die Zeitungen zur Anmeldung fordern. Unabdingbar
sind die Eingabe einer Emailadresse sowie ein Passwort. Dann kommt es wieder zu
unterschiedlichen Handhabungen und unterschiedlichen Lösungen. Manche Zeitungen
verlangen die Nennung von Klarnamen und Benutzernamen, manche nur Klarnamen,
manche nur Benutzernamen. Einige Portale geben zusätzliche Hinweise zur
Registrierung. 


\begin{landscape}\footnotesize
  \begin{longtable}{l|p{28mm}p{20mm}p{20mm}p{90mm}}
  \caption{Die Registrierung: Die Angabe einer Emailadresse und eines Passworts wird immer verlangt. In der Tabelle sind weitere obligatorische Angaben aufgeführt.} \\

\bfseries Portal & \bfseries alternative \mbox{Anmeldung} & \bfseries Klarname & \bfseries Benutzer\-name & \bfseries Sonstiges \\\hline
\endfirsthead

\bfseries Portal & \bfseries alternative \mbox{Anmeldung} & \bfseries Klarname & \bfseries Benutzer\-name & \bfseries Sonstiges \\ \hline
\endhead

\hline \multicolumn{5}{r}{\emph{Fortsetzung auf der nächsten Seite}}
\endfoot

\hline
\endlastfoot

  bild.de &
    mypass, Facebook & ja & ja &
    Volljährigkeit bzw. Einverständnis der Erziehungsberechtigten bei
    Minderjährigen \\\hline

  spiegel.de & % Spalte 2 bei ''mein spiegel'' als Abonnent oder Nicht-Abonnent
    Facebook & ja & ja\footnote{Es wird darauf hingewiesen, dass der Benutzername angezeigt wird.\label{foot:angezeigt}} &
    \\\hline

  faz.net & % Spalte 3 bei ''Mein FAZ.NET'' mit
    keine & ja & nein &
    \\\hline

  focus.de & % Spalte 4 bei FOCUS online
    Facebook & ja\footnote{Es wird darauf hingewiesen, dass der Name im gesamten Internet recherchierbar ist.} & nein &
    \\\hline

  diewelt.de & % Spalte 5 bei WELT DIGITAL über
    mypass, Disqus, Facebook, Twitter, Google & ja & ja &
    als Gast schreiben \\\hline

  derwesten.de & % Spalte 6 auf derwesten.de mit
    keine & ja & ja\footref{foot:angezeigt} &
    jeder ist zugangs- und teilnahmeberechtigt \\\hline

  rp-online.de & % Spalte 9 bei mein RP ONLINE mit
    keine & nein & ja &
    auch Minderjährige, wenn sie sich über Nutzung bewusst sind bzw. mit
    Einverständnis der Erziehungsberechtigten \\\hline

  handelsblatt.de & % Spalte 10 auf handelsblatt.de mit
    keine & ja & nein &
    Volljährigkeit \\\hline

  suedkurier.de & % Spalte 11auf suedkurier.de mit
    keine & ja & ja &
    Anrede, Land, Adresse (Angaben werden auf Richtigkeit geprüft (teilweise
    auch telefonisch); Einverständnis nach sechs Monaten Inaktivität wird
    Registrierung/Benutzername gesperrt/freigegeben \\\hline

  zeit.de & % Spalte 12 auf zeit.de mit
    keine & nein & ja &
    \\\hline

  badische-zeitung.de & % Spalte 13 bei Meine BZ mit
    keine & ja & nein &
    Anrede, Löschen des Accounts bei Wegwerf-Emailadresse
    \\\hline

  stuttgarter-zeitung.de & % Spalte 14 auf stuttgarterzeitung.de
    Facebook & ja & nein &
    \\\hline

  merkur.de & % Spalte 15 bei Merkur-Online
    Disqus, Facebook, Twitter, Google & nein & ja\footnote{Ein Benutzername ist erwünscht und soll angemessen sein.} &
    als Gast schreiben \\\hline

  hna.de & % Spalte 16 mit HNA-Login
    Disqus & nein & ja\footnote{Der Benutzername soll angemessen und nicht beleidigend/anstößig sein.} &
    \\\hline

  mopo.de & % Spalte 17
    Disqus, Facebook, Twitter, Google & nein & nein &
    \\\hline

  mainpost.de & % Spalte 18 auf Mainpost.de
    keine & ja & ja\footnote{Der Benutzername soll nicht beleidigend, ehrverletzend, hetzerisch sein.} &
    Adresse \\\hline

  tagesspiegel.de & % Spalte 19 auf tagesspiegel.de
    keine & nein & ja\footnote{Der Benutzername soll angemessen sein.} &
    \\\hline

  swp.de & % Spalte 20
    keine & ja & ja &
  augsburger-allgemeine.de &%spalte 21
  & ja &ja & Wohnort 
    \\ \hline

\end{longtable}
\end{landscape}

% vim: set ai si et tw=80 sts=2 ts=2 sw=2:

\paragraph{Kommentar: inhaltliche Regeln und Hinweise}

Wie sie sich die Diskussion auf ihren Plattformen vorstellen und wie nicht,
beschreiben die Online-Zeitungen in der Netiquette, den AGB oder den
Nutzungsbedingungen. Auf manchen Portalen muss der Nutzer bei der Registrierung
dazu aktiv zustimmen und damit bestätigen, die Regeln zumindest zu akzeptieren.

Für die Regeln verwenden die Redaktionen eigene Formulierungen mit
unterschiedlichem Umfang. Es geht erst einmal darum, zu erklären, was man unter
einem \glqq guten Ton\grqq\ versteht. Dann wird erläutert, was nicht toleriert
wird und wann und wo ein Eingreifen seitens der Moderation erfolgt.

Gerade für Foren ohne Moderation sind inhaltliche Vorgaben besonders wichtig. Da
es niemanden gibt, der die Beiträge vorab oder danach liest und gegebenenfalls
einschreitet, müssen die Nutzer darüber informiert werden, was erlaubt ist und
was nicht. Außerdem werden sie zur Selbstregulierung aufgefordert - zum Beispiel
über die Möglichkeit Verstöße zu melden - und brauchen Handlungsvorgaben.

Es gibt inhaltliche Regeln zu den Kommentaren, die auf allen Plattformen zu
finden sind (Ausnahmen werden entsprechend gekennzeichnet).  Das sind zunächst
rechtliche Hinweise. Der Nutzer bestätigt der Urheber seiner Beiträge zu sein
oder die Urheberrechte zu besitzen, d.h. er muss Gewähr leisten, dass fremde
Inhalte zur Verbreitung freigegeben sind. Er trägt somit die volle Verantwortung
für die eingestellten Beiträge und stellt sicher, dass keine Rechte Dritter oder
Urheberrechte oder Persönlichkeitsrechte oder sonstige Rechte verletzt werden.
Außerdem gilt die Rechteeinräumung. Der Nutzer stimmt damit zu, dass die
entsprechende Zeitung sein Beiträge \glqq benutzen\grqq\ kann (z.B. vervielfältigen,
modifizieren, anpassen, übersetzten, bearbeiten, verbreiten, verwerten,
hervorheben, bewerten, archivieren, usw.).

Es gibt infolgedessen auch einen Haftungsausschluss des Anbieteres. Dieser
haftet nicht für den Inhalt von Nutzerbeiträgen. Dasselbe gilt für die Inhalte
fremder Seiten durch Verlinkung. Entsteht ein Schaden haftet der Nutzer.

Was auf keinen Fall geduldet wird ist Werbung in irgendeiner Form.
Diskussionsforen dürfen nicht für kommerzielle Zwecke missbraucht werden.
Diskussionsforen sind keine Werbefläche für Webseiten oder Dienste (Spamming).

Auch dem Datenschutz müssen die Nutzer zustimmen. Damit ist z.B. das Über\-prü\-fen
der Emails oder Abfragen auf Viren gemeint oder das Einhalten von gesetzlichen,
behördlichen und technischen Vorschriften. Das Passwort soll geheim gehalten und
Vertraulichkeit gewahrt werden.

Jedes Portal weist ausdrücklich darauf hin, dass Kommentare {\bfseries
themenbezogen} sein sollen.  Es gibt Inhalte, die kategorisch bei allen
Zeitungen nicht erlaubt sind. Dazu gehören Beleidigungen und Beiträge mit
se\-xi\-sti\-schen/sit\-ten\-wi\-dri\-gen/por\-no\-gra\-phi\-schen/obs\-zö\-nen/grob
anstößigen Inhalten und Rassismus. (Diese eben genannten inhaltlichen Verbote
werden in der Tabelle nicht mehr aufgeführt, weil sie in allen Online-Zeitungen
genannt werden.)

Auch das Verbot von Diskriminierungen zählen die meisten Nachrichtenportale in
irgendeiner Form auf.

Die Redaktionen wählen unterschiedliche Formulierungen, um unerwünschte Beiträge
zu beschreiben. Viele Formulierungen sind sich im Inhalt ähnlich  und
unterscheiden sich um Nuancen (z.B. Beleidigung und Beschimpfung). Trotzdem
werden sie hier bewusst aufgelistet und nicht zusammengefasst, um deutlich zu
machen, was die Zeitungen extra erwähnt haben wollen und was nicht.

\begin{landscape}\footnotesize
\begin{longtable}{lp{170mm}}
\caption{Inhaltliche Regeln auf Nachrichtenportalen}\\
\bfseries Portal & \multicolumn{1}{c}{\textbf{Inhaltliche Regeln}}\\ \hline
\endfirsthead
\bfseries Portal & \multicolumn{1}{c}{\textbf{Inhaltliche Regeln}}\\ \hline
\endhead
\hline \multicolumn{2}{r}{\emph{Forsetzung auf der nächsten Seite}}
\endfoot
\hline
\endlastfoot


bild.de & Nutzungsbedingungen: allgemeine und besondere (Zustimmung verlangt
	bei Registrierung); Netiquette.

	sachlich, höflich bleiben, andere respektieren, nicht dagegen
	argumentieren, Angriffe versuchen zu ignorieren; wie man selbst
	behandelt werden möchte, keine unangemessenen Beiträge wie
	Beschimpfungen/Belästigungen/Drohungen/Diskriminierungen, keine Beiträge
	mit nicht-themenbezogen/antisemitische Inhalten; keine privaten
	Angaben\footnote{Angaben von Postadresse und/oder Telefonnummer und/oder
	Emailadresse oder Angaben über Dritte verbreiten; keine automatisierte
	Nutzung; kein Mobbing; keine Links zu Werbung/Chats/Foren; Datensicherung vom
	Nutzer selbst; keine Trolle; kein Spam}\tabularnewline\hline

spiegel.de & Nutzungsbedingungen: allgemeine und für Foren (Zustimmung)

	faire, sachliche, angenehme, offene, freundschaftliche, respektvolle
	Diskussion (auch bei Streit), obwohl es sich um verbale
	Auseinandersetzung handeln soll; keine Beiträge mit
	strafbaren/inakzeptablen/nicht-themenbezogenen Inhalten;\tabularnewline\hline

faz.net & Nutzungsbedingungen: allgemein (Zustimmung); ``wie Sie mit
	diskutieren''-Button

	keine Beiträge mit links- und
	rechtsradikalen/verleumderischen/ruf-/geschäftsschädigenden Inhalten;
	keine falschen/nicht nachprüfbaren Behauptungen; keine
	Hyperlinks\tabularnewline\hline

focus.de & AGB (Zustimmung), Netiquette (Zustimmung)
	sachliche, freundliche,
	respektvolle, tolerante Diskussion, keine Fremdtexte; keine privaten
	Angaben (Adresse, Telefonnummer); keine Links zu Werbung; keine
	Diskriminierungen jeder Art\footnote{Diskriminierungen aufgrund von
	Herkunft, Nationalität, Religion, sexueller Orientierung, Alter,
	Geschlecht, usw.}; keine Beiträge mit  demagogischen Inhalten; keine
	Schadsoftware verwenden, kein Missbrauch von Daten; Datensicherung vom
	Nutzer selbst;\tabularnewline\hline

diewelt.de & Nutzungsbedingungen, veraltete Netiquette

	fair, höflich, verständlich, kritische Kommentare erwünscht, keine
	Be\-schim\-pfung\-en/Dis\-kri\-mi\-nie\-run\-gen/Pro\-vo\-ka\-tio\-nen/Ent\-wür\-di\-gung\-en/Auf\-ruf zu
	Demonstrationen oder Gewalt; Gesetze beachten; keine Angaben über Dritte
	verbreiten; keine Hasspropaganda; kein Hinweis auf
	Haftungsausschluss\tabularnewline\hline

derwesten.de & Nutzungsbedingungen (Zustimmung), Netiquette

	keine Beschimpfungen/Kränkungen; das Forum ist kein
	Ver\-an\-stal\-tungs\-ka\-len\-der/keine Ter\-min\-an\-kün\-di\-gun\-gen; Zitate müssen
	Quellenangabe enthalten; keine Beiträge mit
	ge\-walt\-ver\-herr\-lich\-en\-den/an\-ti\-se\-mi\-ti\-schen/ge\-set\-zes\-wi\-dri\-gen
	Inhalten\tabularnewline\hline

% Spalte 9
rp-online & AGB

	keine Beiträge mit nicht-themenbezogen/sinnlosen Inhalten/Inhalten,
	die die Diskussion stören; sich so verhalten, wie man selbst behandelt
	werden möchte; keine Anschuldigungen/Tatsachenbehauptungen; keine
	Beiträge mit strafbaren Inhalten/üble
	Nachrede/Urheberrechtsverstöße/Drohungen/volksverhetzende
	Äußerungen/Aufforderung zu Gewalt; Verbot von Inhalten, die dem Ansehen
	von Verstorbenen und deren Angehörigen schaden könnten/die doppeldeutig
	sind oder anderweitige Darstellungen, deren Rechtswidrigkeit vermutet
	wird, aber nicht abschließend festgestellt werden kann; keine
	unwahren/unsachlichen Beiträge; keine Schadsoftware\tabularnewline\hline

% Spalte 10
handelsblatt & Nutzungshinweise (Zustimmung), Netiquette

	mit zynischen/ironischen Äußerungen vorsichtig sein; guter Ton; nicht
	persönlich werden; keine persönlichen Angriffe/Diskriminierungen jeder
	Art; keine Beiträge mit
	verleumderischen/ruf-/geschäftsschädigenden/strafrechtlich relevanten
	Inhalten; keine Veröffentlichung Daten Dritter; sich bewusst machen,
	welche eigenen Daten frei zugänglich ins Internet gestellt werden;
	ignorieren von Provokationen/Trollen; keine
	Junkmails/Spam/Scraping/sonstige rechtswidrige Kommunikationsformen;
	keine Nennung von Produktnamen/Dienstleistern/Marken/Produzenten; kein
	Hinweis auf Nutzungsrechte von handelsblatt.de\tabularnewline\hline

% Spalte 11
suedkurier & Nutzungsbedingungen (Zustimmung), Netiquette

	faire, sachliche, offene, gehaltvolle Diskussion; keine nicht-belegbaren
	Be\-haup\-tun\-gen/Ver\-leum\-dun\-gen/Dif\-fam\-ier\-un\-gen/Dro\-hun\-gen/Dis\-kri\-mi\-nie\-run\-gen
	aller Art/Hetze/Ge\-walt\-ver\-herr\-li\-chung/Vul\-gär\-aus\-drücke; keine Beiträge mit
	ruf-/geschäftsschädigenden Inhalten; keine Veröffentlichung Daten
	Dritter\tabularnewline\hline


% Spalte 12
zeit.de & AGB (Zustimmung), Netiquette (besonders ausführlich und erklärend)

	Durchlesen vor Abschicken, guter Umgangston, nicht provozieren lassen,
	mit zynischen/ironischen Äußerungen vorsichtig sein; keine
	Diskriminierungen aller Art (auch
	Behinderung/Einkommensverhältnisse)/Diffamierungen/Verleumdungen; nicht
	prüfbare Unterstellungen/Verdächtigungen; nachvollziehbare Aussagen;
	keine Beiträge mit ruf-/geschäftsschädigenden Inhalten;  keine
	Veröffentlichung Daten Dritter; sich bewusst machen, welche persönlichen
	Daten frei zugänglich werden; keine Schadsoftware\tabularnewline\hline

% Spalte 13
badische zeitung & AGB (Zustimmung), Netiquette

	sachliche, niveauvolle, faire, offene Diskussion; freundlich, tolerant
	sein; guter Umgangston, andere so behandeln, wie man es selber möchte;
	keine persönlichen Angriffe; keine Beiträge mit
	vulgärem/hetzerischem/gewaltverherrlichendem Inhalt; keine privaten
	Daten\tabularnewline\hline

% Spalte 14
stuttgarter zeitung & AGB (Zustimmung), Kommentarregeln = Netiquette

	engagiert, fair; akzeptable, respektvolle Wortwahl; sachkritisch,
	seriös; keine
	Schmä\-hun\-gen/Dis\-kri\-mi\-nie\-run\-gen/Volks\-ver\-het\-zung/Pro\-pa\-gan\-da,
	keine Beiträge mit
	ju\-gend\-ge\-fähr\-den\-den/an\-ti\-se\-mi\-ti\-schen/straf\-ba\-ren/ver\-leum\-de\-ri\-schen/ruf-/ge\-schäfts\-schä\-di\-gen\-den/men\-schen\-ver\-ach\-ten\-den/ge\-gen
	die guten Sitten verstoßenden Inhalten; keine Trolle; keine
	Junkmails/Spam/Kettenbriefe; keine privaten Daten; keine
	Veröffentlichung Daten Dritter;  nur für private Zwecke (keine
	Vervielfältigung)\tabularnewline\hline

% Spalte 15
merkur & AGB (Zustimmung), Netiquette (Zustimmung)

	sachlich, freundlich, verständlicher Umgangston, man soll Spaß haben und
	sich wohl fühlen,  keine persönlichen Angriffe/Beschimpfungen, andere
	Meinungen akzeptieren; keine Beiträge mit
	un\-wah\-ren/un\-sach\-lichen/ju\-gend\-ge\-fähr\-den\-den/ver\-leum\-de\-ri\-schen/ver\-fas\-sungs\-feind\-li\-chen/ex\-tre\-mis\-ti\-schen/il\-le\-ga\-len/ethisch-mo\-ra\-lisch-pro\-ble\-ma\-ti\-schen/
	Inhalten; keine Schadsoftware; keine privaten Daten; nur für private
	Zwecke; keine Junkmails/Spam/Kettenbriefe\tabularnewline\hline

% Spalte 16
hna & AGB (Zustimmung), Netiquette (Zustimmung)

	sachlich, freundlich, verständlicher Umgangston, man soll Spaß haben und
	sich wohl fühlen, keine Beschimpfungen/persönlichen Angriffe; andere
	Meinungen akzeptieren; keine Beiträge mit
	un\-wah\-ren/un\-sach\-li\-chen/ju\-gend\-ge\-fähr\-den\-den/straf\-ba\-ren/ver\-leum\-de\-ri\-schen/ver\-fas\-sungs\-feind\-li\-chen/ex\-tre\-mis\-ti\-schen/il\-le\-ga\-len/ethisch-mo\-ra\-lisch-pro\-ble\-ma\-ti\-schen
	Inhalten; keine Schadsoftware; keine privaten Daten; kein Hinweis auf
	Nutzungsrechte von hna.de, kein Hinweis auf
	Haftungsausschluss\tabularnewline\hline

% Spalte 17
mopo &  keine\tabularnewline\hline

% Spalte 18
mainpost& Netiquette

	faire, akzeptable, respektvolle Wortwahl; keine verbalen Angriffe,
	privaten Details aus dem Leben anderer; mit zynischen/ironischen
	Äußerungen vorsichtig sein; kein Aufruf zu Straftaten; keine privaten
	Daten; keine Beiträge mit ehrverletzenden/gewaltverherrlichenden
	Inhalten/Aufrufen zur Gewalt; Löschen von Beiträgen, die gegen das
	Gesetz verstoßen; kein Hinweis auf Nutzungsrechte von mainpost.de, kein
	Hinweis auf Haftungsausschluss\tabularnewline\hline

% Spalte 19
tagesspiegel & Richtlinien für Community

	sachlich, respektvoll; fairer Umgang, angenehme Atmosphäre; bewusst
	machen, dass K. öffentlich werden; keine Angaben, die nicht für die
	Öffentlichkeit sind; keine pauschale/persönliche Herabwürdigung; keine
	Beiträge mit
	pietätlosen/menschenverachtenden/gewaltverherrlichenden/verleumderischen
	Inhalten; kein Geschichtsrevisionismus; keine Trolle; keine
	Unterstellungen/Kampagnen/Diskriminierungen aller Art (auch aufgrund von
	Weltanschauung/sozialem Status); Links zu anderen Webinhalten/Kritik an
	Artikeln mit denselben inhaltlichen Regeln wie K.; keine
	Mehrfachaccounts; kein Hinweis auf Nutzungsrechte von
	tagesspiegel.de\tabularnewline\hline

% Spalte 20
swp & Netiquette

	sachlich, fair, freundlich, wie man selbst behandelt werden möchte;
	keine het\-ze\-ri\-schen/ge\-walt\-ver\-herr\-li\-chen\-den Töne; keine privaten Adressen;
	keine Hinweise zu Rechten\tabularnewline

\end{longtable}
\end{landscape}

\section{Kategorie: \glqq formale Regeln\grqq}

mit  den Unterkategorien
\begin{itemize}
\item\glqq {\bf Zeichenbegrenzung}\grqq: Gibt es eine Zeichenbegrenzung oder kann der Nutzer schreiben, so viel er möchte?


\item\glqq{\bfÜberschrift}\grqq: Wird die Eingabe einer Überschrift verlangt?


\item\glqq{\bf Sonstiges}\grqq: Welche formalen Hinweise geben die Online-Zeitungen zusätzlich?
\end{itemize}


Beobachtungen bei den Nachrichtenportalen:\\
Die Zeitungen geben auch Hinweise zum formalen Umgang mit Kommentaren. Die Hälfte gibt eine Zeichenbegrenzung vor. Diese reicht von 800 bis 3000 Zeichen. 
Ebenso fordert ungefähr die Hälfte der Redaktionen die Eingabe eines Betreffs. Die sonstigen formalen Regeln haben unterschiedlichen Inhalt. Interessant ist die Möglichkeit bei mainpost.de, tagesspiegel.de und augsburger-allgemeine.de den eingegebenen Text sogar hervorzuheben. Diese Funktion geht über die übliche Art von Kommentaren hinaus. Bei augsburger-allgemeine.de kann der Kommentar vollständig formatiert werden.


\begin{landscape} \footnotesize
  \begin{longtable}{l|llp{100mm}}

  & \multicolumn{3}{c}{\bfseries formale Regeln} \\
  & Zeichenbegrenzung & Überschrift (Pflichtfeld) & Sonstiges \\\hline
  \endfirsthead

  & \multicolumn{3}{c}{\bfseries formale Regeln} \\
  & Zeichenbegrenzung & Überschrift (Pflichtfeld) & Sonstiges \\\hline
  \endhead

  \hline \multicolumn{4}{r}{\emph{Fortsetzung auf der nächsten Seite}}
  \endfoot

  \hline
  \endlastfoot

%Kommentar: formale Regeln\\
%  & Zeichenbegrenzung & Überschrift (Pflichtfeld) & Sonstiges
%
%Zeile 1: Portale	 %
%Spalte 1		
bild.de			& keine & keine & vorsichtig mit Großbuchstaben, Zitate kennzeichnen \\\hline
%Spalte 2	
spiegel.de			& keine & optional & keine Bilder posten, keine langen Kopien von Quellen (Links verwenden) \\\hline
%Spalte 3		
faz.net			& 1000 Zeichen & ja, 100 Zeichen & \\\hline
%Spalte 4		
focus.de			& 800 Zeichen & ja & reiner Text ohne besondere Kennzeichnungen (z.B. keine Smilies, Hervorhebungen, Chat-Symbole, nur Kleinschreibung, usw.), korrektes Deutsch, auf Rechtschreibung/Interpunktion achten, Absätze machen \\\hline
% Spalte 5		
welt.de			& keine & keine & Zitate kennzeichnen; keine Fremdsprachen, keine Links zu externen Webseiten (seriöse Ausnahmen möglich) \\\hline
% Spalte 6	
derwesten.de		& keine & keine & Zitate müssen Quellenangabe enthalten\\\hline
% Spalte 9		
rp-online.de		& keine & ja & deutsche Sprache; Links möglich (keine Links zu Werbung/strafbare Inhalte) \\\hline
% Spalte 10	
handelsblatt.de		& 2000 & keine & Großbuchstaben (Schreien) vermeiden; Absätze machen und strukturieren; Wortwahl überprüfen; auf Rechtschreibung achten; 	Zitate/Quellen kennzeichnen \\\hline
% Spalte 11		
suedkurier.de		& 1000 Zeichen & ja & Zitate kennzeichnen mit Quellenangabe \\\hline
% Spalte 12		
zeit.de			& 1500 Zeichen & ja, mindestens 5 Zeichen & Absätze machen; auf Rechtschreibung achten; vorsichtig mit Großbuchstaben; Zitate kennzeichnen, wenig verwenden, Quellenangabe machen, nur als Ergänzung verwenden; Links möglich \\\hline
% Spalte 13	
badische-zeitung.de	& keine & keine & \\\hline
% Spalte 14	
stuttgarter-zeitung.de	& keine & ja & Links möglich (keine Links zu Werbung/kommerziellen Angeboten/Chats/Foren/strafbaren Inhalten) \\\hline
% Spalte 15	
merkur.de			& keine & keine & deutsche Sprache; Links nicht erwünscht (falls doch distanziert sich merkur-online von Inhalten der gelinkten Seiten) \\\hline
% Spalte 16	
hna.de			& keine & keine & deutsche Sprache; Beiträge in Fremdsprachen werden gegebenenfalls entfernt, da für größten Teil der Nutzer nicht verständlich; Links nicht erwünscht (falls doch distanziert sich hna.de von Inhalten der gelinkten Seiten) \\\hline
% Spalte 17		
mopo.de			& keine & keine & \\\hline
% Spalte 18		
mainpost.de		& 1000 Zeichen & ja & auf deutsche Rechtschreibung achten; korrekte Interpunktion, Absätze machen; Hervorhebungen möglich: fett, kursiv, unterstrichen; markieren von Links, Zitaten möglich; einfügen von Emoticons möglich (Grinsen, Zwinkern, traurig sein); \\\hline
% Spalte 19	
tagesspiegel.de		& 2000 Zeichen & ja & auf Rechtschreibung/Grammatik achten; Hervorhebungen möglich: fett, kursiv; markieren von Quellen durch Links; markieren von Links/Zitaten; keine Links zu anderen Webinhalten; Zitate als Ergänzung, nicht alleinstehend \\\hline
% Spalte 20		
swp.de			& 3000 Zeichen & ja & \\ \hline
%Spalte 21		
augsburger-allgemeine.de	& keine & optional & sämtliche Formatierungsmöglichkeiten, auch Emoticons; Vorschau möglich!; Button zum Zitieren; Anhängen von Daten möglich (begrenzte Größe)\\ \hline

\end{longtable}
\end{landscape}


% vim: set ai si et tw=80 sts=2 ts=2 sw=2:

\section{Kategorie: Funktionen beim Kommentar}

mit den Unterkategorien:
\begin{itemize}
\item\glqq{\bf Bewerten}\grqq:\\
Gibt es die Möglichkeit für einen bereits geschriebenen und veröffentlichten Kommentar eine Bewertung abzugeben, ob man ihn gut (oder schlecht) findet?

\item\glqq{\bf Antworten}\grqq: \\
Gibt es die Möglichkeit auf einen bereits geschriebenen und veröffentlichten Kommentar zu antworten?


\item\glqq{\bf Sonstiges}\grqq:\\
Was gibt es für weitere Funktionen beim Kommentar selbst? Was kann sich der Nutzer anzeigen lassen? 


\end{itemize}

Beobachtungen bei den Nachrichtenportalen:\\
Weniger als die Hälfte aller Zeitungen geben die Möglichkeit eine
Bewertung des Kommentars abzugeben. Wenn eine Bewertung gemacht werden kann,
dann eher nur als positive Bewertung.\\
Die meisten Oberflächen geben die Möglichkeit auf einen Kommentar
zu antworten. \\
Bei sonstigen Funktionen wird häufig das
Ordnen nach bestimmten Kriterien angeboten. 


\begin{landscape} \footnotesize
  \begin{longtable}{l|p{40mm}p{40mm}p{80mm}}

  & \multicolumn{3}{c}{\bfseries Funktionen beim Kommentar} \\
  & Bewerten & Antworten & Sonstiges \\\hline
  \endfirsthead

  & \multicolumn{3}{c}{\bfseries Funktionen beim Kommentar} \\
  &  Bewerten & Antworten & Sonstiges \\\hline
  \endhead

  \hline \multicolumn{4}{r}{\emph{Fortsetzung auf der nächsten Seite}}
  \endfoot

  \hline
  \endlastfoot



\hline

  % Zeile 1
  bild.de &
    positiv & & Ordnen nach beliebteste, älteste, neueste K.\\
    & \multicolumn{3}{c}{}\\\hline

  % Zeile 2
  spiegel.de &
  & ja\footnote{auf was man antwortet wird in Zitate gesetzt} & \\
  & \multicolumn{3}{c}{}\\\hline

  % Zeile 3
  faz.net & positiv & & dem Kommentator folgen \\
          & \multicolumn{3}{c}{}\\\hline

  % Zeile 4
  focus.de & positiv und negativ & ja & \\
           & \multicolumn{3}{c}{}\\\hline

  % Spalte 5
  diewelt.de &
    positiv\footnote{Klicks werden gezählt; eine Bewertung ist nach Anmeldung oder als ``Gast'' möglich.}
    und negativ\footnote{Bewertung ist nach Anmeldung möglich.} &
    ja &
    ``Teilen''-Button (= diese Diskussion auf Twitter/Facebook teilen, nach
    Anmeldung dort), ``Teilen''-Button für einzelnen K., ``Empfehlen''-Button (=
    diese Diskussion empfehlen); Ordnen nach beste, neueste, älteste Kommentare;
    ``Abonnieren'', um Mittleiungen dieser Diskussion per Email zu erhalten;
    Disqus auf der eigenen Seite hinzufügen; Datenschutz\\
    & \multicolumn{3}{c}{Oberfläche von Disqus}\\\hline

  % Spalte 6
  derwesten.de & & ja\footnote{Anzahl Antworten wird angegeben.} & \\
               & \multicolumn{3}{c}{}\\\hline

  % Spalte 9
  rp-online.de & positiv & & Ordnen nach älteste; Button für ``mehr K.''\\
            & \multicolumn{3}{c}{}\\\hline

  % Spalte 10
  handelsblatt.de & & ja & \\
               & \multicolumn{3}{c}{}\\\hline

  % Spalte 11
  suedkurier.de & & ja &
    Ordnen nach älteste, neueste, beste Bewertung; zur Auswahl: ``informiert
    bleiben'' (bei jedem neuen Beitrag der Diskussion erhält man
    Benachrichtigung)\\
    & \multicolumn{3}{c}{}\\\hline

  % Spalte 12
  zeit.de & & ja &
    Reaktionen/Antworten auf diesen K. anzeigen; Ordnen nach neuesten,
    empfohlenen, allen K.; Empfehlungen aussprechen; Empfehlungen der Redaktion
    (bedeutet aber nicht, dass die Redaktion der Meinung des Lesers zustimmt)\\
    & \multicolumn{3}{c}{}\\\hline

  % Spalte 13
  badische-zeitung.de & & & \\
                   & \multicolumn{3}{c}{} \\\hline

  % Spalte 14
  stuttgarter-zeitung.de & & ja & Ordnen nach älteste, neueste K.; Empfehlen\\
                      & \multicolumn{3}{c}{}\\\hline

  % Spalte 15
  merkur.de &  & & \\
         & \multicolumn{3}{c}{Oberfläche von Disqus: siehe diewelt.de}\\\hline


  % Spalte 16
  hna.de & & & \\
      & \multicolumn{3}{c}{Oberfläche von Disqus: siehe diewelt.de}\\\hline

  % Spalte 17
  mopo.de & & & \\
         & \multicolumn{3}{c}{Oberfläche von Disqus: siehe diewelt.de}\\\hline

  % Spalte 18
  mainpost.de & & ja & Ordnen nach älteste, neueste, best bewertete K.\\
           & \multicolumn{3}{c}{}\\\hline

  % Spalte 19
  tagesspiegel.de & & ja\footnote{Man kann wählen, ob man die Antworten sehen möchte.} &
    Ordnen nach neueste, älteste K., chronologisch\\
    & \multicolumn{3}{c}{}\\\hline

  % Spalte 20
    swp.de & positiv & ja & \\
        & \multicolumn{3}{c}{}\\\hline \\

  %Spalte 21
    augsburger-allgemeine.de &
      & ja\footnote{Anhänge möglich, Vorschau vorhanden, Speichern möglich} & \\
      & \multicolumn{3}{c}{}\\\hline

\end{longtable}
\end{landscape}



%
%Funktionen im Kommentar	\\
%``Bewerten''\\ ``Antworten''\\ Sonstiges\\
%
%% Spalte 1
%\\
%  positiv\\ \\ Ordnen nach beliebteste, älteste, neueste K. &
%% Spalte 2
%\\
%  \\ ja: auf was man antwortet wird in Zitate gesetzt\\ \\ &
%% Spalte 3
%\\
%  positiv\\ \\ dem Kommentator folgen\\ &
%% Spalte 4
%\\
%  positiv und negativ\\ ja\\ \\ &
%% Spalte 5
%Oberfläche von Disqus\\
%  positiv (Klicks werden gezählt), nach Anmeldung oder als ``Gast'' möglich; negativ (nach Anmeldung)\\ ja\\ ``Teilen''-Button (= diese Diskussion auf Twitter/Facebook teilen, nach Anmeldung dort), ``Teilen''-Button für einzelnen K., ``Empfehlen''-Button (= diese Diskussion empfehlen); Ordnen nach beste, neueste, älteste Kommentare; ``Abonnieren'', um Mittleiungen dieser Diskussion per Email zu erhalten; Disqus auf der eigenen Seite hinzufügen; Datenschutz\\ &
%% Spalte 6
%\\
%  \\ ja, mit Anzahl Antworten\\ \\ &
%% Spalte 9
%\\
%  positiv\\ \\ Ordnen nach älteste; Button für ``mehr K.''\\ &
%% Spalte 10
%\\
%  \\ ja\\ \\ &
%% Spalte 11
%\\
%  \\ ja \\ Ordnen nach älteste, neueste, beste Bewertung; zur Auswahl: ``informiert bleiben'' (bei jedem neuen Beitrag der Diskussion erhält man Benachrichtigung)\\ &
%% Spalte 12
%\\
%  \\ ja\\ Reaktionen/Antworten auf diesen K. anzeigen; Ordnen nach neuesten, empfohlenen, allen K.; Empfehlungen aussprechen; Empfehlungen der Redaktion (bedeutet aber nicht, dass die Redaktion der Meinung des Lesers zustimmt)\\ &
%% Spalte 13
%\\
%  \\ \\ \\ &
%% Spalte 14
%\\
%  \\ ja\\ Ordnen nach älteste, neueste K.; Empfehlen\\ &
%% Spalte 15
%Oberfläche von Disqus: siehe die Welt.de\\
%  \\ \\ \\ &
%% Spalte 16
%Oberfläche von Disqus: siehe die Welt.de\\
%  \\ \\ \\ &
%% Spalte 17
%Oberfläche von Disqus: siehe die Welt.de\\
%  \\ \\ \\ &
%% Spalte 18
%\\
%  \\ ja\\ Ordnen nach älteste, neueste, best bewertete K.\\ &
%% Spalte 19
%\\
%  \\ ja; zur Auswahl: Antworten anzeigen\\ Ordnen nach neueste, älteste K., chronologisch\\ &
%% Spalte 20
%\\
%  positiv\\ ja\\ \\ &
%% Spalte 21
%\\
%  \\ ja; Anhänge möglich, Vorschau vorhanden, Speichern möglich\\ \\ &
%
%%%%

% vim: set ai si et tw=80 sts=2 ts=2 sw=2:

\section{Zusammenfassung}

Moderation: was für eine Moderation wird gemacht\\
Prä-M. = Prä-Moderation bzw. st. Prä-M. = stichprobenartige Prä-Moderation\\
Post-M. = Post-Moderation bzw. st. Post-M. = stichprobenartige Post-Moderation\\




Registrierung: Welche Angaben sind nötig zu einer Registrierung\\
KN = Klarnamen\\
BN = Benutzername\\
alt = alternative Anmeldemöglichkeiten

Bewerten: Hat der Nutzer die Option den Kommentar zu bewerten?

Antwort: Gibt es die Option, direkt auf einen Kommentar zu antworten?

%alternative Anmeldemöglichkeiten: Fb = Facebook, Dis = Disqus, Tw = Twitter, G = Google

Community: gibt es eine Community?

\begin{landscape} \footnotesize
\begin{longtable}{l|lllll}
\caption{Zusammenfassung}
\\
\bfseries Portal 		&\bfseries Moderation &\bfseries Registr. &\bfseries Bewerten  &\bfseries Antwort &\bfseries Community\\ \hline
\endfirsthead
\bfseries Portal 		&\bfseries Moderation &\bfseries Bedingungen &\bfseries Registrierung 				&\bfseries Community\\ \hline
\endhead
\hline \multicolumn{6}{r}{\emph{Fortsetzung auf der nächsten Seite}}
\endfoot
\hline
\endlastfoot

%Spalte 1			&Moderation 		&Registrierung 		&Bewerten	 &Antworten	&Community
bild.de			& keine			&KN, BN, alt		&ja		&ja		& ja		 \tabularnewline \hline
%Spalte 2             
spiegel.de			& Prä-M.			& KN, BN, alt		&nein	&ja			& ja 			\tabularnewline \hline
%Spalte 3                
faz.net			& Prä-M.			& KN 			&ja		&ja		& ja 			\tabularnewline \hline
%Spalte 4               
focus.de			& st. Prä-M.		& KN, alt			&ja		&ja			& ja 			\tabularnewline \hline
% Spalte 5              
welt.de			& Prä-M.			& KN, BN, alt  		&ja		& ja			&ja	 	\tabularnewline \hline
% Spalte 6              
derwesten.de		& keine			& KN, BN 		 	&nein	&ja			& nein 		\tabularnewline \hline
% Spalte 9              
rp-online.de		& keine			& BN 			&ja		&nein		& nein 		\tabularnewline \hline
% Spalte 10             
handelsblatt.com	& st. Post-M.		& KN 			&nein	&ja			& nein 		\tabularnewline \hline
% Spalte 11              
suedkurier.de		& st. Post-M.		& KN, BN 			&ja 		&ja			& nein		 \tabularnewline \hline
% Spalte 12           
zeit.de			& Prä-M.			& BN 			&ja		&ja			& nein		 \tabularnewline \hline
% Spalte 13             
badische-zeitung.de	& st. Post-M.		& KN 			&nein	&nein		& nein 		\tabularnewline \hline
% Spalte 14             
stuttgarter-zeitung.de	& Prä-M.			& KN, alt			&nein	&ja			& nein 		\tabularnewline \hline
% Spalte 15             
merkur.de			& keine			& BN, alt 			&ja		&ja			& ja 			\tabularnewline \hline
% Spalte 16             
hna.de			& keine			& BN, alt 			&ja		&ja			& ja 			\tabularnewline \hline
% Spalte 17             
mopo.de			& keine			& Dis, alt 			&ja		&ja			& ja			 \tabularnewline \hline
% Spalte 18             
mainpost.de		& Prä-M.			& KN, BN 			&ja	&ja			& ja			 \tabularnewline \hline
% Spalte 19             
tagesspiegel.de		& Prä-M.			& BN 			&nein	&ja			& nein 		\tabularnewline \hline
% Spalte 20             
swp.de			& keine/st. Post-M.	& KN, BN 			&ja		&ja			& nein 		\tabularnewline \hline
% Spalte 21            
augsburger-allgemeine.de& Post-M. 		& KN, BN 			&nein	&ja			& ja			 \tabularnewline \hline
% REGISTRIERUNG (mypass = Anmeldung über ``mypass'', Fb = Facebook, Tw = Twitter, G = Google, Dis = Disqus)

\end{longtable}
\end{landscape}


\chapter{Ausblick}\label{Ausblick}



Ziel und Beantwortung der Forschungsfragen\footnote{FF1: Wie werden die Nutzerkommentare auf
deutschen Nachrichtenportalen gehandhabt?\hfill\\
FF2: An welche Vorgaben müssen sich die Nutzer halten?} ist es, einen Überblick zu schaffen und
darzustellen, wie das Kommentarmanagement bei deutschen Online-Zeitungen aussieht. Infolgedessen ist
auch eine Beschreibung von Kommentaren im Allgemeinen entstanden.

Kommentarmanagement heißt, welche Bedingungen die Zeitungen verlangen, um überhaupt schreiben zu
können, was es für Regeln gibt und welche Funktionen zusätzlich vorhanden sind. Im Sinne einer
Dokumentenanalyse wurde Kategorien erarbeitet, die das Kommentarmanagement beschreiben. Diese
Kategorien sind in Tabellen zusammengefasst und dargestellt. Bei den jeweiligen Tabellen stehen die
entsprechenden Beobachtungen zu den Kategorien.  Es gibt noch weitere Beobachtungen in welche
Richtung das Kommentarmanagement gerade geht.

\begin{itemize}
  \item Kommentare lassen sich {\bfseries formatieren}. Gerade die Online-Ausgabe der Augsburger
    Allgemeinen bietet in der Kommentarfunktion ein volles Textprogramm zur Formatierung an.  Hier
    ist die Frage, ob diese Individualisierung für das Schreiben von  Kommentaren nötig ist. Oder
    hoffen die Zeitungen damit auf bessere Beiträge? Wenn ja, würden auch ein paar wenige
    Formatierungsfunktionen ausreichen?

  \item Die Technik der Kommentarfunktion wird von einem externen Unternehmen zur Verfügung
    gestellt. Dieses externe interaktive Kommentarsystem nennt sich \glqq{\bfseries
    Disqus}\grqq\footnote{Die Disqus-Kommentierungsfunktion wird von der Big Head Labs, Inc., San
    Francisco/USA als Dienstleistung zur Verfügung gestellt.} und wird von mehreren Online-Zeitungen
    genutzt. Es wäre interessant zu untersuchen, ob sich diese Tendenz bestätigt und das
    Kommentarsystem von weiteren Portalen genutzt wird.

    Die Kommentarregeln können bei Disqus selbst bestimmt werden. Disqus moderiert ausdrücklich
    nicht und verwaltet auch nicht die Communities. Es gibt ein paar \emph{Basic Rules}, die
    allgemein für Disqus gelten. Für den Nutzer liegt der Vorteil bei der Verwendung von Disqus
    darin, dass er sich sofort auf der Oberfläche auskennt. Außerdem kann er sich über Disqus
    anmelden und muss sich nicht mehr auf dem Portal registrieren. Die Nutzer können sich auch über
    bestehende Accounts bei Facebook, Twitter und Google anmelden. Dabei geht aber ein Stück
    Individualität einer Zeitung verloren.  Andererseits kann man sich fragen, ob eine individuelle
    Kommentarfunktion überhaupt nötig ist.  Schließlich wollen die Nutzer nur kurz ihre spontanen
    Gedanken loswerden.

%Vorteil: der Nutzer kennt sich sofort auf der Oberfläche aus; er benötigt nur den Zugang vom Anbieter\\
%Nachteil: Individualität geht verloren

  \item Eine andere Tendenz ist der Aufbau von {\bfseries Communities}, d.h.  der Nutzer kann u.a.
    ein Profil anlegen, persönliche Rankings erstellen oder sich in Rankings wiederfinden (Disqus,
    siehe oben,  bietet ebenfalls eine Community an). Auch hier ist das Ziel des
    Kommentarmanagements, dass gute Kommentare verfasst werden. Außerdem findet eine
    Selbstregulierung durch aktive und vertrauenswürdige Nutzer statt. Man kann eher konstruktive
    Beiträge von den Mitgliedern erwarten.  Es entstehen keine Nachteile, da man das Profil auch
    ignorieren kann.  Die Online-Ausgabe der Augsburger Allgemeinen sticht mit ihrer Community und
    den Nutzerbeiträgen auch hier besonders hervor. Neben einem großen Profil gibt es ein
    Textprogramm zur Formatierung (siehe oben). Außerdem kann man Beiträge im Forum nach Themen und
    Aktualität suchen. Es gibt Aufstellungen über Themen, Beiträge, Antworten, Aufrufe. Dieses
    äußerst umfangreiche Forum ist interessant, schießt aber unter Umständen über das Ziel hinaus.
    Schließlich stellt sich noch die Frage, ob das Vorhandensein einer Community die Kommentare
    beeinflusst und ob es dadurch zu einem besseren Umgangston kommt?

  \item Was immer stärker von den Nachrichtenportalen genutzt wird ist das {\bfseries Anmelden über
    andere} soziale Netzwerke, allen voran Facebook und Twitter (Auch bei Disqus ist das Standard,
    siehe oben.). Das ist besonders nutzerfreundlich, da man den Zugang nehmen kann, den man unter
    Umständen eh schon hat. Die Zeitungen schlagen hier zwei Fliegen mit einer Klappe. Sie machen es
    dem Nutzer leicht, da die Registrierung wegfällt, und stellen eine Verbindung in die genannten
    Netzwerke her. 

  \item Inwieweit die Zahl der Kommentare einer Zeitung insgesamt Einfluss auf die Art des
    Kommentarmanagements hat und umgekehrt, wurde nicht untersucht. Die regionalen Zeitungen aus dem
    Süden (z.B. suedkurier.de und swp.de) haben nämlich extrem wenig Kommentare auf ihren Seiten zu
    verzeichnen.  Daraus ergeben sich neue Fragen am Rande. Sind die Nutzer einfach nur zu träge zum
    Kommentieren oder ist die Kommentarfunktion nicht attraktiv genug? Ist es von den Zeitungen
    gewollt, dass wenig kommentiert wird? 
\end{itemize}

Es gibt immer wieder neue Ideen und Modelle für das Kommentarmanagement. Hier sind zwei Beispiele.
\begin{itemize}
	\item Bei dem Online-Magazin \glqq Tablet\grqq{} muss man fürs Kommentar-Schreiben {\bfseries bezahlen}
	\autocite{schenz}.
Damit soll versucht werden, den Missbrauch zu umgehen. Ob die Nutzer bereits sind, Geld fürs Kommentieren zu bezahlen, wird sich zeigen. Genauso, ob sich Störenfriede von Geld abhalten lassen.


\item Oder die Kommentarfunktion wird komplett {\bfseries ausgelagert}, d.h. dass man nicht mehr auf der Seite der
	Online-Zeitung kommentieren kann. Bei sueddeutsche.de wird man auf \glqq Rivva\grqq{} (Rivva filtert das Social Web nach am meisten empfohlenen Artikeln und debattierten Themen) weitergeleitet, wo man die Diskussion verfolgen und über die entsprechenden sozialen Netzwerke auch selber kommentieren kann. Sueddeutsche.de versucht auf diesem Weg die Schwächen des Kommentarmanagements zu umgehen. 

\end{itemize}

% vim: set ai si et tw=100 sts=2 ts=2 sw=2:



Hwang, Hyunseo. 2008. “Why Does Incivility Matter When Communicating Disagreement?:
Examining the Psychological Process of Antagonism in Political Discussion.” Dissertation,
University of Wisconsin-Madison

Loke, J. (2013) “Readers’ Debate a Local Murder Trial: ‘Race’ in the Online
Public Sphere.” Communication, Culture \& Critique 6 (1): 179–200.
doi:10.1111/j.1753-9137.2012.01139.x.

Paulussen, S. 2011. ``Inside the Newsroom Journalists’ motivations and organizational structures'' In: Practices


Robinson, S. 2010. “Traditionalists vs. Convergers: Textual Privilege, Boundary Work, and the
Journalist – Audience Relationship in the Commenting Policies of Online NewsSites.”
Convergence: The International Journal of Research into New Media Technologies 16 (1):
125–143. doi:10.1177/1354856509347719.

Loke, J. (2013) “Readers’ Debate a Local Murder Trial: ‘Race’ in the Online
Public Sphere.” Communication, Culture \& Critique 6 (1): 179–200.
doi:10.1111/j.1753-9137.2012.01139.x.



Bereicherung der public agenda? Die Nachrichten werden einem präsentiert, aber
nun gibt es auch die Möglichkeit darüber zu sprechen. Denn je mehr berichtet
wird, umso mehr wird diskutiert. Dadurch dass die Kommentare unterbunden werden
können, trifft das nicht mehr zu.

 

„[\ldots] we can actually talk to these people now! And they can talk to us! And
this is great!“ (Reich, 2011, S. 105).

„In addition, media gatekeepers  turned older participation channels into
exclusive spaces: Only those citizens whom the gatekeepers decided were
worth hearing were allowed a public voice. Comment threads, in contrast,
are inclusive spaces; most comments that do not break explicit rules of
participation are included.“ (Reich, 2011, S. 97)

\end{document}

