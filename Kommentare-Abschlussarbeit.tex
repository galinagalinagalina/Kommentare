% !TEX TS-program = pdflatex
% !TEX encoding = UTF-8 Unicode

% This is a simple template for a LaTeX document using the "article" class.
% See "book", "report", "letter" for other types of document.

\documentclass[12pt,parskip=full]{scrreprt} % use larger type; default would be 10pt

\usepackage[onehalfspacing]{setspace}
\AfterTOCHead{\singlespacing}

\KOMAoptions{DIV=last}

\usepackage[utf8]{inputenc} % set input encoding (not needed with XeLaTeX)
\usepackage[ngerman]{babel}

% Maybe colorize links?
\usepackage[hidelinks]{hyperref}

%%% Examples of Article customizations
% These packages are optional, depending whether you want the features they provide.
% See the LaTeX Companion or other references for full information.

%%% PAGE DIMENSIONS
\usepackage{geometry} % to change the page dimensions
\geometry{a4paper} % or letterpaper (US) or a5paper or....
% \geometry{margin=2in} % for example, change the margins to 2 inches all round
% \geometry{landscape} % set up the page for landscape
%   read geometry.pdf for detailed page layout information

\usepackage{graphicx} % support the \includegraphics command and options

% \usepackage[parfill]{parskip} % Activate to begin paragraphs with an empty line rather than an indent

%%% PACKAGES
\usepackage{booktabs} % for much better looking tables
% \usepackage{array} % for better arrays (eg matrices) in maths
\usepackage{paralist} % very flexible & customisable lists (eg. enumerate/itemize, etc.)
\usepackage{verbatim} % adds environment for commenting out blocks of text & for better verbatim
\usepackage{subfig} % make it possible to include more than one captioned figure/table in a single float
% These packages are all incorporated in the memoir class to one degree or another...

%%% HEADERS & FOOTERS
\usepackage{fancyhdr} % This should be set AFTER setting up the page geometry
\pagestyle{fancy} % options: empty , plain , fancy
\renewcommand{\headrulewidth}{0pt} % customise the layout...
\lhead{}\chead{}\rhead{}
\lfoot{}\cfoot{\thepage}\rfoot{}

%%% SECTION TITLE APPEARANCE
\usepackage{sectsty}
\allsectionsfont{\sffamily\mdseries\upshape} % (See the fntguide.pdf for font help)
% (This matches ConTeXt defaults)

%%% ToC (table of contents) APPEARANCE
\usepackage[nottoc,notlof,notlot]{tocbibind} % Put the bibliography in the ToC
\usepackage[titles,subfigure]{tocloft} % Alter the style of the Table of Contents
\renewcommand{\cftsecfont}{\rmfamily\mdseries\upshape}
\renewcommand{\cftsecpagefont}{\rmfamily\mdseries\upshape} % No bold!

%%% END Article customizations

%%% The "real" document content comes below...

%\setkomafont{date}{\large}
%\setkomafont{author}{\large}
\title{Kommentarmanagement auf deutschen Nachrichtenseiten}
\subtitle{Abschlussarbeit im Aufbaustudiengang Journalistik}
\author{Nadine Ambrosch}

%\date{} % Activate to display a given date or no date (if empty),
         % otherwise the current date is printed 

\begin{document}

\begin{spacing}{1}
  \maketitle
\end{spacing}

\chapter{Nutzerbeteiligung im Journalismus: Kommentare als Nutzerbeteiligung}
\label{kap:nutzerbeteiligung}

\section{Vom Leser zum Nutzer zur Nutzerbeteiligung}

Mit der Digitalisierung ändert sich der Kommunikationsprozess. Der Leser kommt
weg von seiner Rolle als reiner Zuhörer. Er muss nicht mehr warten, bis ihm
etwas angeboten wird. Der Leser wird zum Nutzer der Medienangebote, denen er
sich auch mitteilen kann. Er kann sie nicht nur nutzen, sondern auch benutzen.
Er kann sich richtig in den Kommunikationsprozess einschalten und zwar direkt
und unmittelbar.

\begin{quote}
\glqq A great many other people also contribute content, representing their own
interests, ideas, observations and opinions. That content comes in a steadily
expanding volume and variety of forms and formats – words, images and sounds,
alone or in combination.\grqq{} \autocite[S.~1]{participatory}
\end{quote}

Im Englischen werden u.a. die Ausdrücke „user generated content“, „citizen
journalism“ oder „participatory journalism“ \autocite[S.~2]{participatory} dafür
verwendet. Sie alle beinhalten, dass Medienschaffende und Nutzer miteinander
kommunizieren und die Nutzer ihren Teil zur Bildung von Nachrichten und
Gemeinschaft dazu tun.

Ausdrucksformen dieser Nutzerbeteiligung sind vor allem Beiträge in
(Diskussions-)Foren und sozialen Netzwerken, Blogs, Berichte, Beurteilungen,
Bewertungen, Kommentare, hochgeladene Fotos und Videos und noch viele mehr.
„Indeed, new participatory formats appear all the time.“ \autocite[S.~2 und
S.~17]{participatory}


\section{Kommentare als Nutzerbeteiligung}

In dieser Arbeit stehen die Nutzerkommentare im Mittelpunkt. Als Kommentare
bezeichnet man „views on a story or other online item, which users typically
submit by filling in a form on the bottom of the item.“
\autocite[S.~17]{participatory}.  Sie markieren eine neue Stufe in der
Nutzerbeteiligung und sie sind überaus beliebt, was auch immer wieder in Studien
bestätigt wird \autocite[S.~97, siehe 3.1]{reich}.

Weil die Funktion bei Onlinezeitungen stark genutzt wird, kommt es zu einer Flut
von Material. Mit diesem Material haben die Redaktionen zu kämpfen. Sie sehen
zwar die Vorteile, die die Kommentarfunktion bringt. In der täglichen Arbeit
führt es jedoch immer wieder zu schlechten Erfahrungen.

Als nächstes wird aufgezeigt, was die Kommentare besonders macht, warum sie so
populär sind, was die positiven Aspekte sind und wie sie den
Kommunikationsprozess bereichern. Danach werden die Probleme und die Eingriffe,
das Kommentarmanagement, erörtert.

\begin{quote}
User-generated posts attached to a published item, typically an article or blog
entry, on a media website. Most news organizations moderate or screen user
comments, either before or after publication;
\autocite[S.~204, Glossar]{participatory}
\end{quote}

% vim: set ai si et tw=80 sts=2 ts=2 sw=2:




\chapter{Kommentare als Nutzerbeteiligung}

In dieser Arbeit stehen die Nutzerkommentare im Mittelpunkt. Als Kommentare
bezeichnet man „views on a story or other online item, which users typically
submit by filling in a form on the bottom of the item.“ (Singer, 2011, S. 17).


User-generated posts attached to a published item, typically an article or blog
entry, on a media website. Most news organizations moderate or screen user
comments, either before or after publication; Singer, 2011, S. 204, Glossar

Sie markieren eine neue Stufe in der Nutzerbeteiligung und sie sind überaus
beliebt, was auch immer wieder in Studien bestätigt wird (Reich, 2011, S. 97)
(siehe 2.1.1)

Weil die Funktion bei Onlinezeitungen stark genutzt wird, kommt es zu einer Flut
von Material. Mit diesem Material haben die Redaktionen zu kämpfen. Sie sehen
zwar die Vorteile, die die Kommentarfunktion bringt. In der täglichen Arbeit
führt es jedoch immer wieder zu schlechten Erfahrungen.

Als nächstes wird aufgezeigt, was die Kommentare besonders macht, warum sie so
populär sind, was die positiven Aspekte sind und wie sie den
Kommunikationsprozess bereichern. Danach werden die Probleme und Maßnahmen
erörtert.


  \chapter{Positive Aspekte der Kommentarfunktion}

Was macht Kommentare wertvoll und was bringen sie eigentlich? Dieser Frage wird
im folgenden Abschnitt nachgegangen.


\section{Beliebtheit der Kommentare} \label{sec:beliebtheit}
Obwohl nur ein kleiner Teil der Leser Kommentare verfasst und ein ebenso kleiner
Teil diese auch liest \autocite[96ff]{reich}, schlägt diese Art der
Nutzerbeteiligung wie eine Bombe ein. Das merken die Redaktionen, die von nicht
mehr zu bewältigenden Kommentaren erreicht werden. Dies ist dem Umstand
geschuldet, dass es auf einmal (2005 entstehen die ersten Kommentare) die
Möglichkeit gibt, rauszulassen, was einem gerade durch den Kopf geht beim Lesen
eines Artikels.

Es ist aber nicht nur die Gelegenheit spontan etwas zum eben Gelesenen zu sagen,
was die Kommentare attraktiv macht. Menschen haben selten die Chance, ihren
Unmut kund zu tun oder Zustimmung auszusprechen, weil sie auf das nur Zuhören
beschränkt sind \autocite[S.~99]{reich}. Diese auferlegte Passivität wird
aufgehoben.\\
Für manche Leser werden die Kommentare sogar genauso wichtig, wie die Nachricht
selbst. Und manchen Leser gefällt es einfach, sich selbst veröffentlicht zu
sehen.


\section{Öffentlicher Diskurs}
Eine ganz wichtige Rolle nehmen die Kommentare als ein Platz oder Ort ein, wo
sich jeder zum Diskutieren treffen und wo man auch die Breite und
Verschiedenheit der Ansichten erfahren kann. Für den demokratischen Prozess ist
es wichtig, dass bestimmte Themen besprochen werden. Jede Stimme soll gehört
werden, auch diejenigen, die sonst leicht untergehen, wie Minderheiten oder
Leute, die sich nicht trauen, oder weil es an Unterstützung fehlt oder weil die
Angelegenheit einfach kontrovers ist \autocite[S.~12]{santana:2014}. Außerdem
gibt es erst mal keine Beschränkungen, wer mit sprechen darf und wer nicht.  Zu
einer gesunden Demokratie gehört das einfach dazu: \glqq By allowing citizens to
participate, journalists behave ethically, and hence they democratize journalism
and the web.\grqq\- \autocite[S.~125]{singer}

Dieser neu geschaffene öffentliche Ort soll auch dazu dienen, herauszufinden,
wie die Leute im Moment \glqq ticken\grqq\- (\glqq serving as a gauge of society’s pulse\grqq\-
 \autocite[S.~181]{loke}. Man kann herauslesen, wo die neuralgischen Punkte in
der Gesellschaft liegen und auf diesen Zug ausspringen, wenn nötig.

Über die Kommentare können Einspruch erhoben oder Bedenken geäußert werden.
Ebenso funktionieren Kommentare als Anregung für weitere Diskussionen oder
greifen korrigierend  in die Redaktionen ein. Auf diesem Weg erhalten die
Journalisten auch direktes Feedback: \glqq Comments can confirm that the website is
doing a good job [\ldots] they can help improve accuracy [\ldots]\grqq\-
\autocite[S.~105]{reich}

Ufern die Diskussionen jedoch ins Unendliche aus oder werden aggressiv und
beleidigend, dann wird aus diesem liberalen Aspekt ein Problem der Kommentare.
\glqq[\ldots] free expression and exposure to differing views can hold deliberative
potential only when participants were respectful toward each other\grqq\-
\autocite[S.~7]{santana:2011}. Das nächste Kapitel wird sich damit beschäftigen.





\section{Leserbindung}
\begin{quote}
\glqq Newspaper websites compete in a marketplace where a rival news source is simply
a click away, so gaining and retaining the attention of readers is more
important than ever. [\ldots] It’s not just getting the eyes on your site
[\ldots] It’s getting them to stay on your site.\grqq\- \autocite[S.~144]{singer}
\end{quote}

Schaffen es die Online-Redaktionen, dass die Leser dabei bleiben, dann erreichen
sie ihre Ziele. Und die Kommentare sind dabei ein ganz wesentlicher Faktor. Die
Teilnahme am Kommentieren kann sich zu einem Wir-Gefühl entwickeln. Das steigert
wiederum die Besuche auf der Website, sowie das Vertrauen und die
Glaubwürdigkeit der Seite \autocite[S.~215]{meyer-carey}. So bekommen die
Medienhäuser den gewünschten Verkehr im Netz und große Klickzahlen. Die Leser
entwickeln eine gewisse Loyalität gegenüber ihrer Zeitung. Das Medienprodukt
wird zur (begehrten) Marke und zieht positive wirtschaftliche Aspekte mit sich.\\
Alle positiven Aspekte können sich durch eine große Masse an Kommentaren jedoch
auch umkehren, z.B. werden die Leser der Online-Zeitung den Rücken zukehren,
wenn sie von den Äußerungen der anderen regelrecht erschlagen werden.

\section{Neue oder zusätzliche Inhalte}
Die Kommentierenden begeistern sich für ihr Thema, lassen sich als Experten
erkennen, liefern Hinweise und sprechen ganz einfach aus, was ihnen wichtig ist.
Dort kann der Journalist ansetzen und solche Beiträge als \glqq journalistisches
Werkzeug\grqq{} \autocite[S.~133]{robinson} benutzen.\\
Kommentare dienen als Informationsquelle oder als Inspiration für Themen.
Manche Journalisten sehen sie als die Möglichkeit, Kontext herzustellen (durch
Hyperlinks) oder um bestimmte Punkte zu klären. Andere Perspektiven tun sich
auf und der Journalist wird sensibilisiert, was die Leser hören wollen
\autocite[S.~12]{santana:2014}.

Dafür müssen die Schreiber jedoch alle Kommentare durchforsten: entweder nur von
ihren eigenen Texten oder von der gesamten Redaktion. Es liegt nahe, dass dies
sehr schnell ausarten kann. \glqq If a journalist were to read all 300 comments, he
wouldn’t be able to do anything else [\ldots]\grqq\- \autocite[S.~84]{domingo}.  So
kommt es, dass nur wenige Journalisten Kommentare überhaupt lesen
\footnote{\glqq Only 35 percent of national journalists and 36 percent of local
  journalists said they have a positive view of citizens posting news content on
  news organizations' websites.  [\ldots] Even fewer said they take the time to
  read the comments and submissions they receive.\grqq\- \autocite[S.
216]{meyer-carey}}
und dass sie eine schlechte Meinung darüber entwickeln.\footnote{The study finds
  that the more comments a news website receives, the more negative attitudes
  the journalists who work at those sites have toward the value and utility of
  comments. \autocite[S.~214]{meyer-carey}}

\section{Secondary gatekeeper} \label{sec:secondary-gate}
Den Nutzern geht es nicht darum, hauptsächlich neue Inhalte zu schaffen. Sie
bedienen sich nur an der Nachrichtenauswahl und verändern dann die Gewichtung.
Sie suchen sich aus, was sie für gut finden und teilen dies anderen mit. Das
Material wird ein zweites Mal veröffentlicht und verteilt
\autocite[S.~66]{singer:2014}. Damit übernehmen die Nutzer neue Rollen und zwar
die des \glqq active redistributor\grqq, des \glqq active recipient\grqq\- und in einer gewissen
Weise die des \glqq gatekeepers\grqq\- \autocite[S.~57]{singer:2014}. Dadurch, dass ihnen
aber die Autonomie fehlt, was wirklich zur Veröffentlichung frei gegeben wird,
werden sie immer hinter dem eigentlichen \emph{gatekeeper}, dem Journalisten,
stehen. Auch weil sie sich an Regeln halten müssen (wie diese Regeln konkret
aussehen, wird in den folgenden Abschnitten beschrieben). Diese Regeln
bekräftigen den eigentlichen \emph{gatekeeper}, indem sie ihm eine neues \glqq gate\grqq\-
liefern \autocite[S.~13]{santana:2014}.\\
Der Journalist besitzt zu Recht aufgrund seiner Professionalität die
Entscheidungsgewalt. Aber der Nutzer wird zumindest zu einem \glqq secondary
gatekeeper\grqq\- \autocite[S.~5]{santana:2014}, der Inhalte verändern kann, indem er
sie hervorhebt, markiert, mit anderen teilt, weiter verschickt, zugänglich macht
\autocite[S.~57]{singer:2014}.\\
Es gibt im Internet unglaublich viele Nachrichten und Informationen. Diese sind
da, aber irgendwo \glqq da draußen\grqq. Deswegen ist das \glqq
Sichtbarmachen\grqq\- von Nachrichten umso wichtiger geworden.

% vim: set ai si et tw=80 sts=2 ts=2 sw=2:

  \chapter{Probleme mit Kommentaren}

Die Nutzerbeteiligung „Kommentare“ führt allerdings nicht nur zu einer
Bereicherung innerhalb der Medien. So hat süddeutsche.de die Kommentarfunktion
auf ihren Internetseiten entfernt bzw. ausgelagert. Dabei entsteht der Eindruck,
dass die Nachteile die Vorteile überwiegen. Die Entscheidung ist in der
Redaktion sicherlich nicht ohne heftige Diskussion gefallen. Am Ende wurden doch
die nachteiligen Erscheinungen der Kommentare für schwerwiegender empfunden. Was
jedoch wiederum für die Kommentare spricht, ist, dass sie nicht ganz gelöscht,
sondern eben nur auf andere Plattformen verschoben wurden. Ausgewählte
Kommentare schaffen es sogar auf die Nachrichtenseiten. Außerdem kann den
Nachteilen mit entsprechenden Maßnahmen entgegen gewirkt werden. Wie diese
Maßnahmen aussehen, wird dann im darauffolgenden Kapitel erklärt

Was die Probleme genau sind, soll nun betrachtet werden.

\section{Kommentarmanagement ist mit Zeitaufwand verbunden}

Es wurde bereits erwähnt, dass die Kommentare Überhand nehmen und nicht mehr
bewältigt werden können. Manchmal schafft es der Autor nicht, die Kommentare zu
seinem eigenen Artikel zu lesen, geschweige denn, die seiner Mitstreiter. Es
werden in manchen Redaktionen externe Mitarbeiter angestellt, die sich
ausschließlich mit den Kommentaren beschäftigen. Das belastet die Redaktionen
zusätzlich: ``[\ldots] the overall volume of user input demanding to be filtered
or factchecked was a burden.'' \autocite[S.~172]{quandt}


\section{Schlechte Qualität der Kommentare und „incivilty“}

Es ist nicht nur die reine Menge der Kommentare, sondern auch deren mindere
Qualität, die bei den Journalisten schlecht ankommt und Arbeit verursacht. Die
Kommentare müssen gesichtet und sortiert werden. ``Very few of them make
intelligent comments or have intelligent things to say. It's very deceiving
[\ldots]'' \autocite[S.~ 103]{reich}. Und dabei schreiben und lesen sowieso nur
sehr wenige Nutzer Kommentare, wie in Kapitel \ref{kap:nutzerbeteiligung}
bemerkt!

Wenn die Beiträge einfach nur schlecht sind, dann hinterlässt das sicherlich
keinen guten Eindruck und die Qualität der Zeitung leidet insgesamt. Aber
Kommentare arten teilweise aus und es kommt zu beleidigenden oder rassistischen
Ausdrücken und irrationalen Argumenten. Die Kommentierenden erniedrigen und
schuldigen an \autocite[S.~103]{reich}.

Diese Verstöße gegen einen angemessenen Diskussionston (\glqq discursive
civility\grqq{} versus \glqq discursive incivility\grqq\footnote{Hwang defines
  ``discursive civility'' as arguing the justice of one's own view while
  admitting and respecting the justice of others' views. Conversely,
  ``discursive incivility'' is defined as expressing disagreement that denies
and disrespects the justice of others' views \autocite{hwang}
\autocite[S.~6/7]{santana:2014}} werden durch die Anonymität des Internets
gefördert. Die Hemmschwelle etwas zu sagen, das unter die Gürtellinie zielt,
sinkt ganz einfach, wenn man sich nicht zu erkennen geben muss. Andererseits ist
gerade diese Anonymität der Schlüsselfaktor, der zu mehr Beteiligung am Diskurs
führt, siehe Kapitel \ref{kap:nutzerbeteiligung}.

Selten wird nicht nur gegen den guten Ton verstoßen, sondern auch gegen das
Gesetz. Hassrede zum Beispiel ist ein bekanntes Problem und in Deutschland
verboten.

Es gibt bestimmte Themen, die sehr kontrovers sind und die Gemüter erhitzen,
weil verhärtete entgegengesetzte Meinungen aufeinandertreffen. ``Certain news
stories do not receive even a single comment, but those that do [\ldots] could
provide a peek into `the community’s heartbeat'.'' \autocite[S.~181]{loke} Die
Zeitungen beobachten Themengebiete, bei denen es immer wieder zu „discursive
incivility” der Kommentierenden kommt. Dazu gehören vor allem sämtliche
Angelegenheiten über Religion\footnote{Ein aktuelles Beispiel ist der {\slshape
shitstorm}, mit dem sich Christiane Florin auseindersetzen musste, weil sie eine
Anzeige in einem christlichen Blatt ablehnte. Es ging „nur“ um eine Anzeige und
die Journalistin hat die Absage auch begründet. Trotzdem erreichten sie
hasserfüllte Leserbriefe, die sie schließlich öffentlich gemacht hat
\url{http://www.christundwelt.de/detail/artikel/sie-kotzen-mich-an}} und Rasse,
Einwanderung und soziale Themen, obwohl gerade in diesen Bereichen ein
öffentlicher Diskurs und verschiedene Meinungen wichtig sind! Sie sind
notwendig, um sämtliche Aspekte abzubilden. Sonst greift auch hier die
Schweigespirale.\footnote{„[\ldots] would-be commenters might feel remiss to
register their opinion if they perceive themselves to hold a minority opinion
[\ldots]“ \autocite[S.~12]{santana:2014}}

Verschiedene Redaktionen reagieren darauf, indem sie zu Themen, mit denen sie
schlechte Erfahrungen gemacht haben, keine Kommentarfunktion mehr zu lassen oder
sie schließen, wenn die Reaktionen zu hitzig werden
\autocite[S.~4]{santana:2014}. Dem gegenüber gibt es Beobachtungen, wie die von
\textcite{loke}\footnote{``Loke found that online discussions of culturally
sensitive issues, such as race, have allowed newsreaders the opportunity to
amplify socially regressive views where they would not have previously. She
argues that despite the hateful dialogue that some sensitive topics attract, the
forums still serve a useful function in allowing the public, as well as
journalists, to get a glimpse into the consciousness of the community, however
unappealing.'' (Santana, 2014, \autocite[S.~12]{santana:2014}}, dass auch bei
kontroversen Themen die Leser aus den Diskussionen etwas über die Gegner
mitnehmen und ihren Blick auf die andere Seite öffnen. Loke tritt dafür ein,
dass alle Kommentare veröffentlicht werden, weil alle in gewisser Weise etwas
aussagen.

In Deutschland jedenfalls wünschen sich die Journalisten und Nutzer einen
angemessenen Umgangston, was sich in den Beschwerden beim deutschen Presserat
widerspiegelt. Aus diesem Grund werden Maßnahmen zur Regulierung von Kommentaren
ergriffen. Diese Maßnahmen werden im nächsten Punkt erläutert.

% vim: set ai si et tw=80 sts=2 ts=2 sw=2:



\end{document}
