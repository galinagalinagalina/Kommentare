\chapter{Nutzerbeteiligung im Journalismus: vom Leser zum Nutzer zur
Nutzerbeteiligung}

Mit der Digitalisierung ändert sich der Kommunikationsprozess. Der Leser kommt
weg von seiner Rolle als reiner Zuhörer. Er muss nicht mehr warten, bis ihm
etwas angeboten wird. Der Leser wird zum Nutzer der Medienangebote, denen er
sich auch mitteilen kann. Er kann sie nicht nur nutzen, sondern auch benutzen.
Er kann sich richtig in den Kommunikationsprozess einschalten und zwar direkt
und unmittelbar.

\begin{quote}
A great many other people also contribute content, representing their own
interests, ideas, observations and opinions. That content comes in a steadily
expanding volume and variety of forms and formats – words, images and sounds,
alone or in combination (Singer, 2011, S. 1).
\end{quote}

Im Englischen werden u.a. die Ausdrücke „user generated content“, „citizen
journalism“ oder „participatory journalism“ (Singer, 2011, S. 2) dafür
verwendet. Sie alle beinhalten, dass Medienschaffende und Nutzer miteinander
kommunizieren und die Nutzer ihren Teil zur Bildung von Nachrichten und
Gemeinschaft dazu tun.

Ausdrucksformen dieser Nutzerbeteiligung sind vor allem Beiträge in
(Diskussions-)Foren und sozialen Netzwerken, Blogs, Berichte, Beurteilungen,
Bewertungen, Kommentare, hochgeladene Fotos und Videos und noch viele mehr.
„Indeed, new participatory formats appear all the time“ (Singer, 2011, S. 2 und
S. 17).


