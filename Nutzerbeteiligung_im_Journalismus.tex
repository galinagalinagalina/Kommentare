\chapter{Nutzerbeteiligung im Journalismus: die Kommentare}
\label{kap:nutzerbeteiligung}

\section{Vom Leser zum Nutzer zur Nutzerbeteiligung}

Mit der Digitalisierung ändert sich der Kommunikationsprozess. Der Leser kommt
weg von seiner Rolle als reiner Zuhörer. Er muss nicht mehr warten, bis ihm
etwas angeboten wird. Der Leser wird zum Nutzer der Medienangebote, denen er
sich auch mitteilen kann. Er kann sie nicht nur nutzen, sondern auch benutzen.
Er kann sich richtig in den Kommunikationsprozess einschalten und zwar direkt
und unmittelbar.

\begin{quote}
\glqq A great many other people also contribute content, representing their own
interests, ideas, observations and opinions. That content comes in a steadily
expanding volume and variety of forms and formats – words, images and sounds,
alone or in combination.\grqq{} \autocite[S.~1]{participatory}
\end{quote}

Im Englischen werden u.a. die Ausdrücke „user generated content“, „citizen
journalism“ oder „participatory journalism“ \autocite[S.~2]{participatory} dafür
verwendet. Sie alle beinhalten, dass Medienschaffende und Nutzer miteinander
kommunizieren und die Nutzer ihren Teil zur Bildung von Nachrichten und
Gemeinschaft dazu tun.

Ausdrucksformen dieser Nutzerbeteiligung sind vor allem Beiträge in
(Dis\-kus\-si\-ons)\-Foren und sozialen Netzwerken, Blogs, Berichte, Beurteilungen,
Bewertungen, Kommentare, hochgeladene Fotos und Videos und noch viele mehr.
„Indeed, new participatory formats appear all the time.“ \autocite[S.~2 und
S.~17]{participatory}


\section{Kommentare als Nutzerbeteiligung}

In dieser Arbeit stehen die Nutzerkommentare im Mittelpunkt. Als Kommentare
bezeichnet man „views on a story or other online item, which users typically
submit by filling in a form on the bottom of the item.“
\autocite[S.~17]{participatory}.  Sie markieren eine neue Stufe in der
Nutzerbeteiligung und sie sind überaus beliebt, was auch immer wieder in Studien
bestätigt wird \autocite[S.~97]{reich}, siehe auch \ref{sec:beliebtheit}.

Weil die Funktion bei Onlinezeitungen stark genutzt wird, kommt es zu einer Flut
von Material. Mit diesem Material haben die Redaktionen zu kämpfen. Sie sehen
zwar die Vorteile, die die Kommentarfunktion bringt. In der täglichen Arbeit
führt es jedoch immer wieder zu schlechten Erfahrungen.

Als nächstes wird aufgezeigt, was die Kommentare besonders macht, warum sie so
populär sind, was die positiven Aspekte sind und wie sie den
Kommunikationsprozess bereichern. Danach werden die Probleme und die Eingriffe,
das Kommentarmanagement, erörtert.

\begin{quote}
User-generated posts attached to a published item, typically an article or blog
entry, on a media website. Most news organizations moderate or screen user
comments, either before or after publication;
\autocite[S.~204, Glossar]{participatory}
\end{quote}

% vim: set ai si et tw=80 sts=2 ts=2 sw=2:
