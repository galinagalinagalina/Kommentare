

\chapter{Kommentare als Nutzerbeteiligung}

In dieser Arbeit stehen die Nutzerkommentare im Mittelpunkt. Als Kommentare
bezeichnet man „views on a story or other online item, which users typically
submit by filling in a form on the bottom of the item.“ (Singer, 2011, S. 17).


User-generated posts attached to a published item, typically an article or blog
entry, on a media website. Most news organizations moderate or screen user
comments, either before or after publication; Singer, 2011, S. 204, Glossar

Sie markieren eine neue Stufe in der Nutzerbeteiligung und sie sind überaus
beliebt, was auch immer wieder in Studien bestätigt wird (Reich, 2011, S. 97)
(siehe 2.1.1)

Weil die Funktion bei Onlinezeitungen stark genutzt wird, kommt es zu einer Flut
von Material. Mit diesem Material haben die Redaktionen zu kämpfen. Sie sehen
zwar die Vorteile, die die Kommentarfunktion bringt. In der täglichen Arbeit
führt es jedoch immer wieder zu schlechten Erfahrungen.

Als nächstes wird aufgezeigt, was die Kommentare besonders macht, warum sie so
populär sind, was die positiven Aspekte sind und wie sie den
Kommunikationsprozess bereichern. Danach werden die Probleme und Maßnahmen
erörtert.

