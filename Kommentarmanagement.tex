\chapter{Kommentarmanagement}

\section{Was beinhaltet das Kommentarmanagement?}
Im vorherigen Kapitel wurde dargestellt, warum es irgendeiner Form bedarf, die
Kommentare zu regeln. Es kommt eben vor, dass sich Kommentierende nicht an einen
angemessenen Diskussionston halten. Diese Nutzer schaden sowohl den anderen
Nutzern, als auch der Zeitung. Die Nutzerschaft will ja eine angenehme Umgebung
vorfinden \autocite[S.~217]{meyer-carey} und alle wollen \glqq gute\grqq{}
Kommentare lesen, obwohl insgesamt das Diskussionsniveau im Internet gesunken
ist.\footnote{"The Internet has lowered the discourse in general – the brevity,
the speed, this sense of ‘why should I make an effort?' [\ldots] It is
problematic for the society, problematic for the democracy, problematic in every
sense." \autocite[S.~130]{singer}} Die Nutzer sollten aus diesem Grund auf
{\bfseries Empfehlungen oder Richtlinien zum Kommentieren} hingewiesen werden
(\glqq Netiquette\grqq).


\begin{quote}
"[\ldots] a `good' comment entailed one that obeyed the policy rules (i.e. no
ranting or trash talk), stayed on topic (introduced and framed by the news
article), praised the original journalism, confirmed the facts in that story,
and informed other readers by adding new information."
\autocite[S.~134]{robinson}
\end{quote}

Im schlimmsten Fall kann es sogar zu Gesetzesverstößen kommen. Es gibt noch
andere Gründe, die zu einer Reglementierung von Kommentaren führen. Es können
sowohl strategische Faktoren sein, als auch das politische Klima, oder die
Themen selbst. Natürlich spielt auch das journalistische Verständnis und die
Vorgaben innerhalb der Redaktion eine Rolle \autocite[S.~106f]{reich}. Es geht
jedoch immer um die grundsätzliche Frage, ob ein Kommentar \glqq rein\grqq{}
darf oder ob er besser \glqq draußen\grqq{}  bleibt.

Diese kategorische Entscheidung hat natürlich eine negative Komponente.
Kommentare werden so zuerst nach einer Regelverletzung bewertet und nicht nach
ihrem eigentlichen Wert. Außerdem geht diese Art der Bewertung weg von
journalistischen Gesichtspunkten hin zu wirtschaftlichen, wie zum Beispiel das
Erreichen großer Klickzahlen.

Im folgenden Abschnitt werden die Möglichkeiten des Kommentarmanagements
skizziert. Es ist vorab noch zu bemerken, dass jede Zeitung irgendeine Form des
Kommentarmanagements macht und dass die meisten Kommentare zugelassen werden.
Die geschätzten Angaben der interviewten Journalisten aus aller Welt in
\textcite[S.~106]{singer} reichen von 40 bis 90 Prozent der frei gegebenen
Kommentare.

Die zwei Säulen des Kommentarmanagements sind Moderieren und Registrieren. Es
gibt jeweils verschieden Arten der Moderation und verschiedene Möglichkeiten der
Registrierung. Zusätzlich gibt es immer wieder neue Funktionen, die ganz neue
Möglichkeiten schaffen.  Die Nutzer wollen ja vor allem Inhalte teilen und
sichtbar machen (siehe Abschnitt~\ref{sec:secondary-gate}).
Dazu braucht es die technischen Voraussetzungen.

Eine {\bfseries Moderation} ist vor oder nach einer Freischaltung möglich (\glqq
pre-moderation\grqq{} vs. \glqq post-moderation\grqq) und wird entweder vom
Journalisten durchgeführt oder jemand anderem des Medienbetriebs. Sie kann ganz
ausgegliedert werden. Oder der Nutzer übernimmt die Moderation, in manchen
Redaktionen zusammen mit dem Journalisten, als \glqq super-user\grqq{}
\autocite[S.~112]{reich}. Diese Zusammenarbeit (\glqq collaborative
moderation\grqq{} \autocite[S.~109]{reich} geht in Richtung einer
Selbstregulierung von Kommentaren, was als erstrebenswert angesehen werden
kann.\footnote{``We need to tend towards this self-regulation of users by other
users because it is the more logical thing to do. It is them, in the end, who
know what do they want, what information is more useful and what is less, and
what bothers them.'' \autocite[S.~112]{reich}} Auch hier kommt der Nutzer als
\emph{secondary gatekeeper} zum Einsatz.

Derjenige, der die Moderation übernimmt, ist der \glqq comment moderator\grqq{}
\autocite[S.~68]{paulussen}. Er sichtet und sortiert aus. Wichtig für einen
{\slshape comment moderator} ist eine klare Vorgabe, was erlaubt ist und was
nicht. An diese {\bfseries Richtlinien} sollen sich einheitlich die Moderatoren
halten, denn jeder hat unterschiedliche Ansichten über eine mögliche
Grenzziehung.\footnote{Es gibt nicht nur unterschiedliche Ansichten zu der
Qualität, sondern zu Kommentaren überhaupt. \textcite{robinson} unterscheidet
zwischen \glqq traditionalist\grqq{} und \glqq converger\grqq{}. Die ersten
sind eher älter und länger im Unternehmen und sehen sich als Autorität. Sie
wollen eine gewisse Verantwortung des Medienbetriebs auch online aufrecht
erhalten. Die anderen sind eher jünger und kürzer dabei, legen weniger Wert
auf Registrierung und wollen vor allem mit den Nutzern interagieren.} Es gibt
aber auch Überlegungen, ob jeder Journalist für seinen Beitrag selbst die
Kommentierregeln bestimmt \autocite[S.~127]{singer}.

\begin{itemize}
  \item[-] {\bfseries Prä-Moderation}
    bedeutet, dass alles gesichtet wird, bevor es online geht (mit Hilfe
    entsprechender Software). Das entspricht einem klassischen journalistischen
    Verständnis, ist aber mit hohen finanziellen und zeitlichen Kosten
    verbunden.  Diese \glqq proaktive\grqq{} Herangehensweise
    \autocite[S.~108]{reich} kann unter Umständen die Diskussion verzerren. Die
    Journalisten sind damit jedoch auf der sicheren (legalen) Seite. Denn
    unangemessene Kommentare treten auch bei Artikeln auf, von denen es man
    nicht gedacht hätte. Obendrein können \glqq Trolle\grqq{} die Moderation
    stören oder umgehen. Für die Nutzer ist eher demotivierend, denn die meisten
    wollen ihren Beitrag veröffentlicht sehen.\footnote{``You can't really beat
    hitting 'submit' and seeing your comment there before you go away. It
    encourages you to come back. You feel you've engaged.''
    \autocite[S.~109]{reich}}

  \item[-] {\bfseries Post-Moderation}
    findet statt, nachdem der Kommentar bereits veröffentlich wurde, und es
    irgendeinen Grund gibt, einzugreifen. Dieser Umgang mit Kommentaren ist viel
    offener und entspannter und in diesem Fall \glqq reaktiv\grqq. Einer
    Post-Moderation geht jedoch fast immer eine Registrierung voraus. Bei
    heiklen Themen (siehe
    \fxnote*[author=VP]{Referenz richtig?}{Kapitel~\ref{kap:probleme}})
    wird die Post-Moderation allerdings problematisch. Hier kann mit einem
    Umschwenken auf Prä-Moderation dagegen gesteuert werden.
\end{itemize}

Eine {\bfseries Registrierung} ist meistens die Voraussetzung, um kommentieren
zu können (vor allem bei Post-Moderation). Dabei muss man seine persönlichen
Daten (be\-stä\-tig\-te Emailadresse und Klarnamen) angeben, um einen Zugang zu
erhalten. Das verhindert natürlich, dass die Nutzer unkontrolliert drauf los
schreiben. Aber es verhindert ebenso manchen Kommentar, der eigentlich
geschrieben werden möchte. Registrierungen können unterschiedlich gehandhabt
werden. Oft genügt die Angabe einer Emailadresse. Und auch wenn mehr als das
verlangt wird, also die Angabe von Namen, dann kann die tatsächliche Identität
einer Person damit nicht bestätigt werden. Der Name kann erfunden sein (auch bei
Klarnamenpflicht) oder die Nutzer verwenden Spitz- oder Fantasienamen. Ebenfalls
besteht die Befürchtung, dass Kommentare allein deswegen nicht geschrieben
werden, weil man mehr als die Emailadresse angeben muss.  {\bfseries Klarnamen}
schaffen zwar Vertrauen (z. B. die wertvollste Rezension bei Amazon stammt
meistens von Nutzern mit Angabe von vollständigen Namen), machen aber auch Angst
(z.B. vor Ärger, den die Meinungsäußerung bei Nachbarn oder beim Chef
möglicherweise mit sich bringt). Mittlerweile sind die Nutzer jedoch daran
gewöhnt, ihre Meinung zu sagen und ihren Namen zu nennen, den jeder sehen kann,
oft sogar mit einem Foto verbunden (wie bei Facebook oder Twitter).


Es gibt {\bfseries zusätzliche  Funktionen} für das Management von Kommentaren.

\begin{itemize}

  \item[-] Dazu gehört zum Beispiel der  \glqq{\bfseries report abuse button
    (Melden-Button)}\grqq{} \autocite[S.~110f]{reich}. Andere Nutzer können auf
    diese Weise melden, wenn sie unangebrachte Kommentare lesen. In der
    aktuellen Untersuchung von \textcite[S.~63]{singer:2014} verwenden drei
    viertel der Online-Zeitungen diese Funktion.

  \item[-] Nutzer, die bereits negativ aufgefallen sind, weil sie das System
    ausnutzen, werden markiert. Deren Kommentare müssen dann vorher kontrolliert
    oder ganz gestoppt werden, entweder endgültig oder nur für eine bestimmte
    Zeit.

  \item[-] Man bietet eine  {\bfseries freiwillige Registrierung} an, und wer
    das tut, bekommt zusätzliche Privilegien (z.B. mehr Platz zum Kommentieren,
    Kommentare ohne Moderation, Möglichkeit für oder gegen andere Kommentare
    abzustimmen).

  \item[-] Die Option des {\bfseries \glqq Bewerten\grqq} ist überhaupt eine
    gute Lösung, die besten Kommentare zu filtern und die Selbstregulierung zu
    unterstützen. Nutzer können so Kommentare oder andere Kommentatoren bewerten
    und/oder weiter empfehlen \autocite[S.~63f]{singer:2014}.

  \item[-] Hilfreich sind dabei die {\bfseries neuen technischen Möglichkeiten}
    des Kommentarmanagements. Damit können automatisch Nutzerprofile erstellt
    und Kommentare in soziale Netzwerke gestellt oder auf diese verwiesen
    werden. Die Beliebtheit bei anderen Nutzern kann dargestellt werden.
    {\slshape Social Bookmarking} wird möglich. Man kann Artikel {\bfseries
    weiter schicken}. Welche dieser Optionen nutzen die Online-Zeitungen?
    Benutzen die Leser diese {\slshape features}? Erfinden die Redaktionen neue?

   % \fxnote*[author=VP]{Absatz Teil vom letzen Punkt oder besser außerhalb der
    %Auflistung?}{%
    

\section{Umsetzung der Dokumentenanalyse}
    
    Gerade wurde beschrieben, wie ein Kommentarmanagement ablaufen
    kann. Daraus ergeben sich auch die Kategorien (siehe oben fettgedruckt),
    die das Kommentarmanagement ausmachen.
   Der nächste Schritt ist die tatsächliche Analyse des Kommentarmanagements im 
   Internet. 
    
    Dazu wurden die entsprechenden Nachrichtenportale aufgerufen und die
    Kommentarfunktionen aufgesucht. Ich habe mich mit der Online-Zeitung vertraut
    gemacht und mit der Funktion, Kommentare zu schreiben. 
    Dann habe ich versucht, Antworten auf die vorgefertigten Kategorien zu finden,
    entweder durch Beobachten (z.B. Welche Funktionen gibt es rund um den Bereich,
    wo man Kommentare verfassen kann?) oder Ausprobieren (z.B. sich selber
    Registrieren, sich in der Community umschauen) oder Suchen.
    
    Fragen nach Moderation und inhaltlichen Richtlinien konnte ich ausschließlich 
    über die Allgemeinen Geschäftsbedingungen, Nutzungsbedingungen, Richtlinien,
    und/oder Netiquette herausfinden. Diese sind oft bei der Registrierung hervorgehoben,
    und/oder direkt verlinkt. Teilweise muss man den Bedingungen auch zustimmen bei einer
    Registrierung. Einige Online-Zeitungen haben ihre Bedingungen aber auch im 
    \glqq Kleingedruckten\grqq\- stehen.
    
    Ich habe die Informationen benutzt, die von den Nachrichtenportalen veröffentlicht
    worden sind. Bei der Kategorie \glqq Moderation\grqq\- wäre es eigentlich noch besser, 
    die Redaktionen direkt nach ihrer Vorgehensweise zu befragen. Ich finde die 
    verfügbaren Antworten beim Thema Moderation ungenügend,
    gerade was die \glqq stichprobenartige\grqq\- Moderation betrifft: Was heißt stichprobenartig genau?
    Wie viele Stichproben werden gemacht? Sitzt da jetzt jemand, ein Moderator, oder macht übernimmt  
    ein Computer die Aufgabe?
    
    Bei der Beschäftigung mit dem Kommentarmanagement auf den entsprechenden Internetseiten
    bin ich auch auf Funktionen gestoßen, die nicht bei Punkt 5.1
    besprochen worden sind, z.B.  \glqq alternative Anmeldung\grqq\- anstelle von Registrierung.
    Dies ist im Sinne der Dokumentenanalyse, die offen für neue Kategorien ist.
    
    Nachdem die ganzen Informationen zum Kommentarmanagement gesammelt waren, 
    habe ich Tabellen für die Kategorien angelegt und die Ergebnisse eingetragen.
    Während des Eintragens und auch danach habe ich versucht, die Ergebnisse zu 
    vereinheitlichen, damit ein Vergleich und Überblick der Online-Zeitungen möglich ist. 
    
    Zu Beginn der Arbeit war es ein Ziel, alle Kategorien in einer Tabelle darzustellen.
    Dies ist jedoch durch den begrenzten Platz nicht möglich. Es sind also mehrere
    Tabellen entstanden. Am Schluss gibt es eine gekürzte Zusammenfassung von ausgewählten
    Kategorien.


    
   % Die Ergebnisse werden tabellarisch wieder gegeben und entsprechend
    %kommentiert.
    
    
    
    
    
    

\end{itemize}


% vim: set ai si et tw=80 sts=2 ts=2 sw=2:
