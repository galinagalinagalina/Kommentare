Im vorherigen Kapitel wurde dargestellt, warum es irgendeiner Form bedarf, die Kommentare zu regeln. 
Es kommt eben vor, dass sich Kommentierende nicht an einen angemessenen Diskussionston halten. Diese Nutzer schaden sowohl den anderen Nutzern, als auch der Zeitung. Die Nutzerschaft will ja eine angenehme Umgebung vorfinden (Meyer \& Carey, 2014, S 217) und alle wollen "gute" Kommentare lesen, obwohl insgesamt das Diskussionsniveau im Internet gesunken ist.\footnote{"The Internet has lowered the
discourse in general – the brevity, the speed, this sense of ‘why should I make
an effort?' [\ldots] It is problematic for the society, problematic for the democracy, problematic in every sense." (Singer, 2011, S. 130)}
Die Nutzer sollten aus diesem Grund auf {\bf Empfehlungen oder Richtlinien zum Kommentieren} hingewiesen werden ("Netiquette"). 


\begin{quote}
"[\ldots]a ‘good’ comment entailed one that obeyed the policy rules (i.e. no
ranting or trash talk), stayed on topic (introduced and framed by the news article), praised
the original journalism, confirmed the facts in that story, and informed other readers by
adding new information." (Robinson, 2010, S. 134)
\end{quote}

Im schlimmsten Fall kann es sogar zu Gesetzesverstößen kommen. Es gibt noch andere Gründe, die zu einer Reglementierung von Kommentaren führen. Es können sowohl strategische Faktoren sein, als auch das politische Klima, oder die Themen selbst. Natürlich spielt auch das journalistische Verständnis und die Vorgaben innerhalb der Redaktion eine Rolle (Reich, 2011, S. 106f).
Es geht jedoch immer um die grundsätzliche Frage, ob ein Kommentar \glqq rein" darf oder ob er besser \glqq draußen"  bleibt.

Diese kategorische Entscheidung hat natürlich eine negative Komponente. Kommentare werden so zuerst nach einer Regelverletzung bewertet und nicht nach ihrem eigentlichen Wert. Außerdem geht diese Art der Bewertung weg von journalistischen Gesichtspunkten hin zu wirtschaftlichen, wie zum Beispiel das Erreichen großer Klickzahlen.  

Im folgenden Abschnitt werden die Möglichkeiten des Kommentarmanagements skizziert. Es ist vorab noch zu bemerken, dass jede Zeitung irgendeine Form des Kommentarmanagements macht und dass die meisten Kommentare zugelassen werden. Die geschätzten Angaben der interviewten Journalisten aus aller Welt in Singer (2011) reichen von 40 bis 90 Prozent der frei gegebenen Kommentare (S. 106). 


Die zwei Säulen des Kommentarmanagements sind Moderieren und Registrieren. Es gibt jeweils verschieden Arten der Moderation und verschiedene Möglichkeiten der Registrierung. Zusätzlich gibt es immer wieder neue Funktionen, die ganz neue Möglichkeiten schaffen. Inzwischen geht es nicht mehr nur darum, bestimmte Kommentare auszusortieren. Die Nutzer wollen ja als gatekeeper Inhalte teilen und sichtbar machen. Dazu braucht es die technischen Voraussetzungen.

Eine {\bf Moderation} ist vor oder nach einer Freischaltung möglich (\glqq pre-moderation" vs. \glqq post-moderation") und wird entweder vom Journalisten durchgeführt oder jemand anderem des Medienbetriebs. Sie kann ganz ausgegliedert werden. Oder der Nutzer übernimmt die Moderation, in manchen Redaktionen zusammen mit dem Journalisten, als \glqq super-user" (Reich, 2011, S. 112). Diese Zusammenarbeit ("collaborative moderation" (Reich, 2011, S. 109) geht in Richtung einer Selbstregulierung von Kommentaren, was als erstrebenswert angesehen werden kann.\footnote{\glqq We need to tend towards this self-regulation of users by
other users because it is the more logical thing to do. It is them, in the end, who
know what do they want, what information is more useful and what is less, and
what bothers them.” (Reich, 2011, S. 112)}
Auch hier kommt der Nutzer als gatekeeper zum Einsatz.



Derjenige, der die Moderation übernimmt, ist der \glqq comment moderator" (Paulussen, 2011, S. 68). Er sichtet und sortiert aus. Wichtig für einen {\slshape comment moderator} ist eine klare Vorgabe, was erlaubt ist und was nicht. An diese {\bf Richtlinien} sollen sich einheitlich die Moderatoren halten, denn jeder hat unterschiedliche Ansichten über eine mögliche 
Grenzziehung. \footnote{Es gibt nicht nur unterschiedliche Ansichten zu der Qualität, sondern zu Kommentaren überhaupt. Robinson (2010) unterscheidet zwischen "traditionalist \grqq und "converger". Die ersten sind eher älter und länger im Unternehmen und sehen sich als Autorität. Sie wollen eine gewisse Verantwortung des Medienbetriebs auch online aufrecht erhalten. Die anderen sind eher jünger und kürzer dabei, legen weniger Wert auf Registrierung und wollen vor allem mit den Nutzern interagieren.} Es gibt aber auch Überlegungen, ob jeder Journalist für seinen Beitrag selbst die Kommentierregeln bestimmt (Singer, 2011, S. 127).  






\begin{itemize}
\item[-] {\bf Prä-moderation} bedeutet, dass alles gesichtet wird, bevor es online geht (mit Hilfe entsprechender Software). Das entspricht einem klassischen journalistischen Verständnis, ist aber mit hohen finanziellen und zeitlichen Kosten verbunden. Diese \glqq proaktive" Herangehensweise (Reich, 2011, S. 108) kann unter Umständen die Diskussion verzerren. Die Journalisten sind damit jedoch auf der sicheren (legalen) Seite. Denn unangemessene Kommentare treten auch bei Artikeln auf, von denen es man nicht gedacht hätte. Obendrein können \glqq Trolle" die Moderation stören oder umgehen. Für die Nutzer ist  eher demotivierend, denn die meisten wollen ihren Beitrag veröffentlicht sehen. \footnote{\glqq You can’t really beat hitting ‘submit’ and seeing your comment there before you go
away. It encourages you to come back. You feel you’ve engaged." (Reich, 2011, S. 109)}
\item[-] {\bf Post-moderation} findet statt, nachdem der Kommentar bereits veröffentlich wurde, und es irgendeinen Grund gibt, einzugreifen. Dieser Umgang mit Kommentaren ist viel offener und entspannter und in diesem Fall \glqq reaktiv". Einer Post-Moderation geht jedoch fast immer eine Registrierung voraus. Bei heiklen Themen (siehe label) wird die Post-Moderation allerdings problematisch. Hier kann mit einem Umschwenken auf Prä-Moderation dagegen gesteuert werden.
\end{itemize}
Eine {\bf Registrierung} ist meistens die Voraussetzung, um kommentieren zu können (vor allem bei Post-Moderation). Dabei muss man seine persönlichen Daten (bestätigte Emailadresse und Klarnamen) angeben, um einen Zugang zu erhalten. Das verhindert natürlich, dass die Nutzer unkontrolliert drauf los schreiben. Aber es verhindert ebenso manchen Kommentar, der eigentlich geschrieben werden möchte. 
Registrierungen können unterschiedlich gehandhabt werden. Oft genügt die Angabe einer Emailadresse. Und auch wenn mehr als das verlangt wird, also die Angabe von Namen, dann kann die tatsächliche Identität einer Person damit nicht bestätigt werden. Der Name kann erfunden sein (auch bei Klarnamenpflicht) oder die Nutzer verwenden Spitz- oder Fantasienamen. Ebenfalls besteht die Befürchtung, dass Kommentare allein deswegen nicht geschrieben werden, weil man mehr als die Emailadresse angeben muss. 
{\bf Klarnamen} schaffen zwar Vertrauen (z. B. die wertvollste Rezension bei Amazon stammt meistens von Nutzern mit Angabe von vollständigen Namen), machen aber auch Angst (z.B. vor Ärger, den die Meinungsäußerung bei Nachbarn oder beim Chef möglicherweise mit sich bringt). Mittlerweile sind die Nutzer jedoch daran gewöhnt, ihre Meinung zu sagen und ihren Namen zu nennen, den jeder sehen kann, oft sogar mit einem Foto verbunden (facebook, twitter). 


Es gibt {\bf zusätzliche  Funktionen} für das Management von Kommentaren. 
\begin{itemize}
\item[-] Dazu gehört zum Beispiel der  \glqq{\bf report abuse button}" (Reich, 2011, S. 110f). Andere Nutzer können auf diese Weise melden, wenn sie unangebrachte Kommentare lesen. In der aktuellen Untersuchung von Singer (2014, S. 63) verwenden drei viertel der Online-Zeitungen diese Funktion.  
\item[-] Nutzer, die bereits negativ aufgefallen sind, weil sie das System ausnutzen, werden markiert. Deren Kommentare müssen dann vorher kontrolliert oder ganz gestoppt werden, entweder endgültig oder nur für eine bestimmte Zeit. 
\item[-] Man bietet eine  {\bf freiwillige Registrierung} an, und wer das tut, bekommt zusätzliche Privilegien (z.B. mehr Platz zum Kommentieren, Kommentare ohne Moderation, Möglichkeit für oder gegen andere Kommentare abzustimmen). 
\item[-] Die Option des  {\bf "Voting\grqq} ist überhaupt eine gute Lösung, die besten Kommentare zu filtern und die Selbstregulierung (siehe \glqq Moderation") zu unterstützen. Nutzer können so Kommentare oder andere Kommentatoren bewerten und/oder weiter empfehlen (Singer, 2014, S. 63f).
 \item[-] Hilfreich sind dabei die {\bf neuen technischen Möglichkeiten} des Kommentarmanagements. Damit können automatisch Nutzerprofile erstellt und Kommentare in soziale Netzwerke gestellt oder auf diese verwiesen werden. Die Beliebtheit bei anderen Nutzern kann dargestellt werden. {\slshape Social Bookmarking} wird möglich. Man kann Artikel weiter \bf verschicken. Welche dieser Optionen nutzen die Online-Zeitungen? Benutzen die Leser diese {\slshape features}? Erfinden die Redaktionen neue?

\end{itemize}


Some newspapers also had begun requiring users to
comment through Facebook, an intriguing step that removes many of the problems created
by anonymous postings while also helping generate social network traffic. Such
trends richly deserve the attention that journalism scholars have begun to afford them.
(Singer, 2014, S. 69)




