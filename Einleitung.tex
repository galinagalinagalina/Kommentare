
\chapter{Einleitung}

\section{Relevanz des Themas}

\glqq Here comes everybody\grqq, so lautet der Satz von Clay Shirky, um die neue
Medienwelt zu beschreiben: Jeder kommt also dran! Jeder kann sich heute schnell
und direkt in den Kommunikationsprozess einklinken.  Und niemand muss mehr
hoffnungsvoll warten, ob es der Leserbrief über die Hürde der Redaktionen
schafft und veröffentlicht wird.  Die Kommentarfunktion ist eine Möglichkeit des
sich \glqq Einschaltens\grqq. Man kann dabei ohne größeren Aufwand zu einem
bestimmten Thema etwas sagen und loswerden, was einem beim Lesen eines Artikels
durch den Kopf schießt.

Es ist also geschafft, was sich sowohl Leser als auch Medienschaffende gewünscht
haben. \glqq Journalists’ desire to maintain communication with readers has been
historically, and remains today one of the basic tenets of journalism\grqq{}
\autocite[S.~67]{santana:2011}.%(Santana, 2011, S. 67).

Der Leser beteiligt sich an der journalistischen Arbeit, er nimmt aktiv am
demokratischen Prozess teil.  Aber dadurch, dass man sich jederzeit einschalten
kann, kommt es zu einer Flut von Äußerungen, deren Qualität nicht garantiert
ist. Die Redakteure sind hin und her gerissen zwischen Nutzen und Nutzlosigkeit.
Sie sehen auch, dass Kommentare auf ihren Plattformen schädlich für das Ansehen
der Zeitung sein können.  Ein einziger \glqq schlechter\grqq\- Kommentar macht
50 gute wertlos, berichtet ein Redakteur in
\textcite[S.~131]{participatory}.

Eine Regulierung ist unumgänglich. Darüber tauchen viele Fragen und Diskussionen
unter den Journalisten zur Organisation von Kommentaren auf.

Das größte Problem stellt der Missbrauch seitens der Kommentierenden dar, die
unter dem Schutz der Anonymität beleidigen und beschimpfen. Das geht
mittlerweile soweit, dass sogar der deutsche Presserat Handlungsbedarf sieht, da
sich Beschwerden mittlerweile zu 60\% auf online erschienene Texte beziehen:

\begin{quote}
„Aber die Zunahme von Beschwerden, unter anderem zu Leserkommentaren und
Online-Archiven, zeigt, dass wir die Publizistischen Grundsätze an einigen
Stellen ergänzen sollten, um den digitalen Ver\-öffentlichungs\-for\-men besser
gerecht zu werden.“ Jahrespressekonferenz, 19.02.2014, Presserat.
\end{quote}

Süddeutsche.de hat sich Mitte des Jahres 2014 für Einschnitte entschieden
\autocite{mark}, um dann die Kommentare ganz weg zu lassen \autocite{fromm},
genauso wie \url{reuters.de} \autocite{standard}.


Wie sieht es bei den anderen deutschen Online-Zeitungen aus? Wie gehen die
Redaktionen mit den Kommentaren um? Lassen sie den Nutzern komplett freie Hand?
Ist eine Anmeldung erforderlich (mit oder ohne Klarnamenpflicht) oder bleiben
die Kommentare anonym? Werden die Kommentare moderiert (vorher oder danach)?
Gibt es Verhaltensrichtlinien für die Nutzer? Wo ist Kommentieren möglich? Gibt
es Verweise auf social media?


\section{Forschungsziel- und fragen}

Mit dieser Arbeit soll der Frage nachgegangen werden, wie das
Kommentarmanagement der größten Nachrichtenportale jeweils aussieht. Es wird
eine Bestandsaufnahme gemacht, wie die Redaktionen Anfang des Jahres 2015 die
Kommentare handhaben und welche Richtlinien dort vorgegeben werden. In diesem
Zuge wird ein Überblick des aktuellen Kommentarmanagements entstehen.

\begin{itemize} \em
  \item FF1: Wie werden die Nutzerkommentare auf deutschen Nachrichtenportalen
    gehandhabt?
  \item FF2: An welche Vorgaben müssen sich die Nutzer halten?
\end{itemize}


\section{Methodische Umsetzung und Kategorien}

Da es sich beim Analysematerial um schriftliche Dokumente handelt, wird zur
methodischen Umsetzung dieses Überblicks eine Dokumentenanalyse herangezogen.
Das heißt auch, dass eine qualitative Inhaltsanalyse gemacht wird, in Abgrenzung
zur quantitativen Inhaltsanalyse, die eigentlich zur Analyse großer Mengen an
Medieninhalten entwickelt wurde. Nur von qualitativer Inhaltsanalyse zu sprechen
lehnt \textcite{kunzler} ab, da es zu kurz greift bzw.~\textcite{mayring} diesen
„Begriff (bereits) für drei von ihm entwickelte Auswertungsverfahren
[Zusammenfassung, Strukturierung bzw.~Kategorisierung, Explikation] geprägt hat“.

Hier liefern die jeweiligen Nachrichtenseiten das Material. Dieses Material wird
bestimmten Kategorien, welche auf Grundlage des theoretischen Vorwissens
angelegt wurden, zugeordnet. Die Kategorien müssen vorher genau definiert werden
(siehe \textcite{mayring}, deduktive Kategorienanwendung).

Es wird aber auch mit Kategorien gearbeitet, die nicht empirisch gehaltvoll
sind, die also zunächst \glqq leer\grqq{} sind (\glqq heuristische
Rahmenkonzepte\grqq{} oder \glqq sensitizing concepts\grqq{}).  Sie dienen als
\glqq Platzhalter\grqq{}, um sich erst später in empirisch gehaltvolle
Kategorien zu entwickeln. Sie dienen als \glqq Starthilfe", da die Dokumente
erst entdeckt und verstanden werden müssen. Somit wird das deduktive Verfahren
mit dem induktiven kombiniert.

Bei solchen heuristischen Rahmenkonzepten gibt es also eine Kombination von
theoretischem \glqq Fachwissen des Forschers\grqq{} \autocite[S.~65]{strauss}
und empirisch generiertem Wissen. Es ist eine Zusammensetzung von zwei Typen von
Kategorien \autocite[S.~64f]{strauss}, den soziologisch konstruierten und den
natürlichen Kodes. Die soziologisch konstruierten Kodes entstehen erst im Laufe
der Beschäftigung mit dem Material, die anderen werden vorausgesetzt.


\section{Auswahl der Nachrichtenportale}

Im Sinne der Forschungsfrage und der Dokumentenanalyse wird nicht zufällig auf
irgendwelche Portale zugegriffen. Nicht die statistische Repräsentativität ist
das Ziel der Analyse, sondern das kriteriengesteuerte Sammeln von Dokumenten
\autocite{kunzler}.

Es kommen die \glqq größten\grqq{} Online-Zeitungen in die Auswahl, d.h. die
Portale mit den meisten Besuchern. Auf diesen Portalen gibt es, meistens am Ende
von Artikeln, das Angebot, Kommentare zu verfassen. Die Rahmenbedingungen und
Funktionalitäten, um dort seine Meinung wieder zu geben, sind das Material für
die Dokumentenanalyse (Quellenbeschreibung).

Die Quellenbeschreibung ist Teil der Quellenkritik, die wiederum ein
Gütekriterium der Dokumentanalyse ist. Weitere Bestandteile der Quellenkritik
sind: Die Nachrichtenportale sind im Internet frei verfügbar und für jeden
zugänglich (Textsicherung). Die Nachrichtenportale werden in einem bestimmten
Zeitraum betrachtet (äußere Kritik).

Für den vergleichenden Überblick werden deutsche Online-Zeitungen mit den
meisten Besuchern ausgewählt. Diese Zeitungen stellen den Nutzern jeweils eine
Kommentarfunktion für ihre Beiträge zur Verfügung.  Diese Funktionen werden dann
analysiert.

Die IVW listet sämtliche Online-Angebote nach der Gesamt-Besucherzahl auf.
Laut den Messungen 1/2015 sind die \glqq Top 10 Online-Zeitungen\grqq{} folgende:

\href{http://www.Bild.de}{Bild.de},
\href{http://www.spiegel.de}{SPIEGEL ONLINE},
\href{http://www.FAZ.NET}{FAZ.NET},
\href{http://www.focus.de}{FOCUS ONLINE},
\href{http://www.welt.de}{Die Welt}, 
\href{http://www.derwesten.de}{Der Westen},
\href{http://www.stern.de}{Stern},
\href{http://www.rp-online.de}{RP ONLINE},
\href{http://www.Handelsblatt.com}{Handelsblatt.com},
\href{http://www.südkurier.de}{SÜDKURIER Online},
\href{http://www.zeit.de}{ZEIT ONLINE}, 
\href{http://www.Badische-Zeitung.de}{Badische Zeitung},
\href{http://www.stuttgarter-zeitung.de}{Stuttgarter Zeitung},
\href{http://www.merkur-online.de}{Merkur},
\href{http://www.hna.de}{HNA},
\href{http://www.mopo.de}{Hamburger Morgenpost},
\href{http://www.mainpost.de}{Mainpost},
\href{http://www.tagesspiegel.de}{Tagesspiegel},
\href{http://www.swp.de}{Südwestpresse},
\href{http://www..de}{Augsburger Allgemeine}

% vim: set ai si et tw=80 sts=2 ts=2 sw=2:
