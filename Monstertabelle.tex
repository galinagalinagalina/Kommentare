\begin{landscape} \small
\begin{tabular}{ccc}

\hline
%Zeile 1: Portale
		&
		%Spalte 1
		bild.de &
		%Spalte 2
		spiegel.de &
		%Spalte 3
		faz.net &
		%Spalte 4
		focus.de 
		% Spalte 5
		diewelt.de 
		% Spalte 6
		derwesten.de
		% Spalte 9
		rp-online
		% Spalte 10
		handelsblatt
		% Spalte 11
		suedkurier
		% Spalte 12
		zeit.de
		% Spalte 13
		badische zeitung
		% Spalte 14
		stuttgarter zeitung
		% Spalte 15
		merkur
		% Spalte 16
		hna
		% Spalte 17
		mopo
		% Spalte 18
		mainpost
		% Spalte 19
		tagesspiegel
		% Spalte 20
		swp
		\\ \hline

% Zeile 2
Kommentare\footnote{werden in der Regel mit Sprechblasen angekündigt; mit Angabe der Anzahl der abgegebenen K.; auch eine Zeitangabe ist üblich entweder mit Datum und Uhrzeit oder ``vor ... Stunden''; K. stehen unter dem Artikel in vielen Fällen nach Werbung (in eigener Sache)  \\
abweichende Angaben und Sonstiges
&		% Spalte 1
		\\
		K. nicht zu allen Artikeln möglich 
		&
		% Spalte 2
		\\
		Startseite: keine Sprechblase sondern Hinweis auf [Forum]; beim Beitrag: Sprechblase mit Ausrufungszeichen\\
		K. werden durchnummeriert
		&
		% Spalte 3
		\\
		zeitliche Begrenzung um K. zu schreiben; Hinweis, dass die K. im Internet recherchierbar sind; K. = Lesermeinung
		&
		% Spalte 4
		 \\
		 auch Videos kommentierbar 
		&
		% Spalte 5
		\\
		über Disqus verwaltet; Schließung nach zwei/drei Tagen oder früher bei Regelverstößen
		&
		% Spalte 6
		\\
		keine Ankündigung der K., keine Sprechblasen; alle Beiträge können kommentiert werden; K. werden durchnummeriert
		&
		% Spalte 9
		\\
		fast jedes Thema kommentierbar; 
		&
		% Spalte 10
		Kommentierzeiten: 7.30 - 21 Uhr, bis zu sieben Tage lang; K. werden (u.U. gekürzt) multimedial verbreitet
		\\
		&
		% Spalte 11
		\\
		K. nicht zu allen Artikeln möglich
		&
		% Spalte 12
		\\
		keine Sprechblase; Hinweis mit  [Anzahl + Kommentare]; K. werden durchnummeriert;  K. nicht zu allen Artikeln möglich; Definition von K.: ``kürzere Textbeiträge, die Sie unter vorhandenen Artikeln, Videos, Fotostrecken oder anderen Multimedia-Inhalten abgeben können''
		&
		% Spalte 13
		&
		% Spalte 14
		für alle Nutzer, K. geben nicht die Meinung der Stuttgarter Zeitung wieder
		&
		% Spalte 15
		über Disqus verwaltet, für alle Nutzer
		&
		% Spalte 16
		über Disqus verwaltet, für alle Nutzer; freie Meinungsäußerung; bei Verstößen Aufruf eine Email zu schreiben 
		&
		% Spalte 17
		über Disqus verwaltet; Kommentierzeiten: 8 - 21 Uhr
		&
		% Spalte 18
		K. unter Artikel und neben Artikel (als neuer Tab!); Problem bei Mainpost.de: begrenzter Zugang, nur bestimmte Zahl an Artikeln im Monat kostenlos lesbar
		&
		% Spalte 19
		keine Sprechblase; Hinweis mit [Anzahl + Kommentare]
		&
		% Spalte 20
		kein Hinweis auf K.; Problem bei swp: begrenzter Zugang, nur bestimmte Zahl an Artikeln im Monat kostenlos lesbar
		&
		
		\\ \hline
		
		
		&
		
Moderation: bei Verstößen/Missbrauch/NIcht-Beachten der Regelen
&		% Spalte 1
		keine; Verantwortung der Inhalte beim Nutzer; Verstöße bitte melden; bei Verstößen Entfernen der Beiträge und Sperren des Nutzers\\ 
		&
		% Spalte 2
		Prä-Moderation (zeitliche Verzögerungen); Forumsmoderator mit Namen sysop/forum@spiegel.de; bei Verstößen Entfernen/keine Veröffentlichung der Beiträge; Redaktion kann bearbeiten, verschieben, Diskussionen schließen; keine Benachrichtigung bei Nicht-Erscheinen; Urheberreicht beim Nutzer; Verstöße melden
		&
		% Spalte 3
		Prä-Moderation (zeitliche Verzögerungen) nach Prüfung den Richtlinien entsprechend; bei Verstößen Sperren des Nutzers; K. werden evtl. gekürzt/verändert
		&
		% Spalte 4		
		eingeschränkte Prä-Moderation: keine umfassende Prüfung, aber Stichproben vom Digitalvermarkter; Moderator: TOMORROW FOCUS Media GmbH/TOMORROW FOCUS NEWS+ GmbH/Beauftragte; bei Verstößen Ändern/Entfernen der Beiträge und Sperren des Nutzers
		&
		% Spalte 5
		Prä-Moderation (zeitliche Verzögerungen); Moderator = Team von ``Welt''-Mitarbeitern; Kritik an Moderationsweise per Email; 				bestimmte Moderationszeiten, Moderation gemäß Nutzungsregeln; bei Verstößen (vom Beiträgen/Benutzernamen/Profilfoto) Ändern der Beiträge/keine Veröffentlichung; Beleidigungen/Beschimpfungen werden entfernt; Sperren des Nutzers (zeitweise/dauerhaft) 
		&
		% Spalte 6
		keine, automatische Veröffentlichung; bei Missbrauch entfernt Community Management Beiträge; keine inhaltliche Prüfung: Hinweis, dass Beiträge falsche Tatsachen enthalten, Rechte 	Dritter verletzen, in die Irre führen, täuschen können; Pflichten der Nutzer in Bezug auf ihre Beiträge; bei Verstößen Entfernen der Beiträge (zeitweise/ganz)
		&
		% Spalte 9
		keine; bei Melden von Missbrauch erfolgt Löschen der Beiträge/Sperrung des Nutzers; auch die Antworten dazu; Abbruch der K.-Funktion bei Verstößen oder bei nicht themenbezogen K. (bis dahin veröffentlichte Beiträge bleiben); Beiträge werden nach eigenem Ermessen entfernt/bearbeitet
		% Spalte 10
		eingeschränkte Post-Moderation: keine umfassende Prüfung, aber Eingriff bei Verstößen;  sofortiges Löschen des Beitrags oder Sperren des Nutzers bei Beanstandungen; man setzt sich mit dem Nutzer in Verbindung; nicht eindeutige Sachverhalte müssen abgeklärt werden; verstoßen einzelnen Abschnitte gegen Regeln, werden diese entfernt und der Eingriff kenntlich gemacht; wird Kommentar komplett entfernt wird dies auch kenntlich gemacht; verstoßen viele Kommentare gegen die Regeln wird die Funktion deaktiviert
		&
		% Spalte 11
		eingeschränkte Post-Moderation: keine umfassende Prüfung; alle Rechte zur Verbreitung der Inhalte müssen beim Nutzer liegen; Bearbeitung des Textes oder komplettes Löschen bei Verstößen; Löschen wenn nicht themenbezogen; Löschen von Trollen; Nutzer, die regelmäßig gegen Regeln verstoßen werden per Email ermahnt; schwere/wiederholte Verstöße führen zum Ausschluss der Community; bei gehäuften Verstößen wird Funktion abgeschaltet; Verantwortung der Inhalte beim Nutzer; keine systematische Prüfung externer Links; keine Verantwortung von südkurier für externe Links; Löschen externer Links bei Verstößen oder bei Links, die auf rechtswidrige Inhalte weiterleiten
		&
		% Spalte 12
		Prä-Moderation (wie bei Leserbriefen); Kürzen/Löschen bei Missbrauch mit Begründung, warum eingeschritten wurde mit Anmerkungen und Kennzeichnungen; schwere/wiederholte Verstöße führen zur Sperrung; Abbruch der K.-Funktion bei gehäuften Verstößen/nicht themenbezogen; Verantwortung der Inhalte beim Nutzer (d.h. Rechte zur Verbreitung liegen beim Nutzer, sicherstellen, dass fremde Inhalte zur Verbreitung freigegeben sind, beachten, dass Gedichte urheberrechtlich geschützt sind); keine Verantwortung von zeit.de für verlinkte Inhalte; Löschen externer Links bei Verstößen
		&
		% Spalte 13
		eingeschränkte Post-Moderation: keine umfassende Prüfung; Verantwortung der Inhalte beim Nutzer; Bearbeitung des Textes oder komplettes Löschen bei Verstößen; schwere/wiederholte Verstöße führen zum Ausschluss der Community; keine Crosspostings; Löschen externer Links bei Verstößen oder bei Links, die auf rechtswidrige Inhalte weiterleiten
		&
		% Spalte 14
		Prä-Moderation; Bearbeitung des Textes oder komplettes Löschen/Sperrung des Nutzers wenn nötig; zeitweise/gänzliche Sperrung des Nutzers; Verantwortung der Inhalte beim Nutzer
		&
		% Spalte 15
		keine; Verantwortung der Inhalte beim Nutzer; bei Missbrauch erfolgt Löschen des kompletten Beitrags (kein Bearbeiten)/Sperren des Nutzers; Kommentarlose Löschung wenn nicht themenbezogen/privater Chat
		&
		% Spalte 16
		keine; Verantwortung der Inhalte beim Nutzer; kommentarloses Löschen bei Missbrauch/Sperren des Nutzers
		&
		% Spalte 17
		keine
		&
		% Spalte 18
		Prä-Moderation
		&
		% Spalte 19
		Prä-Moderation: Rahmen für sachlichen Austausch von Argumenten schaffen
		&
		% Spalte 20
		keine/eingeschränkte Post-Moderation; Verantwortung der Inhalte beim Nutzer; Versuch unerwünschte Beiträge fernzuhalten; keine umfassende Überprüfung möglich; Löschen bei Missbrauch 
		&
		\\ \hline
		
		
Registrierung/Anmeldung/Login  \\
Bedingungen (außer Email und Passwort)
 &		% Spalte 1
		bei BILD.de-Community über Anmeldedienst ``mypass'', über Facebook Account; Profil anlegen möglich\\
		Klarname, (angemessener) Benutzername (angezeigt), Volljährigkeit bzw. Einverständnis der Erziehungsberechtigten bei Minderjährigen
		&
		% Spalte 2
		bei ''mein spiegel'' als Abonnent oder Nicht-Abonnent, Facebook, automatisches Login möglich\footnote{Internetbrowser merkt sich das Login mit Cookie, welches den Benutzernamen und das Passwort enthält} \\
		Klarname, Benutzername (angezeigt)
		&
		% Spalte 3
		bei ''Mein FAZ.NET'' mit\\
		Klarname, Profilbild möglich
		&
		% Spalte 4
		bei FOCUS online, Facebook\\
		Klarname (Hinweis: im gesamten Internet recherchierbar)
		&
		% Spalte 5
		bei WELT DIGITAL über Anmeldedienst ``mypass'', Disqus, Facebook, Twitter, Google\\
		Klarname, Benutzername, Profilfoto möglich;`als Gast schreiben
		Whitelist: Nutzer können ohne Moderation kommentieren (Redaktion und Community-Mitglieder, die auffallend positiv 					kommentieren)
		&
		% Spalte 6
		auf derwesten.de mit\\
		Klarname, Benutzername (angezeigt), jeder ist zugangs- und teilnahmeberechtigt
		&
		% Spalte 9
		bei mein RP ONLINE mit\\
		Benutzername; auch Minderjährige, wenn sie sich über Nutzung bewusst sind bzw. mit Einverständnis der Erziehungsberechtigten
		% Spalte 10
		auf handelsblatt.de mit\\
		Klarname, Volljährigkeit
		&
		% Spalte 11
		auf suedkurier.de mit \\
		Klarname, Benutzername, Anrede, Land, Adresse (Angaben werden auf Richtigkeit geprüft (teilweise auch telefonisch); Einverständnis nach sechs Monaten Inaktivität wird Registrierung/Benutzername gesperrt/freigegeben
		&
		% Spalte 12
		auf zeit.de mit\\
		Benutzername; Profil anlegen möglich
		&
		% Spalte 13
		bei Meine BZ mit\\
		Klarname, Anrede, Profilbild (optional); Löschen des Accounts bei Wegwerf-Emailadresse
		&
		% Spalte 14
		auf stuttgarterzeitung.de, über Facebook Account \\
		Klarname
		&
		% Spalte 15
		bei Merkur-Online, Disqus, Facebook, Twitter, Google\\
		Benutzername (erwünscht, aber angemessen), Profilbild (optional; angemessen, Verbot von rassistischen, pornografischen, menschenverachtenden, beleidigenden oder gegen die guten Sitten verstoßenden Abbildungen); als Gast schreiben
		&
		% Spalte 16
		mit HNA-Login, Disqus\\
		Benutzername (angemessen, nicht beleidigend/anstößig), Profilbild (optional; Urheberrecht beachten/Recht auf eigenes Bild; keine Werbung/Logos; kein beleidigendes, verletzendes Motiv; Verbot von rassistischen, pornografischen, menschenverachtenden, beleidigenden oder gegen die guten Sitten verstoßenden Abbildungen)
		&
		% Spalte 17
		\\
		Disqus, Facebok, Twitter, Google
		&
		% Spalte 18
		auf Mainpost.de
		Klarname, Benutzername (nicht beleidigend/ehrverletzend/hetzerisch), Adresse
		&
		% Spalte 19
		auf tagesspiegel.de
		Benutzername (angemessen); Profil anlegen möglich (Profilbild optional/angemessen), persönliche Nutzer-Statistik)
		&
		% Spalte 20
		Klarnamen, Benutzername, (Sicherheitsfrage)
		&
		\\ \hline
		
		
Artikel teilen auf sozialen Netzwerken (Facebook ``teilen'' und/oder ``empfehlen'', Twitter, g+) und durch Versenden (Briefsymbol)	
&		% Spalte 1
		tumbl, Pinterest\\
		K. gleichzeitig auf Facebook veröffentlichen möglich 
		&
		% Spalte 2
		Xing, LinkedIn, Tumbl, Pinterest, deli.cio.us, Diggy, reddit
		&
		% Spalte 3
		&
		% Spalte 4		
		Facebook ``gefällt mir'', Xing
		&
		% Spalte 5 
		&
		% Spalte 6
		&
		% Spalte 9
		&
		% Spalte 10
		Xing, Email schreiben
		&
		% Spalte 11
		&
		% Spalte 12
		(keine Symbole/Verweise auf Startseite)
		&
		% Spalte 13
		(keine Symbole/Verweise auf Startseite) kein g+; Versenden (kein Briefsymbol), Verlinken
		&
		% Spalte 14
		&
		% Spalte 15
		Youtube; Briefsymbol bedeutet Feedback geben (kein Versenden)
		&
		% Spalte 16
		&
		% Spalte 17
		&
		% Spalte 18
		Double Click notwendig für Teilen auf sozialen Netzwerken, Versenden nur nach Registrierung möglich
		&
		% Spalte 19
		Double Click notwendig für Teilen auf sozialen Netzwerken, Versenden mit Sicherheitsabfrage
		&
		% Spalte 20
		&
		\\ \hline
		
		
Wo Regeln/Hinweise zum Kommentieren stehen (bei Nutzungsbedingungen, Richtlinien, AGB) und ob eine Netiquette vorhanden ist:
&		% Spalte 1
		Nutzungsbedingungen: allgemeine und besondere (Zustimmung verlangt bei Registrierung); Netiquette
		&
		% Spalte 2
		Nutzungsbedingungen: allgemeine und für Foren (Zustimmung)
		&
		% Spalte 3
		Nutzungesbedingungen: allgemein (Zustimmung); ``wie Sie mit diskutieren''-Button
		&
		% Spalte 4
		AGB (Zustimmung), Netiquette (Zustimmung)
		&
		% Spalte 5
		Nutzungsbedingungen, veraltete Netiquette
		&
		% Spalte 6
		Nutzungsbedingungen (Zustimmung), Netiquette
		&
		% Spalte 9
		AGB
		&
		% Spalte 10
		Nutzungshinweise (Zustimmung), Netiquette 
		&
		% Spalte 11
		Nutzungsbedingungen (Zustimmung), Netiquette
		&
		% Spalte 12
		AGB (Zustimmung), Netiquette (besonders ausführlich und erklärend)
		&
		% Spalte 13
		AGB (Zustimmung), Netiquette
		&
		% Spalte 14
		AGB (Zustimmung), Kommentarregeln = Netiquette
		&
		% Spalte 15
		AGB (Zustimmung), Netiquette (Zustimmung)
		&
		% Spalte 16
		AGB (Zustimmung), Netiquette (Zustimmung)
		&
		% Spalte 17
		\\
		&
		% Spalte 18
		Netiquette 
		&
		% Spalte 19
		Richtlinien für Community
		&
		% Spalte 20
		Netiquette 
		&
		\\ \hline
		
		
Kommentar: formale Regeln\\
Zeichenbegrenzung\\
Überschrift (Pflichtfeld)\\
Sonstiges 
&		% Spalte 1
		\\
		keine\\
		keine\\
		vorsichtig mit Großbuchstaben, Zitate kennzeichnen	
		&
		% Spalte 2
		\\
		keine\\
		optional\\
		keine Bilder posten, keine langen Kopien von Quellen (Links verwenden)\\
		&
		% Spalte 3
		\\
		1000 Zeichen\\
		ja, 100 Zeichen\\
		\\
		&
		% Spalte 4
		\\
		800 Zeichen\\
		ja\\
		reiner Text ohne besondere Kennzeichnungen (z.B. keine Smilies, Hervorhebungen, Chat-Symbole, nur Kleinschreibung, usw.), korrektes Deutsch, auf Rechtschreibung/Interpunktion achten, Absätze machen
		&
		% Spalte 5
		\\
		keine\\
		keine\\
		Zitate kennzeichnen; keine Fremdsprachen, keine Links zu externen Webseiten (seriöse Ausnahmen möglich)\\
		&
		% Spalte 6
		\\
		keine\\
		keine\\
		&
		% Spalte 9
		\\
		keine\\
		Betreff\\
		deutsche Sprache; Links möglich (keine Links zu Werbung/strafbare Inhalte)
		&
		% Spalte 10
		\\
		2000\\
		keine\\
		Großbuchstaben (Schreien) vermeiden; Absätze machen und strukturieren; Wortwahl überprüfen; auf Rechtschreibung achten; 				Zitate/Quellen kennzeichnen
		&
		% Spalte 11
		\\
		1000 Zeichen\\
		ja\\
		Zitate kennzeichnen mit Quellenangabe\\
		&
		% Spalte 12
		\\
		1500 Zeichen\\
		ja, mindestens 5 Zeichen\\
		Absätze machen; auf Rechtschreibung achten; vorsichtig mit Großbuchstaben; Zitate kennzeichnen, wenig verwenden, Quellenangabe machen, nur als Ergänzung verwenden; Links möglich \\
		&
		% Spalte 13
		\\
		keine\\
		keine\\
		\\
		&% Spalte 14
		\\
		keine\\
		Betreff\\
		Links möglich (keine Links zu Werbung/kommerziellen Angeboten/Chats/Foren/strafbaren Inhalten\\
		&% Spalte 15
		\\
		keine\\
		keine\\
		deutsche Sprache; Links nicht erwünscht (falls doch distanziert sich merkur-online von Inhalten der gelinkten Seiten)\\
		&
		% Spalte 16
		\\
		keine\\
		keine\\
		deutsche Sprache; Beiträge in Fremdsprachen werden gegebenenfalls entfernt, da für größten Teil der Nutzer nicht verständlich; Links nicht erwünscht (falls doch distanziert sich hna.de von Inhalten der gelinkten Seiten)\\
		&
		% Spalte 17
		\\
		keine\\
		keine\\
		\\
		&
		% Spalte 18
		\\
		1000 Zeichen\\
		ja\\
		auf deutsche Rechtschreibung achten; korrekte Interpunktion, Absätze machen; Hervorhebungen möglich: fett, kursiv, unterstrichen; markieren von Links, Zitaten möglich; einfügen von Emoticons möglich(Grinsen, Zwinkern, traurig sein); \\
		&% Spalte 19
		\\
		2000 Zeichen\\
		Titel\\
		auf Rechtschreibung/Grammatik achten; Hervorhebungen möglich: fett, kursiv; markieren von Links, Zitaten\\
		&
		% Spalte 20
		\\
		3000 Zeichen\\
		Betreff\\
		\\
		&
		\\ \hline
		
		
Kommentar: inhaltliche Regeln und Hinweise
\footnote{Kommentare sollen immer themenbezogen sein. Keine kommerzielle/werbliche Nutzung. keine Beleidigungen. keine Beiträge mit sexistischen/sittenwidrigen/pornographischen/obszönen/grob anstößigen und rassistischen Inhalten. Keine werblichen/kommerziellen Inhalte}
&		% Spalte 1
		sachlich, höflich bleiben, andere respektieren, nicht dagegen argumentieren, Angriffe versuchen zu ignorieren; wie man selbst behandelt werden möchte, keine unangemessenen Beiträge wie Beschimpfungen/Belästigungen/Drohungen/Diskriminierungen, keine Beiträge mit nicht-themenbezogen/antisemitische Inhalten; keine privaten Angaben\footnote{Angaben von Postadresse und/oder Telefonnummer und/oder Emailadresse) oder Angaben über Dritte verbreiten; keine automatisierte Nutzung; Urheberrechte beachten; kein Mobbing; keine Links zu Werbung/Chats/Foren; Datensicherung vom Nutzer selbst; keine Trolle; kein Spam
		&
		% Spalte 2
		faire, sachliche, angenehme, offene, freundschaftliche, respektvolle Diskussion (auch bei Streit), obwohl es sich um verbale Auseinandersetzung handeln soll; keine Beiträge mit strafbaren/inakzeptablen/nicht-themenbezogenen Inhalten; 
		&
		% Spalte 3
		keine Beiträge mit links- und rechtsradikalen/verleumdnerischen/ruf- und geschäftsschädigenden Inhalten;  keine falschen/nicht nachprüfbaren Behauptungen
		&
		% Spalte 4
		sachliche, freundliche, respektvolle, tolerante Diskussion, keine Fremdtexte; keine privaten Angaben (Adresse, Telefonnummer); keine Links zu Werbung; keine Diskriminierungen jeder Art\footnote{Diskriminierungen aufgrund von Herkunft, Nationalität, Religion, sexueller Orientierung, Alter, Geschlecht, usw.}; keine Beiträge mit  demagogischen Inhalten; keine Schadsoftware verwenden, kein Missbrauch von Daten; Datensicherung vom Nutzer selbst; 
		&
		% Spalte 5
		fair, höflich, verständlich, kritische Kommentare erwünscht, keine Beschimpfungen/Diskriminierungen/Provokationen/Entwürdigungen/Aufruf zu Demonstrationen oder Gewalt; Gesetze beachten; keine Angaben über Dritte verbreiten; keine Hasspropaganda
		&
		% Spalte 6
		keine Beschimpfungen/Kränkungen; das Forum ist kein Veranstaltungskalender/keine Terminankündigungen; Zitate müssen Quellenangabe enthalten; keine Beiträge mit gewaltverherrlichenden/antisemitischen/gesetzeswidrigen Inhalten
		&
		% Spalte 9
		keine Beiträge mit nicht-themenbezogen/sinnlosen Inhalten/Inhalten, die die Diskussion stören; sich so verhalten, wie man selbst behandelt werden möchte; keine Anschuldigungen/Tatsachenbehauptungen; keine Beiträge mit strafbaren Inhalten/üble Nachrede/Urheberrechtsverstöße/Drohungen/volksverhetzende Äußerungen/Aufforderung zu Gewalt; Verbot von Inhalten, die dem Ansehen von Verstorbenen und deren Angehörigen schaden könnten/die doppeldeutig sind oder anderweitige Darstellungen, deren Rechtswidrigkeit vermutet wird, aber nicht abschließend festgestellt werden kann; keine unwahren/unsachlichen Beiträge; keine Verletzung von Urheberrechten/Rechten Dritter; keine Schadsoftware
		&
		% Spalte 10
		mit zynischen/ironischen Äußerungen vorsichtig sein; nicht persönlich werden; keine persönlichen Angriffe/Diskriminierungen jeder Art; keine Beiträge mit verleumderischen/ruf-/geschäftsschädigenden/diskriminierenden/strafrechtlich relevanten Inhalten; keine Veröffentlichung Daten Dritter; sich bewusst machen, welche eigenen Daten frei zugänglich ins Internet gestellt werden; guter Ton; ignorieren von Provokationen/Trollen; keine Junkmails/Spam/Scraping/sonstige rechtswidrige Kommunikationsformen; keine Nennung von Produktnamen/Dienstleistern/Marken/Produzenten
		&
		% Spalte 11
		faire, sachliche, offene, gehaltvolle Diskussion; kein nicht-belegbaren Behauptungen/Verleumdungen/Diffamierungen/Drohungen/Diskriminierungen aller Art/Hetze/Gewaltverherrlichung/Vulgärausdrücke; keine Beiträge mit ruf-/geschäftsschädigenden Inhalten; keine Veröffentlichung Daten Dritter
		&
		% Spalte 12
		Durchlesen vor Abschicken, guter Umgangston, nicht provozieren lassen, mit zynischen/ironischen Äußerungen vorsichtig sein; keine Diskriminierungen aller Art (auch Behinderung/Einkommensverhältnisse)/Diffamierungen/Verleumdungen; nicht prüfbare Unterstellungen/Verdächtigungen; nachvollziehbare Aussagen; keine Diskriminierungen; keine Beiträge mit ruf-/geschäftsschädigenden Inhalten; keine Verletzung von Rechten Dritter/von geltendem Recht;  keine Veröffentlichung Daten Dritter; sich bewusst machen, welche persönlichen Daten frei zugänglich werden; keine Schadsoftware
		&
		% Spalte 13
		sachliche, niveauvolle, faire, offene Diskussion; freundlich, tolerant sein; guter Umgangston, andere so behandeln, wie man es selber möchte; keine persönlichen Angriffe; keine Beiträge mit vulgären/hetzerische/gewaltverherrlichende Inhalt; keine privaten Daten; Urheberrechte beachten
		&
		% Spalte 14
		engagiert, fair; akzeptable, respektvolle Wortwahl; sachkritisch, seriös; keine Schmähungen/Diskriminierungen/Volksverhetzung/Propaganda, keine Beiträge mit jugendgefährdenden/antisemitischen/strafbaren/verleumderischen/ruf-/geschäftsschädigenden/menschenverachtenden/gegen die guten Sitten verstoßenden Inhalten; keine Trolle; keine Junkmails/Spam/Kettenbriefe; keine privaten Daten; keine Veröffentlichung Daten Dritter; Urheberrechte beachten; nur für private Zwecke (keine Vervielfältigung)
		&
		% Spalte 15
		sachlich, freundlich, verständlicher Umgangston, man soll Spaß haben und sich wohl fühlen,  keine persönlichen Angriffe/Beschimpfungen, andere Meinungen akzeptieren; keine Beiträge mit unwahren/unsachlichen/jugendgefährdenden/verleumderischen/verfassungsfeindlichen/extremistischen/illegalen/ethisch-moralisch-problematischen/ Inhalten; keine Schadsoftware; keine privaten Daten; nur für private Zwecke; keine Junkmails/Spam/Kettenbriefe 
		&
		% Spalte 16
		 sachlich, freundlich, verständlicher Umgangston, man soll Spaß haben und sich wohl fühlen, keine Beschimpfungen/persönlichen Angriffe; andere Meinungen akzeptieren; keine Beiträge mit unwahren/unsachlichen/jugendgefährdenden/strafbaren/verleumderischen/verfassungsfeindlichen/extremistischen/illegalen/ethisch-moralisch-problematischen Inhalten; keine Schadsoftware; keine privaten Daten
		&
		% Spalte 17
		keine
		&
		% Spalte 18
		faire, akzeptable, respektvolle Wortwahl; keine verbalen Angriffe, privaten Details aus dem Leben anderer; mit zynischen/ironischen Äußerungen vorsichtig sein; kein Aufruf zu Straftaten; keine privaten Daten; keine Beiträge mit ehrverletzenden/gewaltverherrlichenden Inhalten/Aufrufen zur Gewalt; Löschen von Beiträgen, die gegen das Gesetz verstoßen
		&
		% Spalte 19
		sachlich; fairer Umgang, angenehme Atmosphäre;  bewusst machen, dass K. öffentlich werden; keine Angaben, die nicht für die Öffentlichkeit sind; keine pauschale/persönliche Herabwürdigung; keine obszönen, pietätlosen, menschenverachtenden, gewaltverherrlichenden Inhalte; keine Verleumdungen; keine Trolle; keine Beleidigungen/Unterstellungen
		&
		% Spalte 20
		sachlich, fair, freundlich, wie man selbst behandelt werden möchte; keine beleidigenden, rassistischen, sexistischen, vulgären, hetzerischen oder gewaltverherrlichenden Töne; keine privaten Adressen;
		&
	
		\\ \hline
		
		
Funktionen im Kommentar	\\
``Melden''\\
``Bewerten''\\
``Antworten''\\
Sonstiges\\
&		% Spalte 1
		\\
		ja, mit Angabe von vier Möglichkeiten (Spam, Copyright, beleidigend, anderer Grund), kurze Begründung möglich\\
		positiv\\
		\\
		Ordnen nach beliebteste, älteste, neueste K.
		&
		% Spalte 2
		\\
		ja\\
		\\
		ja: auf was man antwortet wird in Zitate gesetzt\\
		\\
		&
		% Spalte 3
		\\
		ja\\
		positiv\\
		dem Kommentator folgen\\
		&
		% Spalte 4
		\\
		ja, besonders hervorgehoben\\
		positiv und negativ\\
		ja\\
		\\
		&
		% Spalte 5
		Oberfläche von Disqus\\
		Fähnchen-Button (nicht immer sichtbar)\\
		positiv (Klicks werden gezählt), nach Anmeldung oder als ``Gast'' möglich; negativ (nach Anmeldung)\\
		ja\\
		``Teilen''-Button (= diese Diskussion auf Twitter/Facebook teilen, nach Anmeldung dort), ``Teilen''-Button für einzelnen K., 		``Empfehlen''-Button (= diese Diskussion empfehlen); Ordnen nach beste, neueste, älteste Kommentare; ``Abonnieren'', um Mittleiungen dieser Diskussion per Email zu erhalten; Disqus auf der eigenen Seite hinzufügen; Datenschutz\\
		&
		% Spalte 6
		\\
		ja\\
		\\
		ja, mit Anzahl Antworten\\
		\\
		&
		% Spalte 9
		\\
		ja (mit Begründung mit Name/Email, auch ohne Registrierung möglich)\\
		positiv\\
		\\
		Ordnen nach älteste; Button für ``mehr K.''\\
		% Spalte 10
		\\
		ja (auch Email oder telefonisch möglich)\\
		\\
		ja\\
		\\
		&
		% Spalte 11
		\\
		ja (mit Name, Emailadresse, Grund an Community Manager)\\
		\\
		ja\\
		Ordnen nach älteste, neueste, beste Bewertung; zur Auswahl: ``informiert bleiben'' (bei jedem neuen Beitrag der Diskussion erhält man Benachrichtigung)\\
		&
		% Spalte 12
		\\
		ja\\
		\\
		ja\\
		Reaktionen/Antworten auf diesen K. anzeigen; Ordnen nach neuesten, empfohlenen, allen K.; Empfehlungen aussprechen; Empfehlungen der Redaktion (bedeutet aber nicht, dass die Redaktion der Meinung des Lesers zustimmt)\\
		&
		% Spalte 13
		\\
		ja\\
		\\
		\\
		\\
		&% Spalte 14
		\\
		positiv und negativ\\
		ja\\
		ja\\
		Ordnen nach älteste, neueste K.; Empfehlen\\
		&% Spalte 15
		Oberfläche von Disqus\\
		Fähnchen-Button (nicht immer sichtbar)\\
		positiv (Klicks werden gezählt), nach Anmeldung oder als ``Gast'' möglich; negativ (nach Anmeldung)\\
		ja\\
		Empfehlen; Ordnen nach beste, neueste, älteste K.; ``Abonnieren'', um Mittleiungen dieser Diskussion per Email zu erhalten; Disqus auf der eigenen Seite hinzufügen; Datenschutz \\
		&
		% Spalte 16
		Oberfläche von Disqus\\
		Fähnchen-Button (nicht immer sichtbar)\\
		positiv (Klicks werden gezählt), nach Anmeldung oder als ``Gast'' möglich; negativ (nach Anmeldung)\\
		ja\\
		Empfehlen; Ordnen nach beste, neueste, älteste K.; ``Abonnieren'', um Mittleiungen dieser Diskussion per Email zu erhalten; Disqus auf der eigenen Seite hinzufügen; Datenschutz\\
		&
		% Spalte 17
		Oberfläche von Disqus\\
		Fähnchen-Button (nicht immer sichtbar)\\
		positiv (Klicks werden gezählt), nach Anmeldung oder als ``Gast'' möglich; negativ (nach Anmeldung)\\
		ja\\
		Empfehlen; Ordnen nach beste, neueste, älteste K.; ``Abonnieren'', um Mittleiungen dieser Diskussion per Email zu erhalten; Disqus auf der eigenen Seite hinzufügen; Datenschutz\\
		&
		% Spalte 18
		\\
		ja, mit Begründung mit Name/Email wegen Rückfragen\\
		\\
		ja\\
		Ordnen nach älteste, neueste, best bewertete K.\\
		&
		% Spalte 19
		\\
		\\
		\\
		ja; zur Auswahl: Antworten anzeigen\\
		Ordnen nach neueste, älteste K., chronologisch\\
		&
		% Spalte 20
		\\
		ja\\
		positiv\\
		ja\\
		\\
		&
		\\ \hline
		
		
Kommentar in der Community/Forum	
& 		% Spalte 1
		Ranglisten der Nutzer; Chronologie der Kommentare der Nutzer; kommentiert letzte 24 h (Top 5)
		&
		% Spalte 2
		meistkommentierte Themen (Top 5); eigene Beiträge anzeigen; Sichtbarmachen der Emailadresse für andere Teilnehmer
		&
		% Spalte 3
		jünste/älteste Lesermeinung, viel/wenig diskutiert, viel/wenig empfohlen, TOP-Argumente; K. verwalten/von der Veröffentlichung zurückziehen
		&
		% Spalte 4
		Chronologie; aktivste Kommentatoren (des Monats, gesamt, top 50), Kommentar des Tages; Videofavoriten der Leser 					(meistkommentiert, top 20)
		&
		% Spalte 5
		DIE WELT auf Disqus: neueste K.; Top Kommentatoren
		&
		% Spalte 6
		&
		% Spalte 9
		&
		% Spalte 10
		&
		% Spalte 11
		&
		% Spalte 12
		&
		% Spalte 13
		&
		% Spalte 14
		keine Community
		&
		% Spalte 15
		Merkur Online Community auf Disqus: neueste K.; Top Kommentatoren
		&
		% Spalte 16
		HNA Community auf Disqus: neueste K.; Top Kommentatoren
		&
		% Spalte 17
		mopo.de Community auf Disqus: neueste K.; Top Kommentatoren
		&
		% Spalte 18
		Auswahl auf Profilseite: niemandem/allen/nur Mitgliederen zeigen
		&
		% Spalte 19
		&
		% Spalte 20
		&
		
		\\ \hline

		
Sonstiges\\ 
Funktionen beim Artikel\\
Felder auf Startseite\\
Außergewöhnliches
&		% Spalte 1
		\\
		Korrektur-Button\footnote{Formular zum Versenden an die Redaktion mit Hinweisen auf Fehler oder anderes}\\
		\\
		``Reaktionen'' möglich: Lachen, Weinen, Wut, Staunen, Wow (zur Auswahl, welche Reaktion man zu dem Beitrag empfindet)
		&
		% Spalte 2
		\\
		Button ``merken'' (auf die Merkliste im Benutzerprofil setzen); Button ``feedback'' (Feedback an die Redaktion über Formular) \\
		\\
		\\
		&
		% Spalte 3
		\\
		Beitrag empfehlen, Beitrag merken, Permalink, Drucker\\
		\\
		sämtliche Buttons/Symbole/Funktionen mit Hilfe/Erklärungen
		&
		% Spalte 4
		\\
		Fehler-Melden; Beitrag ``Bewerten'' mit Sternen (Anzahl Bewertungen)\\
		\\
		Leserbericht schreiben (zusätzlich zum Kommentar mit mehr Zeichen (4000), persönliche Erfahrungen), Kommentare abonnieren\\
		&
		% Spalte 5
		\\
		\\
		\\
		Hinweis bei Löschung: ``Dieser Kommentar wurde entfernt'' (Antworten darauf aber noch sichtbar)\\
		&
		% Spalte 6
		\\
		\\
		\\
		\\
		&
		% Spalte 9
		\\
		Beitrag empfehlen, Drucken, Schriftgröße ändern\\
		\\
		Kontakt mit der Zeitung über Email an den Chefredakteur, Newsletter, Leserbrief schreiben (über Formular)
		&
		% Spalte 10
		\\
		Beitrag ``Merken''\\
		\\
		\\
		&
		% Spalte 11
		\\
		\\
		``Meistkommentiert''  (Top 3), \\
		Leserreporter-Beitrag schreiben \\
		&
		% Spalte 12
		\\
		Drucken, als PDF speichern\\
		``Meistgelesen''/''Meistkommentiert'' (Top 5)\\
		Leserartikel schreiben = ausführliche Meinungsbeiträge und Erfahrungsberichte (meistgelesene/meistkommentierte Leserartikel, Top 3 auf Leserartikel-Seite), Debattenkultur: ``Aus den Kommentaren'' (Höhepunkte aus den Leserdebatten mit neuer Fragestellung), Kommentarkultur: ``Bitte weichen Sie vom Thema ab'' (Experiment: Kommentieren ohne Artikel), Empfehlungen bei Facebook (aktuelle Empfehlungen aus Facebook-Freundeskreis), Tweets von ZEIT ONLINE Politik\\
		&
		% Spalte 13
		\\
		Drucken, Vorlesen, Fehler-Melden\\
		``Meistkommentiert'' (Top 5)/''zuletzt kommentiert''\\
		Nutzer registriert seit [...] + Anzahl der bereits geschriebenen K. vom Nutzer; Vorschau möglich: man kann K. sehen, wie er online aussehen wird\\
		&
		% Spalte 14
		\\
		\\
		\\
		\\
		&
		% Spalte 15
		\\
		\\
		\\
		\\
		&
		% Spalte 16
		\\
		\\
		\\
		\\
		&
		% Spalte 17
		\\
		\\
		\\
		\\
		&
		% Spalte 18
		\\
		zur Auswahl: ``ich möchte bei neuen K. per Email benachrichtigt werden'' \\
		``aktuelle Leserkommentare'', ``kommentiert'' (Top 5), Teilen auf Youtube\\
		Anzahl der bereits geschriebenen K. vom Nutzer, Kontakt mit Redaktion, Sicherheitsfrage\\
		&
		% Spalte 19
		\\
		Newsletter abonnieren, Drucken, Lesezeichen setzen\\
		\\
		Community-Statistik\\
		&
		% Spalte 20
		\\
		\\
		\\
		\\
		&
		
		\\ \hline

\end{tabular}
\end{landscape}
