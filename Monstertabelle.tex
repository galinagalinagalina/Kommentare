\begin{landscape} \small
\begin{tabular}{ccc}

\hline
%Zeile 1: Portale
		&
		%Spalte 1
		bild.de &
		
		%Spalte 2
		spiegel.de &
		
		%Spalte 3
		faz.net &
		
		%Spalte 4
		focus.de 
		
		% Spalte 5
		diewelt.de 
		
		% Spalte 6
		derwesten.de
		\\ \hline

% Zeile 2
Kommentare\footnote{werden in der Regel mit Sprechblasen angekündigt; mit Angabe der Anzahl der abgegebenen Kommentare; auch eine Zeitangabe ist üblich entweder mit Datum und Uhrzeit oder ``vor ... Stunden'' \\
abweichende Angaben und Sonstiges
&
		% Spalte 1
		\\
		Kommentare nicht zu allen Artikeln möglich 
		&
		
		% Spalte 2
		\\
		Startseite: keine Sprechblase sondern Hinweis auf [Forum]; beim Beitrag: Sprechblase mit Ausrufungszeichen\\
		Kommentare werden durchnummeriert
		&
		
		% Spalte 3
		\\
		zeitliche Begrenzung um Kommentar zu schreiben\\
		Hinweis, dass die Kommentare im Internet recherchierbar sind\\
		Kommentar = Lesermeinung
		&
		
		% Spalte 4
		 \\
		 auch Videos kommentierbar 
		&
		
		% Spalte 5
		\\
		über Disqus verwaltet\\
		Schließung nach zwei/drei Tagen oder früher bei Regelverstößen
		&
		
		% Spalte 6
		keine Ankündigung der Kommentare, keine Sprechblasen\\
		alle Beiträge können kommentiert werden\\
		Kommentare werden durchnummeriert
		\\ \hline
		
		
		&
		
Moderation 
&
		% Spalte 1
		keine \\ 
		&
		
		% Spalte 2
		Prä-Moderation (zeitliche Verzögerungen)\\
		Forumsmoderator mit Namen SYSOP
		&
		
		% Spalte 3
		Prä-Moderation (zeitliche Verzögerungen)
		&
		
		% Spalte 4		
		keine, aber Stichproben vom Digitalvermarkter
		&
		
		% Spalte 5
		Prä-Moderation (zeitliche Verzögerungen); Moderator = Team von ``Welt''-Mitarbeitern; Kritik an Moderationsweise per Email; 				bestimmte Moderationszeiten, Moderation gemäß Nutzungsregeln
		&
		
		% Spalte 6
		keine, automatische Veröffentlichung, keine inhaltliche Prüfung: Hinweis, dass Beiträge falsche Tatsachen enthalten, Rechte 				dritter verletzen, in die Irre führen, täuschen können; Pflichten der Nutzer in Bezug auf ihre Beiträge
		
		\\ \hline
		
		
Registrierung/Anmeldung/Login  \\
Bedingungen (außer Email und Passwort)
 &	
		% Spalte 1
		BILD.de-Community über Anmeldedienst ``mypass'', Facebook\\
		Klarname, Benutzername (angemessen), Volljährigkeit bzw. Einverständniserklärung bei Minderjährigen
		&
		
		% Spalte 2
		''mein spiegel'' als Abonnent oder Nicht-Abonnent, Facebook, automatisches Login möglich\footnote{Internetbrowser merkt sich 				das Login mit Cookie, welches den Benutzernamen und das Passwort enthält}\\
		Klarname, Benutzername
		&
		
		% Spalte 3
		''Mein FAZ.NET''\\
		Klarname (Nennung bei Kommentaren), Benutzername möglich für andere Aktionen
		&
		
		% Spalte 4
		bei FOCUS online, Facebook\\
		Klarnamen (ausdrücklicher Hinweis: Name wird im gesamten Internet recherchierbar)
		&
		
		% Spalte 5
		WELT DIGITAL über Anmeldedienst ``mypass'', Disqus, Facebook, Twitter, g+, ``ich schreibe als Gast''; anonymes Kommentieren 			möglich unter Pseudonym; Whitelist: Nutzer 	können ohne Moderation kommentieren (Redaktion und Community-Mitglieder, die 				auffallend positiv kommentieren)\\
		Klarname, Profilfoto möglich
		&
		
		% Spalte 6
		Klarname, Benutzername, jeder ist zugangs- und teilnahmeberechtigt
		
		\\ \hline
		
		
soziale Netzwerke (außer Facebook ``teilen'', Twitter, g+) 	
&
		% Spalte 1
		Facebook ``Empfehlen'', tumbl, Pinterest\\
		Kommentar gleichzeitig auf Facebook veröffentlichen möglich 
		&
		
		% Spalte 2
		Xing, LinkedIn, Tumbl, Pinterest, deli.cio.us, Diggy, reddit
		&
		
		% Spalte 3
		Facebook ``Empfehlen''
		&
		
		% Spalte 4		
		Facebook ``gefällt mir'', Xing
		&
		
		% Spalte 5
		nur Facebook ``Empfehlen'' 
		&
		
		% Spalte 6
		Facebook ``Empfehlen''
		\\ \hline
		
		
Nutzungsbedingungen \\
Richtlinien \\
Netiquette \\
AGB\\
&
		% Spalte 1
		allgemeine und besondere (Zustimmung)\\
				\\
		vorhanden \\
		\\
		&
		
		% Spalte 2
		allgemeine und für Foren (Zustimmung)\\
		\\
		\\
		\\
		&
		
		% Spalte 3
		allgemein (Zustimmung)\\
		``wie Sie mit diskutieren''-Button\\
		nein\\
		\\
		&
		
		% Spalte 4
		\\
		\\
		Zustimmung\\
		Zustimmung
		&
		
		% Spalte 5
		vorhanden\\
		\\
		veraltet\\
		\\
		&
		
		% Spalte 6
		Zustimmung\\
		\\
		ja\\
		\\
		
		\\ \hline
		
		
Kommentar: formale Regeln\\
Zeichenbegrenzung\\
Überschrift\\
Sonstiges 
&
		% Spalte 1
		\\
		keine\\
		keine\\
		Großbuchstaben vermeiden, Zitate kennzeichnen	
		&
		
		% Spalte 2
		\\
		keine\\
		optional\\
		\\
		&
		
		% Spalte 3
		\\
		1000 Zeichen\\
		ja, 100 Zeichen\\
		\\
		&
		
		% Spalte 4
		\\
		800 Zeichen\\
		ja\\
		reiner Text ohne besondere Kennzeichnungen (z.B. keine Smilies, Hervorhebungen, Chat-Symbole, nur Kleinschreibung, usw.) 
		&
		
		% Spalte 5
		\\
		keine\\
		keine\\
		Zitate kennzeichnen; keine Fremdsprachen\\
		&
		
		% Spalte 6
		\\
		keine\\
		keine\\
		
		\\ \hline
		
		
Kommentar: inhaltliche Regeln und Hinweise
&
		% Spalte 1
		sachlich, höflich bleiben, andere respektieren, nicht dagegen argumentieren, keine unangemessenen Beiträge (= Beleidigungen, 			Beschimpfungen, Belästigungen, Drohungen, Diskriminierungen, Spam, Schadsoftware, Trolle, nicht themenbezogen) 
		&
		
		% Spalte 2
		angenehmes Diskussionsklima, fair, sachlich bleiben, obwohl es sich um verbale Auseinandersetzung handeln soll
		&
		
		% Spalte 3
		&
		
		% Spalte 4
		korrektes Deutsch nach Rechtschreibung, Interpunktion; keine Fremdtexte; freundlich, respektvoll, tolerant, sachlich
		&
		
		% Spalte 5
		kritische Kommentare erwünscht, keine Beschimpfungen/Beleidigungen/Diskriminierungen/Provokationen, höflich, verständlich
		&
		
		% Spalte 6
		keine Beschimpfungen, Beleidigungen, Kränkungen; keine kommerziellen Angebote oder Gesuche; das Forum ist kein Veranstaltungskalender/keine Terminankündigungen; Zitate müssen Quellenangabe enthalten
		
		
		\\ \hline
		
		
Funktionen im Kommentar	\\
``Melden''\\
``Bewerten''\\
``Antworten''\\
Sonstiges\\
&
		% Spalte 1
		\\
		ja\\
		positiv\\
		\\
		Ordnen nach beliebteste, älteste, neueste Kommentare
		&
		
		% Spalte 2
		\\
		\\
		ja\\
		\\
		ja: auf was man antwortet wird in Zitate gesetzt\\
		&
		
		% Spalte 3
		\\
		ja\\
		positiv\\
		dem Kommentator folgen\\
		&
		
		% Spalte 4
		\\
		ja, besonders hervorgehoben\\
		positiv und negativ\\
		ja\\
		\\
		&
		
		% Spalte 5
		\\
		Fähnchen-Button (nicht immer sichtbar)\\
		positiv (Klicks werden gezählt), nach Anmeldung oder als ``Gast'' möglich; negativ (nach Anmeldung)\\
		ja, eingerückt dargestellt\\
		``Teilen''-Button (= diese Diskussion auf Twitter/Facebook teilen, nach Anmeldung dort), ``Teilen''-Button für einzelnen Kommentar, 		``Empfehlen''-Button (= diese Diskussion empfehlen); Ordnen nach beste, neueste, älteste Kommentare\\
		&
		
		% Spalte 6
		ja\\
		\\
		ja, mit Anzahl Antworten\\
		\\
		\\
		
		
		\\ \hline
		
		
Kommentar in der Community/Forum	
& 
		% Spalte 1
		Ranglisten der Nutzer; Chronologie der Kommentare der Nutzer; meistgeklickt; kommentiert letzte 24 h
		&
		
		% Spalte 2
		meistdiskutierte Themen; eigene Beiträge anzeigen
		&
		
		% Spalte 3
		jünste/älteste Lesermeinung, viel/wenig diskutiert, viel/wenig empfohlen, TOP-Argumente 
		&
		
		% Spalte 4
		Chronologie; aktivste Kommentatoren (des Monats, gesamt, top 50), Kommentar des Tages; Videofavoriten der Leser 					(meistkommentiert, top 20)
		&
		
		% Spalte 5
		&
		
		% Spalte 6
		
		
		\\ \hline



Sonstige Funktionen beim Beitrag (Beitrag verschicken ist immer möglich, in der Regel über Briefsymbol)
& 
		% Spalte 1
		Korrektur-Button\footnote{Formular zum Versenden an die Redaktion mit Hinweisen auf Fehler oder anderes}
		&
		
		% Spalte 2
		Button ``merken'' (auf die Merkliste im Benutzerprofil setzen); Button ``feedback'' (Feedback an die Redaktion über Formular) 
		&
		
		% Spalte 3
		Symbol/Button ``Stern'' (= Empfehlen auf faz.net) und ``Haken'' (= Merken), Permalink, Drucker
		&
		
		% Spalte 4
		 Fehler-Melden-Funktion; Beitrag ``Bewerten'' mit Sternen (Anzahl Bewertungen)
		 &
		 
		 % Spalte 5
		&
		
		% Spalte 6
		 \\ \hline
		
Sonstiges 
&
		 % Spalte 1
		``Reaktionen'' möglich: Lachen, Weinen, Wut, Staunen, Wow (anklickbar, welche Reaktion man zu dem Beitrag empfindet)
		&
		
		% Spalte 2
		&
		
		% Spalte 3
		sämtliche Buttons/Symbole/Funktionen mit Hilfe/Erklärungen
		&
		
		% Spalte 4
		Kommentare abonnieren\\
		Leserbericht schreiben (zusätzlich zum Kommentar mit mehr Zeichen (4000), persönliche Erfahrungen) 
		&
		
		% Spalte 5
		Hinweis bei Löschung: ``Dieser Kommentar wurde entfernt'' (Antworten darauf aber noch sichtbar)
		&
		
		% Spalte 6
		
		
		\\ \hline

\end{tabular}
\end{landscape}
