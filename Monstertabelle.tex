\begin{landscape} \small
\begin{tabular}{ccc}

\hline
%Zeile 1: Portale
		&
		%Spalte 1
		bild.de &
		%Spalte 2
		spiegel.de &
		%Spalte 3
		faz.net &
		%Spalte 4
		focus.de 
		% Spalte 5
		diewelt.de 
		% Spalte 6
		derwesten.de
		% Spalte 9
		rp-online
		% Spalte 10
		handelsblatt
		% Spalte 11
		suedkurier
		% Spalte 12
		zeit.de
		% Spalte 13
		badische zeitung
		% Spalte 14
		stuttgarter zeitung
		% Spalte 15
		
		% Spalte 16
		
		% Spalte 17
		
		% Spalte 18
		
		% Spalte 19
		
		% Spalte 20
		
		\\ \hline

% Zeile 2
Kommentare\footnote{werden in der Regel mit Sprechblasen angekündigt; mit Angabe der Anzahl der abgegebenen Kommentare; auch eine Zeitangabe ist üblich entweder mit Datum und Uhrzeit oder ``vor ... Stunden'' \\
abweichende Angaben und Sonstiges
&		% Spalte 1
		\\
		K. nicht zu allen Artikeln möglich 
		&
		% Spalte 2
		\\
		Startseite: keine Sprechblase sondern Hinweis auf [Forum]; beim Beitrag: Sprechblase mit Ausrufungszeichen\\
		K. werden durchnummeriert
		&
		% Spalte 3
		\\
		zeitliche Begrenzung um K. zu schreiben; Hinweis, dass die K. im Internet recherchierbar sind; K. = Lesermeinung
		&
		% Spalte 4
		 \\
		 auch Videos kommentierbar 
		&
		% Spalte 5
		\\
		über Disqus verwaltet; Schließung nach zwei/drei Tagen oder früher bei Regelverstößen
		&
		% Spalte 6
		\\
		keine Ankündigung der K., keine Sprechblasen; alle Beiträge können kommentiert werden; K. werden durchnummeriert
		&
		% Spalte 9
		\\
		fast jedes Thema kommentierbar; 
		&
		% Spalte 10
		\\
		&
		% Spalte 11
		\\
		K. nicht zu allen Artikeln möglich
		&
		% Spalte 12
		\\
		keine Sprechblase; Hinweis mit ``Anzahl + Kommentare''; K. werden durchnummeriert;  K. nicht zu allen Artikeln möglich; Definition von K.: ``kürzere Textbeiträge, die Sie unter vorhandenen Artikeln, Videos, Fotostrecken oder anderen Multimedia-Inhalten abgeben können''
		&
		% Spalte 13
		&
		% Spalte 14
		für alle Nutzer
		&
		% Spalte 15
		&
		% Spalte 16
		&
		% Spalte 17
		&
		% Spalte 18
		&
		% Spalte 19
		&
		% Spalte 20
		&
		
		\\ \hline
		
		
		&
		
Moderation 
&		% Spalte 1
		keine \\ 
		&
		% Spalte 2
		Prä-Moderation (zeitliche Verzögerungen)\\
		Forumsmoderator mit Namen SYSOP
		&
		% Spalte 3
		Prä-Moderation (zeitliche Verzögerungen)
		&
		% Spalte 4		
		keine, aber Stichproben vom Digitalvermarkter
		&
		% Spalte 5
		Prä-Moderation (zeitliche Verzögerungen); Moderator = Team von ``Welt''-Mitarbeitern; Kritik an Moderationsweise per Email; 				bestimmte Moderationszeiten, Moderation gemäß Nutzungsregeln
		&
		% Spalte 6
		keine, automatische Veröffentlichung, keine inhaltliche Prüfung: Hinweis, dass Beiträge falsche Tatsachen enthalten, Rechte 				dritter verletzen, in die Irre führen, täuschen können; Pflichten der Nutzer in Bezug auf ihre Beiträge
		&
		% Spalte 9
		keine; bei Melden von Missbrauch erfolgt Löschen der Beiträge/Sperrung des Nutzers; auch die Antworten dazu; Abbruch der K.-Funktion bei Verstößen/nicht themenbezogen (bis dahin veröffentlichte Beiträge bleiben); Beiträge werden nach eigenem Ermessen entfernt/bearbeitet
		% Spalte 10
		eingeschränkte Post-Moderation: keine umfassende Prüfung, aber Eingriff bei Meldungen/Verstößen;  bei Beanstandung wird der Beitrag sofort gesperrt oder gelöscht und sich mit dem Nutzer in Verbindung gesetzt; nicht eindeutige Sachverhalte müssen abgeklärt werden; verstoßen einzelnen Abschnitte gegen Regeln, werden diese entfernt und der Eingriff kenntlich gemacht; wird Kommentar komplett entfernt wird dies auch kenntlich gemacht; verstoßen viele Kommentare gegen die Regeln wird die Funktion deaktiviert
		&
		% Spalte 11
		eingeschränkte Post-Moderation: Bearbeitung des Textes oder komplettes Löschen bei Verstößen; Löschen wenn nicht themenbezogen; Löschen von Trollen; Nutzer, die regelmäßig gegen Regeln verstoßen werden per Email ermahnt; schwere/wiederholte Verstöße führen zum Ausschluss der Community; bei gehäuften Verstößen wird Funktion abgeschaltet
		&
		% Spalte 12
		Prä-Moderation (wie bei Leserbriefen); Kürzen/Löschen bei Regelverstoß mit Begründung, warum eingeschritten wurde mit Anmerkungen und Kennzeichnungen; schwere/wiederholte Verstöße führen zur Sperrung; Abbruch der K.-Funktion bei gehäuften Verstößen/nicht themenbezogen; 
		&
		% Spalte 13
		eingeschränkte Post-Moderation: keine umfassende Prüfung; Nutzer trägt für Beiträge Verantwortung; Bearbeitung des Textes oder komplettes Löschen bei Verstößen; schwere/wiederholte Verstöße führen zum Ausschluss der Community; keine Crosspostings; 
		&
		% Spalte 14
		Prä-Moderation; Bearbeitung des Textes oder komplettes Löschen/Sperrung des Nutzers wenn nötig
		&
		% Spalte 15
		&
		% Spalte 16
		&
		% Spalte 17
		&
		% Spalte 18
		&
		% Spalte 19
		&
		% Spalte 20
		&
		\\ \hline
		
		
Registrierung/Anmeldung/Login  \\
Bedingungen (außer Email und Passwort)
 &		% Spalte 1
		bei BILD.de-Community über Anmeldedienst ``mypass'', über Facebook Account; Profil anlegen möglich\\
		Klarname, (angemessener) Benutzername (angezeigt), Volljährigkeit bzw. Einverständnis der Erziehungsberechtigten bei Minderjährigen
		&
		% Spalte 2
		bei ''mein spiegel'' als Abonnent oder Nicht-Abonnent, Facebook, automatisches Login möglich\footnote{Internetbrowser merkt sich das Login mit Cookie, welches den Benutzernamen und das Passwort enthält}\\
		Klarname, Benutzername (angezeigt)
		&
		% Spalte 3
		bei ''Mein FAZ.NET''\\
		Klarname, Profilbild möglich
		&
		% Spalte 4
		bei FOCUS online, Facebook\\
		Klarname (Hinweis: im gesamten Internet recherchierbar)
		&
		% Spalte 5
		bei WELT DIGITAL über Anmeldedienst ``mypass'', Disqus, Facebook, Twitter, g+\\
		Klarname, Benutzername, Profilfoto möglich;``ich schreibe als Gast';
		Whitelist: Nutzer können ohne Moderation kommentieren (Redaktion und Community-Mitglieder, die auffallend positiv 					kommentieren)
		&
		% Spalte 6
		mit\\
		Klarname, Benutzername (angezeigt), jeder ist zugangs- und teilnahmeberechtigt
		&
		% Spalte 9
		bei mein RP ONLINE\\
		Benutzername; auch Minderjährige, wenn sie sich über Nutzung bewusst sind bzw. mit Einverständnis der Erziehungsberechtigten
		% Spalte 10
		mit \\
		Klarname, Volljährigkeit
		&
		% Spalte 11
		mit\\
		Benutzername, Anrede, Klarname, Land, Adresse (Angaben werden auf Richtigkeit geprüft (teilweise auch telefonisch); Einverständnis nach sechs Monaten Inaktivität wird Registrierung/Benutzername gesperrt/freigegeben
		&
		% Spalte 12
		mit\\
		Benutzername, Profil anlegen möglich
		&
		% Spalte 13
		bei Meine BZ\\
		Anrede, Klarname, Profilbild (optional)
		&
		% Spalte 14
		mit \\
		Klarname, über Facebook Account
		&
		% Spalte 15
		&
		% Spalte 16
		&
		% Spalte 17
		&
		% Spalte 18
		&
		% Spalte 19
		&
		% Spalte 20
		&
		\\ \hline
		
		
Artikel teilen auf sozialen Netzwerken (Facebook ``teilen'' und/oder ``empfehlen'', Twitter, g+) und durch Versenden (Briefsymbol)	
&		% Spalte 1
		tumbl, Pinterest\\
		K. gleichzeitig auf Facebook veröffentlichen möglich 
		&
		% Spalte 2
		Xing, LinkedIn, Tumbl, Pinterest, deli.cio.us, Diggy, reddit
		&
		% Spalte 3
		&
		% Spalte 4		
		Facebook ``gefällt mir'', Xing
		&
		% Spalte 5 
		&
		% Spalte 6
		&
		% Spalte 9
		&
		% Spalte 10
		Xing, Email schreiben
		&
		% Spalte 11
		&
		% Spalte 12
		(keine Symbole/Verweise auf Startseite)
		&
		% Spalte 13
		(keine Symbole/Verweise auf Startseite) kein g+; Versenden (kein Briefsymbol), Verlinken
		&
		% Spalte 14
		&
		% Spalte 15
		&
		% Spalte 16
		&
		% Spalte 17
		&
		% Spalte 18
		&
		% Spalte 19
		&
		% Spalte 20
		&
		\\ \hline
		
		
Wo Regeln/Hinweise zum Kommentieren stehen (bei Nutzungsbedingungen, Richtlinien, AGB) und ob eine Netiquette vorhanden ist:
&		% Spalte 1
		Nutzungsbedingungen: allgemeine und besondere (Zustimmung verlangt bei Registrierung); Netiquette
		&
		% Spalte 2
		Nutzungsbedingungen: allgemeine und für Foren (Zustimmung)
		&
		% Spalte 3
		Nutzungesbedingungen: allgemein (Zustimmung); ``wie Sie mit diskutieren''-Button
		&
		% Spalte 4
		AGB (Zustimmung), Netiquette (Zustimmung)
		&
		% Spalte 5
		Nutzungsbedingungen, veraltete Netiquette
		&
		% Spalte 6
		Nutzungsbedingungen (Zustimmung), Netiquette
		&
		% Spalte 9
		AGB
		&
		% Spalte 10
		Nutzungshinweise (Zustimmung), Netiquette 
		&
		% Spalte 11
		Nutzungsbedingungen (Zustimmung), Netiquette
		&
		% Spalte 12
		AGB (Zustimmung), Netiquette (besonders ausführlich und erklärend)
		&
		% Spalte 13
		AGB (Zustimmung), Netiquette
		&
		% Spalte 14
		AGB (Zustimmung), Kommentarregeln = Netiquette
		&
		% Spalte 15
		&
		% Spalte 16
		&
		% Spalte 17
		&
		% Spalte 18
		&
		% Spalte 19
		&
		% Spalte 20
		&
		\\ \hline
		
		
Kommentar: formale Regeln\\
Zeichenbegrenzung\\
Überschrift (Pflichtfeld)\\
Sonstiges 
&		% Spalte 1
		\\
		keine\\
		keine\\
		vorsichtig mit Großbuchstaben, Zitate kennzeichnen	
		&
		% Spalte 2
		\\
		keine\\
		optional\\
		\\
		&
		% Spalte 3
		\\
		1000 Zeichen\\
		ja, 100 Zeichen\\
		\\
		&
		% Spalte 4
		\\
		800 Zeichen\\
		ja\\
		reiner Text ohne besondere Kennzeichnungen (z.B. keine Smilies, Hervorhebungen, Chat-Symbole, nur Kleinschreibung, usw.), korrektes Deutsch, auf Rechtschreibung/Interpunktion achten
		&
		% Spalte 5
		\\
		keine\\
		keine\\
		Zitate kennzeichnen; keine Fremdsprachen\\
		&
		% Spalte 6
		\\
		keine\\
		keine\\
		&
		% Spalte 9
		\\
		keine\\
		Betreff\\
		deutsche Sprache
		&
		% Spalte 10
		\\
		keine\\
		keine\\
		Großbuchstaben (Schreien) vermeiden; Absätze machen und strukturieren; Wortwahl überprüfen; auf Rechtschreibung achten; 				Zitate/Quellen kennzeichnen
		&
		% Spalte 11
		\\
		1000 Zeichen\\
		ja\\
		\\
		&
		% Spalte 12
		\\
		1500 Zeichen\\
		ja, mindestens 5 Zeichen\\
		Absätze machen, auf Rechtschreibung achten, vorsichtig mit Großbuchstaben, Zitate kennzeichnen, wenig verwenden, Quellenangabe\\
		&
		% Spalte 13
		\\
		keine\\
		keine\\
		\\
		&% Spalte 14
		\\
		\\
		Betreff\\
		\\
		&% Spalte 15
		\\
		\\
		\\
		\\
		&
		% Spalte 16
		\\
		\\
		\\
		\\
		&
		% Spalte 17
		\\
		\\
		\\
		\\
		&
		% Spalte 18
		\\
		\\
		\\
		\\
		&% Spalte 19
		\\
		\\
		\\
		\\
		&
		% Spalte 20
		\\
		\\
		\\
		\\
		&
		\\ \hline
		
		
Kommentar: inhaltliche Regeln und Hinweise
&		% Spalte 1
		sachlich, höflich bleiben, andere respektieren, nicht dagegen argumentieren, keine unangemessenen Beiträge (= Beleidigungen, 			Beschimpfungen, Belästigungen, Drohungen, Diskriminierungen, Spam, Schadsoftware, Trolle, nicht themenbezogen) 
		&
		% Spalte 2
		angenehmes Diskussionsklima, fair, sachlich bleiben, obwohl es sich um verbale Auseinandersetzung handeln soll
		&
		% Spalte 3
		&
		% Spalte 4
		freundlich, respektvoll, tolerant, sachlich, keine Fremdtexte
		&
		% Spalte 5
		kritische Kommentare erwünscht, keine Beschimpfungen/Beleidigungen/Diskriminierungen/Provokationen, höflich, verständlich
		&
		% Spalte 6
		keine Beschimpfungen, Beleidigungen, Kränkungen; keine kommerziellen Angebote oder Gesuche; das Forum ist kein 					Veranstaltungskalender/keine Terminankündigungen; Zitate müssen Quellenangabe enthalten
		&
		% Spalte 9
		wie man selbst behandelt werden möchte; keine Beleidigungen, Anschuldigungen, Tatsachenbehauptungen, strafbare Inhalte; 				Verbot von Inhalten, die dem Ansehen von Verstorbenen und deren Angehörigen schaden könnten; Verbot von Inhalten, die 				doppeldeutig sind  oder anderweitige Darstellungen, deren Rechtswidrigkeit vermutet wird, aber nicht abschließend festgestellt 				werden kann; keine unwahren/unsachlichen
		% Spalte 10
		mit zynischen/ironischen Äußerungen vorsichtig sein; nicht persönlich werden; sich bewusst machen, welche eigenen Daten frei 			zugänglich ins Internet gestellt werden; guter Ton; ignorieren von Provokationen/Trolls
		&
		% Spalte 11
		offene, faire, sachliche, gehaltevolle Diskussion
		&
		% Spalte 12
		Durchlesen vor Abschicken, guter Umgangston, nicht provozieren lassen, mit zynischen/ironischen Äußerungen vorsichtig sein; keine Beleidigungen, Diskriminierungen, Diffamierungen, Verleumdungen, geschäfts- und rufschädigende Äußerungen, nicht prüfbare Unterstellungen/Verdächtigungen
		&
		% Spalte 13
		sachliche, niveauvolle, faire, offene Diskussionskultur; freundlich, tolerant sein; guter Umgangston, andere so behandeln, wie man es selber möchte; keine persönlichen Angriffe, keine Beleidigungen, rassistische, sexistische, vulgäre, hetzerische, gewaltverherrlichende Töne
		&
		% Spalte 14
		engagiert, fair; akzeptable, respektvolle Wortwahl; sachkritisch, seriös; keine Trolls
		&
		% Spalte 15
		&
		% Spalte 16
		&
		% Spalte 17
		&
		% Spalte 18
		&
		% Spalte 19
		&
		% Spalte 20
		&
	
		\\ \hline
		
		
Funktionen im Kommentar	\\
``Melden''\\
``Bewerten''\\
``Antworten''\\
Sonstiges
&		% Spalte 1
		\\
		ja\\
		positiv\\
		\\
		Ordnen nach beliebteste, älteste, neueste K.
		&
		% Spalte 2
		\\
		\\
		ja\\
		\\
		ja: auf was man antwortet wird in Zitate gesetzt\\
		&
		% Spalte 3
		\\
		ja\\
		positiv\\
		dem Kommentator folgen\\
		&
		% Spalte 4
		\\
		ja, besonders hervorgehoben\\
		positiv und negativ\\
		ja\\
		\\
		&
		% Spalte 5
		\\
		Fähnchen-Button (nicht immer sichtbar)\\
		positiv (Klicks werden gezählt), nach Anmeldung oder als ``Gast'' möglich; negativ (nach Anmeldung)\\
		ja, eingerückt dargestellt\\
		``Teilen''-Button (= diese Diskussion auf Twitter/Facebook teilen, nach Anmeldung dort), ``Teilen''-Button für einzelnen K., 		``Empfehlen''-Button (= diese Diskussion empfehlen); Ordnen nach beste, neueste, älteste Kommentare\\
		&
		% Spalte 6
		ja\\
		\\
		ja, mit Anzahl Antworten\\
		\\
		\\
		&
		% Spalte 9
		\\
		ja (mit Begründung mit Name/Email, auch ohne Registrierung möglich)\\
		positiv\\
		\\
		Ordnen nach älteste; Button für ``mehr K.''\\
		% Spalte 10
		\\
		ja (auch Email oder telefonisch möglich)\\
		\\
		ja\\
		\\
		&
		% Spalte 11
		\\
		ja (mit Name, Emailadresse, Grund an Community Manager)\\
		\\
		ja\\
		Ordnen nach älteste, neueste, beste Bewertung; anklickbar: ``informiert bleiben'' (bei jedem neuen Beitrag der Diskussion erhält man Benachrichtigung)\\
		&
		% Spalte 12
		ja\\
		Empfehlungen aussprechen; Empfehlungen der Redaktion (bedeutet aber nicht, dass die Redaktion der Meinung des Lesers zustimmt)\\
		ja\\
		ja\\
		Reaktionen/Antworten auf diesen K. anzeigen; Ornden nach neuesten, empfohlenen, allen K.\\
		&
		% Spalte 13
		\\
		ja\\
		\\
		\\
		\\
		&% Spalte 14
		\\
		\\
		\\
		ja\\
		Ordnen nach älteste, neueste K.\\
		&% Spalte 15
		\\
		\\
		\\
		\\
		\\
		&
		% Spalte 16
		\\
		\\
		\\
		\\
		\\
		&
		% Spalte 17
		\\
		\\
		\\
		\\
		\\
		&
		% Spalte 18
		\\
		\\
		\\
		\\
		\\
		&
		% Spalte 19
		\\
		\\
		\\
		\\
		\\
		&
		% Spalte 20
		\\
		\\
		\\
		\\
		\\
		&
		\\ \hline
		
		
Kommentar in der Community/Forum	
& 		% Spalte 1
		Ranglisten der Nutzer; Chronologie der Kommentare der Nutzer; kommentiert letzte 24 h (Top 5)
		&
		% Spalte 2
		meistkommentierte Themen (Top 5); eigene Beiträge anzeigen
		&
		% Spalte 3
		jünste/älteste Lesermeinung, viel/wenig diskutiert, viel/wenig empfohlen, TOP-Argumente 
		&
		% Spalte 4
		Chronologie; aktivste Kommentatoren (des Monats, gesamt, top 50), Kommentar des Tages; Videofavoriten der Leser 					(meistkommentiert, top 20)
		&
		% Spalte 5
		&
		% Spalte 6
		&
		% Spalte 9
		&
		% Spalte 10
		&
		% Spalte 11
		&
		% Spalte 12
		&
		% Spalte 13
		Nutzer registriert seit [...] + Anzahl K. von Nutzer
		&
		% Spalte 14
		keine Community
		&
		% Spalte 15
		&
		% Spalte 16
		&
		% Spalte 17
		&
		% Spalte 18
		&
		% Spalte 19
		&
		% Spalte 20
		&
		
		\\ \hline



Sonstige Funktionen beim Artikel 
& 		% Spalte 1
		Korrektur-Button\footnote{Formular zum Versenden an die Redaktion mit Hinweisen auf Fehler oder anderes}
		&
		% Spalte 2
		Button ``merken'' (auf die Merkliste im Benutzerprofil setzen); Button ``feedback'' (Feedback an die Redaktion über Formular) 
		&
		% Spalte 3
		Beitrag empfehlen, Beitrag merken, Permalink, Drucker
		&
		% Spalte 4
		 Fehler-Melden; Beitrag ``Bewerten'' mit Sternen (Anzahl Bewertungen)
		 &
		% Spalte 5
		&
		% Spalte 6
		&
		% Spalte 9
		Beitrag empfehlen, Drucken, Schriftgröße ändern
		&
		% Spalte 10
		Beitrag ``Merken''
		&
		% Spalte 11
		&
		% Spalte 12
		Drucken, als PDF speichern
		&
		% Spalte 13
		Drucken, Vorlesen, Fehler-Melden
		&
		% Spalte 14
		&
		% Spalte 15
		&
		% Spalte 16
		&
		% Spalte 17
		&
		% Spalte 18
		&
		% Spalte 19
		&
		% Spalte 20
		&
		 \\ \hline
		
Sonstiges 
&		% Spalte 1
		``Reaktionen'' möglich: Lachen, Weinen, Wut, Staunen, Wow (anklickbar, welche Reaktion man zu dem Beitrag empfindet)
		&
		% Spalte 2
		&
		% Spalte 3
		sämtliche Buttons/Symbole/Funktionen mit Hilfe/Erklärungen
		&
		% Spalte 4
		Kommentare abonnieren\\
		Leserbericht schreiben (zusätzlich zum Kommentar mit mehr Zeichen (4000), persönliche Erfahrungen) 
		&
		% Spalte 5
		Hinweis bei Löschung: ``Dieser Kommentar wurde entfernt'' (Antworten darauf aber noch sichtbar)
		&
		% Spalte 6
		&
		% Spalte 9
		Kontakt mit der Zeitung über Email an den Chefredakteur, Newsletter, Leserbrief schreiben (über Formular)
		&
		% Spalte 10
		&
		% Spalte 11
		``Meistkommentiert'' auf Startseite (Top 3); Leserreporter-Beitrag schreiben 
		&
		% Spalte 12
		``Meistgelesen''/''Meistkommentiert'' (Top 5) auf Startseite; Leserartikel schreiben = ausführliche Meinungsbeiträge und Erfahrungsberichte (meistgelesene/meistkommentierte Leserartikel, Top 3 auf Leserartikel-Seite); Debattenkultur: ``Aus den Kommentaren'' (Höhepunkte aus den Leserdebatten mit neuer Fragestellung); Kommentarkultur: ``Bitte weichen Sie vom Thema ab'' (Experiment: Kommentieren ohne Artikel); Empfehlungen bei Facebook (aktuelle Empfehlungen aus Facebook-Freundeskreis); Tweets von ZEIT ONLINE Politik
		&
		% Spalte 13
		``Meistkommentiert'' (Top 5)/''zuletzt kommentiert'' auf Startseite; Vorschau möglich: man kann K. sehen, wie er online aussehen wird
		&
		% Spalte 14
		K. geben nicht die Meinung der Stuttgarter Zeitung wieder
		&
		% Spalte 15
		&
		% Spalte 16
		&
		% Spalte 17
		&
		% Spalte 18
		&
		% Spalte 19
		&
		% Spalte 20
		&
		
		\\ \hline

\end{tabular}
\end{landscape}
