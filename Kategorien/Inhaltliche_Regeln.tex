
\section{Kategorie: \glqq inhaltliche Regeln und Hinweise\grqq}

Wie sie sich die Diskussion auf ihren Plattformen vorstellen und wie nicht,
beschreiben die Online-Zeitungen in der Netiquette, den AGB oder den
Nutzungsbedingungen. Auf manchen Portalen muss der Nutzer bei der Registrierung
dazu aktiv zustimmen und damit bestätigen, die Regeln zumindest zu akzeptieren.

Für die Regeln verwenden die Redaktionen eigene Formulierungen mit
unterschiedlichem Umfang. Es geht erst einmal darum, zu erklären, was man unter
einem \glqq guten Ton\grqq\ versteht. Dann wird erläutert, was nicht toleriert
wird und wann und wo ein Eingreifen seitens der Moderation erfolgt.

Gerade für Foren ohne Moderation sind inhaltliche Vorgaben besonders wichtig. Da
es niemanden gibt, der die Beiträge vorab oder danach liest und gegebenenfalls
einschreitet, müssen die Nutzer darüber informiert werden, was erlaubt ist und
was nicht. Außerdem werden sie zur Selbstregulierung aufgefordert - zum Beispiel
über die Möglichkeit Verstöße zu melden - und brauchen Handlungsvorgaben.

Es gibt inhaltliche Regeln zu den Kommentaren, die auf allen Plattformen zu
finden sind (Ausnahmen werden entsprechend gekennzeichnet).  Das sind zunächst
rechtliche Hinweise. Der Nutzer bestätigt der Urheber seiner Beiträge zu sein
oder die Urheberrechte zu besitzen, d.h. er muss Gewähr leisten, dass fremde
Inhalte zur Verbreitung freigegeben sind. Er trägt somit die volle Verantwortung
für die eingestellten Beiträge und stellt sicher, dass keine Rechte Dritter oder
Urheberrechte oder Persönlichkeitsrechte oder sonstige Rechte verletzt werden.
Außerdem gilt die Rechteeinräumung. Der Nutzer stimmt damit zu, dass die
entsprechende Zeitung seine Beiträge \glqq benutzen\grqq\ kann (z.B. vervielfältigen,
modifizieren, anpassen, übersetzten, bearbeiten, verbreiten, verwerten,
hervorheben, bewerten, archivieren, usw.).

Es gibt infolgedessen auch einen Haftungsausschluss des Anbieteres. Dieser
haftet nicht für den Inhalt von Nutzerbeiträgen. Dasselbe gilt für die Inhalte
fremder Seiten durch Verlinkung. Entsteht ein Schaden, haftet der Nutzer.

Was auf keinen Fall geduldet wird, ist Werbung in irgendeiner Form.
Diskussionsforen dürfen nicht für kommerzielle Zwecke missbraucht werden.
Diskussionsforen sind keine Werbefläche für Webseiten oder Dienste (Spamming).
Auch dem Datenschutz müssen die Nutzer zustimmen. Damit ist z.B. das
Über\-prü\-fen der Emails oder Abfragen auf Viren gemeint oder das Einhalten von
gesetzlichen, behördlichen und technischen Vorschriften. Das Passwort soll
geheim gehalten und Vertraulichkeit gewahrt werden.

Jedes Portal weist ausdrücklich darauf hin, dass Kommentare {\bfseries
themenbezogen} sein sollen.  Es gibt Inhalte, die kategorisch bei allen
Zeitungen nicht erlaubt sind. Dazu gehören Beleidigungen und Beiträge mit
se\-xi\-sti\-schen/sit\-ten\-wi\-dri\-gen/por\-no\-gra\-phi\-schen/obs\-zö\-nen/grob
anstößigen Inhalten und Rassismus. (Diese eben genannten inhaltlichen Verbote
werden in der Tabelle nicht mehr aufgeführt, weil sie in allen Online-Zeitungen
genannt werden.)

Auch das Verbot von Diskriminierungen zählen die meisten Nachrichtenportale in
irgendeiner Form auf.

Die Redaktionen wählen unterschiedliche Formulierungen, um unerwünschte Beiträge
zu beschreiben. Viele Formulierungen sind sich im Inhalt ähnlich  und
unterscheiden sich um Nuancen (z.B. Beleidigung und Beschimpfung). Trotzdem
werden sie hier bewusst aufgelistet und nicht zusammengefasst, um deutlich zu
machen, was die Zeitungen extra erwähnt haben wollen und was nicht.

Eine lange Liste von Richtlinien kann als besonders sorgfältige Beschäftigung
mit den Inhalten von Nutzern gesehen werden. Inwieweit die Nutzer sich damit
auseinandersetzen, weiß man nicht. Sie müssen zwar bei den meisten Zeitungen den
Richtlinien zustimmen. Ob sie diese auch tatsächlich lesen, sei dahingestellt.
Auf alle Fälle wird bei Zustimmung direkt auf diese Regeln verwiesen oder
verlinkt, was dem Nutzer entgegenkommt.

In der ersten Zeile (schräggedruckt) steht, wo diese Regeln zu finden sind und
wo eine Zustimmung mit Anklicken gefordert wird.
\newpage

\begingroup
\footnotesize
\begin{longtable}{p{28mm}p{110mm}}
  \caption{Kategorie \glqq inhaltliche Regeln\grqq} \label{inhaltliche Regeln} \\ \\

  \toprule
  \bfseries Portal & \multicolumn{1}{c}{\textbf{Inhaltliche Regeln}}\\\midrule[\heavyrulewidth]
  \endfirsthead

  \toprule
  \bfseries Portal & \multicolumn{1}{c}{\textbf{Inhaltliche Regeln}}\\\midrule[\heavyrulewidth]
  \endhead

  \multicolumn{2}{r}{\emph{Fortsetzung auf der nächsten Seite}}
  \endfoot

  \bottomrule
  \endlastfoot


bild.de & \emph{Nutzungsbedingungen: allgemeine und besondere (Zustimmung
  verlangt bei Registrierung); Netiquette}

  Diskussion: sachlich, höflich, respektvoll, nicht dagegen argumentieren,
  Angriffe ignorieren, wie man selbst behandelt werden möchte

  keine: Beschimpfungen\footnote{Hier und im Folgenden: Unter Beschimpfungen fallen auch
  Diskriminierungen, Erniedrigungen, Verleumdungen, Ruf- und
  Geschäftsschädigungen}, Belästigungen, Drohungen; privaten
  Angaben\footnote{Hier und im Folgenden: Private Angaben sind Angaben von Postadresse und/oder
  Telefonnummer und/oder Emailadresse} oder Angaben über Dritte; automatisierte
  Nutzung; Links\footnote{Hier und im Folgenden: keine Links, die auf Werbung, Chats, Foren, strafbare Inhalte, u.ä.
  weiterleiten.}; Trolle; Schadsoftware;

  kein Mobbing
  \\*\midrule

spiegel.de & \emph{Nutzungsbedingungen: allgemeine und für Foren (Zustimmung)}

  Diskussion: fair, sachlich, angenehm, offen, freundschaftlich, respektvoll

  keine: Beiträge mit strafbaren, inakzeptablen  Inhalten
  \\*\midrule

faz.net & \emph{Nutzungsbedingungen: allgemein (Zustimmung); \glqq wie Sie
  mitdiskutieren\grqq\--Button}

  Diskussion: sich bewusst machen, dass die Beiträge oder persönlichen Daten
  frei zugänglich ins Internet gestellt werden und dort auch recherchierbar sind
  (= bewusst machen)

  keine: Beschimpfungen; Beiträge mit links- und rechtsradikalen Inhalten;
  falsche, nicht nachprüfbare Behauptungen; Links
  \\*\midrule

focus.de & \emph{AGB (Zustimmung), Netiquette (Zustimmung)}

  Diskussion: sachlich, freundlich, respektvoll, tolerant

  keine: Fremdtexte; privaten Angaben; Links; Diskriminierungen jeder
  Art\footnote{Hier und im Folgenden: Unter Diskriminierungen jeder Art versteht man Diskriminierungen
  aufgrund von Herkunft, Nationalität, Religion, sexueller Orientierung, Alter,
  Geschlecht, usw.}; Beiträge mit  demagogischen Inhalten; Schadsoftware

  kein Missbrauch von Daten
  \\*\midrule

welt.de & \emph{Nutzungsbedingungen, veraltete Netiquette}

  Diskussion: fair, höflich, verständlich, kritisch, zivil, kontrovers, gehaltvoll

  keine Beschimpfungen, Provokationen, Entwürdigungen, Aufruf zu Demonstrationen
  oder Gewalt; Angaben über Dritte; Hasspropaganda

  Gesetze beachten; auf Haftungsausschluss wird nicht hingewiesen
  \\*\midrule

derwesten.de & \emph{Nutzungsbedingungen (Zustimmung), Netiquette}

  keine: Beschimpfungen, Beleidigungen; Beiträge mit gewaltverherrlichenden,
  antisemitischen, gesetzeswidrigen Inhalten

  das Forum ist kein Veranstaltungskalender, keine Terminankündigungen
  \\*\midrule


% Spalte 9
rp-online.de & \emph{AGB}

  Diskussion: wie man selbst behandelt werden möchte

  keine: Anschuldigungen, Tatsachenbehauptungen; Beschimpfungen, unwahren,
  sinnlosen, störenden Beiträge oder Beiträge mit strafbaren Inhalten,
  Urheberrechtsverstöße, Drohungen, volksverhetzende Äußerungen, Aufforderung zu
  Gewalt; Schadsoftware

  Verbot von Inhalten, die dem Ansehen von Verstorbenen und deren Angehörigen
  schaden könnten, die doppeldeutig sind oder anderweitige Darstellungen, deren
  Rechtswidrigkeit vermutet wird, aber nicht abschließend festgestellt werden
  kann \\*\midrule

% Spalte 10
handelsblatt.com & \emph{Nutzungshinweise (Zustimmung), Netiquette}

  Diskussion: vorsichtig mit zynischen, ironischen Äußerungen, guter Ton, nicht
  persönlich werden, Provokationen ignorieren, sich bewusst machen

  keine: Diskriminierungen jeder Art;  Beschimpfungen, Beiträge mit
  strafrechtlich relevanten Inhalten; Angaben über Dritte;
  %sich bewusst machen, welche eigenen Daten frei zugänglich ins Internet
  %gestellt werden (steht jetzt bei Allgemeines);
  Trolle; Schadsoftware; Nennung von Produktnamen, Dienstleistern, Marken,
  Produzenten

  auf Nutzungsrechte von handelsblatt.de wird nicht
  hingewiesen\\*\midrule

% Spalte 11
suedkurier.de & \emph{Nutzungsbedingungen (Zustimmung), Netiquette}

  Diskussion: fair, sachlich, offen, gehaltvoll

  keine: nicht-belegbaren Behauptungen, Beschimpfungen, Diffamierungen,
  Drohungen, Diskriminierungen aller Art, Hetze, Gewaltverherrlichung,
  Vulgärausdrücke;  Angaben über Dritte
  \\*\midrule

% Spalte 12
zeit.de & \emph{AGB (Zustimmung), Netiquette (besonders ausführlich und erklärend)}

  Diskussion: vorsichtig mit zynischen, ironischen Äußerungen, guter Ton,
  Provokationen ignorieren, nachvollziehbar argumentieren; sich bewusst machen
  und Durchlesen vor Abschicken

  keine: Beschimpfungen, Diskriminierungen aller Art (auch Behinderung,
  Einkommensverhältnisse), Diffamierungen; Verdächtigungen; Angaben über Dritte;
  Schadsoftware
  \\*\midrule

% Spalte 13
badische-zeitung.de & \emph{AGB (Zustimmung), Netiquette}

  Diskussion: sachlich, niveauvoll, fair, offen, freundlich, tolerant, wie man
  es selber möchte, nicht persönlich werden

  keine: Beiträge mit hetzerischem, gewaltverherrlichendem Inhalt; privaten
  Daten
  \\*\midrule

% Spalte 14
stuttgarter-zeitung.de & \emph{AGB (Zustimmung), Kommentarregeln = Netiquette}

  Diskussion: engagiert, fair, respektvoll, sachkritisch, seriös

  keine: Beschimpfungen, Schmähungen, Volksverhetzung, Propaganda, Beiträge mit
  jugendgefährdenden, antisemitischen, strafbaren, menschenverachtenden, gegen
  die guten Sitten verstoßenden Inhalten; Trolle; Schadsoftware; privaten Daten;
  Angaben über Dritte
  \\*\midrule

% Spalte 15
merkur.de & \emph{AGB (Zustimmung), Netiquette (Zustimmung)}

  Diskussion: sachlich, freundlich, verständlich, nicht persönlich werden, man
  soll Spaß haben und sich wohl fühlen, andere Meinungen akzeptieren

  keine: Beiträge mit unwahren, unsachlichen, jugendgefährdenden,
  verleumderischen, verfassungsfeindlichen, extremistischen, illegalen,
  ethisch-moralisch-problematischen Inhalten; Schadsoftware; privaten
  Daten
  \\*\midrule

% Spalte 16
hna.de & \emph{AGB (Zustimmung), Netiquette (Zustimmung)}

  Diskussion: sachlich, freundlich, verständlich, nicht persönlich werden, man
  soll Spaß haben und sich wohl fühlen, andere Meinungen akzeptieren

  keine: Beiträge mit unwahren, unsachlichen, jugendgefährdenden,
  verleumderischen, verfassungsfeindlichen, extremistischen, illegalen,
  ethisch-moralisch-problematischen Inhalten; Schadsoftware; keine privaten
  Daten

  auf Nutzungsrechte und Haftungsausschluss von hna.de wird nicht
  hingewiesen
  \\*\midrule

% Spalte 17
mopo.de &  \emph{keine eigenen; Basic Rules von Disqus}\footref{BasicRules}
\\*\midrule

% Spalte 18
mainpost.de& \emph{Netiquette}

  Diskussion: fair, respektvoll, nicht persönlich werden, vorsichtig mit
  zynischen, ironischen Äußerungen

  keine: Angaben über Dritte; privaten Daten; Beiträge mit ehrverletzenden,
  gewaltverherrlichenden Inhalten oder Aufrufen zur Gewalt

  kein Aufruf zu Straftaten; auf Nutzungsrechte und Haftungsausschluss von
  mainpost.de wird nicht hingewiesen
  \\*\midrule

% Spalte 19
tagesspiegel.de & \emph{Richtlinien für Community}

  Diskussion: sachlich, respektvoll, fair, angenehme Atmosphäre; bewusst machen

  keine: Beschimpfungen, Diskriminierungen aller Art (auch aufgrund von
  Weltanschauung, sozialem Status), Verdächtigungen, Beiträge mit pietätlosen,
  menschenverachtenden, gewaltverherrlichenden, Inhalten; Trolle;
  %keine Links zu anderen Webinhalten, (steht jetzt bei formale Regeln)
  kein Geschichtsrevisionismus; auf Nutzungsrechte von tagesspiegel.de wird
  nicht hingewiesen
  \\*\midrule

% Spalte 20
swp.de & \emph{Netiquette}

  Diskussion: sachlich, fair, freundlich, wie man selbst behandelt werden möchte

  keine: hetzerischen, gewaltverherrlichenden Töne; privaten Daten

  auf Rechte wird nicht hingewiesen
  \\*\midrule

% Spalte 21
augsburger-allgemeine.de &
  \emph{AGB, Nutzungsbedingungen der Community (Zustimmung) (äußerst
  umfangreiche Erklärung, was erwünscht ist und was nicht, was er für Fälle
  geben kann und wie damit umgegangen werden soll}

  keine: Beschimpfungen; Beiträge mit menschenverachtenden, abscheulichen,
  bedrohlichen, belästigenden, strafbaren, jugendgefährdenden,
  verfassungsfeindlichen, extremistischen Inhalten (Anzeige gegen den Nutzer);
  Inhalte von verfassungsfeindlichen Gruppierungen; Inhalte, die dem Ansehen des
  Forums Schaden zufügen oder stören können;  Angaben über Dritte; Schadsoftware

  das Folgen von Tipps, Ratschlägen folgt auf eigene Gefahr
  \\*\midrule

% Spalte 22
abendzeitung-muenchen.de &
  \emph{Nutzungsbedingungen, Kommentarregeln}
	
	Diskussion: sachkritisch, seriös
	
	
  keine: Beschimpfungen, Drohungen, Diskriminierungen, strafbare Äußerungen, sinnlose 
  Inhalte, Provokationen; Links
  
  Verbot: von Schadsoftware, freien Zugang für Nutzer einzuschränken/zu unterbinden

  \\*\midrule

% Spalte 23
fr-online.de &
  \emph{Datenschutz, \glqq Wir über uns\grqq: Informationen zu Registrierung und Moderation; Basic Rules von Disqus\footnote{Basic Rules: keine Beschimpfungen, privaten Daten, anstößigen Bilder beim Profilbild, kein Spam; Missbrauch melden}\label{BasicRules}}

\end{longtable}
\endgroup

% vim: set ai si et tw=80 sts=2 ts=2 sw=2:
