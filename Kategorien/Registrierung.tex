

\section{Kategorie: \glqq Registrierung\grqq}

Kommentare können immer gelesen werden, um selbst welche zu schreiben, bedarf es
jedoch einer Anmeldung. Darin sind sich alle Anbieter einig (Ausnahme ist die Abendzeitung
Online, bei der Kommentieren ohne Registrierung möglich ist). Dies ist die erste
Maßnahme, um Missbrauch bei der Kommentarfunktion zu vermeiden. Man kann nicht
einfach drauf los schreiben, sondern muss den Prozess einer Registrierung durch
gehen. In der Regel bieten die Online-Zeitungen eine eigene Registrierung an.
Man ist dann auf dem Portal angemeldet und sieht es auch (außer Hamburger
Morgenpost und Südwest Presse). Manche Zeitungen ermöglichen auch eine Anmeldung
über Dritte Anbieter. Das ist dann eine Kompromisslösung für die Nutzer. Eine
Anmeldefunktion wird nämlich auch als Hemmschwelle zum Schreiben diskutiert.
Kann man sich über ein Konto anmelden, das man bereits hat, dann gilt dieses
Argument nicht mehr. Der Trend, den \textcite[S.~69]{singer:2014} beschreibt
bestätigt sich also: \glqq Some newspapers also had begun requiring users to
comment through Facebook, an intriguing step that removes many of the problems
created by anonymous postings while also helping generate social network
traffic. Such trends richly deserve the attention that journalism scholars have
begun to afford them.\grqq

Durch eine Anmeldung wird sicher gestellt, dass die Person existiert und
gegebenenfalls zur Verantwortung gezogen werden kann.  Natürlich ist auch hier
Missbrauch nicht ausgeschlossen. Der Nutzer wird darauf hingewiesen,
wahrheitsgemäße Angaben zu machen. 

Interessant ist, welche Angaben die Zeitungen zur Anmeldung fordern. Unabdingbar
sind die Eingabe einer Emailadresse sowie ein Passwort. Dann kommt es wieder zu
unterschiedlichen Handhabungen und unterschiedlichen Lösungen. Manche Zeitungen
verlangen die Nennung von Klarnamen und Benutzernamen, manche nur Klarnamen,
manche nur Benutzernamen. Einige Portale geben zusätzliche Hinweise zur
Registrierung. Bei manchen Zeitungen kann man sich über ein anderes Portale oder
soziales Netzwerk anmelden.

Daraus ergeben sich folgende Unterkategorien zur Registrierung:

\begin{itemize}
  \item \glqq{\bf alternative Anmeldung}\grqq:\\ Kann man sich über ein anderes
    Portal anmelden, um dann eigene Inhalten einzustellen? Welches Portal ist
    das?

  \item\glqq{\bf Klarname}\grqq{} (KN): \\ Ist die Angabe von Zu- und Nachnamen
    notwendig?

  \item\glqq{\bf Benutzername}\grqq{} (BN):\\ Ist die Angabe eines
    Benutzernamens, d.h.  eines frei gewählten Namens oder Spitzname möglich?

  \item\glqq{\bf Sonstiges}\grqq:\\ Welche Angaben werden von der Online-Zeitung
    zusätzlich verlangt zur Registrierung?
\end{itemize}


%\begin{landscape}
\begingroup
  \footnotesize
  \begin{longtable}{p{24mm}p{20mm}p{10mm}p{10mm}p{60mm}}

  \caption{Kategorie \glqq Registrierung\grqq}
  % FIXME Footnote einbauen!
  %\footnote{Die Angabe einer Emailadresse und eines Passworts wird immer
  %verlangt. In der Tabelle sind weitere obligatorische Angaben, Unterkategorien,
  %aufgeführt.}
  \\
  \toprule
  \bfseries Portal & \bfseries alt. Anmeld. &
  \centerline{\bfseries KN} & \centerline{\bfseries BN} & \bfseries Sonstiges\\
  \midrule[\heavyrulewidth]
  \endfirsthead

  \toprule
  \bfseries Portal & \bfseries alt. Anmeld. & \centerline{\bfseries KN}
  & \centerline{\bfseries BN} & \bfseries Sonstiges\\
  \midrule[\heavyrulewidth]
  \endhead

  \multicolumn{5}{r}{\emph{Fortsetzung auf der nächsten Seite}}
  \endfoot

  \bottomrule
  \endlastfoot

bild.de
& mypass, Facebook
& \centerline{ja}
& \centerline{ja}
& Volljährigkeit bzw. Einverständnis der Erziehungsberechtigten bei
  Minderjährigen
\\*\midrule

spiegel.de % Spalte 2 bei ''mein spiegel'' als Abonnent oder Nicht-Abonnent
& Facebook
& \centerline{ja}
& \centerline{ja\footnote{Es wird darauf hingewiesen, dass der Benutzername
  angezeigt wird.\label{foot:angezeigt}}}
&
\\*\midrule

faz.net % Spalte 3 bei ''Mein FAZ.NET'' mit
& keine
& \centerline{ja}
& \centerline{nein}
&
\\*\midrule

focus.de % Spalte 4 bei FOCUS online
& Facebook
& \centerline{ja\footnote{Es wird darauf hingewiesen, dass der Name im gesamten
  Internet recherchierbar ist.}}
& \centerline{nein}
&
\\*\midrule

welt.de % Spalte 5 bei WELT DIGITAL über
& mypass, Disqus, Facebook, Twitter, Google
& \centerline{ja}
& \centerline{ja\footnote{Die Nutzungsbedingungen (siehe Tabelle 6.3 \glqq inhaltliche Regeln\grqq) gelten auch für den Benutzernamen.}}
& als Gast schreiben möglich
\\*\midrule

derwesten.de % Spalte 6 auf derwesten.de mit
& keine
& \centerline{ja}
& \centerline{ja\footref{foot:angezeigt}}
& jeder ist zugangs- und teilnahmeberechtigt
\\*\midrule

rp-online.de % Spalte 9 bei mein RP ONLINE mit
& keine
& \centerline{nein}
& \centerline{ja}
& auch Minderjährige, wenn sie sich über Nutzung bewusst sind bzw. mit
  Einverständnis der Erziehungsberechtigten
\\*\midrule

handelsblatt.com % Spalte 10 auf handelsblatt.de mit
& keine
& \centerline{ja}
& \centerline{nein}
& Volljährigkeit
\\*\midrule

suedkurier.de % Spalte 11auf suedkurier.de mit
& keine
& \centerline{ja}
& \centerline{ja}
& Anrede, Land, Adresse (Angaben werden auf Richtigkeit geprüft (teilweise auch
  telefonisch); Einverständnis nach sechs Monaten Inaktivität wird
  Registrierung/Benutzername gesperrt/freigegeben
\\*\midrule

zeit.de % Spalte 12 auf zeit.de mit
& keine
& \centerline{nein}
& \centerline{ja}
&
\\*\midrule

badische-zeitung.de % Spalte 13 bei Meine BZ mit
& keine
& \centerline{ja}
& \centerline{nein}
& Anrede, Löschen des Accounts bei Wegwerf-Emailadresse
\\*\midrule

stuttgarter-zeitung.de % Spalte 14 auf stuttgarterzeitung.de
& Facebook
& \centerline{ja}
& \centerline{nein}
& für alle Nutzer
\\*\midrule

merkur.de % Spalte 15 bei Merkur-Online
& Disqus, Facebook, Twitter, Google
& \centerline{nein}
& \centerline{ja\footnote{Ein Benutzername ist erwünscht und soll angemessen
  sein.}}
& für alle Nutzer; als Gast schreiben
\\*\midrule

hna.de % Spalte 16 mit HNA-Login
& Disqus
& \centerline{nein}
& \centerline{ja\footnote{Der Benutzername soll angemessen und nicht
  beleidigend/anstößig sein.}}
& für alle Nutzer
\\*\midrule

mopo.de % Spalte 17
& Disqus, Facebook, Twitter, Google
& \centerline{nein}
& \centerline{nein}
&
\\*\midrule

mainpost.de % Spalte 18 auf Mainpost.de
& keine
& \centerline{ja}
& \centerline{ja\footnote{Der Benutzername soll nicht beleidigend,
  ehrverletzend, hetzerisch sein.}}
& Adresse
\\*\midrule

tagesspiegel.de % Spalte 19 auf tagesspiegel.de
& keine
& \centerline{nein}
& \centerline{ja\footnote{Der Benutzername soll angemessen sein.}}
&
\\*\midrule

swp.de % Spalte 20
& keine
& \centerline{ja}
& \centerline{ja}
&
\\*\midrule

augsburger-allgemeine.de %spalte 21
& keine
& \centerline{ja}
& \centerline{ja}
& Wohnort, Ausschluss bei Mehrfach-/Scheinregistrierungen oder unter dem Namen
  anderer, bei Minderjährigen Einverständnis der Erziehungsberechtigten
\\*\midrule

abendzeitung-muenchen.de %spalte 21
& keine
& \centerline{nein}
& \centerline{ja}
& für alle Nutzer
\end{longtable}
%\end{landscape}
\endgroup

% vim: set ai si et tw=80 sts=2 ts=2 sw=2:
