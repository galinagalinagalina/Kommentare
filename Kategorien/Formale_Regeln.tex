\section{Kategorie: \glqq formale Regeln\grqq}

mit den Unterkategorien
\begin{itemize}
  \item\glqq {\bfseries Zeichenbegrenzung}\grqq: Gibt es eine Zeichenbegrenzung oder kann der Nutzer schreiben so viel
    er möchte?

  \item\glqq{\bfseries Überschrift}\grqq: Wird die Eingabe einer Überschrift verlangt?

  \item\glqq{\bfseries Sonstiges}\grqq: Welche formalen Hinweise geben die Online-Zeitungen zusätzlich?
\end{itemize}

Beobachtungen bei den Nachrichtenportalen:\\
Die Zeitungen geben auch Hinweise zum formalen Umgang mit Kommentaren. Die Hälfte gibt eine Zeichenbegrenzung vor. Diese
reicht von 800 bis 3000 Zeichen.  Ebenso fordert ungefähr die Hälfte der Redaktionen die Eingabe eines Betreffs. Die
sonstigen formalen Regeln haben unterschiedlichen Inhalt. Interessant ist die Möglichkeit bei mainpost.de,
tagesspiegel.de und augsburger-allgemeine.de, den eingegebenen Text sogar hervorzuheben. Diese Funktion geht über die
übliche Art von Kommentaren hinaus. Bei augsburger-allgemeine.de kann der Kommentar vollständig formatiert werden.

\begingroup
  \footnotesize
  \begin{longtable}{p{28mm}p{17mm}p{20mm}p{65mm}}

    \caption{Kategorie \glqq formale Regeln\grqq} \\
    \toprule
    \bfseries Portal & \bfseries Zei\-chen\-be\-gren\-zung &
    \bfseries Überschrift (Pflichtfeld) & \bfseries Sonstiges \\
    \midrule[\heavyrulewidth]
    \endfirsthead

    \toprule
    \bfseries Portal & \bfseries Zei\-chen\-be\-gren\-zung &
    \bfseries Überschrift (Pflichtfeld) & \bfseries Sonstiges \\
    \midrule[\heavyrulewidth]
    \endhead

    \multicolumn{4}{r}{\emph{Fortsetzung auf der nächsten Seite}}
    \endfoot

    \bottomrule
    \endlastfoot

    bild.de
    & keine
    & keine
    & vorsichtig mit Großbuchstaben, Zitate kennzeichnen
    \\*\midrule

    spiegel.de
    & keine
    & optional
    & keine Bilder posten, keine langen Zitate (Links verwenden)
    \\*\midrule

    faz.net
    & 1000
    & ja\footnote{mindestens 100 Zeichen}
    &
    \\*\midrule

    focus.de
    & 800
    & ja
    & reiner Text ohne besondere Kennzeichnungen (z.B. keine Smilies, Hervorhebungen, Chat-Symbole, nur Kleinschreibung,
    usw.), korrektes Deutsch, auf Rechtschreibung/Interpunktion achten, Absätze machen
    \\*\midrule

    welt.de
    & keine
    & keine
    & Zitate kennzeichnen; keine Fremdsprachen, keine Links zu externen Webseiten (seriöse Ausnahmen möglich)
    \\*\midrule

    derwesten.de
    & keine
    & keine
    & Zitate nur mit Quellenangabe
    \\*\midrule

    rp-online.de
    & keine
    & ja
    & deutsche Sprache; Links möglich (keine Links zu Werbung/strafbare Inhalte)
    \\*\midrule

    handelsblatt.com
    & 2000
    & keine
    & Großbuchstaben (Schreien) vermeiden; Absätze machen und strukturieren; Wortwahl überprüfen; auf Rechtschreibung
    achten; 	Zitate/Quellen kennzeichnen
    \\*\midrule

    suedkurier.de
    & 1000
    & ja
    & Zitate kennzeichnen mit Quellenangabe
    \\*\midrule

    zeit.de
    & 1500
    & ja\footnote{mindestens 5 Zeichen}
    & Absätze machen; auf Rechtschreibung achten; vorsichtig mit Großbuchstaben; Zitate kennzeichnen, wenig verwenden,
    mit Quellenangabe, nur als Ergänzung verwenden; Links möglich
    \\*\midrule

    badische-zeitung.de
    & keine
    & keine
    &
    \\*\midrule

    stuttgarter-zeitung.de
    & keine
    & ja
    & Links möglich (keine Links zu Werbung/kommerziellen Angeboten/Chats/Foren/strafbaren Inhalten)
    \\*\midrule

    merkur.de
    & keine
    & keine
    & deutsche Sprache; Links nicht erwünscht (falls doch distanziert sich merkur-online von Inhalten der gelinkten
    Seiten)
    \\*\midrule

    hna.de
    & keine
    & keine
    & deutsche Sprache; Beiträge in Fremdsprachen werden gegebenenfalls entfernt, da für größten Teil der Nutzer nicht
    verständlich; Links nicht erwünscht (falls doch distanziert sich hna.de von Inhalten der gelinkten Seiten)
    \\*\midrule

    mopo.de
    & keine
    & keine
    &
    \\*\midrule

    mainpost.de
    & 1000
    & ja
    & auf deutsche Rechtschreibung achten; korrekte Interpunktion, Absätze machen; Hervorhebungen möglich: fett, kursiv,
    unterstrichen; markieren von Links, Zitaten möglich; einfügen von Emojis möglich (Grinsen, Zwinkern, traurig sein);
    \\*\midrule

    tagesspiegel.de
    & 2000
    & ja
    & auf Rechtschreibung/Grammatik achten; Hervorhebungen möglich: fett, kursiv; markieren von Quellen durch Links;
    markieren von Links/Zitaten; keine Links zu anderen Webinhalten; Zitate als Ergänzung, nicht alleinstehend
    \\*\midrule

    swp.de
    & 3000
    & ja
    &
    \\*\midrule

    augsburger-allgemeine.de
    & keine
    & optional
    & sämtliche Formatierungsmöglichkeiten, auch Emojis; Vorschau möglich!; Button zum Zitieren; Anhängen von Daten
    möglich (begrenzte Größe)
    \\*\midrule

    abendzeitung-muenchen.de
    & 2000
    & keine\footnote{Ein Betreff wird als Pflichtfeld markiert, man kann den Kommentar allerdings auch ohne Überschrift
      senden.}
    & Links möglich, aber Benutzung auf eigene Gefahr
    
      \\*\midrule

    fr-online.de
    & keine
    & keine
    & 

  \end{longtable}
\endgroup

% vim: set ai si et tw=120 sts=2 ts=2 sw=2:
