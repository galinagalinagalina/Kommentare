\chapter{Kategoriensystem}

%1. Art der Moderation
%2. Registrierung
%3. Inhaltliche Regeln
%4. Zusammenfassung

\section{Einführung}

Sprechblasen bei den einzelnen Artikeln der Online-Zeitungen weisen in der Regel darauf hin, dass kommentiert wurde
(mit Angabe der Anzahl der abgegebenen Kommentare) und/oder ein Kommentar
geschrieben werden kann. Keine Sprechblasen machen zeit.de und tagesspiegel.de. Sie verlinken mit dem Wort
\glqq Kommentar\grqq. Swp.de verwendet weder Sprechblasen noch benutzt diese Zeitung einen anderen 
Link direkt zu Kommentaren\footnote{Ist das der Grund, warum extrem wenig kommentiert wird? Die Nutzer 
müssen selbst bis zum Ende
des Artikels scrollen, um zur Kommentarfunktion zu gelangen. Darüberhinaus hat swp.de einen begrenzten Zugang. 
Den Nutzern wird es schwer gemacht einen Kommentar abzugeben.}.
Beim Kommentar selbst ist eine Zeitangabe üblich
entweder mit Datum und Uhrzeit oder \glqq vor \ldots Stunden\grqq.  Kommentare
stehen unter dem Artikel in vielen Fällen nach Werbung (in eigener Sache).

Auf den Nachrichtenportalen wurden folgende Besonderheiten beobachtet:
Auch auf formaler Ebene werden Kommentare beschränkt. Es gibt ganz
unterschiedliche Handhabungen. Handelsblatt, Hamburger Morgenpost, Die Welt und Frankfurter Rundschau bieten Kommentierzeiten an. 
Nur dann gibt es die Möglichkeit zu schreiben. Außerhalb dieser Zeiten ist es nicht
möglich (handelsblatt.com: 7.30 - 21 Uhr (bis zu sieben Tage lang kann man Kommentare schreiben), mopo.de: 8 - 21 Uhr, 
welt.de: werktags von 6 - 23 Uhr, samstags, sonntags und feiertags von 7 - 23 Uhr und fr-online.de: 7 - 22.30 Uhr).

Faz.net und welt.de\footnote{Welt.de schließt die Kommentarfunktion nach zwei/drei Tagen oder früher bei
  Regelverstößen.} schließen die Kommentarfunktion nach einer bestimmten
Zeit. Dann können die verfassten Kommentare zwar noch gelesen werden, aber keine
neuen mehr geschrieben. 

Bei  bild.de, suedkurier.de und zeit.de wird in den Richtlinien erwähnt, dass nicht jeder Artikel
kommentiert werden kann.

Mainpost.de, swp.de und augsburger-allgemeine.de haben das Problem eines begrenzten
Zugangs, d.h. es kann nur eine bestimmte Anzahl an Artikeln im Monat kostenlos gelesen werden. 

%Faz.net und handelsblatt.de weisen darauf hin, dass Kommentare im Internet recherchierbar sind und frei zugänglich (steht jetzt bei inhaltliche Regeln).
Stuttgarter-zeitung.de, hna.de und abendzeitung-muenchen.de betonen, dass es sich bei den Kommentaren um eine freie 
Meinungsäußerung der Nutzer handelt. 

Der Anbieter {\slshape Disqus} stellt eine Oberfläche bzw. Technik zur Verwaltung von Kommentaren zur Verfügung. 
Die Nutzungsbedingungen können vom Kunden selbst gewählt werden, die Form bleibt jedoch 
bei allen Nutzern von {\slshape Disqus} gleich. Für diese Art der Kommentarfunktion haben sich diese
Zeitungen entschieden: welt.de, merkur.de, hna.de, mopo.de, fr-online


%Zeile 1
%bild.de &
  %K. nicht zu allen Artikeln möglich (themenunabhängig) \\\midrule

%Zeile 2
%spiegel.de &
 % keine Sprechblase sondern Hinweis auf [Forum] auf Startseite; beim Artikel: Sprechblase mit
%  Ausrufungszeichen; K. durchnummeriert \\\midrule

%Zeile 3
%faz.net &
  %K.-Funktion kann irgendwann eingestellt werden; %zeitliche Begrenzung um K. zu schreiben; 
 % Hinweis, dass die K. im Internet recherchierbar sind\\\midrule

%Zeile 4
%focus.de &
  %auch Videos kommentierbar \\\midrule

% Zeile 5
%diewelt.de &
  %über Disqus verwaltet; %Schließung nach zwei/drei Tagen oder früher bei Regelverstößen \\\midrule

% Zeile 6
%derwesten.de &
 % alle Artikel kommentierbar %aber kein direkter Link/keine Sprechblasen; K.
%  durchnummeriert \\\midrule

% Zeile 9
%rp-online.de &
%  (fast) alle Artikel kommentierbar%fast jedes Thema kommentierbar
  %\\\midrule

% Zeile 10
%handelsblatt.de &
%  Kommentierzeiten: 7.30 - 21 Uhr, bis zu sieben Tage lang; 
%K.  werden (u.U.  gekürzt) multimedial verbreitet \\\midrule

% Zeile 11
%suedkurier.de &
 % K. nicht zu allen Artikeln möglich (themenunabhängig) \\\midrule

% Zeile 12
%zeit.de &
 % keine Sprechblase aber [Anzahl + Kommentare]; K. werden durchnummeriert;  
%  K. nicht zu allen Artikeln möglich; %Definition von K.: ``kürzere Textbeiträge,
 % die Sie unter vorhandenen Artikeln, Videos, Fotostrecken oder anderen
 % Multimedia-Inhalten abgeben können'' \\\midrule

% Zeile 13
%badische-zeitung.de & 
%K. geben nicht unbedingt die Meinung der Zeitung wieder
  %\\\midrule

% Zeile 14
%stuttgarter-zeitung.de &
 % für alle Nutzer, %K. geben nicht die Meinung der Stuttgarter Zeitung
 % wieder\\\midrule

% Zeile 15
%merkur.de &
%  über Disqus verwaltet, 
 % für alle Nutzer
  %\\\midrule

% Zeile 16
%hna.de &
 % über Disqus verwaltet,
 % für alle Nutzer; %freie Meinungsäußerung; %bei Verstößen
  %Aufruf eine Email zu schreiben \\\midrule

% Zeile 17
%mopo.de &
 % über Disqus verwaltet; %Kommentierzeiten: 8 - 21 Uhr\\\midrule

% Zeile 18
%mainpost.de &
  %K. unter Artikel und neben Artikel (als neuer Tab!); 
  %begrenzter Zugang\footnote{Das heißt, dass nur eine bestimmte Zahl an Artikeln im Monat
  %kostenlos lesbar sind.}\\\midrule

% Zeile 19
%tagesspiegel.de &
%  keine Sprechblase aber [Anzahl + Kommentare]\\\midrule

% Zeile 20
%swp.de &
  %kein Link auf K.; begrenzter Zugang\\\midrule

%Zeile 21
%augsburger-allgemeine.de &
  %begrenzter Zugang; %K. durchnummeriert
