\section{Zusammenfassung}
<<<<<<< Updated upstream

Im Folgenden werden in Tabelle \ref{tab:zus} nochmals ausgewählte Kategorien kompakt dargestellt.

Die Spalte {\bfseries Moderation} sagt, was für eine Moderation gemacht wird.
Die jeweiligen Moderationsarten werden wie folgt abgekürzt:
\begin{itemize}
  \item Prä-M. = Prä-Moderation bzw. st.~Prä-M. = stichprobenartige Prä-Moderation
  \item Post-M. = Post-Moderation bzw. st.~Post-M. = stichprobenartige Post-Moderation
\end{itemize}

Die Spalte {\bfseries Registrierung} sagt, welche Angaben nötig sind zu einer Registrierung:

\begin{itemize}
  \item KN = Klarnamen,
  \item BN = Benutzername oder
  \item alt = alternative Anmeldemöglichkeiten (Facebook, Disqus, Twitter oder google+)
=======
Ausgewählte Kategorien werden hier gekürzt 
dargestellt.\\
Die Spalte {\bf Moderation} sagt, was für eine Moderation gemacht wird.
Die jeweiligen Moderationsarten werden wie folgt abgekürzt:\\
Prä-M. = Prä-Moderation bzw. st. Prä-M. = stichprobenartige Prä-Moderation\\
Post-M. = Post-Moderation bzw. st. Post-M. = stichprobenartige Post-Moderation\\
Die Spalte {\bf Registrierung} sagt, welche Angaben nötig sind zu einer Registrierung: KN = Klarnamen, BN = Benutzername, alt = alternative Anmeldemöglichkeiten oder Disqus, wenn man sich nur über Disqus kommentieren kann.\\
In der Spalte {\bf Bewerten} bzw. {\bf Antworten} geht es darum, ob der Nutzer die Option hat, den Kommentar zu bewerten bzw. auf einen 
Kommentar zu antworten. \\
Bei {\bf Community} geht es darum, ob eine solche vorhanden ist. dis gibt an, dass es sich um die Disqus-Community handelt. 
>>>>>>> Stashed changes
%alternative Anmeldemöglichkeiten: Fb = Facebook, Dis = Disqus, Tw = Twitter, G = Google
\end{itemize}

In der Spalte {\bfseries Bewerten} bzw. {\bfseries Antworten} geht es darum, ob der Nutzer die Option hat, den Kommentar
zu bewerten bzw. auf einen Kommentar zu antworten.

Bei {\bfseries Community} geht es darum, ob eine solche vorhanden ist. dis gibt an, dass es sich um die Disqus-Community
handelt.



%\begin{landscape}
\begin{table}
  \footnotesize
  \caption{Zusammenfassung\label{tab:zus}}
  \begin{tabular}{p{28mm}*{5}{l}}

    \\\toprule
    \bfseries Portal & \bfseries Moderation & \bfseries Registr. & \bfseries Bewerten & \bfseries Antwort & \bfseries Community\\*\toprule

bild.de
& keine
& KN, BN, alt
& ja
& ja
& ja
\\*\midrule

spiegel.de
& Prä-M.
& KN, BN, alt
& nein
& ja
& ja
\\*\midrule

faz.net
& Prä-M.
& KN
& ja
& ja
& ja
\\*\midrule

focus.de
& st. Prä-M.
& KN, alt
& ja
& ja
& ja
\\*\midrule

welt.de
& Prä-M.
& KN, BN, alt
& ja
& ja
& dis
\\*\midrule

derwesten.de
& keine
& KN, BN
& nein
& ja
& nein
\\*\midrule

rp-online.de
& keine
& BN
& ja
& nein
& nein
\\*\midrule

handelsblatt.com
& st. Post-M.
& KN
& nein
& ja
& nein
\\*\midrule

suedkurier.de
& st. Post-M.
& KN, BN
& ja
& ja
& nein
\\*\midrule

zeit.de
& Prä-M.
& BN
& ja
& ja
& nein
\\*\midrule

badische-zeitung.de
& st. Post-M.
& KN
&nein
&nein
& nein
\\*\midrule

stuttgarter-zeitung.de
& Prä-M.
& KN, alt
& nein
& ja
& nein
\\*\midrule

merkur.de
& keine
& BN, alt
& ja
& ja
& dis
\\*\midrule

hna.de
& keine
& BN, alt
& ja
& ja
& dis
\\*\midrule

mopo.de
& keine
& Disqus
& ja
& ja
& ja
\\*\midrule

mainpost.de
& Prä-M.
& KN, BN
& ja
& ja
& ja
\\*\midrule

tagesspiegel.de
& Prä-M.
& BN
& nein
& ja
& nein
\\*\midrule

swp.de
& keine/st. Post-M.
& KN, BN
& ja
& ja
& nein
\\*\midrule

augsburger-allgemeine.de
& Post-M.
& KN, BN
& nein
& ja
& ja
\\*\midrule

abendzeitung-muenchen.de
& keine/Post-M.
& keine
& ja
& ja
& nein

\\*\midrule

fr-online.de
& keine/Prä-M.
& Disqus
& ja
& ja
& dis

\\*\bottomrule
% FIXME Legende?
% REGISTRIERUNG (mypass = Anmeldung über ``mypass'', Fb = Facebook, Tw = Twitter, G = Google, Dis = Disqus)

  \end{tabular}
\end{table}
%\end{landscape}

% vim: set ai si et tw=120 sts=2 ts=2 sw=2:
