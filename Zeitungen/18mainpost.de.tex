#Sprechblase (Anzahl) nach Einleitung Artikel

INTERESSANT: Kommentar steht als neuer Tab neben Artikel ODER beim Tab Artikel unter dem Artikel (aber ohne weitere Möglichkeiten)
Kommentar als eigener Tab: Überschrift und Einleitung des Artikels; dann #Sortierung nach ``Älteste zuerst'', Neueste zuerst'', ``best bewertete zuerst''
weitere Möglichkeiten: #``Antwort schreiben'' #``Kommentar melden'' (über Formular, warum man melden will, Angabe Email wegen Rückfragen: 

#In dem Kommentar sind ehr- oder persönlichkeitsverletztende Äußerungen enthalten? Oder rassistische, gewaltverherrlichende oder Aufrufe zur Gewalt? #Kommentare, die gegen Buchstabe und Geist des Grundgesetzes der Bundesrepublik Deutschland verstoßen, werden von der Redaktion gelöscht.

#mit Datum + Uhrzeit

#beim Tab Artikel vor Kommentar: Verweise auf facebook, g+, twitter (mit DOUBLE CLICK); ``kontakt REdaktion (über Formular), ``an Bekannten versenden'' (NUR MIT REGISTRIERUNG möglich)

#Registrierung:
#mit Benutzername, Passwort, Email, Vorname, Nachname + Adresse (KLARNAMENPFLICHT)

#AUSWAHL PROFILSEITE: ``niemandem/allen/nur Mitgliedern zeigen'', 

#Kommentar selbst: Nutzername (Anzahl KOmmentare)

#rechte Spalte unten: ``aktuelle Leserkommentare''
# rechte Spalte noch weiter unten: folgen sie uns auf : facebook, g+, twitter, YOUTUBE, RSS-FEED
# rechte Spalte: ``kommentiert'' (top 5

#SIcherheitsfrage!

#PRÄMODERATION (veröffentlichung nach Prüfung durch die Redaktion)

#KOmmentar selbst: TITEL + TEXT
#HERVORHEBUNGEN MÖGLICH: fett, kursiv, unterstrichen, LINK, EMAIL, ZITAT, EMOTICONS (GRINSEN, ZWINKERN; TRAURIG)

#1000 ZEICHEN
#Anklicken: ``Ich möchte bei neuen Kommentaren per Email benachrichtigt werden

#bitte beachten sie die Verhaltensregeln = Netiquette

Netiquette:
#+ faire, akzeptable, respektvolle Wortwahl
#+ keine verbalen Angriffe, private Details aus dem Leben anderer
#+ kein Aufruf zu Straftaten, Urheberrecht!
#+ Pseudonyme/Nicknames erlaubt (nicht ehrverletzenden, beleidigenden oder hetzerischen )
#+ keine privaten DAtenb
#+ Ironie/andere Stilmittel sind missverständlich
#+ deutsche Rechtschreibung; korrekte Interpunktion/Absätze erleichtern Lesen
#+ keine WErbung (ansonsten Löschung)

#KEINE Nutzungsregeln bei AGB

#PROBLEM: kein unbegrenzter Zugang, nur bestimmte Zahl an Artikels im MOnat kostenlos lesbar und somit auch die Kommentare































Mainpost legt großen Wert auf Datenschutz (siehe Double Click, Auswahl zur Profilseite)