#Sprechblasen (Anzahl) bei Überschrift oder im Bild

#Verweise auf soziale Netzwerke  gleich ganz oben links neben MERKUR-ONLINE.de: twitter, facebook , YOUTUBE, g+

# jeder Artikel mit Bild: unmittelbar darunter: Sprechblase (Anzahl), facebook (Anzahl Empfehlungen), twitter (Anzahl tweets), g+ (Anzahl), BRIEFSYMBOL =  #FEEDBACK (KEIN Versenden)

#Sortieren nach: Best, newest, oldest (ENGLISCH!)
#außerdem: Recommend (Herzsymbol) Diskussion empfehlen, Share = teilen 

#Voten möglich: up /down (down nur nach Registrierung
#Reply = antworten
#Share auf sozialen Netzwerken

#LOGIN möglich mit: Disqus, facebook, twitter, google

andere Kommentare von komplett anderen Themen unter dem angeklickten Kommentar

#Zeitangabe: vor soundsoviel Stunden/Tagen

auf Disqus: Top Diskussionen, Aktuell, Top Kommentatoren

wo stehen Kommentare: unter Artikel, direkt, dazwischen nur ein kurzes ``Mehr zum Thema''

#keine Zeichenbegrenzung



#Registrierung:
#Benutzername + Emailadresse (keine Klarnamenpflicht) Nicknames erwünscht, auch Bild möglich, aber keine unerwünschten Nicknames; keine Werbung
#das Motiv keinen anderen beleidigen oder verletzen könnte. Rassistische, pornografische, menschenverachtende, beleidigende oder gegen die guten Sitten #verstoßende Abbildungen sind verboten.



Netiquette:
#+ keine Beleidigungen, Beschimpfungen
# + sachlich
#+ keine persönlichen Angriffe, andere Meinungen akzeptieren 

+ ie Einstellung folgender Inhalte ist nicht zulässig:

    (als Meinung getarnte) Werbung
    Inhalte, die vorsätzlich unsachlich oder unwahr sind
    Inhalte, die Urheberrechte oder sonstige Rechte Dritter verletzen oder verletzen könnten
    Inhalte, die pornographische, sittenwidrige oder sonstige anstößige Elemente enthalten Inhalte, die Jugendliche gefährden, beeinträchtigen oder nachhaltig schädigen könnten
    Inhalte, die strafbarer oder verleumderischer Art sind
    Inhalte, die verfassungsfeindlicher oder extremistischer Natur sind oder von verbotenen Gruppierungen stammen
    Inhalte, die Viren oder sonstige schädliche Bestandteile enthalten
    Links zu verlagsfremden Internetseiten; falls entsprechende Inhalte dennoch publiziert werden, distanziert sich merkur-online.de ausdrücklich von allen Inhalten auf den gelinkten Seiten und macht sich deren Inhalt nicht zu eigen
    Inhalte, die private Daten wie Namen, Postadressen, Telefonnummern oder E-Mail-Adressen enthalten
    Beschimpfungen, Beleidigungen, illegale und ethisch-moralisch problematische Inhalte
    
#+ DEUTSCH diskutieren
+ Entfernen, wenn kein Zusammenhang, privater Chat
#+ Verantwortung für Inhalte beim Nutzer
#+ MELDEN = Email schreiben (kein Formular, Mailprogramm öffnet sich)
#+ Nutzer können gesperrt werden
#+ wenn entfernt, dann alles, nur Teile ist nicht möglich
#+ freundlich, verständlicher Umgangston; man soll Spaß haben und sich wohl fühlen


AGB:
#+ alle Nutzer teilnahmeberechtigt
+ nur für persönliche Zwecke, keine Vervielfältigung (Urheberrechte, Rechte an Inhalten)
+ kein Recht auf VEröffentlichung
+ Der Anbieter macht sich deren Inhalt und deren Aussagen nicht zu eigen und distanziert sich ausdrücklich von allen nutzergenerierten oder sonst erkennbar fremden, d.h. nicht vom Anbieter stammenden Inhalten.

+ Pflichten des Nutzers:
Ausdrücklich untersagt ist die Einstellung von rassistischen, pornographischen, menschenverachtenden, beleidigenden, zu Straftaten anleitenden und gegen die guten Sitten verstoßenden Beiträgen. Ausdrücklich verboten ist die Verbreitung von Inhalten, mit denen zum Hass gegen Teile der Bevölkerung aufgerufen wird (Volksverhetzung) oder mit denen Propaganda für eine verfassungsfeindliche Organisation betrieben wird, sowie verleumderische, beleidigende oder ruf- oder geschäftsschädigende bzw. persönlichkeitsrechtsverletzende Äußerungen sowie Junkmails, Spam, Kettenbriefe und andere Inhalte mit werbendem Charakter. Ebenfalls untersagt ist die Einstellung von Beiträgen, die gegen die Grundsätze der Datensicherheit verstoßen (z.B. mit Viren, Würmern, Trojanern u.ä. behaftete Beiträge).

+ Nutzungsrechte
Der Nutzer räumt dem Anbieter mit Einstellung seines Beitrages (Text, Bild, Audio-Datei/Podcast, Video usw.) das Recht ein, diesen unentgeltlich, sowie zeitlich und räumlich unbeschränkt zu veröffentlichen und/oder veröffentlichen zu lassen, zu verbreiten und/oder verbreiten zu lassen, sowie auf sonstige Weise Dritten öffentlich zugänglich zu machen. Von der Rechteeinräumung umfasst ist die Möglichkeit, Beiträge zum Abruf durch Dritte zur Verfügung zu stellen und Beiträge zu archivieren. Ebenfalls räumt der Nutzer dem Anbieter das Recht ein, Beiträge in anderen Medien der Mediengruppe Münchner Merkur / tz, im Print-, Online-, Hörfunk- oder TV-Bereich, und durch alle anderen Nutzungsarten – auch in gekürzter oder in sonstiger Weise bearbeiteter Form - zu verwenden und/oder zu veröffentlichen bzw. verwenden und/oder veröffentlichen zu lassen.

#+ Löschen bei Missachtungen
