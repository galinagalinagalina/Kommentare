#nicht alle Artikel lassen sich kommentieren, d.h. bei manchen Artikeln ist nur ``Ihre Reaktion zu diesem Thema'' möglich 
``Ihre Reaktion zu diesem Thema'': man kann anklicken: ``Lachen'', ``Weinen'', ``Wut'', ``Staunen'', ``wow''


Wo wird angezeigt, dass es eine Kommentiermöglichkeit gibt?: links vom Artikel: ``Kommentare'' und facebook ``gefällt mir''/''teilen'' und twittern 
Möglichkeit zu kommentieren: unter Artikel, nach Werbung in eigener Sache
#SPRECHBLASE IM BILD taucht auf, wenn ein Kommentar möglich ist. NICHT ALLE ARTIKEL/Videos können kommentiet werden! 
#n der Sprechblase: Zahl, wieviele Kommentare es zu dem Thema bereits gibt

#Mit Abschicken eines Kommentars kann man diesen auch gleichzeitig auf facebook veröffentlichen

#``Daumen hoch''-Button und ``Melden''-Button


#KEINE ZEICHENBEGRENZUNG

#NETIQUETTE

#Es gibt: ``meistgeklickte Artikel'' und darunter ``kommentiert letzte 24h'' (auf der Startseite)
bei Kommentare: Ordnen nach ``beliebteste'', ``älteste'', ``neueste'' zuerst

#Briefsymbol: Link zum Artikel kann über eigenes Mailprogramm verschickt werden
#Ausrufezeichen: Formular, um Korrekturen/Tipps anzugeben

#Buttons:
facebook Teilen
Twitter Twittern
g+
t = tumblr
P = Pinterest
#Briefumschlag = direkt Email schreiben
#Sprechblasen = Anzahl der Kommentare


#Reaktion zu diesem Thema möglich: anklickbar: Lachen, Weinen, Wut, Stauen, Wow


#Regeln: Richtlinien zum Kommentieren:
#	für angenehmes Diskussions-Klima: Höflichkeit, Respekt, andere so behandeln, wie man es selber gerne möchte; bei Angriffen: versuchen zu ignorieren, #nicht 			dagegen argumentieren, sachlich bleiben; andere Kommentare als Beispiel anschauen
	#bitte keine privaten Angaben veröffentlichen
	#Verbot Angaben über dritte weiterzugeben oder sie dazu aufzufordern
	nicht erwünscht sind Kommentare mit folgenden Inhalten (siehe Kapitel)
	
	zusätzlich: terroristische oder extremistische politische Vereinigung werbend, zu einer Straftat auffordernd und/oder ehrverletzende Äußerung enthalten, holocaust leugnende


	#Verantwortung des Nutzers für eingestelltes Material (keine Verletzung von Urheber-Marken-Persönlichkeitsrechten)

	#keine Werbung (keine Verbreitung von Werbung irgendeiner Form)
	Die automatisierte Nutzung der Dienste mithilfe von „Robots", „Crawlern" oder anderer Software ist verboten.
	
	Haftungsausschluss für Inhalte fremder Seiten, 
	
	
	#keine Links zu Werbung/kommerzielle Angebote/Chats/Foren
	#Datensicherung (Nutzer muss sich selbst um die Sicherung seiner Daten kümmern)
	
	#KEINE KONTROLLE (Nutzer ist für den Inhalt seiner Äußerungen verantwortlich, Bild.de distanziert sich)
	
	Rechteeinräumung (siehe Kapitel)
	
	Datenschutz
	
	
!!!!!!!!	UNANGEMESSENE BEITRÄGE:
	kein Mobbing erwünscht; wenn jemand nicht mehr angesprochen/kontaktiert werden möchte, dann bitte respektieren
	
	Löschung (oder temporäre oder dauerhafte Sperrung) von unangemessenen Beiträgen, insbesondere:     
	sexuelle Belästigungen, 
	persönliche Beleidigungen und Beschimpfungen,
    	Drohungen,
    	Diskriminierungen,
    	antisemitische und rassistische Aussagen,
    	jede Art von strafbaren Äußerungen.
	ohne sinnvollten Inhalt (Trolle, off-topic)
	bei Spam/Junkmails, Massenpostings
	bei Schadsoftware (keine Daten verwenden, die dem Computersystem schaden)
	
	#Aufforderung Missbrauch anzuzeigen
	
	#Der Nutzer hat keinen Rechtsanspruch auf Veröffentlichung der eingesandten Inhalte.

	
	
	
	``besondere Nutzungsbedingungen'':
			#Verantwortung für Vollständikeit/Rechtmäßigkeit liegt beim Nutzer
			#bei Verstößen: Ausschluss und Löschung der Beiträge
			#kein Rechtsanspruch auf Nutzung der Dienste/Veröffentlichung der eingestellten Inhalte
			
			
			
			
	Rechteeinräumung, wo: bei den AGBs: 
	unbegrenztes Speichern der Kommentare, unentgeltliches Abrufen zum Bearbeiten, Speichern, Ausdrucken, Verlinken, dasselbe für dritte Unternehmer; 			Kommentare dürfen bearbeitet, vervielfältigt, öffentlich gemacht, gesendet, genutzt oder verwertet, redaktionell dargestellt, hervorgehoben, bewertet werden.
	Einverständniserklärung, wo:
	
#Moderation der Kommentare: KEINE




#Registrierung:  notwendig! wahrheitsgemäß Klarnamen, Emailadresse, Passwort, VOLLJÄHRIGKEIT; NICKNAMEN als zusätzlicher Dienst nach ordentlicher Registrierung: kann für Beiträge verwendet werden (ohne jemanden zu beleidigen/verärgern; keine Spitznamen, die rassistisch, gewaltverherrlichend, pornografisch sind oder andere Namen imitieren; Löschung falls notwendig)
	 Klarnamenpflicht: eine Registrierung ist nur mit der Angabe von Klarnamen möglich: Vorname und Nachname
	 über facebook: ja
	 über ``mypass'': ein Anmeldedienst 	
	 kann gelöscht werden, Beiträge bleiben anonymisiert erhalten
	 Einverständnisverklärung
	 

#Buttons für Kommentare: 
	positiv bewerten: ``Daumen hoch'''
	
	

#Buttons für Kommentatoren:  (wurde durch Zustimmung AGBs erlaubt)
	Ranglisten: durch Klick auf den Button kommt man auf eine Rangliste von Kommentatoren (wer wieviel kommentiert hat)
	chronologische Ordnung: durch Klick auf den Button kommt man auf die Chronologie der Kommentare bestimmter Kommentatoren

#NEtitquette:
#Damit sich jeder innerhalb der BILD.de-Community wohlfühlt, gelten auch hier bestimmte Umgangsformen. Nur gegenseitiger Respekt und Höflichkeit sorgen für #eine rege Diskussion.

#Es werden keine Beiträge toleriert, die andere wegen ihres Geschlechts, ihres Alters, ihrer Sprache, ihrer Abstammung, ihrer religiösen Zugehörigkeit oder ihrer #Weltanschauung diskriminieren oder gegen Gesetze verstoßen. Entsprechende Einträge und Userprofile werden umgehend gelöscht.

#Texte, Bilder und Videos dürfen nur hochgeladen werden, wenn die Urheberrechte beim Verfasser liegen. #Zitate sind als solche zu kennzeichnen.

#Eine Nutzung der BILD.de-Community zu kommerziellen Zwecken ist nicht erlaubt. Beiträge, die werblichen, strafbaren, beleidigenden oder anderweitig inakzeptablen Inhalts sind, werden umgehend gelöscht.

#Das Veröffentlichen von Privatadressen, Telefonnummern oder E-Mail-Adressen ist grundsätzlich nicht erlaubt.

#Sachliche Kritik ist in der BILD.de-Community erwünscht, Beschimpfungen, anstößige Inhalte und ähnliches nicht.

#Großbuchstaben werden innerhalb einer Community als „Geschrei“ und äußerst aggressiv empfunden.

#Sie haben einen User-Beitrag gesehen, der der BILD.de-Netiquette nicht entspricht? Dann haben Sie die Möglichkeit, uns mit der „MELDEN“-Funktion darauf hinzuweisen.

#Dazu klicken Sie einfach unter dem Beitrag auf „MELDEN“. In dem neuen Fenster haben Sie die vier Möglichkeiten, den Beitrag einzuordnen, und auch noch eine kurze Begründung zu schreiben.
