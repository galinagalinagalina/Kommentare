#KEINE Sprechblase, nach Einleitung steht ``Anzahl + Kommentare''

#durchnummeriert

Ranglisten:

#auf Startseite:
#rechte Spalte: ``meistgelesen'' + ``meistkommentiert''

#auf Anmeldeseite:
#``meistgelesene Leserartikel''
#``meistkommentierte Leserartikel''

Ranglistenübersicht:
``meistgelesen''
``meistkommentiert''
``meistempfohlen''
``facebook''
``meistgesucht''


nach Anmeldung:
#Empfehlen
#als bedenklich melden
#reaktionen auf diesen Kommentar anzeigen, Antworten auf diesen Kommentar anzeigen
#antwort schreiben


Leserkommentare: 
#Button ``zu den neuesten Kommentaren''
#Button ``alle Kommentare''
#Button ``Leserempfehlungen
#Redaktionsempfehlungen

#unter den Kommentaren:
#``Antwort auf'' (einen anderen Kommentar)

#1500 Zeichen
#Überschrift (mindestens 5 Zeichen)

#Anmelden/Registrieren:
#Automatisch für digitales Abo

#BENUTZERNAME + EMAIL-ADRESSE (Geschlecht fakultativ) #(akzeptieren der AGBs) Profilbild

#Schreiben eines LESERARTIKELS möglich (Ausführliche Meinungsbeiträge und Erfahrungsberichte)
Leserartikel = schreiben aus eigener Erfahrung, persönlicher Blickwinkel; Ziel: Inhalte bereichern, zusätzliche Sichtweisen, Erfahrungsberichte, Meinungen; Leser wissen, wovon sie sprechen, weil sie es selbst erlebt haben od. unmittelbar betroffen sind od. weil sich sich in einem bestimmten Thema sehr gut auskennen.


Nutzungsbedingungen:
#keine Werbung, keine Schadsoftware verwenden

#Behandlung von Kommentaren wie LESERBRIEFE: also Überprüfung, Kürzung, Bearbeitung
#Urheberrechte
#sämtliche Nutzungsrechte gehen an zeit.de
#Diskriminierende, rassistische oder pornographische Beiträge werden auch dann nicht toleriert, wenn Sie im Einzelfall nicht gegen geltendes Recht verstoßen #sollten.
#Sie sind dafür verantwortlich, dass Ihre Beiträge keine Rechte Dritter (insbesondere Persönlichkeitsrechte, Urheber- und Leistungsschutzrechte, Markenrechte) #verletzen und auch im Übrigen nicht gegen geltendes Recht verstoßen.
#Sperrung/Löschung bei Nichtbeachten

#Empfehlungen bei Facebook
#Hier werden aktuelle Empfehlungen aus Ihrem Facebook-Freundeskreis angezeigt.

#Tweets von ZEIT ONLINE Politik

#Artikel: teilen, speichern, drucken

#NETIQUETTE
Themenbezogen
begründet
#guter Umgangston 
#nicht provozieren lassen
#Rechtschreibung
#Absätze
#Vorsicht mit Großbuchstaben 
#ÜBERSCHRIFT wählen
#Vorsicht mit Stilmitteln (Ironie, Zynismus)
#REDAKTIONSEMPFEHLUNG (kann ausgesprochen werden, spiegelt nicht unbedingt die Meinung der Redaktion wieder)
Regeln, die von allen eingehalten werden müssen:
http://www.zeit.de/administratives/2010-03/netiquette/seite-2
 
 Es gibt einige Regeln, die alle Diskussionsteilnehmer einhalten müssen:
 
 #Beleidigungen haben in den Diskussionen keinen Platz. Wenn Sie einem Artikel oder Kommentar widersprechen, kritisieren Sie dessen Inhalte und greifen nicht #den Verfasser an.

#Diskriminierung und Diffamierung anderer Nutzer und sozialer Gruppen aufgrund ihrer Religion, Herkunft, Nationalität, Behinderung, Einkommensverhältnisse, #sexuellen Orientierung, ihres Alters oder ihres Geschlechts sind ausdrücklich nicht gestattet.

#Verleumdungen sowie geschäfts- und rufschädigende Äußerungen dürfen nicht verbreitet werden.

#Nicht prüfbare Unterstellungen und Verdächtigungen, die durch keine glaubwürdigen Argumente oder Quellen gestützt werden, entfernen wir. Bitte bemühen Sie #sich um nachvollziehbare Aussagen.

#Werbung und andere kommerzielle Inhalte haben nichts in den Diskussionen zu suchen. Verzichten Sie auch darauf, in jedem Ihrer Kommentare auf Ihr Blog #oder Ihre Website hinzuweisen. Dazu dient Ihr Benutzerprofil.

#Persönliche und personenbezogene Daten anderer dürfen nicht von Ihnen veröffentlicht werden. Bitte überlegen Sie auch gut, welche Ihrer eigenen Daten Sie #frei zugänglich ins Internet stellen.

#Die Rechte zur Verbreitung der von Ihnen veröffentlichten Inhalte müssen bei Ihnen liegen. Sollten Sie fremde Inhalte wiedergeben, stellen Sie sicher, dass diese #zur Verbreitung freigegeben sind. Auch Gedichte sind in der Regel urheberrechtlich geschützt. Veröffentlichen Sie daher nie mehr als kurze Ausschnitte.

#Zitate müssen eindeutig als solche gekennzeichnet sein und sollten so sparsam wie möglich gebraucht werden. Geben Sie stets die Quelle Ihres Zitats an. #Verwenden Sie Zitate außerdem immer als Ergänzung Ihrer eigenen Aussagen, statt Ihnen einen Großteil Ihres Beitrags einzuräumen.

#Links dürfen in Kommentaren gerne verwendet werden, zum Beispiel um auf weiterführende Informationen zu einem Thema hinzuweisen. ZEIT ONLINE ist nicht #für die verlinkten Inhalte verantwortlich und prüft diese nicht systematisch. Wir behalten uns aber vor, Links zu entfernen, falls die verlinkten Inhalte gegen die hier #aufgeführten Regeln verstoßen.



MODERATION:
#entfernen/kürzen, wenn gegen die Regeln verstoßen wird (auch nur die entsprechenden Passagen)(mit Begründung, warum eingeschritten wurde, mit Anmerkungen und Kennzeichnungen)
#Abbrechen von Diskussionen, wenn vom Thema weg oder zu viele Regelverstöße
#AUSGEWÄHLTE Artikel ohne Kommentarfunktion
#SPERREN von Nutzern bei Wiederholten/schweren Verstößen
#BITTE um Durchlesen vor Abschicken

#``Debattenkultur: ``Aus den Kommentaren'': Was im Kommentarbereich passiert, bleibt nicht im Kommentarbereich. Wir stellen Höhepunkte aus den  #Leserdebatten vor und versehen sie mit einer neuen Fragestellung.


#``Kommentarkultur: Bitte weichen Sie vom Thema ab: Kommentare, die nicht den Artikelinhalt diskutieren, werden entfernt. Aber was passiert, wenn ein Artikel gar keinen Inhalt hat? Probieren Sie es aus. Ich bin kein Artikel. Ich habe kein Thema für Sie, liebe Leser. Ich beschreibe nichts, ich vermittle keine Informationen, ich äußere keine Meinung.

Einen Kommentarbereich habe ich trotzdem. Und ich bin gespannt, was dort passieren wird. Ob dort etwas passieren wird. Kommentieren Sie, wenn Sie wollen. Worüber Sie wollen. Egal, worüber Sie schreiben, es wird nicht themenfern sein, oder off topic, wie Sie es gern nennen, denn es gibt ja kein Thema, von dem Sie sich entfernen könnten. Deshalb ist jeder Kommentar, den Sie hier verfassen, automatisch passend.
Anzeige

Dies ist ein Experiment. Sie selbst bestimmen, wo es hinführt. Nur auf eines sollten Sie dabei achten: Mit Ausnahme der Gestattung themenferner Kommentare gilt auch hier die Netiquette.

















