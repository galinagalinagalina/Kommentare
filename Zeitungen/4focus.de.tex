
#keine Moderation, keine Kontrolle der Kommentare! Nehmen sich aus der Verantwortung für Schäden
#wenn auf Rechtverletzungen hingewiesen wird, dann kommt es zur Sperrung (nach Prüfung) (vorübergehend oder dauerhaft von der Nutzung des Angebots)
#deswegen auch fett gedruckter Achtung-Ausrufe-Button zum Melden von Missbrauch! Nutzer stecken in der Rolle des Moderators; Selbstregulierung!



#KEIN ``feedback''-Button



#Kommentierbarkeit welcher Themen: Artikel, Videos
wo stehen Kommentare? Artikel, ``aus unserem Netzwerk'' (Werbung in eigener Sache), Kommentiermöglichkeit
Ausschluss von Kommentaren UND Kommentatoren mit demagogischen, sexistischen oder rassistischen Äußerungen



#Länge der Kommentare: 800 Zeichen

#Richtlinien zum Kommentieren (Netiquette, AGBs, Rechteeinräumung, Einverständniserklärung)
#wo stehen Richtlinien? beim login


AGBs: Sperrung und Entfernung von Inhalten 
			Verantwortung des Nutzers für eingestelltes Material (keine Verletzung von Urheber-Marken-Persönlichkeitsrechten); Rechteeinholung, wenn er anderes als seines 					benutzt
			sonstiges Regeln: keine Schadsoftware verwenden, keine Werbung, kein Missbrauch von Daten, Datensicherung 
			Folgen von Verstößen: 
			
			
			
	Rechteeinräumung, wo: bei den AGBs: unbegrenztes Speichern der Kommentare, unentgeltliches Abrufen zum Bearbeiten, Speichern, Ausdrucken, Verlinken, 										dasselbe für dritte Unternehmer; Kommentare dürfen bearbeitet, vervielfältigt, öffentlich gemacht, gesendet, genutzt oder 										verwertet, redaktionell dargestellt, hervorgehoben, bewertet werden.

Regeln: 
	Allgemeines (Regeln eines guten und respektvollen Umgangs miteinander einhalten; nicht gegen gesetzliche Verbote, die guten Sitten, Rechte Dritter verstoßen; 				ausdrücklich nicht erlaubt ist pornografische, jugendbeeinträchtigende, gewaltverherrlichende, volkverhetzende Inhalte zu verbreiten, zu Straftaten aufzurufen, 				Anleitungen hierfür darzustellen oder politische, weltanschauliche oder religiöse Anschauungen Dritter zu verletzen
	Hinweis auf Netiquette
	
	keine Schadsoftware (keine Daten verwenden, die dem Computersystem schaden)
	keine Werbung (keine Verbreitung von Werbung irgendeiner Form)
	kein MIssbrauch von Daten (bekannt werdende Informationen über andere Nutzer oder Inhalte sollen vertraulich behandelt werden, außer sie sind veröffentlicht)
	Datensicherung (Nutzer muss sich selbst um die Sicherung seiner Daten kümmern)
	Folgen von Verstößen: Haftungsausschluss von focus online; Haftung für entstandenen Schaden beim Nutzer: ``FOCUS Online haftet nicht für den Inhalt von Nutzerbeiträgen und verweist auf die Eigenverantwortung der Leser für ihre Beiträge.''

	
#Netiquette: 
	gewünscht ist reiner Text (ohne besondere Kennzeichen, Kennzeichnungen, plain text), korrektes Deutsch gemäß Rechtschreibung und Interpunktion; freundlicher respektvoller toleranter sachlicher Ton; keine Fremdtexte, ARTIKELbezogen kommentieren
	
	
	Regeln: 
	Bitte beachten Sie die Rechtschreibung und Interpunktion. Beiträge mit zu vielen sprachlichen Fehlern, falschem Satzbau, durchgehender Klein- oder Großschreibung, Hervorhebungen, übertriebener Zeichensetzung, fehlenden Abständen, unüblichen Abkürzungen, Smilies oder anderen Chat-Symbolen können wir leider nicht veröffentlichen.
Besonderen Wert legen wir auf einen sachlichen Stil, einen freundlichen Ton, Toleranz und den Respekt vor anderen Meinungen.


Texte sollten von Ihnen stammen, in deutscher Sprache verfasst sein und sich am Artikelthema orientieren.
Unzulässig sind Kontaktadressen, Telefonnummern, Weblinks und gewerbliche oder werbende Hinweise.

	Löschung (oder temporäre oder dauerhafte Sperrung): Demagogische, sexistische oder rassistische Äußerungen, Diskriminierung jeder Art
	

#Moderation der Kommentare: 
keine umfassende Prüfung, Stickproben werden durchgeführt
Beiträge werden geprüft, gegebenenfalls editiert oder abgelehnt, stichprobenartig (?)
Moderator: TOMORROW FOCUS Media GmbH, der TOMORROW FOCUS NEWS+ GmbH und ihrer Beauftragten entscheiden, welche Verhaltensweisen oder Inhalte gegen die genannten Grundsätze verstoßen


#Registrierung 
	mit Klarnamenpflicht 
	auch über facebook, Einverständniserklärung darüber
	Aufforderung richtige und vollständige Angaben zu machen
	
	
	
#Buttons für Kommentare: 
	Melden: ja (besonders hervorgehoben)
	positiv bewerten: ja
	negativ bewerten: ja
	direkt antworten auf Kommentare: ja ``Über die Antwort-Funktion können Sie anderen Usern Ihre Meinung zu deren Kommentar mitteilen und sich mit diesen über das kommentierte Thema austauschen.''


#Buttons für Kommentatoren: 
	Ranglisten (chronologische Ordnung) (es gibt auch Rankings von den aktivsten Kommentatoren (des Monats und gesamt) (top 50); 
	Kommentar des Tages; 
	(Video-) Favoriten der Leser (meistkommentiert)(top 20)

#Buttons für soziale Netzwerke 
	facebook ``teilen'': ja
	twitter ``twittern'': ja
	google ``öffentlich auf google empfehlen'': ja
	xing ``Kontakten auf Xing empfehlen'': ja
	wo stehen die? über dem Artikel und unter dem Artikel, aber vor dem Kommentar


---

#Kommentare abonnieren: ja

#Leserbericht (zusätzlich zum Kommentar mit mehr Zeichen): ja mit 4000 Zeichen mögliche (persönliche Erfahrungen)


#Symbolleiste nach der Überschrift, vor dem Artikel
	Fehler melden
	Buttons für facebook ``gefällt mir'' ``teilen'', g+, Xing, Bewerten mit bis zu 5 Sternen möglich
	Symbole Sprechblase: Kommentar möglich
	Symbol Brief: Seite versenden
	Symbol Drucker: Drucken

#Ausgliederung der Kommentare: nein

#Fremdverwaltung der Kommentare: vom Digitalvermarkter