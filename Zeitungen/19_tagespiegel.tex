#KEINE Sprechblase: nach Einleitung ARtikel in eckigen Klammern: Anzahl Kommentare

#Startseite: oberste Querspalte: Teilen auf sozialen Netzwerken

#Bei Artikel soziale Netzwerke (facebook empfehlen, twitter, g+) über Double Click anklickbar

#Wo steht Kommentar: unter Artkel andere Sachen (``aktuelle Beiträge, Bilder, videos, Anzeigen) dazwischen

#über Artikel: Kommentare: Anzahl, Symbole für Newsletter abonnieren, Artikel versenden (mit Formular und Sicherheitsabfrage), Drucken, Lesezeichen)

#Kommentare selbst: Sortieren nach: ``neueste, älteste, chronologisch''

#Antworten auf Kommentar anzeigbar, man kann drauf klicken
#Button zum Klicken für Antworten

#DAtum + Uhrzeit

#Pseudonym/Nickname

#Kommentar: mit Titel + TExt
#Hervorhebungen möglich: fett, kursiv, Link, Zitat

#2000 Zeichen

#Registrierung notwendig zum Kommentieren + Bewerten (gibt es nicht mehr) mit Benutzername + Email + Passwort

#KEINe BEWERTUNG möglich

#Benutzerprofil: mit was der Kommentator zuletzt kommentiert hat; User-STatistik: ... war seit ... Tagen dabei, das letzte Mal aktiv vor ... Minuten, ... hat .... KOmmentare geschrieben und das Profil wurde ... Mal angesehen



#Prämoderation (redaktionell geprüft und veröffentlicht)

#Richtlinien für ``Community'' = Netiquette (wird nicht so bezeichnet)
#fairer Umgang, angenehme Atmosphäre
#Kommentare werden öffentlich gemacht, dass soll man sich bewusst machen!
#keine Angaben, die NICHT für die Öffentlichkeit bestimmt sind 
#Veröffentlichugn nicht garantiert

#1. mit Respekt; Ziel unserer Moderation ist es, den Rahmen für einen sachlichen Austausch von Argumenten zu schaffen.

#2. Kampagnen und Stigmatisierungen aufgrund von Abstammung, Weltanschauung, religiöser Zugehörigkeit, Nationalität, Geschlecht, sexueller Orientierung sowie #sozialem Status sind nicht zulässig. Kommentare, die auf eine pauschale oder persönliche Herabwürdigung abzielen, werden nicht veröffentlicht.

#3. Sie verpflichten sich, keine obszönen, pietätlosen, menschenverachtenden oder gewaltverherrlichenden Inhalte zu verfassen.

#4. Verleumdungen und andere justitiable Äußerungen wie auch offensichtlicher Geschichtsrevisionismus werden nicht veröffentlicht.

#5. KEINE TROLLE/STÖRENFRIEDEKommentare, die lediglich darauf abzielen, provokativ den sachlichen Austausch von Argumenten innerhalb einer Debatte zu stören (Trolle!), schalten wir nicht frei.

#6. THEMENBEZOGEN, RECHTSCHREIBUNG, GRAMMATIK: Beiträge sollen thematisch zur Debatte passen, verständlich formuliert sein und möglichst den Grundregeln der Rechtschreibung und Grammatik folgen.

7. Mit Abgabe eines Kommentars verpflichten Sie sich, nur Inhalte zu veröffentlichen, für die Sie die Nutzungsrechte besitzen. Das Zitieren von urheberrechtlich geschützten Werken ist nur erlaubt, wenn Sie kürzere Auszüge davon verwenden und dies entsprechend durch Nennung des Urhebers kenntlich machen. Nutzen Sie die Möglichkeit, Quellen zu verlinken. Zitate sollten nie alleine stehen, sondern Ihren Kommentar sinnvoll ergänzen.

#8. KEINE WERBUNG: Inhalte, die vordergründig dem Zweck der Werbung dienen, werden nicht veröffentlicht.

#9. Kritik an unseren Artikeln wissen wir zu würdigen, doch auch diese sollte sachlich formuliert sein und keine Unterstellungen oder beleidigenden Inhalte gegen den Autor enthalten.

#10. Diese aufgeführten Kriterien gelten auch für die von Ihnen eingestellten Links zu anderen Webinhalten.

#11. Mehrfachaccounts bergen die Gefahr einer Diskussionsverzerrung und sind deshalb nicht erwünscht. Bei Nichtbeachtung kann dies zum Ausschluss aus der Community führen

#12. Bitte beachten Sie auch bei der Wahl des Benutzernamens und Profilbilds die geltende Richtlinie.

keine weiteren Regeln / AGBs gefunden!?