rechts vom ARtikel: Symbolleiste mit facebook, Twitter, g+, Zoom, ``Brief'' ARtikel versenden, Sprechblase Kommentare, Drucken

Kommentare stehen am Ende des Artikels (nach ``Video-Empfehlungen'', Anzeigen)
#werden durchgezählt/durchnummeriert



#Kommentare mit Datum und Uhrzeit
#``melden'' möglich
#``antworten'' möglich (nach Anmeldung): wurde auf einen Kommentar geantwortet, dann steht ``Anzahl Antworten''
#Kommentar schreiben nach Registrierung:
	verpflichtend
	Anrede
	Vorname
	NachnameK
	Benutzername (Pseudonym)
	email
	Passwort
	jeder ist zugangs- und teilnahmeberechtigt
	

#alle Nachrichten können kommentiert werden 
#automatische Veröffentlichung
Community Management löscht Kommentare bei Missbrauch

kein Recht auf Veröffentlichung
#KEINE inhaltliche Prüfung auf wahrheitsgehalt
#Beiträge können also: falsche Tatsachen enthalten, Rechte dritter verletzen, in die Irre führen, täuschen 
#offensichtliche rechtswidrige Inhalte werden entfernt, Nutzer wird informiert

Datenschutz

#Pflichten der Nutzer in Bezug auf ihre Beiträge
Der Nutzer ist rechtlich verantwortlich für die von ihm eingestellten Inhalte und hat sicherzustellen, dass keine Informationsangebote mit rechts- oder sittenwidrigen Inhalten aufgenommen werden oder die Aufnahme solcher Inhalte ermöglicht wird, derartiger Inhalt gespeichert, verbreitet, zugänglich gemacht oder auf ein Angebot mit solchem Inhalt hingewiesen wird.
6.2 Der Nutzer verpflichtet sich insbesondere, die von ihm eingestellten Inhalte freizuhalten von Informationen, die der Volksverhetzung dienen; zu Straftaten anleiten oder Gewalt verherrlichen oder verharmlosen; sexuell anstößig sind; gemäß § 184 StGB pornografisch bzw. geeignet sind, Kinder oder Jugendliche sittlich schwer zu gefährden oder in ihrem Wohl zu beeinträchtigen und/oder; die Persönlichkeitsrechte und/oder Schutzrechte Dritter verletzen oder beeinträchtigen; rassistische oder sexistische Äußerungen enthalten; Viren, Umgehungsvorrichtungen im Sinne des Zugangskontrolldienstegesetzes oder unaufgeforderte Massensendungen („Spam)“ enthalten, bzw. auf sonstige Art und Weise gegen die guten Sitten und/oder gegen anerkannte Umgangsformen und Verhaltensregeln in Internet (Netiquette , Chatiquette) verstoßen.
6.3 Der Nutzer  ist allein und umfassend für die von ihm bei DerWesten.de eingestellten Inhalte verantwortlich und ist damit verpflichtet FUNKE MEDIEN sämtlichen Schaden, einschließlich aller Aufwendungen, zu ersetzen, welche dieser aufgrund schuldhafter Verstöße gegen diese Nutzungsbedingungen entstehen.
6.4 Der Nutzer stellt FUNKE MEDIEN ausdrücklich auch von allen Ansprüchen Dritter und den Kosten der damit verbundenen Rechtsverfolgung frei, die gegen FUNKE MEDIEN aufgrund der vom Nutzer eingestellten Beiträge und deren Inhalten geltend gemacht werden.

Nutzungsrechte von Funke Medien 

Haftungsbeschränkungen



Verlinken auf:
NETIQUETTE (unsere vollständigen Nutzungsbedingungen finden Sie hier = roter Balken) 
bei Nichtbeachten: zeitweise/völlige Löschung
    #Beschimpfungen, Beleidigungen oder Kränkungen werden nicht toleriert.
    Pornografische, gewaltverherrlichende, antisemitische, rassistische oder gesetzeswidrige Äußerungen sind in jeglicher Form zu unterlassen. Extremistische Äußerungen werden ohne Ansehen der politischen Orientierung nicht geduldet.
    Fremd- und Parteiwerbung und Links zu Webseiten mit pornografischen, gewaltverherrlichenden, antisemitischen, rassistischen oder gesetzeswidrigen Inhalten sind ebenfalls nicht gestattet.
    Das Verlinken auf Webseiten, wenn diese rechtswidrige Inhalte enthalten oder wenn damit Eigentumsrechte Dritter verletzt werden, ist nicht erlaubt.
    #Für kommerzielle Angebote oder Gesuche ist in den Kommentaren kein Platz. Gerne helfen wir Ihnen bei der Suche nach dem geeigneten Ansprechpartner weiter.
    #Das Forum ist KEIN Veranstaltungskalender. KEINE Terminankündigungen!
    Persönliche Daten wie Adressen, Telefonnummern sind aus Sicherheitsgründen in den Beiträgen nicht erwünscht.
    Mehrfachbeiträge gleichen Inhalts gelten als Spam und werden nicht toleriert.
    Respektieren Sie bitte Urheberrechte. Mit anderen Worten: Posten Sie keine urheberrechtlich geschützten Inhalte, wenn Sie dafür nicht die Genehmigung des Urhebers haben. Besonders bei Bildern achten Sie bitte darauf, dass Bildmaterial, dessen Rechte Sie nicht inne haben lizenzfrei zu verwenden ist.
    #Zitate (zum Beispiel Auszüge aus Zeitungsartikeln) müssen eine Quellenangabe enthalten, sie dürfen inhaltlich nicht verändert werden. Zulässig sind Zitate nur in dem Umfang, in dem sie als Belegstelle oder Grundlage die eigenen Ausführungen dienen. Die darüber hinausgehende Verwendung fremder Werke ist unzulässig. Komplette Texte dürfen nicht in Beiträge kopiert werden. Online-Texte können - soweit Sie den Forenregeln entsprechen - verlinkt werden.
    Missbräuchlicher Gebrauch von Doppelnicks kann zur Sperrung Ihrer Profile führen.