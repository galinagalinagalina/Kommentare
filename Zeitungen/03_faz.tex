Artikel auf Startseite: nach Einleitung: Sprechblase mit Anzahl Kommentare, Sternchen mit Anzahl Empfehlungen

Artikel: 
#rechte Spalte: Sprechblase ``Meinungen'' (Anzahl), Haken ``Merken'', Drucker ``Drucken'', Stern ``Empfehlen (Anzahl), Symbol ``Permalink'', Brief ``Mail'' (= Artikel per E-Mail versenden (über Formular)
Twittern (Anzahl), g+1, 

unter Artikel: dieselben Symbole wie oben rechte Spalte
unter Artikel (OHNE Werbung dazwischen): ``Lesermeinungen''

#Auswahl: ``Vollansicht''/''Kurzansicht'' (vollständiger Kommentar/nur Überschrift)
#Auswahl: Sortieren nach: 
	jüngste/älteste Lesermeinung
	viel/wenig diskutiert
	viel/wenig empfohlen
	
#unter Kommentar: ``Antworten'' ``Verstoß melden'' (nach Anmeldung)
#rechts von Kommentar: bewerten mit Sternchen (Anzahl), dem Kommentator ``folgen'' (mit Anmeldung)

#Kommentar = LESERMEINUNG
rechte Spalte: unregelmäßig auf Webseiten Rubrik ``Lesermeinungen'' mit ausgewählten Kommentaren


#Richtlinien für Lesermeinungen (``Wie sie mit diskutieren'') (DIREKT verlinkt, ausdrücklich Hinweis)
#1000 Zeichen
#Klarnamenpflicht: korrekter Vor- und Nachname (kein Pseudonym/anonym) (mit Begründung warum Klarnamenpflicht)
#Themenbezogen 
#links- und rechtsradikale, pornographische, rassistische, beleidigende, verleumderische sowie ruf- und geschäftsschädigende Inhalte können nicht berücksichtigt #werden, ebenso wenig sachlich falsche oder in angemessener Zeit nicht nachprüfbare Behauptungen. 
#Lesermeinung evtl. kürzen/modifizierne
#Redaktion PRÜFT und veröffentlicht dann (soll Richtlinien entsprechen) 
#Nutzer, die sich nicht daran halten, werden nicht mehr registriert (Einschränken, Beenden)


Rechteeinräumung (Aussagen ganz oder teilweise zu nutzen, zu vervielfältigen, zu modifizieren, anzupassen, zu veröffentlichen, zu übersetzen, zu bearbeiten, zu verbreiten, aufzuführen und darzustellen, Dritten einfache Nutzungsrechte an diesen Aussagen einzuräumen sowie die Aussagen in andere Werke und/oder Medien zu übernehmen.)

#MODERATION: Prä-Moderation (zeitliche Verzögerungen)

#``Mein FAZ.net'': 
	Kommentare können dort verwaltet werden, d.h. auch von der Veröffentlichung ZURÜCKGEZOGEN werden

#HINWEIS, dass Kommentare im gesamten Internet recherchierbar sind

#TOP-ARGUMENTE (werden grün nach dem Namen hervorgehoben)

#DATUM und UHRZEIT, wann Kommentar verfasst wurde

#Registrierung (kostenlos): Klarnamen, Benutzername (wird nicht bei Kommentaren verwendet)‚, Zustimmung zu Datenschutz und Nutzungsbedingung

Datenschutzerklärung

Nutzungsbedingung (im Gegensatz zu Richtlinien):<
- nur für persönliche Zwecke
- keine Weitergabe von Inhalten an Dritte
- keine Hyperlinks
- keine gewerbliche Zwecke
- keine Ansprüche von Dritten (Verantwortung liegt bei Nutzer)
- keine Gewährleistung für Erwartungen des Nutzers, Störungen im ONline-Betrieb, Inhalten von Dritten, Wirtschaftsinhalte, Informationen aus Gesundheitsbereich, für Links auf andere Webseiten
- Haftungsbeschränkung: 


LESERBRIEFE: als Email, werden ausgewählt

#sämlichte Buttons/Symbole/Funktionen werde ERKLÄRT (bei Hilfe)

#es kann nicht ALLES kommentiert werden (wird jedoch nicht ausdrücklich erklärt)

#``dieser Beitrag kann nicht mehr kommentiert werden''






