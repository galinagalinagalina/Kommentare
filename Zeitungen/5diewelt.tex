#benutzt die Kommentarfunktion ``DISQUS''

#Registrieren/Einloggen unter: Disqus, Facebook, Twitter, Google möglich ODER EINGABE von Namen, Email, Passwort (= Anmeldung bei Disqus) ODER ``ich schreibe lieber als Gast''

#``Teilen''-Button: ``Teile diese Diskussion auf twitter, facebook'' (Anmeldung erfolgt dann auf twitter, Facebook)
#``Favorit''-Button: 
#``Gefällt mir''-Button'': zählt die Anzahl der Gefällt-mirs: man muss sich anmelden dafür oder kann als ``gast'' ein Gefällt-mir abgeben
#``Gefällt mir nicht''-Button: (man muss sich anmelden dafür)
#``Antworten''-Button unter dem Kommentar: Antwort-Kommentare werden eingerückt dargestellt; 
#``Teilen''-Button unter dem Kommentar (für facebook, twitter, ?)
#``EMPFEHELN''-Button: Empfehle diese Diskussion
#FÄHNCHEN = als unpassend markieren (
 SORTIEREN: beste, neueste, Älteste

#Fall: ein Kommentar wurde entfernt ``Dieser Kommentar wurde entfernt'', die Antworten darauf sind aber noch sichtbar!

unter einem einleitenden Satz und direkt unter Artikel:
#Sprechblase (mit Anzahl der Kommentare)
#Brief-Symbol: Artikel per Email empfehlen (formular)
#facebook-empfehlen (mit Anzahl der Empfehlungen)
#twittern (mit Anzahl der Twitter-Meldungen)
#g+1 (mit Anzahl der Empfehlungen



LESERKOMMENTARE mit HInweisen zu Datenschutz und MOderation


#MODERATION: PRÄ-MODERATION (Kritik an der MOderationsweise per Email)

#ModerationsZEITEN! Werktags von 6 bis 23 Uhr (samstags, sonntags und feiertags von 7 bis 23 Uhr)

#Moderator: Ein Team von "Welt"-Mita‚rbeitern.

#Moderation erfolgt nach den Nutzungsregeln (das ist keine Zensur, denn Betreiber von Internetseiten sind gesetzlich verpflichtet, beleidigende oder rechtswidrige Beiträge zu entfernen.


#keine Veröffentlichung bei Missachtung der Regelen oder Redigierung, um Löschung zu vermeiden (wird dann angezeigt).
#Folge bei Prä-Moderation: Verzögerungen bei Veröffentlichung
#Kommentarfunktion wird nach ZWEI/DREI Tagen GESCHLOSSEN (oder früher bei Regelverstößen)


#Die Moderation der Kommentare liegt allein bei DIE WELT. Allgemein gilt: Kritische Kommentare und Diskussionen sind willkommen, Beschimpfungen / #Beleidigungen hingegen werden entfernt. Wie wir moderieren, erklären wir in den Nutzungsbedingungen. 


NUTZUNGSREGELN:
#1. Keine Diskriminierungen, Beleidigungen, Provokationen

#Seien Sie gern hart in der Sache, aber höflich und verständlich im Ton. Debatten leben nicht von Grobheiten, sondern davon, dass jeder seinen eigenen #Standpunkt darlegt und sich mit Dritten fair auseinandersetzt. Eine Debatte findet nicht statt, wenn Teilnehmer oder unbeteiligte Dritte diskriminiert, beleidigt oder #provoziert werden. Derartiges lassen wir nicht zu.

#Bei Nichtbeachtung der Nutzungsbedingungen behält die "Welt" sich vor, einzelne Beiträge (ohne vorherige Abstimmung) zu löschen, zu bearbeiten, zu #verschieben, zu schließen oder einzelnen Nutzern zeitweise oder dauerhaft den Zutritt zu den einzelnen Foren zu versperren. Die Regeln gelten im gleichen #Maße für die Verwendung von Benutzernamen und PROFILFOTO.

#2. Relevanz des Kommentars: Themen/Diskussionsbezogen
#3. Zitate und Urheberrecht: Zitat kennzeichnen

Verbreitung/Vervielfältigung von welt.de 

#4. Werbung und personenbezogene Daten: keine werblichen Inhalten, keine Informationen über Personen
#5. Gesetze beachten: Selbstverständlich tolerieren wir keine Nutzerbeiträge, die gegen geltende Gesetze verstoßen.

#Außerdem untersagen wir den Missbrauch der Foren als Werbefläche für Webseiten oder Dienste (Spamming)
# das kommerzielle oder private Anbieten von Waren oder Dienstleistungen
 #Obszönitäten, Pornographie, Hasspropaganda
# Aufforderungen zu Gewalt gegen Personen oder Unternehmen
# Beleidigungen und Entwürdigungen von Personen oder Unternehmen
# Aufruf zu Demonstrationen und Kundgebungen jeglicher politischer Richtung
# Links zu externen Webseiten (Ausnahmen sind möglich bei seriösen journalistischen Quellen und vertretbarem Prüf-Aufwand unseres Moderationsteams)
# fremdsprachige Beiträge
 
 
# ANONYMES Kommentieren möglich! unter Pseudonym
 
 #es gibt eine ``WHITELIST'': darauf stehen Nutzer, die live kommentieren können, also ohne Moderation; das sind die Redaktion und community-Mitglieder, die auffallend positv kommentieren
 
 
#Netiquette (?) veraltet
 
 
 
 
 
 
 
 
 
 
 
 
 
 