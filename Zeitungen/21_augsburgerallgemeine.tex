#Achtung: begrenzte Anzahl der lesbaren Artikel

Kommentare werden mit Sprechblase angezeigt (Anzahl)

Linke Spalte weiter unten: ``Soziale Netzwerke'' (wird auch so bezeichnet!)
Folgen auf g+
Gefällt mir facebook
auf twitter folgen 

Artikel: nach Einleitung Verweise auf soziale netze (gefällt mir/teilen facebook, twittern, g+) 
Hinweis auf DATENSCHUTZ + AGB als BUTTON
unter Artikel: Verweise auf soziale netze (gefällt mir/teilen facebook, twittern, g+) 
Hinweis auf DATENSCHUTZ + AGB als BUTTON
unter ARtikel: Anzeige, dann Kommentar

Kommentar: die neuesten Kommentare, insgesamt ... Beiträge
dann: Datum + Uhrzeit + Kommentator + Anfang des KOmmentars
Wer mehr lesen will oder alle Beiträge sehen/lesen kann klicken und eine neue Seite öffnet sich mit anderem Layout

#POST-MODERATION (Wir bemühen uns die Inhalte, die von Nutzerinnen und Nutzern eingestellt werden, im Nachhinein zu überprüfen. Eine -#hundertprozentige Kontrolle können wir aber nicht gewährleisten.)

#zurückhaltend wie möglich moderieren: Themenbezogen, keine Provokation, wenn gegen Nutzungsbedingung verstoßen wird und/oder rechtlich bedenklich

#Post moderation: Editierung nur für betreffenden Passagen, notfalls ganz

#Es kann zu THREADSCHLIESSUNGEN kommen (bei Verstößen, weg vom Thema, nach ANKÜNDIGUNG oder unmittelbar)

#MELDEN: Meldefunktion oben rechts in Foren, Button ``BEeitrag melden'': Formular (formatierbar) geht auf! nur angemeldete Benutzer können kommentieren

Community (FACEBOOK angelehnt!!): Meine Daten: großes Benutzerprofil (mit Bild, wenn gewünscht, über mich, ich interessiere mich für, ich suche Menschen, die , ich mag... mein Lebensmotto.... (formatierbar)
mein Profil, Blog schreiben, Nachricht schreiben, Album erstellen 
über mich, Einstellungen, Mail-Benachrichtigungen, kennwort ändern
mein Profil, Nachrichten, mein Blog, meine Fotos, meine Bekannten, meine Beiträge, meine Daten, Newsletter, REGELN


Forum - Diskussionen - Plauderecke
Plauderecke:
``ersten ungelesenen Beitrag anzeigen'' ``THema durchsuchen''
Alle Kommentare werden durchnummeriert

Forum: 
Diskussionen: Ordner mit verschiedenen Themen; hinter dem Ordner wird der ``letzte Beitrag'' angezeigt + Anzahl der Themen + Anzahl der Beiträge

Forumseite: Auswahl nach ``neueste/aktuellste Themen'', Benutzerliste, Suche, Reglen, Hilfe

Öffnet man den Ordner wird angezeigt, was alles für Themen im Ordner sind, wieder mit Angabe des ``letzten Beitrags'' + Anzahl Antworten + Anzahl Aufrufe (letzte 30 Tage)

man kann nach Benutzern suchen
--- ganz viele Übersichten , was es für Kommentare gibt, wie viele, über welche Themen, usw. toll!

Auf einen Beitrag antworten: Titel + Antwort (formatierbar) + ANHÄNGE + VORSCHAU + SPEICHERN
Diskutieren Sie mit!: Titel + Text (formatierbar) + Vorschau + Speichern

Mitglieder: mit Bild! Benutzername/Nickname, Registriert seit, Beiträge (Anzahl), Bewertungen zum Kommentator ``Sehr erfahrenes Mitglied'',  ``äußerst erfahrenes Mitglied'' ``erfahrenes Mitglied''
Redaktion = Ehrenmitglied und macht mit dem Artikel den ersten Beitrag

#Registrieren:
#Benutzername + Email + Klarnamenpflicht (wird nicht öffentlich) + WOHNORT (zwingend, nicht öffentlich) + Sicherheitsfrage

rechte Spalte beim Registrieren: ``Aktivste Mitglieder'', ``zuletzt aktive Mitglieder'' ``neues aus der Community'' (mit BILDERN)

Nutzungbedingungen der community: (müssen akzeptiert werden beim Registrieren) (genauso wie Datenschutz):
verbindliche Anerkennung
äußerst umfangreiche Erklärung, was erwünscht ist, und was nicht, was für Fälle es geben kann und wie damit umgegangen werden soll 

1. Präambel

Die Presse-Druck- und Verlags-GmbH, Curt-Frenzel-Str. 2, 86167 Augsburg (nachfolgend: "Betreiberin") stellt eine Online-Plattform (nachfolgend "Community") zur Verfügung, über die Nutzer diskutieren, sich miteinander kontaktieren und austauschen, Bewertungen abgeben, sowie Texte und multimediale Inhalte (nachfolgend kurz "Beiträge" genannt) veröffentlichen können.

erforderliche Alter für den Abschluss eines bindenden Vertrags erreicht haben, dürfen sich ausschließlich bei nachweisbarem Vorliegen einer Einwilligung ihres/ihrer

Lesen der Beiträge auch OHNE Registrierung möglich

Aktivierungsemail

Aussschluss bei Missbrauch (Mehrfachregistrierungen, Scheinregistrierungen, unter den Namen von anderen

Nutzungsrechte gehen an die Betreiberin

HAUSRECHT der Betreiberin = Moderatoren überwachen die Einhaltung der Teilnahmebedingungen, greifen bei Verstößen ein

keine Veröffentlichung von privaten Emails/ persönlichen Daten Dritter 

das Folgen von Tipps/Ratschlägen folgt auf eigene Gefahr

keine Werbung

Keine Haftung für Links zu anderen Websites

ANHÄNGEN von Daten möglich (begrenzte Größe)

Der Nutzer erkennt an, dass die Betreiberin berechtigt ist, allgemeine Regeln und Beschränkungen hinsichtlich der Benutzung der Community, insbesondere hinsichtlich (a) der maximalen Anzahl an Tagen, an denen E-Mail-Nachrichten, Inhalte, die eingegeben wurden, etc. in der Community verbleiben dürfen, (b) der maximalen Anzahl an E-Mail Nachrichten, die von einem Nutzer-Konto in der Community verschickt oder empfangen werden dürfen, (c) der maximalen Größe einer E-Mail-Nachricht, die von einem Nutzer-Konto verschickt oder empfangen werden darf, (d) der maximalen Größe an Festplattenspeicher, der in den Servern der Betreiberin für die Nutzer eingeräumt wird und (e) der maximalen Anzahl der Besuche von Nutzern in der Community innerhalb eines bestimmten Zeitraums (und der maximalen Besuchsdauer) aufzustellen.

7.1. Der Nutzer ist für die von ihm eingestellten Beiträge allein verantwortlich. Die Betreiberin vermittelt lediglich den Zugang zu diesen Beiträgen.

7.2. Mit seiner Registrierung verpflichtet sich der Nutzer insbesondere keine Beiträge mit folgendem Inhalt einzustellen und zu veröffentlichen:

    Inhalte, die hasserregend, ausländerfeindlich, obszön, vulgär, sexuell orientiert, pornografisch, rassistisch, menschenverachtend, abscheulich, bedrohlich, sittenwidrig oder in sonstiger Weise anstößig sind;
    Inhalte, die beleidigend, belästigend, verleumderisch oder in sonstiger Weise strafbar sind;
    Inhalte, die in Rechte Dritter (Persönlichkeitsrechte, Urheberrechte, Leistungsschutzrechte, Eigentumsrechte u.a.) eingreifen bzw. solche Rechte verletzen;
    Inhalte mit verfassungsfeindlichen oder extremistischen Charakter oder die von als verfassungsfeindlich bzw. extremistisch eingestuften Gruppierungen, Parteien, Vereinen etc. stammen;
    gegen geltendes Recht in der Bundesrepublik Deutschland, insbesondere den Jugendschutz, verstoßen;
    Inhalte, die zu widerrechtlichen oder illegalen Handlungen aufzurufen bzw. die Möglichkeit solcher Handlungen in jeglicher Art und Weise begünstigen;

7.3. Dem Nutzer ist es daneben untersagt,

    Kettenbriefe, sog. Spam-Mails, Werbemails zu versenden und Seitenaufrufe künstlich zu generieren,
    Inhalte und Dateien zur Verfügung zu stellen bzw. zum Abruf bereit zu halten, die einen Virus enthalten bzw. in sonstiger Weise die Hard- und Software der Betreiberin und von Dritten beeinträchtigen, beschädigen und zerstören können,
    für eigene oder fremde Produkte und Dienstleistungen zu werben, solche anzubieten bzw. Verlinkungen auf entsprechende Webseiten einzustellen,
    Inhalte einzustellen, die dem Ansehen des Forums Schaden zufügen bzw. den Forenfrieden stören und gefährden können.


7.4. Bei Verstoß gegen die vorgenannten Verpflichtungen und Verbote hat die Betreiberin in Ausübung ihres virtuellen Hausrechts das Recht, Beiträge nach freien Ermessen zu löschen und den Nutzer befristet oder dauerhaft von der Teilnahme am Forum auszuschließen. Etwaige Ansprüche des Nutzers gegen die Betreiberin in diesem Zusammenhang sind ausgeschlossen.

7.5. Im Falle strafrechtlich relevanten Inhalts von Beiträgen behält sich die Betreiberin neben den Sanktionsmöglichkeiten nach Ziff. 6.4. die Erstattung einer strafrechtlichen Anzeige gegen den Nutzer vor. Zum Zwecke der Strafverfolgung werden in diesem Fall auch die entsprechenden Daten (Datum, Uhrzeit, IP-Adresse etc.) an die Strafverfolgungsbehörden weitergegeben.

Nutzer für Beiträge verantwortlich

Haftungsausschluss der Betreiberin










