Kommentare sollen themenbezogen sein

nicht erwünscht sind Kommentare mit folgenden Inhalten

Diskriminierung irgendeiner Art (wegen Geschlecht, Alter, Sprache, Abstammung) religiöse Zugehörigkeit oder Weltanschauung)
demagogische Äußerungen
sexistische Äußerungen/Belästigungen
rassistische Äußerungen 
gegen gesetzliche Verbote
gegen die guten Sitten 
gegen Rechte Dritter
pornografisch
jugendgefährdend
extremistisch
volkverhetzend
gewaltverherrlichend/gewalt verharmlosend
den Holocaust leugnend
grob anstößig
Straftat auffordernd 
ehrverletzende Äußerung
sonstige strafbare Inhalte
verleumdnerisch
ruf- geschäftsschädigend



RECHTEEINRÄUMUNG =
unbegrenztes Speichern der Kommentare, unentgeltliches Abrufen zum Bearbeiten Speichern, Ausdrucken, Verlinken, dasselbe für dritte Unternehmer; Kommentare dürfen bearbeitet, vervielfältigt, öffentlich gemacht, gesendet, genutzt oder verwertet, redaktionell dargestellt, hervorgehoben, bewertet werden.
Nutzer bestätigt Besitz sämlicher Urheber-/sonstigen Rechte


VERANTWORTUNG des Nutzers für eingestelltes Material =
keine Verletzung von Rechten dritter, Urheberrechten, Persönlichkeitsrechten, Patent- und Markenrechten, sonstige Rechten; Vollständigkeit
Rechte zur Verbreitung liegen beim Nutzer, sicherstellen, dass fremde Inhalte zur Verbreitung freigegeben sind, beachten, dass Gedichte urheberrechtlich geschützt sind

KEIN RECHTSANSPRUCH auf Nutzung der Dienste/Veröffentlichung der eingestellten Inhalte

Freistellung des Anbieters von Ansprüchen Dritter  (aufgrund von Verletzungen ihrer Rechte) 

keine WERBUNG/kommerzielle Zwecke =
keine Verbreitung von Werbung in irgendeiner Form, Foren als Werbefläche für Webseiten oder Dienste (Spamming), das kommerzielle/private Anbieten von Waren/Dienstleisungen

DATENSCHUTZ =
 Emails/Abfragen auf Viren überprüfen; gesetzliche/behördliche/technische Vorschriften einzuhalten; Geheimhaltung Passwort; Gewährleistung Vertraulichkeit

HAFTUNGSAUSSCHLUSS für Inhalte fremder Seiten = 
für (direkte/indirekte) Schäden, die durch die Nutzung entstehen; Funktionsfähigkeit/Fehlerfreitheit/Rechtmäßigkeit durch Verlinkung; bei technischen/sonstigen Störungen



Nutzung für privaten Gebrauch gestattet (keine Vervielfältigung, Verbreitung, Änderung)

Einhaltung der Jugendschutzbestimmungen

zeit.de/Netiquette 2. Seite

Es gibt einige Regeln, die alle Diskussionsteilnehmer einhalten müssen:
 
 Beleidigungen haben in den Diskussionen keinen Platz. Wenn Sie einem Artikel oder Kommentar widersprechen, kritisieren Sie dessen Inhalte und greifen nicht den Verfasser an.

Diskriminierung und Diffamierung anderer Nutzer und sozialer Gruppen aufgrund ihrer Religion, Herkunft, Nationalität, Behinderung, Einkommensverhältnisse, sexuellen Orientierung, ihres Alters oder ihres Geschlechts sind ausdrücklich nicht gestattet.

Verleumdungen sowie geschäfts- und rufschädigende Äußerungen dürfen nicht verbreitet werden.

Nicht prüfbare Unterstellungen und Verdächtigungen, die durch keine glaubwürdigen Argumente oder Quellen gestützt werden, entfernen wir. Bitte bemühen Sie sich um nachvollziehbare Aussagen.

Werbung und andere kommerzielle Inhalte haben nichts in den Diskussionen zu suchen. Verzichten Sie auch darauf, in jedem Ihrer Kommentare auf Ihr Blog oder Ihre Website hinzuweisen. Dazu dient Ihr Benutzerprofil.

Persönliche und personenbezogene Daten anderer dürfen nicht von Ihnen veröffentlicht werden. Bitte überlegen Sie auch gut, welche Ihrer eigenen Daten Sie frei zugänglich ins Internet stellen.

Die Rechte zur Verbreitung der von Ihnen veröffentlichten Inhalte müssen bei Ihnen liegen. Sollten Sie fremde Inhalte wiedergeben, stellen Sie sicher, dass diese zur Verbreitung freigegeben sind. Auch Gedichte sind in der Regel urheberrechtlich geschützt. Veröffentlichen Sie daher nie mehr als kurze Ausschnitte.

Zitate müssen eindeutig als solche gekennzeichnet sein und sollten so sparsam wie möglich gebraucht werden. Geben Sie stets die Quelle Ihres Zitats an. Verwenden Sie Zitate außerdem immer als Ergänzung Ihrer eigenen Aussagen, statt Ihnen einen Großteil Ihres Beitrags einzuräumen.

Links dürfen in Kommentaren gerne verwendet werden, zum Beispiel um auf weiterführende Informationen zu einem Thema hinzuweisen. ZEIT ONLINE ist nicht für die verlinkten Inhalte verantwortlich und prüft diese nicht systematisch. Wir behalten uns aber vor, Links zu entfernen, falls die verlinkten Inhalte gegen die hier aufgeführten Regeln verstoßen.