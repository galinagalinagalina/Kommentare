linke Spalte:
	Versenden (Seite kann versendet werden über ein Formular)
	Drucken
	facebook-Symbol (``teilen'', ``empfehlen'')
	twitter-Symbol (``twittern'')
	g+
	Feedback (über Formular)
	
	
unter Artikel
	Feedback an die Redaktion 
	Link kopieren
	Diskussion mit Freunden (facebook, twittern, g+, Briefsymbol ``mailen'' (über Formular Seite versenden)
	rivva-Debattenmonitor (= alle öffentlichen Kommentare, Tweets, posts zu diesem Artikel)
	(evtl. Diskussion mit SZ-Lesern auf facebook)
	
Rivva:
	Links auf den Artikel von Blogs und Nachrichtenseiten
	Facebook-Kommentare zum Artikel 
	Tweets zum Artikel
	
Logo zur Leserbeteiligung: (Ziel: in den Dialog mit dem Leser kommen, klassische Wege verlassen)
	Debatten zu drei Themen des Tages: moderiert
	neue Formate: ``Ihre Post'' (Veröffentlichen von Lesermails
				``Ihre Frage'' (Redakteuren Fragen stellen)
				``die Recherche''
				``Gefahren-Atlas''
	neue Plattform: Disqus (globales Diskussionsnetzwerk) (Anmeldung mit SZ-ID)
	
Diskussionen/Debatten 
	sollen in den sozialen Netzwerken (Disqus, Facebook, Twitter, GooglePlus) stattfinden, wo sie sich ohnehin schon hin verlagert haben.Das ist auch der Grund, 		weshalb Sie künftig unter jedem Artikel angezeigt bekommen, wie dieser in den Social Networks diskutiert wird. Zusammen mit unserem Technikpartner Rivva 		scannen wir Blogs, Twitter und Facebook-Posts auf unserer Facebook-Seite - und lassen auf einer eigenen Rivva-Seite den Stand der Debatte darstellen.
Grund: Schwächen des bisherigen Systems (Diskussion geht durcheinander, Freischalten von Beiträgen dauert zu lange, Pöbler rutschen durch die Moderation)

Ermunterung mitzumachen 

Registrierung: mit SZ-Login
Klarnamenpflicht! bei Anmeldung (später auch Pseudonym möglich) (keine Weitergabe von Daten an dritte)
Urheberrecht beachten! (z.B. Zitate kennzeichnen)
keine Verantwortung für Links


Regeln:
	weg von Kurzkommentaren zu fundierten Arguementationen, die die Diskussion bereichern; konstruktive Beiträge; Einschätzungen zu Sachfragen
	konkretes Zeil mit definiertem Ende
	starker Moderator
	kühlen Kopf bewahren
	niemanden angreifen (keine rassistische, sexistische, homophobe, andere Beleidigungen)
	respektvolle Ansprache (gute/klare Sprache/Ausdrucksweise, guten zwischenmenschlicher Stil
	Rechtschreibung beachten
	Beiträge werden geprüft und selektiert vor der Veröffentlichung
	keine Werbung, Troll-, Spam-Beiträge
	keine mehrfach-Beiträge
	Sperrung von Nutzern bei Nicht-Beachtung
	
	
Netiquette (veraltet)
	
	