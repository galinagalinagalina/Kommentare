unter Einleitung ARtikeL: 
	``Teilen'': facebook, twitter, google+, xing, Email schreiben
	``Merken''
	``Kommentare'' (Anzahl)
	
Kommentare: Auswahl nach Debatten oder Chronologisch
	Kommentatoren mit Namen (kann auch Pseudonym sein (?)) Datum Uhrzeit
	unter Kommentar: Button für Antworten oder als Spam melden
	vorher: Login/Registrieren
	
Registrieren: Email + Passwort (KEINE KLARnamenpflicht)
	volljährig




Kommentare erscheinen NICHT direkt unter dem Artikel: man muss auf den Button zum KOmmentierne klicken und eine neue Seite öffnet sich mit allen Kommentaren und dann auch der Möglichkeit, selber einen Kommentar zu schreiben; der zu kommentierende Artikel erscheint nur mit der Überschrift und der Einleitung


Kommentare werden redaktionell BEGLEITET
bei Missbrauch: Account zeitweise/vollständig sperren
Beiträge, die beleidigend, diskriminierend, rassistisch, pornografisch oder strafrechtlich relevant sind, werden von uns nicht toleriert - Löschung
Junkmails, Spams, Scraping oder sonstige rechtswidrige Kommunikationsformen verwenden. 
keine WErbung
keine Nennung von Produktnamen, Dienstleistern, Marken oder Produzenten 
auffällige Beiträge melden (Email, TELEFON)
Wir gehen der Beanstandung nach, indem wir entweder einen Beitrag sofort sperren oder löschen oder uns mit dem Autor des beanstandeten Beitrags in Verbindung setzen. Das Recht auf Meinungsfreiheit gebietet, dass wir nicht eindeutige Sachverhalte erst ABKLÄREN müssen, bevor wir Sperrungen oder Löschungen vornehmen.

Netiquette:
	Vorsicht bei zynischen/ironischen Äußerungen
	nicht persönlich werden: Persönliche Angriffe gegen andere Nutzer oder soziale Gruppen, Beleidigungen und Diskriminierungen zum Beispiel aufgrund von Religion, Nationalität, sexueller Orientierung, Alter oder Geschlecht sind ausdrücklich nicht gestattet. Gleiches gilt für Verleumdungen sowie geschäfts- und rufschädigende Äußerungen sowie für die Veröffentlichung persönlicher und personenbezogener Daten Dritter. Bitte überlegen Sie zudem gut, welche Ihrer eigenen Daten Sie frei zugänglich ins Internet stellen.
	guter Ton, igonrieren von Provokationen
	Klarnamen (bei Registrierung nicht notwendig (?) (Sperrung/Löschung vorbehalten)
	ABsätze machen, Wortwahl überprüfen
	Großbuchstaben (!)
	keine Werbung, keine kommerziellen Anbieter
	Angabe von ``Zitaten''/QUellen
	Urheberrecht!
	
Editieren/Entfernen bei nicht-Beachten; Entfernen von Textstellen (wird kenntlich gemacht
keine Kommentarfunktion mehr bei vielzahl von Verstößen