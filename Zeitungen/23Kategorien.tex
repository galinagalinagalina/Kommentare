

Es ist noch kein Königsweg gefunden worden, wie man die Moderation von Kommentaren am besten durchführt. Außerdem ist sie von vielen Faktoren abhängig und fällt deswegen bei jeder Zeitung anders aus. In einer Dokumentenanalyse von den meist gelesenen Online-Zeitungen Deutschlands soll eine Übersicht erstellt werden, wie diese Zeitungen mit Kommentaren umgehen. Die Kategorien für die Dokumentenanalyse leiten sich aus den Möglichkeiten des Kommentarmanagements ab. Diese wurden im vorhergehenden Kapitel erläutert und werden nun zusammengefasst. 

Kommentierbarkeit welcher Themen
Richtlinien zum Kommentieren 
Moderation der Kommentare 
Prä-Moderation
Post-Moderation
Registrierung
Registrierung freiwillig
Klarnamenpflicht
Melden von Kommentaren
Bewertungen von Kommentaren
Bewertungen von Kommentierenden
Verweise  auf soziale Netzwerke
Möglichkeit zum Weiterverschicken
Ausgliederung der Kommentare




-------
Kommentare:
	Kommentierbarkeit welcher Themen: was kann alles kommentiert werden? Alle Themen? Artikel, Videos?
	wo befindet sich die Kommentierfunktion?
	Symbole für Kommentarfunktion?
	Länge der Kommentare: wieviele Zeichen stehen dem Kommentator zur Verfügung?
	Leserbericht: zusätzlich zum Kommentar mit mehr Zeichen
	Kommentare abonnieren?
	Ausgliederung der Kommentare
	Fremdverwaltung
	

Regeln: Richtlinien zum Kommentieren:
	Netiquette, wo: Hinweise zu einem guten Umgangston
	AGBs, wo: 
			Sperrung und Entfernung von Inhalten 
			Verantwortung des Nutzers für eingestelltes Material; Rechteeinholung, wenn er anderes als seines benutzt
			sonstiges Regeln: keine Schadsoftware verwenden, keine Werbung, kein Missbrauch von Daten, Datensicherung 
			Folgen von Verstößen: 
			
			
			
	Rechteeinräumung, wo: bei den AGBs: unbegrenztes Speichern der Kommentare, unentgeltliches Abrufen zum Bearbeiten, Speichern, Ausdrucken, Verlinken, 										dasselbe für dritte Unternehmer; Kommentare dürfen bearbeitet, vervielfältigt, öffentlich gemacht, gesendet, genutzt oder 										verwertet, redaktionell dargestellt, hervorgehoben, bewertet werden.
	Einverständniserklärung, wo:
	
Moderation der Kommentare:  (wurde durch Zustimmung AGBs erlaubt)
	Prä-Moderation: bevor ein Kommentar online geht wird er von einem Moderator überprüft
	Post-Moderation: die Kommentare werden erst nachdem sie veröffentlicht wurden von einem Moderator überprüft
	stichprobenartig: nicht alle Kommentare werden gelesen; 
	ohne: alle Kommentare werden unkontrolliert online gestellt

Registrierung:  (wurde durch Zustimmung AGBs erlaubt)
	 freiwillig (mit Privilegien): man muss sich nicht registrieren, um einen Kommentar schreiben zu können; wenn man sich aber registriert bekommt man Privilegien 
	 Klarnamenpflicht: eine Registrierung ist nur mit der Angabe von Klarnamen möglich
	 über facebook: man meldet sich über seinen facebook-Account an und muss keine neue Registrierung über die Online-Zeitung machen
	 Einverständniserklärung darüber)

Buttons für Kommentare: (wurde durch Zustimmung AGBs erlaubt)
	Melden: man zeigt einen Kommentar an, der negativ auffällt
	positiv bewerten: man sagt, dass man einen Kommentar gut findet
	negativ bewerten: man sagt, dass man einen Kommentar schlecht findet
	direkt antworten auf Kommentare (mit vorheriger Registrierung?): einen neuen Kommentar schreiben, der sich wiederum auf einen Kommentar bezieht (und nicht 														direkt auf den Artikel)

Buttons für Kommentatoren:  (wurde durch Zustimmung AGBs erlaubt)
	Ranglisten: durch Klick auf den Button kommt man auf eine Rangliste von Kommentatoren (wer wieviel kommentiert hat)
	chronologische Ordnung: durch Klick auf den Button kommt man auf die Chronologie der Kommentare bestimmter Kommentatoren

Buttons für soziale Netzwerke:
	facebook ``teilen''
	twitter ``twittern''
	google ``öffentlich auf google empfehlen'', 
	xing ``Kontakten auf Xing empfehlen'')
	wo befinden sich diese Buttons? (neben, unter, über der Kommentierfunktion?)


---

Kommentare abonnieren 

Leserbericht (zusätzlich zum Kommentar mit mehr Zeichen) 



Symbole (für Kommentare, für Email schreiben)

Ausgliederung der Kommentare

Fremdverwaltung der Kommentare
