``Beitrag melden''

``Antworten/Zitieren''

Möglichkeit zu Kommentieren unter dem Artikel: dann wird man zum ``Forum'' weitergeleitet: dann kommen die bereits veröffentlichten Kommentare, dann kann man selber einen Kommentar schreiben

Registrierung: Unterscheidung zwischen Registrierung für Abonennten und Nicht-Abonennten; notwendig; verpflichtend: Name, Email, Passwort; auch über facebook möglich; automatische Registrierung

Überschrift optional

Keine Einschränkung bei Länge des K.

Buttons: über Artikel
	facebook ``empfehlen''
	facebook ``teilen''
	``twittern''
	g+
	
linke Spalte: 
	Datum
	Drucken/Senden/Merken (EMAIL schreiben über Formular) (in mein Spiegel)
	Nutzungsrechte/Feedback
	KOMMENTIEREN/....Kommentare


	
im ``Forum'': ``Meistdiskutierte Themen''

``feedback''-Button

Nutzungsbedingungen für die Foren:
- angenehmes Diskussionsklima, fair, sachlich, Themenbezogen!
- keine kommerzielle Zwecke
- LÖSCHEN/NICHT FREISCHALTEN von werblichen, strafbaren, beleidigenden, inakzeptablem Inhalten
- Redaktion kann bearbeiten, verschieben, Diskussion schließen
- MODERATION (ausdrücklich) (PRÄ-MODERATION): Folge: Verzögerungen beim Erscheinen 
- FORUMSMODERATOR: mit Namen ``SYSOP'' 
- keine Benachrichtigung bei Nicht-Erscheinen
- Forum = verbale Auseinandersetzung
- Ausschluss bei Verstößen
- Urheberrecht beim Kommentierenden, erlaubt spiegel online die dauerhafte Einstellung



Nutzungsbedingungen allgemein:

- Pflichten: Der Nutzer verpflichtet sich, bei der Nutzung der Dienste nicht gegen geltende Rechtsvorschriften und etwaige vertragliche Bestimmungen zu verstoßen. Er verpflichtet sich insbesondere dazu, dass von ihm verbreitete Inhalte keine Rechte Dritter, insbesondere Urheberrechte, Persönlichkeitsrechte, Patent- und Markenrechte und sonstige Rechte verletzen, dass die geltenden Strafgesetze und Jugendschutzbestimmungen beachtet werden und dass keine rassistischen, den Holocaust leugnenden, grob anstößigen, pornografischen oder sexuellen, jugendgefährdenden, extremistischen, Gewalt verherrlichenden oder verharmlosenden, den Krieg verherrlichenden, für eine terroristische oder extremistische politische Vereinigung werbenden, zu einer Straftat auffordernden, ehrverletzende Äußerung enthaltenden, beleidigenden oder für Minderjährige ungeeigneten oder sonstige strafbaren Inhalte verbreitet werden.

Außerdem verpflichtet sich der Nutzer, zum Schutz der Daten den anerkannten Grundsätzen der Datensicherheit Rechnung zu tragen und die Verpflichtungen der Datenschutzbestimmungen zu beachten. Der Nutzer verpflichtet sich insbesondere, an den Anbieter ausgehende E-Mails und Abfragen mit größtmöglicher Sorgfalt auf Viren zu überprüfen, gesetzliche und technische Vorschriften einzuhalten, Benutzername und Passwort geheim zu halten, nicht weiterzugeben, keine Kenntnisnahme zu dulden oder zu ermöglichen und bei einem Missbrauch oder Verlust dieser Angaben oder einem entsprechenden Verdacht dies dem Anbieter unverzüglich anzuzeigen. Hinweise auf eine missbräuchliche Nutzung der Inhalte des Anbieters sind ebenfalls unverzüglich anzuzeigen.