#rechte Spalte: KONTAKT
	#Leserbrief schreiben (über Formular)
	#Newsletter
	#Mail an Chefredakteur 
	
#über Artikel: 
	#Stern (Anzahl) ``Empfehlen'' (bezieht sich auf Artikel, nicht Kommentar)
	#Sprechblase (Anzahl) ``Zu den Kommentaren''
	
Diskussion = Kommentarforum
über Sprechblase gelangt man direkt zur``Diskussion''; die befindet sich ganz unten, unter ARtikel; vorher ``video-Empfehlungen'', ``mehr aus dem WEb'', ``das könnte sie auch interessieren''


#Auswahl nach:
	#``älteste KOmmentare zuerst''
	#``mehr Kommentare''


# ganz rechte Spalte:  Symbole (von oben nach unten: facebook, twitter, google+, empfehlen (Anzahl) (Artikel), Artikel per Email empfehlen, Drucken, schriftgröße ändern, zu den Kommentaren, zurück


#Datum, Uhrzeit, Bewertungen (mit Sternchen) (für Kommentar)
#unter Kommentar: ``Als unangemessen MELDEN'' (mit BEGRÜNDUNG, warum gemeldet wird mit Name, Email; OHNE Registrierung möglich)
#nur positive Bewertungen möglich (KEINE Negativen)

Registrieren:
#Benutzername, Passwort, Email (wahrheitsgemäß) Minderjährige werden nicht ausgeschlossen, wenn sie sich über die Konsequenzen bewusst sind bzw. die #Erziehungsberechtigten dies zulassen

#Kommtarforum wird u.U. auch ganz geschlossen (bis dahin veröffentlichte Beiträge werden angezeigt; viele Kommentare bis dahin allerdings auch entfernt)


Regeln für Leserkommentare:
#Betreff eingeben 
#fast jedes Thema ist kommentierbar
#keine Plattform für Beleidigungen, Anschuldigungen, Tatsachenbehauptungen
#Wenn Benutzerzugang gelöscht wird, werden auch Kommentare gelöscht
#Beiträge werden gelöscht, wenn der Redaktion Missbrauch gemeldet wird (#KEINE Moderation: RP online vermittelt lediglich den ZUGANG) auch Kommentare, #die sich auf gelöschte beziehen; Nutzer werden gesperrt

#Umgangston: wie man selbst behandelt werden möchte
Strafbare Inhalte: Beleidigungen, üble Nachrede, sexuelle Anspielungen, Urheberrechtsverstöße, Drohungen und rassistische sowie volksverhetzende Äußerungen werden nicht toleriert. Auch können wir Tatsachenbehauptungen, insbesondere, wenn sie Personen betreffen und nicht bewiesen oder nicht #überprüfbar sind, nicht veröffentlichen.
 
# Sprache: Deutsch
 
 Links: keine Haftung von rp-online; keine Werbung, Links möglich, keine Links auf inhalte mit strafbarem Inhalt
 
 Nutzungsrechte gehen an RP-online
 
 Haftungsausschluss: Alle Beiträge im Meinungsforum geben ausschließlich die persönlichen Ansichten und Meinungen der User wieder. Für die Richtigkeit und Vollständigkeit der Inhalte übernimmt RP ONLINE keinerlei Gewähr, da die Inhalte #nicht von RP ONLINE vorab geprüft werden.
 
 Löschen, wenn: 
 
    einer solchen Diskussion nicht förderlich sind
    sich nicht auf die entsprechenden Beiträge beziehen
    ausschließlich Werbung darstellen
    augenscheinlich sinnlos sind
    Drohungen oder Aufforderungen zu Gewalt gegen Personen, Institutionen, Unternehmen oder Sachen enthalten
    pornografischen Inhalt haben
    Urheberrechtsverletzungen beinhalten
    die Persönlichkeitsrechte Dritter beeinträchtigen
    Tatsachenbehauptungen beinhalten, die nicht bewiesen und nicht überprüfbar sind
    sich auf Kommentare beziehen, die aus den oben genannten Gründen entfernt worden sind.


Allgemeine Bedingungen für die Nutzung von RP ONLINE :
	Links und Werbung
	Rechte an den INhalten
	berechtigte Personengruppen/Registrierung
	Gewährleistung/Haftung: Haftungsausschluss
	Datenschutz
	Forum: Es besteht die Verpflichtung, keine beleidigende, obszöne, vulgäre, verleumdende, gewaltverherrlichende oder aus anderen Gründen strafbare Inhalte in diesem Forum zu veröffentlichen. Beiträge, die gegen diese Regeln verstoßen, werden nach Entdeckung sofort gelöscht und der Nutzer gesperrt. Der Rechtsweg ist ausgeschlossen. RP ONLINE wird das Recht eingeräumt, Beiträge #nach eigenem Ermessen zu entfernen, zu bearbeiten, zu verschieben oder zu sperren. Es besteht kein Anspruch auf die Veröffentlichung eines Beitrags. 
	Pflichten der Nutzer/Haftung: Verbot von
	Inhalte und Formulierungen, die in irgendeiner Weise dem Ansehen von #VERSTORBENE oder deren Angehörigen schaden könnten;
    Inhalte, die in Rechte Dritter (z.B. Persönlichkeitsrechte, Urheberrechte, sonstige Eigentumsrechte) eingreifen;
    Inhalte, die pornografisch, sittenwidrig oder in sonstiger Weise als anstößig einzuordnen sind;
    Inhalte verfassungsfeindlicher oder extremistischer Art oder von verbotenen Gruppierungen stammend;
  #  Inhalte sowie DOPPGELDEUTIGe Bezeichnungen und anderweitige Darstellungen, deren Rechtswidrigkeit vermutet wird, aber nicht abschließend festgestellt werden kann;
    Inhalte, die strafbar, insbesondere volksverhetzend und beleidigend sind;
    #Inhalte, die unsachlich oder UNWAHR sind;
    Inhalte, die lediglich zum Zweck der Verbreitung eines politischen, weltanschaulichen oder religiösen Bekenntnisses eingestellt werden;
    Inhalte, die Produkte oder Dienstleistungen zu gewerblichen Zwecken anbieten und dafür werben;
    Inhalte, die Soft- oder Hardware beeinträchtigen, beschädigen oder zerstören können, insbesondere Viren enthaltende Inhalte.
    Haftungsausschluss für eingestellte Inhalte
    Nutzungsrechte an rp-online

#KEIN Netiquette









