#BEGRENZTER ZUGANG 

Kommentar: Ende ARtikel (nach ``meist gelesen'') und facebook, twitter, g+ , Drucken, Briefsymbol ``Link per Email versenden/weiterempfehlen'' über Formular und Sicherheitsfrage
NEBEN Briefsymbol steht die Anzahl der ``MEinungen'' = Kommentare (aber nur, wenn man den Artikel aufgerufen hat). Auf der STartseite ist das nicht ersichtlich, was kommentiert wurde, es gibt kein Sprechblasensymbol oder anderes



KOmmentare: Button zum ``Antworten'', ``Kommentar melden'', ``BEwerten'' (nur positiv = ``guter Kommentar' (Anzahl) nach vorheriger Anmeldung
#DAtum + Uhrzeit
#3000 Zeichen 
#KLARNAMEN



TABS: ``meist gelesen'', ``zuletzt kommentiert'', ``meist kommentiert''


Netiquette:
#fair, sachlich, freundlich, wie man behandelt werden möchte
#Beleidigende, rassistische, sexistische, vulgäre, hetzerische oder gewaltverherrlichende Töne sind im Forum nicht erwünscht! Entsprechende Beiträge werden von #der Redaktion kommentarlos gelöscht.

#Die Redaktion behält sich grundsätzlich das Recht vor, Beiträge zu löschen, die gegen die Netiquette verstoßen.

Werbliche oder kommerzielle Inhalte werden gelöscht.

#Es ist verboten, private Adressen im Forum zu veröffentlichen.

#Wenn Sie die Redaktion darauf hinweisen möchten, dass ein Beitrag einen Fehler enthalten könnte, verwenden Sie bitte die Funktion "Verstoß melden".
 
#Mit der Teilnahme an der Diskussion erkennen Sie die Richtlinien der Netiquette an. Bei Verstößen kann ein  Diskussionsteilnehmer dauerhaft ausgeschlossen #werden. Eine Registrierung unter einem anderen Namen wird nicht gestattet.
   
#swp.de versucht, alle unerwünschten Beiträge fernzuhalten. Trotzdem ist es uns nicht möglich, alle Beiträge zu überprüfen. Die SÜDWEST PRESSE Online-Dienst  #GmbH, kann nicht für den Inhalt der Beiträge verantwortlich gemacht werden. Jeder Nutzer trägt für seine Beiträge die Verantwortung.
#KEINE MODERATION ??    
     
      
 #Keine Angaben in AGB       


#Registrierung: 
#Benuztername + Passwort + Email + echter Vorname + echter Nachname + SICHERHEITSFRAGE