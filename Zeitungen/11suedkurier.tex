Startseite: keine Verweise auf soziale Netzwerke

Artikel:
linke Spalte: Symbole (untereinander) für facebook, twitter, g+, KOMMENTARE (Sprechblase, grün, Anzahl), rss-feeds, Artikel versenden (über Formular möglich = Artikel weiterempfehlen), Artikel drucken)

kleine Sprechblase hinter Überschrift mit Anzahl KOmmentare (wenn KOmmentare vorhanden)

Kommentarbereich:
#Sortieren nach ``Älteste'', ``Neueste'' (automatisch), ``beste Bewertung''
Unter dem Kommentar: 
#``Netiquetten Verstoß melden'': geht nur nach Angabe von Name und Email-Adresse und Grund; Formular geht an Community Manager: ``Südkurier ist nicht #verpflichtet Inhalte von Nutzern zu überwachen bzw. zu überprüfen'' = KEINE MOderation

Kommentator mit Anzahl der Beiträge

Sicherheitsfrage!

#Überschrift
#1000 Zeichen
#INFORMIERT BLEIBEN: man kann anklicken, ob man ``bei jedem neuen Beitrag in dieser Diskussion erhalten Sie eine entsprechende Benachrichtigung''

Communikty-Regeln: (auf der Seite zum Kommentare schreiben)

#REgistrierung (was verpflichtend ist): 
#Benutzername (wie gewünscht)
#Passwort
#Email
#Anrede
#Vorname
#Nachname
#Land und Adresse
#Angaben werden auf Richtigkeit überprüft (teilsweise auch telefonisch)

#THEMEN: nicht zu allen Artikeln wird die Kommentarfunktion aktiviert

#es gibt in der rechten Spalte weit unten die Rubrik: am meisten kommentierte Artikel (Überschrift des Artikels wird angegeben)

#mit Datum + Uhrzeit und   ``Antwort auf den Kommentar von ....''

Wo steht der Kommentar:
ganz unten vom ARtikel, nach ``mehr zum Thema'', nach ``Korrekturhinweis'' , nach ``neu aus diesem Resort'', nach ``die besten Themen'' KEINE WERBUNG dazwischen; dann ``Kommentare `` mit ANzahl 

themenbezogen (wird aber nicht extra hervorgehoben)

Unsere Netiquette:
Beleidigungen, nicht belegbare Behauptungen, Verleumdungen, Diffamierungen, Drohungen, Diskriminierungen (aufgrund von Herkunft, Nationalität, Religion, sexueller Orientierung, Alter, Geschlecht, etc), Hetze, Gewaltverherrlichungen, Pornographie sowie Vulgärausdrücke sind in unserer Community untersagt.
Ruf- und geschäftsschädigende Inhalte dürfen ebenso wenig verbreitet werden wie Beiträge mit werblichem oder kommerziellem Charakter.

* Die Veröffentlichung persönlicher und personenbezogener Daten anderer ist nicht gestattet.

* Wenn Sie Beiträge verfassen, müssen alle Rechte zur Verbreitung der Inhalte bei Ihnen liegen. Zitate müssen entsprechend gekennzeichnet sein und bedürfen einer Quellenangabe.

* SÜDKURIER Online ist nicht verantwortlich für den Inhalt, die Verfügbarkeit, Richtigkeit oder Genauigkeit von verlinkten Seiten und prüft diese nicht systematisch. Externe Links können umgehend entfernt werden, wenn sie auf rechtswidrige Inhalte weiterleiten beziehungsweise gegen unsere Netiquette verstoßen.
ALLE Beiträge zu ÜBERPRÜFEN ist leider nicht möglich. SÜDKURIER Online kann nicht für die den Inhalt der Leserbeiträge verantwortlich gemacht werden. Die Verantwortung liegt allein beim Verfasser.

#Moderation: (Frage nach Moderation wird nicht eindeutig beantwortet; man greift ein, aber man prüft nicht alles)
#Wir greifen ein, wenn ein Beitrag gegen die oben formulierten Regeln verstößt. Dieser kann komplett gelöscht oder dessen Inhalt editiert werden. Darüber hinaus #entfernen wir auch Beiträge, die zwar nicht explizit gegen eine der Regeln verstoßen, aber dennoch nicht wesentlich zur Diskussion beitragen oder diese gar #stören.

#* NICHT zu allen Artikeln auf suedkurier.de wird die Kommentarfunktion aktiviert. Bei gehäuften Verstößen gegen die Netiquette innerhalb eines Artikels kann die #Kommentarfunktion auch im Nachhinein abgeschaltet werden.

#* Nutzer, die regelmäßig gegen unsere Regeln verstoßen, werden PER EMAIL ERMAHNT. Schwere oder wiederholte Verstöße haben den AUSSCHLUSS aus #der Community zur Folge.

#* Nutzer haben die Möglichkeit, dem Community-Management über das Feld „Melden“ (unterhalb eines Kommentars) einen Verstoß gegen die Netiquette #mitzuteilen.


Allgemeine Nutzungsbedingungen:
#nach 6 Monaten INAKTIVITÄT: Registrierung/Benutzernamen kann gesperrt/freigegeben werden
Nutzungsrechte für südkurier
Garantie der Urheberschaft
honorarfreie Vervielfältigung/Verbreitung/Veröffentlichung
#Einverständnis der Eltern bei Minderjährigen
#LESERREPORTER-BEITRÄGE
Haftungsausschluss: 
    Inhalte, die das Persönlichkeitsrecht anderer verletzen
    Rassistische und diskriminierende Inhalte
    Gewaltdarstellung
    Aufruf zur Gewalt
    Pornografische Schriften, Darstellungen und Bilder
    Informationen / Daten welche Urheberrechte, Schutzrechte oder Immaterialgüterrechte verletzen

Bei Zuwiderhandlung kann SÜDKURIER einen Zugang sperren, löschen oder entsprechende Daten an Behörden zur Strafverfolgung weitergeben. 


Alles klar und übersichtlich; Regeln sind leicht zu finden, man wird darauf hin gewiesen, ganz wenig KOmmentare und folglich auch kein Problem mit Missbrauch; stark regional geprägt 
es ist alles da, was modern ist, es gibt eine community

#``Ihre Meinung ist ein fester, wichtiger und vitaler Bestandteil von SÜDKURIER Online. Wir legen großen Wert auf die Qualität Ihrer Beiträge und wollen Ihnen eine Plattform für offene, faire, sachliche, themenbezogene und gehaltvolle Diskussionen bieten. Unsere Netiquette erklärt die Community-Regeln und gibt Hinweise, wie wir bei Verstößen reagieren.'' OFFENE - FAIRE- SACHLICH THEMENBEZOGEN GEHALTVOLL



