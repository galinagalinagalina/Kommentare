

Es ist noch kein Königsweg gefunden worden, wie man die Moderation von Kommentaren am besten durchführt. Außerdem ist sie von vielen Faktoren abhängig und fällt deswegen bei jeder Zeitung anders aus. In einer Dokumentenanalyse von den meist gelesenen Online-Zeitungen Deutschlands soll eine Übersicht erstellt werden, wie diese Zeitungen mit Kommentaren umgehen. Die Kategorien für die Dokumentenanalyse leiten sich aus den Möglichkeiten des Kommentarmanagements ab. Diese wurden im vorhergehenden Kapitel erläutert und werden nun zusammengefasst. 

Kommentierbarkeit welcher Themen
Richtlinien zum Kommentieren 
Moderation der Kommentare
Prä-Moderation
Post-Moderation
Registrierung
Registrierung freiwillig
Klarnamenpflicht
Melden von Kommentaren
Bewertungen von Kommentaren
Bewertungen von Kommentierenden
Verweise  auf soziale Netzwerke
Möglichkeit zum Weiterverschicken
Ausgliederung der Kommentare






neue Kategorien (alle focus.de)

Kommentare abonnieren (focus)
Netiquette (focus)
AGBs (focus)
Keine Moderation/stichprobenartige Überprüfung (focus)
Rechteeinräumung
Fremdverwaltung (vom Digitalvermarkter)
verschiedene Buttons (Missbrauch melden, positiv bewerten, negativ bewerten, Antwort schreiben

focus.de

Registrierung
