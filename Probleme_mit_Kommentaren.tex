\chapter{Probleme mit Kommentaren} \label{kap:probleme}

Die Nutzerbeteiligung „Kommentare“ führt allerdings nicht nur zu einer
Bereicherung innerhalb der Medien. So hat sueddeutsche.de die Kommentarfunktion
auf ihren Internetseiten entfernt bzw. ausgelagert. Dabei entsteht der Eindruck,
dass die Nachteile die Vorteile überwiegen. Die Entscheidung ist in der
Redaktion sicherlich nicht ohne heftige Diskussion gefallen. Am Ende wurden doch
die nachteiligen Erscheinungen der Kommentare für schwerwiegender empfunden. Was
jedoch wiederum für die Kommentare spricht, ist, dass sie nicht ganz gelöscht,
sondern eben nur auf andere Plattformen verschoben wurden. Ausgewählte
Kommentare schaffen es sogar auf die Nachrichtenseiten. Außerdem kann den
Nachteilen mit entsprechenden Maßnahmen entgegen gewirkt werden. Wie diese
Maßnahmen aussehen, wird dann im darauffolgenden Kapitel erklärt.

Was die Probleme genau sind, soll nun betrachtet werden.

\section{Kommentarmanagement ist zeitaufwändig}

Es wurde bereits erwähnt, dass die Kommentare Überhand nehmen und nicht mehr
bewältigt werden können. Manchmal schafft es der Autor nicht, die Kommentare zu
seinem eigenen Artikel zu lesen, geschweige denn, die seiner Mitstreiter. Es
werden in manchen Redaktionen externe Mitarbeiter angestellt, die sich
ausschließlich mit den Kommentaren beschäftigen. Das belastet die Redaktionen
zusätzlich: \glqq[\ldots] the overall volume of user input demanding to be filtered
or factchecked was a burden.\grqq{} \autocite[S.~172]{quandt}


\section{Schlechte Qualität der Kommentare und „incivilty“} \label{sec:schlecht}

Es ist nicht nur die reine Menge der Kommentare, sondern auch deren mindere
Qualität, die bei den Journalisten schlecht ankommt und Arbeit verursacht. Die
Kommentare müssen gesichtet und sortiert werden. \glqq Very few of them make
intelligent comments or have intelligent things to say. It's very deceiving
[\ldots]\grqq\- \autocite[S.~ 103]{reich}. Und dabei schreiben und lesen sowieso nur
sehr wenige Nutzer Kommentare, wie in Abschnitt \ref{sec:beliebtheit}
bemerkt!

Wenn die Beiträge einfach nur schlecht sind, dann hinterlässt das sicherlich
keinen guten Eindruck und die Qualität der Zeitung leidet insgesamt. Aber
Kommentare arten teilweise aus und es kommt zu beleidigenden oder rassistischen
Ausdrücken und irrationalen Argumenten. Die Kommentierenden erniedrigen und
schuldigen an \autocite[S.~103]{reich}.

Diese Verstöße gegen einen angemessenen Diskussionston (\glqq discursive
civility\grqq{} versus \glqq discursive incivility\grqq\footnote{Hwang defines
  \glqq discursive civility\grqq{} as arguing the justice of one's own view while
  admitting and respecting the justice of others' views. Conversely,
  \glqq discursive incivility\grqq{} is defined as expressing disagreement that denies
and disrespects the justice of others' views \autocite{hwang}
\autocite[S.~6/7]{santana:2014}}) werden durch die Anonymität des Internets
gefördert. Die Hemmschwelle etwas zu sagen, das unter die Gürtellinie zielt,
sinkt ganz einfach, wenn man sich nicht zu erkennen geben muss. Andererseits ist
gerade diese Anonymität der Schlüsselfaktor, der zu mehr Beteiligung am Diskurs
führt, siehe Abschnitt \ref{kap:nutzerbeteiligung}.

Selten wird nicht nur gegen den guten Ton verstoßen, sondern auch gegen das
Gesetz. Hassrede zum Beispiel ist ein bekanntes Problem und in Deutschland
verboten.


\section{Heikle Themen} \label{abschnitt:themen}

Es gibt bestimmte Themen, die sehr kontrovers sind und die Gemüter erhitzen,
weil verhärtete entgegengesetzte Meinungen aufeinandertreffen. \glqq Certain
news stories do not receive even a single comment, but those that do [\ldots]
could provide a peek into `the community’s heartbeat'.\grqq{}
\autocite[S.~181]{loke} Die Zeitungen beobachten Themengebiete, bei denen es
immer wieder zu „discursive incivility” der Kommentierenden kommt. Dazu gehören
vor allem sämtliche Angelegenheiten über Religion\footnote{Ein aktuelles
Beispiel ist der {\slshape shitstorm}, mit dem sich Christiane Florin
auseindersetzen musste, weil sie eine Anzeige in einem christlichen Blatt
ablehnte. Es ging „nur“ um eine Anzeige und die Journalistin hat die Absage auch
begründet. Trotzdem erreichten sie hasserfüllte Leserbriefe, die sie schließlich
öffentlich gemacht hat \autocite{christundwelt}.} und Rasse, Einwanderung und
soziale Themen, obwohl gerade in diesen Bereichen ein öffentlicher Diskurs und
verschiedene Meinungen wichtig sind! Sie sind notwendig, um sämtliche Aspekte
abzubilden. Sonst greift auch hier die Schweigespirale.\footnote{„[\ldots]
would-be commenters might feel remiss to register their opinion if they perceive
themselves to hold a minority opinion [\ldots]“ \autocite[S.~12]{santana:2014}}

Verschiedene Redaktionen reagieren darauf, indem sie zu Themen, mit denen sie
schlechte Erfahrungen gemacht haben, keine Kommentarfunktion mehr zu lassen oder
sie schließen, wenn die Reaktionen zu hitzig werden
\autocite[S.~4]{santana:2014}. Dem gegenüber gibt es Beobachtungen, wie die von
\textcite{loke}\footnote{\glqq Loke found that online discussions of culturally
sensitive issues, such as race, have allowed newsreaders the opportunity to
amplify socially regressive views where they would not have previously. She
argues that despite the hateful dialogue that some sensitive topics attract, the
forums still serve a useful function in allowing the public, as well as
journalists, to get a glimpse into the consciousness of the community, however
unappealing.\grqq\- \autocite[S.~12]{santana:2014}}, dass auch bei kontroversen
Themen die Leser aus den Diskussionen etwas über die Gegner mitnehmen und ihren
Blick auf die andere Seite öffnen. Loke tritt dafür ein, dass alle Kommentare
veröffentlicht werden, weil alle in gewisser Weise etwas aussagen.

In Deutschland jedenfalls wünschen sich die Journalisten und Nutzer einen
angemessenen Umgangston, was sich in den Beschwerden beim deutschen Presserat
widerspiegelt. Aus diesem Grund werden Maßnahmen zur Regulierung von Kommentaren
ergriffen. Diese Maßnahmen werden im nächsten Punkt erläutert.

% vim: set ai si et tw=80 sts=2 ts=2 sw=2:
